% !TEX program = xelatex
\def\NN{\mathbb{N}}
\def\RR{\mathbb{R}}
\def\ZZ{\mathbb{Z}}
\def\QQ{\mathbb{Q}}
\def\PP{\mathcal{P}}
\def\SS{\mathcal{S}}
\def\DD{\mathcal{D}}
\def\sub{\setminus}
\def\bld{\mathbf}
\def\lmf{\lim_{n \to \infty}}
% Make the ref command use parenthesis
\let\oldref\ref
\renewcommand{\ref}[1]{(\oldref{#1})}

\newcommand{\ontop}[1]{\overset{\text{#1}}}
\newcommand{\ang}[1]{\langle #1 \rangle}
\newcommand{\pip}[1]{\left| #1 \right|}



% style
\newcommand{\bm}[1]{\displaystyle{#1}}
\def\nl{$ $ \newline}

\ExplSyntaxOn

\NewDocumentCommand{\getenv}{om}
{
  \sys_get_shell:nnN{ kpsewhich ~ --var-value ~ #2 }{}#1
}

\ExplSyntaxOff

%environments

\documentclass{article}
\usepackage[]{amsthm} %lets us use \begin{proof}
\usepackage{amsmath}
\usepackage{mathtools}
\usepackage{enumerate}
\usepackage{xparse}
\usepackage[makeroom]{cancel}
\usepackage[]{amssymb} %gives us the character \varnothing
\usepackage{polyglossia}
\usepackage{fontspec}
% \usepackage[frak=mma]{mathalfa}
\setdefaultlanguage{hebrew}
\setotherlanguage{english}
%\setmainfont{Frank Ruehl CLM}
\setmainfont{David CLM}
\setmonofont{Miriam Mono CLM}
\setsansfont{Simple CLM}
\newfontface\niceee{Brush Script MT}
\DeclarePairedDelimiter\set\{\}
% Use the following if you only want to change the font for Hebrew
%\newfontfamily\hebrewfont[Script=Hebrew]{David CLM}
%\newfontfamily\hebrewfonttt[Script=Hebrew]{Miriam Mono CLM}
%\newfontfamily\hebrewfontsf[Script=Hebrew]{Simple CLM}
\getenv[\ID]{ID}
\newtheorem{lemma}{טענת עזר}
\title{אלגברה לינארית 2 - ממ"ן 12}
\author{אליחי טורקל \ID}
\date\today

\DeclareFontFamily{OT1}{cmrx}{}
\DeclareFontShape{OT1}{cmrx}{m}{n}{<->cmr10}{}
\let\saveLongrightarrow\Longrightarrow
\makeatletter
\renewcommand*{\Longrightarrow}{%
    \mathrel{\rlap{\fontfamily{cmrx}\fontencoding{OT1}\selectfont=}%
    \hphantom{\saveLongrightarrow}%
    \llap{$\m@th\Rightarrow$}}}
\makeatother



%\clearpage %Gives us a page break before the next section. Optional.
%\selectlanguage{english}
	%Section and subsection automatically number unless you put the asterisk next to them.


\begin{document}
	\maketitle %This command prints the title based on information entered above

	\section*{שאלה 1}
	נתונות המטריצות הבאות:
	\begin{align*}
		A_1 = \begin{pmatrix}
			0 & 3i \\
			-3i & 0 \\
		\end{pmatrix},
		A_2 = \begin{pmatrix}
			1 & 2i \\
			0 & 1 \\
		\end{pmatrix},
		A_3 = \begin{pmatrix}
			0 & 1 & 1 \\
			1 & 0 & 1 \\
			1 & 1 & 0
		\end{pmatrix}
	\end{align*}
	\subsection*{סעיף א}
	מצאות את כל המטריצות הנורמליות
	\begin{proof}
		\begin{align*}
			\overline{A_1} = \begin{pmatrix}
				0 & 3i \\
				-3i & 0 \\
			\end{pmatrix},
			\overline{A_2} = \begin{pmatrix}
				1 & 0 \\
				-2i & 1 \\
			\end{pmatrix},
			\overline{A_3} = \begin{pmatrix}
				0 & 1 & 1 \\
				1 & 0 & 1 \\
				1 & 1 & 0
				\end{pmatrix}
		\end{align*}
		ואנו רואים כי $A_1, A_3$ צמודות לעצמן ולכן בהכרח נורמליות. כעת נבדוק את $A_2$:
		\begin{align*}
			&A_2\overline{A_2} =
			\begin{pmatrix}
				1 & 2i \\
				0 & 1 \\
			\end{pmatrix} \cdot
			\begin{pmatrix}
				1 & 0 \\
				-2i & 1 \\
			\end{pmatrix} =
			\begin{pmatrix}
				5 & 2i \\
				-2i & 1 \\
			\end{pmatrix} \\
			&\overline{A_2}A_2 =
			\begin{pmatrix}
				1 & 0 \\
				-2i & 1 \\
			\end{pmatrix} \cdot
			\begin{pmatrix}
				1 & 2i \\
				0 & 1 \\
			\end{pmatrix} =
			\begin{pmatrix}
				1 & 2i \\
				-2i & 5 \\
			\end{pmatrix}
		\end{align*}
		וקיבלנו ש $A_1, A_3$ נורמליות ו $A_2$ לא.
	\end{proof}
	ולכן לפי משפט 3.2.1 המטריצות הנ"ל לכסינות אוניטרית, כעת נמצא את המטריצות המלכסנות אותן:
	\begin{enumerate}
		\item \textbf{בעבור} $\boldsymbol{A_1}$:
		\begin{proof}
			\begin{align*}
				det(tI - A_1)
				= \pip{\begin{pmatrix}
					t & - 3i \\
					3i & t \\
				\end{pmatrix}}
				= t^2 - (3i \cdot (-3i))
				= t^2 - 3^2
				= (t-3)(t+3)
			\end{align*}
			וקיבלנו שהערכים העצמיים של $A_1$ הם $3, -3$ שניהם עם ריבוי אלגברי 1, נמצא כעת את תתי המרחבים שלהם:
			\begin{align*}
				V_3 = \begin{pmatrix}
					3 & - 3i \\
					3i & 3 \\
				\end{pmatrix}
				\overset{R_2 = R_2 - i \cdot R_1}\Rightarrow
				\begin{pmatrix}
					3 & - 3i \\
					0 & 0 \\
				\end{pmatrix}
				\overset{R_1 = \frac{1}{3}R_1}\Rightarrow
				\begin{pmatrix}
					1 & -i \\
					0 & 0 \\
				\end{pmatrix}
			\end{align*}
			וקיבלנו $x - yi = 0$ ומכאן $x = yi$ ז"א $\boxed{V_3 = Sp\set{(i, 1)}}$
			בצורה דומה:
			\begin{align*}
				V_{-3} = \begin{pmatrix}
					-3 & - 3i \\
					3i & -3 \\
				\end{pmatrix}
				\overset{R_2 = R_2 + i \cdot R_1}\Rightarrow
				\begin{pmatrix}
					-3 & - 3i \\
					0 & 0 \\
				\end{pmatrix}
				\overset{R_1 = -\frac{1}{3}R_1}\Rightarrow
				\begin{pmatrix}
					1 & i \\
					0 & 0 \\
				\end{pmatrix}
			\end{align*}
			וקיבלנו: $x + yi = 0$ ומכאן $x = -yi$ ז"א $\boxed{V_{-3} = Sp\set{(-i, 1)}}$, כעת נחשב נורמות:
			\begin{align*}
				&\norm{(i, 1)} = \sqrt{\pip{i}^2 + \pip{1}^2} = \sqrt{1 + 1} = \sqrt{2} \\
				&\norm{(-i, 1)} = \sqrt{\pip{-i}^2 + \pip{1}^2} = \sqrt{1 + 1} = \sqrt{2} \\
				&\ang{(i, 1), (-i, 1)} = \sqrt{i \cdot i + 1 \cdot 1} = 0
			\end{align*}
			וקיבלנו שהוקטורים כבר אורתוגונלים זה לזה (המכפלה הפנימית שלהם היא אפס) והנורמה שלהם שווה ל $\sqrt{2}$ וכעת נציב את הוקטורים המנורמלים כעמודות במטריצה, והיא המטריצה האוניטרית המלכסנת:
			\begin{align*}
				\boxed{
					\begin{pmatrix}
						\frac{i}{\sqrt{2}} & - \frac{i}{\sqrt{2}} \\
						\frac{1}{\sqrt{2}} & \frac{1}{\sqrt{2}} \\
					\end{pmatrix}
				}
			\end{align*}
		\end{proof}

	\item \textbf{בעבור} $\boldsymbol{A_3}$:
	\begin{proof}
		\begin{align*}
			det(tI - A_3)
			&= \pip{\begin{pmatrix}
				t & -1 & -1 \\
				-1 & t & -1 \\
				-1 & -1 & t
			\end{pmatrix}} \\
			&\ontop{לפי שורה 1}=
			t \cdot \pip{\begin{pmatrix}
				t & -1 \\
				-1 & t \\
			\end{pmatrix}}
			+ 1 \cdot \pip{\begin{pmatrix}
				-1 & -1 \\
				-1 & t \\
			\end{pmatrix}}
			- 1 \cdot \pip{\begin{pmatrix}
				-1 & t \\
				-1 & -1
			\end{pmatrix}} \\
			&= t \cdot (t^2 - 1) + 1(-t - 1) - 1(1 + t)
			= t^3 - t -t -1 -1 -t \\
			&= t^3 -3t -2
			= \boxed{(t+1)^2(t-2)}
		\end{align*}
		וקיבלנו את הערכים העצמיים של $A_3$: $-1, 2$, כעת נמצא את תתי המרחבים:
		\begin{align*}
			V_{-1}
			&= \begin{pmatrix}
				-1 & -1 & -1 \\
				-1 & -1 & -1 \\
				-1 & -1 & -1
			\end{pmatrix}
			\overset{R_2 = R_2 + R_1}{\underset{R_3 = R_3 + R_1}\Rightarrow}
			\begin{pmatrix}
				-1 & -1 & -1 \\
				0 & 0 & 0 \\
				0 & 0 & 0
			\end{pmatrix}
			\overset{R_1 = -R_1}\Rightarrow
			\begin{pmatrix}
				1 & 1 & 1 \\
				0 & 0 & 0 \\
				0 & 0 & 0
			\end{pmatrix}
		\end{align*}
		וקיבלנו $x+y+z = 0$ ז"א $x = -y-z$ כעת נוציא מקדמים: \\
		${(-y-z, y, z) \Rightarrow y(-1,1,0) + z(-1,0,1)}$
		ולכן $V_{-1} = Sp\set{(-1,1,0), (-1,0,1)}$
		נשתמש געת בגרם-שמידט ע"מ לבנות בסיס אורתונורמלי:
		\begin{align*}
			&u_1 = (-1,1,0) \\
			&u_2 = (-1,0,1) - \frac{\ang{(-1,0,1), (-1,1,0)}}{\norm{(-1,1,0)}^2} \cdot (-1,1,0)
			= (-1,0,1) - \frac{1}{2} \cdot (-1,1,0)
			= (-\frac{1}{2},-\frac{1}{2},1) \\
			&u_1^{*}
			= \frac{(-1,1,0)}{\norm{(-1,1,0)}}
			= \frac{(-1,1,0)}{\sqrt{2}}
			= \brac{- \frac{1}{\sqrt{2}}, \frac{1}{\sqrt{2}}, 0} \\
			&u_2^{*} = \frac{(-\frac{1}{2},-\frac{1}{2},1)}{\norm{(-\frac{1}{2},-\frac{1}{2},1)}}
			= \frac{(-\frac{1}{2},-\frac{1}{2},1)}{\sqrt{\frac{1}{4} + \frac{1}{4} + 1}}
			= \frac{(-\frac{1}{2},-\frac{1}{2},1)}{\frac{\sqrt{6}}{2}}
			= \brac{-\frac{1}{\sqrt{6}},-\frac{1}{\sqrt{6}},\sqrt{\frac{2}{3}}} \\
			&\text{ Check: Sanity}
			\ang{u_1^{*}, u_2^{*}}
			= \ang{\brac{- \frac{1}{\sqrt{2}}, \frac{1}{\sqrt{2}}, 0}, \brac{-\frac{1}{\sqrt{6}},-\frac{1}{\sqrt{6}},\sqrt{\frac{2}{3}}}}
			= \sqrt{\frac{1}{\sqrt{12}} - \frac{1}{\sqrt{12}} + 0} = 0
		\end{align*}
		וקיבלנו בסיס אורתונורמלי: $\boxed{V_{-1} = Sp\set{\brac{- \frac{1}{\sqrt{2}}, \frac{1}{\sqrt{2}}, 0}, \brac{-\frac{1}{\sqrt{6}},-\frac{1}{\sqrt{6}},\sqrt{\frac{2}{3}}}}}$ \\
		בצורה דומה, נחשב את התת מרחב של הע"ע $2$:
		\begin{align*}
			V_{-1}
			&= \begin{pmatrix}
				2 & -1 & -1 \\
				-1 & 2 & -1 \\
				-1 & -1 & 2
			\end{pmatrix}
			\overset{R_2 = 2R_2 + R_1}{\underset{R_3 = 2R_3 + R_1}\Rightarrow}
			\begin{pmatrix}
				2 & -1 & -1 \\
				0 & 3 & -3 \\
				0 & -3 & 3
			\end{pmatrix}
			\overset{R_3 = R_3 + R_2}\Rightarrow
			\begin{pmatrix}
				2 & -1 & -1 \\
				0 & 3 & -3 \\
				0 & 0 & 0
			\end{pmatrix}
		\end{align*}
		וקיבלנו $\begin{cases}
			2x - y - z = 0 \\
			3y - 3z = 0
		\end{cases}$
		ומכאן נובע ש $y = z$, $x = \frac{y+z}{2} = y$ ז"א: $\boxed{V_2 = Sp\set{(1,1,1)}}$
		זהו וקטור יחיד ולכן קבוצה אורתוגונלית, כעת ננרמל אותו $u_1 = \frac{(1,1,1)}{\norm{(1,1,1)}} = \frac{(1,1,1)}{\sqrt{3}} = (\frac{1}{3}, \frac{1}{3}, \frac{1}{3})$.
		נשים את שלושת הוקטורים כעמודים ע"מ למצוא את המטריצה המלכסנת:
		\begin{align*}
			\boxed{
				\begin{pmatrix}
					-\frac{1}{\sqrt{2}} & - \frac{1}{\sqrt{6}} & \frac{1}{\sqrt{3}}\\
					\frac{1}{\sqrt{2}} & - \frac{1}{\sqrt{6}} & \frac{1}{\sqrt{3}}\\
					0 & \sqrt{\frac{2}{3}} & \frac{1}{\sqrt{3}}
				\end{pmatrix}
			}
		\end{align*}
	\end{proof}
\end{enumerate}

	\pagebreak
	\subsection*{סעיף ב}
	מצאו את כל המטריצות האוניטריות
	\begin{proof}
		$A_2$ לא יכולה להיות אוניטרית מכיוון שאינה נורמלית.
		ממשפט 2.3.6 שורות מטריצה אוניטרית צריכות הינן בסיס אורתונורמלי של $\FF^n$ אז נבדוק את הנורמה של השורה הראשונה של $A_1$:
		\[
		\norm{(0, 3i)} = \sqrt{0 \cdot \overline{0} + 3i \cdot \overline{3i}}
		= \sqrt{\pip{3i}^2} = \pip{3i} = \sqrt{0^2 + 3^2} = 3
		\]
		וקיבלנו שוקטור השורה הראשון של $A_1$ הוא בעל נורמה 3 ולכן לא אורתונורמלי. \\
		בצורה דומה נבדוק את השורה הראשונה של $A_3$
		\[
		\norm{(0,1,1)} = \sqrt{0^2 + 1^2 + 1^2} = \sqrt{2}
		\]
		וקיבלנו שהנורמה של השורה הראשונה אינה אחת ולכן השורות אינן מהוות בסיס אורתונורמלי ל $\FF^3$ ולכן המטריצה אינה אוניטרית.
	\end{proof}

	\subsection*{סעיף ג}
	מצאו את כל המטריצות החיוביות לחלוטין
	\begin{proof}
		המטריצה $A_2$ לא צמודה לעצמה ולכן לא יכולה להיות חיובית לחלוטין.
		ממשפט 3.3.2 אנו יודעים שמטריצה היא חיובית לחלוטין אם"ם כל הערכים העצמיים שלה חיוביים,
		אך הראינו כבר כי $-3$ הוא ע"ע של $A_1$ ו $-1$ הוא ע"ע של $A_3$ ולכן שתיהן לא חיובית לחלוטין.
	\end{proof}
	\pagebreak

	\section*{שאלה 2}
	יהי $V$ מרחב אוניטרי ממימד סופי. תהי $T: V \rightarrow V$ העתקה לינארית צמודה לעצמה ואוניטרית.
	הוכיחו שקיים $W$ תת מרחב של $V$ כך ש $T(v) = v'' -v'$ לכל $v \in V$ המקיים $v = v' + v''$ כאשר $v' \in W$, $v'' \in W^{\perp}$.
	\begin{proof}
		נתון ש $T$ צמודה לעצמה, ז"א $T = T^{\ast}$ וגם אוניטרית ז"א $TT^{\ast} = I$ וביחד $T^2 = I$. \\
		ולכן $T$ היא גם נורמלית.
		אנו צריכים להוכיח כי
		\[T(v) = T(v' + v'') = T(v') + T(v'') = v'' - v'\]
		ממשפט 2.4.3 נובע כי הערכים העצמיים של $T$ הם: $\pm 1$, ולכן נסמן, $W = V_1$ ז"א המרחב העצמי של הע"ע 1.
		נסמן $T(v') = -v'$ ז"א $v' \in W=V_1$, וגם $T(v'') = v''$ ז"א $v'' \in V_2$
		 וממשפט 3.4.1 נובע כי $V_1 \perp V_{-1}$ ז"א $V_{-1} \subseteq W^{\ast}$
		 ולכן מתקיים $T(v') + T(v'') = -v' + v'' = v'' - v'$, כמובן שאם $1$ או $-1$ אינם ע"ע של $T$ אזי המרחב המתאים יהיה ריק והוקטור המתאים יהיה אפס.
		 אך לא יכול להיות ששניהם יהיו אפס כי אז נקבל ש $T$ אינה לכסינה, בסתירה למשפט 3.2.1. ולכן לפחות אחד מ $V_1 \lor V_{-1}$ אינם ריקים. \\
		 ובסה"כ הראינו שקיים מרחב $W = V_1$ כך ש $v' \in W$, $v'' \in V_{-1} \subseteq W^{\ast}$ \\
		 כך ש $T(v' + v'') = v'' - v'$ כנדרש.
	\end{proof}


	\section*{שאלה 3}
	תהי $A \in M^{\RR}_{n\times n}$ הוכיחו כי: $M = 2A^tA + 3AA^t$ ניתנת לליכסון, וכי הערכים העצמיים שלה הם מספרים ממשיים אי שליליים.

	\begin{proof}
		נתחיל בלהראות כי $M$ סימטרית:
		\[
		M^t
		= (2A^tA + 3AA^t)^t
		= 2(A^tA)^t + 3(AA^t)^t
		= 2(A^t)^tA^t + 3(A^t)^tA^t
		= 2AA^t + 3AA^t
		= M
		\]
		ולכן ממשפט 3.2.1 נקבל כי $M$ לכסינה. ובנוסף נקבל כי $A$ נורמלית (מכיוון שהיא סימטרית וממשית ולכן צמודה לעצמה)
		ולכן ממשפט 3.2.6 וקטורים עצמיים של $M$ השייכים לע"ע שונים הם אורתוגונלים זה לזה.
		יהי $v \neq 0$ וקטור עצמי של $M$ ויהי $t \in \RR$ ע"ע של $M$, כך ש $Mv = tv$, אנו בעצם צריכים להוכיח כי $t \geq 0$:
		\begin{align*}
			t \ang{v,v}
			&= \ang{tv, v}
			= \ang{Mv, v}
			= \ang{(2A^tA + 3AA^t)v, v} \\
			&= \ang{(2A^tA)v, v} + \ang{(3AA^t)v, v}
			= 2\ang{A^t(Av), v} + 3\ang{A(A^t)v, v} \\
			&\ontop{צמודה לעצמה}= 2\ang{Av, A^tv} + 3\ang{A^tv, Av}
			\ontop{סימטרית}= 2\ang{Av, Av} + 3\ang{Av, Av} \ontop{חיוביות}\geq 0
		\end{align*}
		ז"א שקיבלנו כי $t \ang{v,v} \geq 0$ אך הגדרנו $v \neq 0$ ומתכונת החיוביות $\ang{v,v} > 0$ ולכן $t \geq 0$ כנדרש.
	\end{proof}

	\pagebreak
	\section*{שאלה 4}
	תהי $U$ מטריצה אוניטרית מסדר $n \times n$ כך ש $U + iI$ צמודה לעצמה.
	הוכיחו כי $U = -iI$
	\begin{proof}
		נסמן $V = U + iI$, נתון לנו כי היא צמודה לעצמה, ז"א $V = V^{\ast}$
		אנו צריכים להראות כי: $U = -iI$ ז"א $U + iI = 0$ ז"א $V = 0$:
		\begin{align*}
			&V = V^{\ast} \\
			&\Rightarrow U + iI = (U + iI)^{\ast} \\
			&\Rightarrow U + iI = U^{\ast} + (iI)^{\ast} \\
			&\Rightarrow U + iI = U^{\ast} - iI
			&&\big\backslash \cdot U \\
			&\Rightarrow (U + iI)U = (U^{\ast} - iI)U \\
			&\Rightarrow U^2 + iU = U^{\ast}U - iU \\
			&\Rightarrow U^2 + iU = I - iU \\
			&\Rightarrow U^2 + 2iU - I = 0 \\
			&\Rightarrow (U + iI)^2 = 0 \\
			&\Rightarrow V^2 = 0
			&&\big\backslash \text{צמודה לעצמה} \\
			&\Rightarrow V V^{\ast} = 0
		\end{align*}
		כעת נראה שלכל $x \in \CC^n$ לפי תכונות המכפלה הפנימית מתקיים והמטריצה הצמודה:
		$\ang{Vx, Vx} = \ang{x, V^{\ast}Vx} = \ang{x, 0x} = \ang{x, 0} = 0$
		ולכן לפי תכונת החיוביות של המכפלה הפנימית נובע כי אם $\ang{Vx, Vx} = 0$ אזי $Vx=0$,
		וזה מתקיים לכל $x \in \CC^n$ שזה יכול להתקיים אם"ם $V=0$ כנדרש.
	\end{proof}

	\pagebreak
	\section*{שאלה 5}
	תהי $A \in S^{\RR}_{n \times n} \subset M^{\RR}_{n \times n}$ ויהי $\lambda$ הע"ע המקסימאלי של $A$

	\subsection*{סעיף א}
	הוכיחו כי לכל $0 \neq x \in \RR^n$ מתקיים $\frac{\ang{Ax, x}}{\ang{x,x}} \leq \lambda$
	\begin{proof}
		ממשפט 3.2.1 נובע שמכיוון ש $A$ סימטרית אזי היא לכסינה אורתוגונלית, וממשפט 3.1.2 קיים בסיס אורתונורמלי $B = \set{b_1, \cdots, b_n}$ של וקטורים עצמיים, כך ש $Sp B = \RR^n$
		ז"א ש $Ab_i = \lambda_i b_i$.
		יהי $0 \neq x \in \RR^n$ קיימים $n$ מקדמים כך ש $x = \sum_{i=1}^n \alpha_i b_i$, ומתקיים:
		\begin{align*}
			\frac{\ang{Ax, x}}{x,x}
			&= \frac{\ang{A\sum_{i=1}^n \alpha_i b_i, x }}{\ang{x,x}}
			\overset{eigenvalues}= \frac{\ang{\sum_{i=1}^n \lambda_i\alpha_i b_i, x }}{\ang{x,x}}
			\overset{linearity}= \frac{\sum_{i=1}^n \ang{ \lambda_i\alpha_i b_i, x}}{\ang{x,x}} \\
			&= \frac{\sum_{i=1}^n \ang{ \lambda_i\alpha_i b_i, \sum_{j=1}^n \alpha_j b_j}}{\ang{x,x}}
			\overset{hermite}= \frac{\sum_{i=1}^n \sum_{j=1}^n \ang{ \lambda_i\alpha_i b_i,  \alpha_j b_j}}{\ang{x,x}}
			= \frac{\sum_{i=1}^n \sum_{j=1}^n \lambda_i\alpha_i\alpha_j\ang{b_i , \alpha_j b_j}}{\ang{x,x}}
		\end{align*}
		($\dagger$)	כעת מכיוון ש $B$ הינו אורתונורמלי, אזי לכל $j \neq i$ מתקיים $\ang{b_i, b_j} = 0$ ולכל $j=i$ מתקיים $\ang{b_i, b_j} = 1$ ולכן:
		\begin{align*}
			\frac{\ang{Ax, x}}{x,x}
			&= \frac{\sum_{i=1}^n \sum_{j=1}^n \lambda_i\alpha_i\alpha_j\ang{b_i , \alpha_j b_j}}{\ang{x,x}}
			= \frac{\sum_{i=1}^n \lambda_i\alpha_i^2}{\ang{x,x}}
			= \frac{\sum_{i=1}^n \lambda_i\alpha_i^2}{\ang{\sum_{i=1}^n \alpha_i b_i,\sum_{j=1}^n \alpha_j b_j}} \\
			&= \frac{\sum_{i=1}^n \lambda_i\alpha_i^2}{\sum_{i=1}^n\ang{\alpha_i b_i,\sum_{j=1}^n \alpha_j b_j}}
			= \frac{\sum_{i=1}^n \lambda_i\alpha_i^2}{\sum_{i=1}^n\sum_{j=1}^n\ang{\alpha_i b_i, \alpha_j b_j}}
			= \frac{\sum_{i=1}^n \lambda_i\alpha_i^2}{\sum_{i=1}^n\sum_{j=1}^n\alpha_i \alpha_j \ang{ b_i,  b_j}} \\
			&\overset{\dagger}= \frac{\sum_{i=1}^n \lambda_i\alpha_i^2}{\sum_{i=1}^n\alpha_i^2}
			\overset{\lambda = max\set{\lambda_i}, \alpha^2 \geq 0}\leq  \frac{\sum_{i=1}^n \lambda\alpha_i^2}{\sum_{i=1}^n\alpha_i^2}
			=  \lambda\frac{\sum_{i=1}^n \alpha_i^2}{\sum_{i=1}^n\alpha_i^2}
			=  \lambda
		\end{align*}
		ובסה"כ קיבלנו ש $\frac{\ang{Ax, x}}{x,x} \leq \lambda$.
	\end{proof}

	\subsection*{סעיף ב}
	הוכיחו כי לכל $0 \neq x \in \RR^n$ המקיים $\frac{\ang{Ax, x}}{\ang{x,x}} = \lambda$ הוא וקטור עצמי של A.
	\begin{proof}
		בסעיף הקודם הגענו לנוסחה:
		\begin{align*}
			\frac{\ang{Ax, x}}{x,x}
			= \frac{\sum_{i=1}^n \lambda_i\alpha_i^2}{\sum_{i=1}^n\alpha_i^2}
			\leq  \frac{\sum_{i=1}^n \lambda\alpha_i^2}{\sum_{i=1}^n\alpha_i^2}
			\Rightarrow \ang{Ax, x} \leq \sum_{i=1}^n \lambda\alpha_i^2
		\end{align*}
		ניתן לראות כי האי שוויון הופך לשוויון כאשר: $\sum_{i=1}^n \lambda\alpha_i^2= \sum_{i=1}^n \lambda_i\alpha_i^2$ ז"א כאשר:
		\[
			\sum_{i=1}^n \lambda\alpha_i^2 - \sum_{i=1}^n \lambda_i\alpha_i^2 = 0
			\Rightarrow \sum_{i=1}^n \alpha_i^2(\lambda - \lambda_i) = 0
		\]
		כלומר לכל $\alpha_i \neq 0$ מתקיים $\lambda_i = \lambda$ ז"א ש $x$ הוא סכום של וקטורי $\lambda \cdot b_i$ ולכן שייך למרחב העצמי $V_\lambda$
	\end{proof}
\end{document}
