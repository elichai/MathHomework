% !TEX program = xelatex
\def\NN{\mathbb{N}}
\def\RR{\mathbb{R}}
\def\PP{\mathcal{P}}
\def\sub{\setminus}

% Make the ref command use parenthesis
\let\oldref\ref
\renewcommand{\ref}[1]{(\oldref{#1})}



% style
\newcommand{\bm}[1]{\displaystyle{#1}}
\def\nl{$ $ \newline}

\ExplSyntaxOn

\NewDocumentCommand{\getenv}{om}
{
  \sys_get_shell:nnN{ kpsewhich ~ --var-value ~ #2 }{}#1
}

\ExplSyntaxOff

%environments

\documentclass{article}
\usepackage[]{amsthm} %lets us use \begin{proof}
\usepackage{amsmath}
\usepackage[a4paper, margin=1.1in]{geometry}
\usepackage{mathtools}
\usepackage{enumerate}
\usepackage{xparse}
\usepackage[makeroom]{cancel}
\usepackage[]{amssymb} %gives us the character \varnothing
\usepackage{polyglossia}
\usepackage{fontspec}
% \usepackage[frak=mma]{mathalfa}
\setdefaultlanguage{hebrew}
\setotherlanguage{english}
%\setmainfont{Frank Ruehl CLM}
\setmainfont{David CLM}
\setmonofont{Miriam Mono CLM}
\setsansfont{Simple CLM}
\newfontface\niceee{Brush Script MT}
\DeclarePairedDelimiter\set\{\}
% Use the following if you only want to change the font for Hebrew
%\newfontfamily\hebrewfont[Script=Hebrew]{David CLM}
%\newfontfamily\hebrewfonttt[Script=Hebrew]{Miriam Mono CLM}
%\newfontfamily\hebrewfontsf[Script=Hebrew]{Simple CLM}
\getenv[\ID]{ID}
\newtheorem{lemma}{טענת עזר}
\title{אלגברה לינארית 2 - ממ"ן 14}
\author{אליחי טורקל \ID}
\date\today

\DeclareFontFamily{OT1}{cmrx}{}
\DeclareFontShape{OT1}{cmrx}{m}{n}{<->cmr10}{}
\let\saveLongrightarrow\Longrightarrow
\makeatletter
\renewcommand*{\Longrightarrow}{%
    \mathrel{\rlap{\fontfamily{cmrx}\fontencoding{OT1}\selectfont=}%
    \hphantom{\saveLongrightarrow}%
    \llap{$\m@th\Rightarrow$}}}
    \makeatother



%\clearpage %Gives us a page break before the next section. Optional.
%\selectlanguage{english}
%Section and subsection automatically number unless you put the asterisk next to them.


\begin{document}
\maketitle %This command prints the title based on information entered above

\section*{שאלה 1}
\subsection*{סעיף א}
מצאו את הדרגה ואת הסימנית של התבנית הריבועית:
\[
q: \RR^3 \rightarrow \RR, \quad q(x_1, x_2, x_3) = 2x_1^2 + 3x_2^2 - x_3^2 + 2x_1x_2 + 2x_1x_3
\]
מצאו בסיס שבו $q$ בעלת צורה אלכסונית וציינו את הצורה האלכסונית.
האם קיים בסיס שבו $q$ מיוצגת ע"י מטריצת היחידה? נמקו.
\begin{proof}
    נתחיל בלהשתמש בשיטת לגראנז' ע"מ למצוא תבנית ריבועית:
    \begin{align*}
        q(x_1, x_2, x_3)
        &= 2x_1^2 + 3x_2^2 - x_3^2 + 2x_1x_2 + 2x_1x_3 \\
        &= 2\brac{x_1 + \frac{x_2}{2} + \frac{x_3}{2}}^2 - \frac{x_2^2}{2} - \frac{x_3^2}{2} - x_2x_3 + 3x_2^2 - x_3^2 \\
        &= 2\brac{x_1 + \frac{x_2}{2} + \frac{x_3}{2}}^2 - \frac{5x_2^2}{2} - \frac{3x_3^2}{2} - x_2x_3 \\
        &= 2\brac{x_1 + \frac{x_2}{2} + \frac{x_3}{2}}^2 + \frac{5}{2}\brac{x_2 - \frac{x_3}{5}}^2 - \frac{8x_3^2}{5} \\
        &\begin{cases}
            t_1 = x_1 + \frac{x_2}{2} + \frac{x_3}{2} \\
            t_2 = x_2 - \frac{x_3}{5} \\
            t_3 = x_3
            \end{cases} \Rightarrow
            \begin{cases}
                x_1 = t_1 - \frac{t_2}{2} - \frac{2t_3}{5} \\
                x_2 = t_2 + \frac{t_3}{5} \\
                x_3 = t_3
            \end{cases} \\
            &\Rightarrow q(t_1, t_2, t_3) = 2t_1^2 + \frac{5}{2}t_2^2 - \frac{8}{5}t_3^2
                \end{align*}
                וקיבלנו את מטריצת המעבר הבאה:
                \begin{align*}
                    M^{-1} = \begin{pmatrix}
                        1 & \frac{1}{2} & \frac{1}{2} \\
                        0 & 1 & -\frac{1}{5} \\
                        0 & 0 & 1
                \end{pmatrix},
                M = \begin{pmatrix}
                    1 & -\frac{1}{2} & -\frac{3}{5} \\
                    0 & 1 & \frac{1}{5} \\
                    0 & 0 & 1
                    \end{pmatrix}
        \end{align*}
        והבסיס הינו עמודות המטריצה: $\set{\brac{1,0,0}, \brac{-\frac{1}{2},1,0}, \brac{-\frac{3}{5},\frac{1}{5},1}}$. \\
        ע"מ לבדוק את הסימנית או החתימה של התבנית, נשים לב כי בצורה האלכסונית יש 2 איברים חיוביים ואחד שלילי, ולכן: $\pi=2$, $\nu=1$ \\
        קיבלנו שדרגת $q$ היא $\rho = 2 + 1 = 3$ והסימנית היא: $\sigma = 2 - 1 = 1$
        לפי משפט ההתמדה הסימנית של התבנית נשמרת בבסיסים שונים, מכיוון שהסימנית של מטריצת היחידה היא 3(שלושה איברים חיוביים)
        והסימנית של $q$ היא 1 אזי לא קיים בסיס שבו $q$ מיוצגת ע"י מטריצת היחידה.
    \end{proof}


    \pagebreak
    \subsection*{סעיף ב}
    מצאו תת מרחב של $\RR^3$ ממימד מקסימאלי שעליו $q$ שלילית לחלוטין. נמקו.
    \begin{proof}
        נסתכל על הבסיס שמצאנו וננסה לבנות תת מרחב: $b_1 = \brac{1,0,0}, b_2 = \brac{-\frac{1}{2},1,0}, b_3 = \brac{-\frac{3}{5},\frac{1}{5},1}$ \\
        נתחיל בלהסתכל על $W = Sp\set{b_1, b_2}$ כל $0 \neq v \in W$ שמיוצג לפי הבסיס $B$ הוא בעצם צ"ל של הראשון והשני ז"א $[v]_B = (\lambda_1, \lambda_2, 0)$
        ולכן $q(v) = 2\lambda_1^2 + \frac{5}{2}\lambda_2^2 - \frac{8}{5} \cdot 0^2$ וניתן לראות שבתת מרחב זה $q$ חיובית לחלוטין. \\
        במקביל אם נסתכל על $U = Sp\set{b_3}$ ובצורה דומה כל $0 \neq v \in U$ הוא צ"ל של $v_3$ ז"א $[v]_B = \brac{0,0, \lambda}$
        ולכן $q(v) = 2\cdot 0^2 + \frac{5}{2}\cdot 0^2 - \frac{8}{5} \cdot \lambda^2$ וניתן לראות כי $q$ כעת שלילית לחלוטין. \\
        נשים לב כי $\RR^3 = W \bigoplus U$ ז"א $W$ ו $U$ אורתוגנלים זה לזה, החיתוך שלהם ריק, ואיחודם שווה למרחב $\RR^3$ כולו.
        ולכן אם קיים תת מרחב אחר ממימד גדול יותר מ $U$ שבו $q$ שלילית לחלוטין, אזי כל וקטור בו יהיה סכום של וקטורים מ $U$ ו $W$ אך בראשון $q$ שלילית לחלוטין ובאחרון היא חיובית לחלוטין, בסתירה לזה ש $q$ צריכה להיות שלילית לחלוטין במרחב הזה. \\
        ובסה"כ קיבלנו ש $U = Sp\set{b_1} \subset \RR^3$ הוא תת מרחב ממימד מקסימלי שעליו $q$ שלילית לחלטין.
    \end{proof}

\section*{שאלה 2}
    יהיו $q_1, q_2$ תבניות ריבועיות מעל $\RR^n$ עם מטריצות סימטריות מייצגות $A, B$ בהתאמה.
    הוכיחו כי אם עבור המטריצה $A$ מתקיים:
    $\Delta_1 < 0$ ו $\Delta_{i-1} \cdot \Delta_i < 0$ לכל $i \in [2,n] \cap \ZZ$ אז ל $A$ ול $B$ יש לכסון סימולטאני.

    \begin{proof}
        מכיוון ש $\Delta_{i-1} \cdot \Delta_i < 0$ אזי הסימן של $\Delta_i$ הפוך משל $\Delta_{i-1}$ ולכן לכל $i$ אי זוגי $\Delta_i < 0$ ולכן $i$ זוגי $\Delta_i > 0$. \\
        נסמן $\Delta_0 = 1$ ונקבל ש: $q_1 = \frac{1}{\Delta_1}y_1^2 + \frac{\Delta_1}{\Delta_2}y_2^2 + \dotsb + \frac{\Delta_{n-1}}{\Delta_n}y_n^2$ היא הצגת יעקובי של $q$.
        נשים לב כי כל המקדמים שליליים מכיוון ש $\Delta_i$ מחליף סימנים ו $\Delta_1 < 0$.
        כעת נתאים את הוכחת משפט 6.5.1' למטריצה שלילית לחלוטין: \\ \\
        תהינה $A,B$ מטריצות סימטריות ממשיות מאותו הסדר, כך ש $A$ שלילית לחלוטין. \\
        ממסקנה 6.2.1 נקבל שקיימת מטריצה הפיכה $P$ כך ש $P^tAP = -I$, \\
        נסתכל על המטריצה $S = P^tBP$ גם היא סימטרית וממשית
        ולכן לפי למה 3.2.4 הפולינום האופייני שלה מתפרק לגורמים ממשיים, ז"א הע"עים שלה ממשיים. \\
        ולכן ממשפט 3.2.1 קיימת מטריצה אורתוגונלית $Q$ כך ש $Q^tSQ = S^t(P^tAP)Q = diag \set{\delta_1, \cdots, \delta_n}$, \\
        נסמן $M=PQ$ ונקבל: $M^tBM = diag \set{\delta_1, \cdots, \delta_n}$
        וגם
        \[ M^tAM = (PQ)^tA(PQ) = Q^t (P^tAP)Q = Q^t(-I)Q = -Q^tQ \overset{(\perp)}= -I \]
        במילים אחרות, מכיוון ששתיהן סימטריות ו $A$ שלילית לחלוטין אזי קיימת מטריצה $M$ המלכסנת גם את $A$ וגם את $B$.
    \end{proof}

\pagebreak
\section*{שאלה 3}
    קבעו האם קיים בסיס שבו שתי התבניות הריבועיות $q_1, q_2$ הן בעלות צורה אלכסונית:
    \begin{equation}
        q_1(x_1, x_2) = -x_1^2 + 4x_1x_2 - 5x_2^2, \qquad q_2(x_1, x_2) = x_1^2 + 6x_1x_2 + 5x_2^2
    \end{equation}
    \begin{equation}
        q_1(x_1, x_2) = x_1^2 + x_1x_2 - x_2^2, \qquad q_2(x_1, x_2) = x_1^2 - 2x_1x_2
    \end{equation}

    \begin{proof}
        \begin{enumerate}
            \item נשתמש בשיטת לגראנז' ע"מ למצוא את הצורה האלכסונית:
            \begin{align*}
                q_1(x_1, x_2)
                = -x_1^2 + 4x_1x_2 - 5x_2^2
                = - (x_1 - 2x_2)^2 + 4x_2^2 - 5x_2^2
                = \boxed{- (x_1 - 2x_2)^2 - x_2^2}
            \end{align*}
            וניתן לראות כי כל המקדמים שליליים ולכן $q_1$ שלילית לחלוטין.
            ולפי שאלה 2 נקבל בצורה שקולה למטריצות, שיש לנו 2 מטריצות סימטריות (תבניות בילינאריות) ואחת מהן שלילית לחלוטין.
            ולכן קיימת מטריצה $M$ שמלכסנת את שתיהן,
            או במילים אחרות קיים בסיס בו שתיהן בעלות צורה אלכסונית.

            \item נציב משתנים ע"מ להביא את שתיהן לצורות ריבועיות לפי אותו הבסיס:
            \begin{align*}
                &\begin{cases}
                    q_1(x_1, x_2) = x_1^2 + x_1x_2 - x_2^2 \\
                    q_2(x_1, x_2) = x_1^2 - 2x_1x_2
                \end{cases},
                \begin{cases}
                    t_1 = x_1 - x_2 \\
                    t_2 = x_2
                \end{cases} \Rightarrow
                \begin{cases}
                    x_1 = t_1 + t_2 \\
                    x_2 = t_2
                \end{cases} \\
                \Rightarrow
                &\begin{cases}
                    q_1(t_1, t_2) = (t_1 + t_2)^2 + (t_1 + t_2)t_2 - t_2^2 \\
                    q_2(t_1, t_2) = (t_1 + t_2)^2 - 2(t_1 + t_2)t_2
                \end{cases}
                = \begin{cases}
                    q_1(t_1, t_2) = t_1^2 + 2t_1t_2 + t_2^2 + t_1t_2 + t_2^2 - t_2^2 \\
                    q_2(t_1, t_2) = t_1^2 + 2t_1t_2 + t_2^2 - 2t_1t_2 - 2t_2^2
                \end{cases} \\
                = &\begin{cases}
                    q_1(t_1, t_2) = t_1^2 + t_2^2 + 3t_1t_2 \\
                    q_2(t_1, t_2) = t_1^2 - t_2^2
                \end{cases}
            \end{align*}
            ולכן קיבלנו את המטריצות המייצגות הסימטריות הבאות:
            \begin{align*}
                A = [q_1] =
                \begin{pmatrix}
                    1 & \frac{3}{2} \\
                    \frac{3}{2} & 1
                \end{pmatrix},
                B = [q_2] =
                \begin{pmatrix}
                    1 & 0 \\
                    0 & -1
                \end{pmatrix}
            \end{align*}
            כעת נניח כי קיים בסיס שבו יש להן צורות אלכסוניות, ז"א קיימת מטריצה הפיכה $M$ שמלכסנת את שתיהן.
            כעת נלכסן אותן ונקבל:
            \begin{align*}
                M^tAM &=
                \begin{pmatrix}
                    \lambda_1 & M_{11}M_{12} + M_{21}M_{22} + \frac{3}{2}(M_{12}M_{21} + M_{11}M{22}) \\
                    \frac{3}{2}(M_{12}M_{21} + M_{11}M{22}) & \lambda_2
                \end{pmatrix} \\
                M^tBM &=
                \begin{pmatrix}
                    \lambda_3 & M_{11}M_{12} - M_{21}M_{22} \\
                    M_{11}M_{12} - M_{21}M_{22} & \lambda_4
                \end{pmatrix}
            \end{align*}
            מכיוון שהן אלכסוניות נקבל:
            \begin{align*}
                \begin{cases}
                    M_{11}M_{12} - M_{21}M_{22} = 0
                    \Rightarrow M_{11}M_{12} = M_{21}M_{22} \\
                    M_{11}M_{12} + M_{21}M_{22} + \frac{3}{2}(M_{12}M_{21} + M_{11}M_{22})  = 0
                \end{cases}
            \end{align*}
            אם $M_{12} = 0$ אזי $M_{21} = 0 \lor M_{22} = 0$ אם $M_{22} = 0$ אזי ישנה עמודת אפסים,
            ואם $M_{21}=0$ אזי מהמשוואה השניה: $0 = M_{11}M_{22}$ ואנו מקבלים ש $M_{11} = 0 \lor M_{22} = 0$
            וקיבלנו שורת אפסים. שתיהם בסתירה להפיכותה של $M$. ומכאן ש $M_{12} \neq 0$. \\
            מהמשוואה הראשונה נגיע ל $M_{11} = \frac{M_{21}M_{22}}{M_{12}}$ ונציב זאת במשוואה השניה: \\

            כעת, אם $M_{21} = 0$ אזי: $0 = M_{11}M_{22}$ ולכן $M_{22} = 0$ ושוב קיבלנו שורת אפסים וסתירה. ולכן גם $M_{21} \neq 0$. \\
            \[
            0
            = M_{21}M_{22} + M_{21}M_{22} + \frac{3}{2} (M_{12}M_{21} + \frac{M_{21}M_{22}^2}{M_{12}} )
            = M_{21} (2M_{22} + \frac{3}{2}M_{12} + \frac{3}{2}\frac{M_{22}^2}{M_{12}} )
            \]
            אם $M_{21} = 0$ נקבל ש $M_{11}M_{12} = 0 \Rightarrow M_{12} = 0$ (מכיוון ש $M_{11} \neq 0$)
            ובצורה דומה להתחלה נקבל שורת אפסים בסתירה.
            ולכן בהכרח:
            \[
            0 = 2M_{22} + \frac{3}{2}M_{12} + \frac{3}{2}\frac{M_{22}^2}{M_{12}}
            = 2M_{22}M_{12} + \frac{3}{2}M_{12}^2+ \frac{3}{2}M_{22}^2
            = (M_{12} + M_{12})^2 + \frac{1}{2}M_{12}^2+ \frac{1}{2}M_{22}^2
            \]
            כל האיברים ריבועיים ולכן כולם חיוביים, וע"מ שהביטוי יהיה שווה לאפס כל המחוברים שלו הינם אפס
            ובפרט $M_{12} = 0$
            בסתירה לזה שמקודם הראינו כי $M_{12} \neq 0$.
            ומכאן שהן שלא קיים בסיס בו שתי התבניות בעלות צורה אלכסונית.
    \end{enumerate}
    \end{proof}


\section*{שאלה 4}
    מצאו את כל הערכים של הפרמטר $a \in \RR$ שעבורם התבנית $q: \RR^3 \rightarrow \RR$:
    \[ q(x_1, x_2, x_3) = 5x_1^2 + x_2^2 + ax_3^2 + 4x_1x_2 - 2x_1x_3 - 2x_2x_3 \]
    היא חיובית לחלוטין.
    \begin{proof}
        נשתמש בשיטת לגרנז' ע"מ להגיע לצורה אלכסונית:
        \begin{align*}
            q(x_1, x_2, x_3)
            &= 5x_1^2 + x_2^2 + ax_3^2 + 4x_1x_2 - 2x_1x_3 - 2x_2x_3 \\
            &= \frac{1}{5}(5x_1 + 2x_2 - x_3)^2 - \frac{1}{5}(2x_2 - x_3)^2 + x_2^2 + ax_3^2 - 2x_2x_3 \\
            &= \frac{1}{5}(5x_1 + 2x_2 - x_3)^2 - \frac{4}{5}x_2^2 + \frac{4}{5}x_2x_3 - \frac{1}{5}x_3^2 + x_2^2 + ax_3^2 - 2x_2x_3 \\
            &= \frac{1}{5}(5x_1 + 2x_2 - x_3)^2 + \frac{1}{5}x_2^2 - \frac{6}{5}x_2x_3 + (a - \frac{1}{5})x_3^2 \\
            &= \frac{1}{5}(5x_1 + 2x_2 - x_3)^2 + 5(\frac{1}{5}x_2 - \frac{3}{5}x_3)^2 - \frac{9}{5}x_3^2  + (a - \frac{1}{5})x_3^2 \\
            &= \frac{1}{5}(5x_1 + 2x_2 - x_3)^2 + 5(\frac{1}{5}x_2 - \frac{3}{5}x_3)^2 + (a - 2)x_3^2 \\
        \end{align*}
        כעת ניתן לראות כי בצורה האלכסונית 2 המקדמים הראשונים חיוביים, והמקדם השלישי חיובי אם"ם $a > 2$ זה נכון לכל בסיס בו התבנית אלכסונית לפי מסקנה 6.2.1.
        ולכן התבנית חיובית לחלוטין אם"ם $a > 2$.
    \end{proof}

\pagebreak
\section*{שאלה 5}
    תהיינה $\ell_1, \ell_2$ העתקות לינאריות: $\ell_i: \RR^n \rightarrow \RR$ לכל $i = 1,2$.
    נגדיר תבנית ריבועית: $q(x) = l_1(x) \cdot l_2(x)$ מצאו את כל הסימניות האפשריות של $q$.
    \begin{proof}
        הטווח של ההעתקות הללו הוא $\RR$ ולכן מימד התמונה שלהן הוא לכל היותר 1, \\
        ומכאן ש $\dim Ker(\ell_i) \in \set{n, n-1}$
        וקל לראות כי לכל $v \in Ker(\ell_i)$ מתקיים $q(v) = \ell_1(v) \cdot \ell_2(v) = 0$, \\
        ולכן בהכרח $\dim ker(q) \in \set{n, n-1}$ (מימדו $n$ אם"ם קיים $\ell_i \equiv 0$).
        ומכאן ש $\rho = \dim Im(q) \in \set{0,1}$
        ולכן לפי משפט ההתמדה $\pi + \nu \in \set{0,1}$ ומכאן שנקבל שהאופציות לחתימה הן $\sigma \in \set{0, 1, -1}$
    \end{proof}
\end{document}
