% !TEX program = xelatex
\def\NN{\mathbb{N}}
\def\RR{\mathbb{R}}
\def\ZZ{\mathbb{Z}}
\def\QQ{\mathbb{Q}}
\def\PP{\mathcal{P}}
\def\SS{\mathcal{S}}
\def\DD{\mathcal{D}}
\def\sub{\setminus}
\def\bld{\mathbf}
\def\lmf{\lim_{n \to \infty}}
% Make the ref command use parenthesis
\let\oldref\ref
\renewcommand{\ref}[1]{(\oldref{#1})}

\newcommand{\ontop}[1]{\overset{\text{#1}}}
\newcommand{\ang}[1]{\langle #1 \rangle}
\newcommand{\pip}[1]{\left| #1 \right|}



% style
\newcommand{\bm}[1]{\displaystyle{#1}}
\def\nl{$ $ \newline}

\ExplSyntaxOn

\NewDocumentCommand{\getenv}{om}
{
  \sys_get_shell:nnN{ kpsewhich ~ --var-value ~ #2 }{}#1
}

\ExplSyntaxOff

%environments

\documentclass{article}
\usepackage[]{amsthm} %lets us use \begin{proof}
\usepackage{amsmath}
\usepackage{mathtools}
\usepackage{enumerate}
\usepackage{xparse}
\usepackage[makeroom]{cancel}
\usepackage[]{amssymb} %gives us the character \varnothing
\usepackage{polyglossia}
\usepackage{fontspec}
% \usepackage[frak=mma]{mathalfa}
\setdefaultlanguage{hebrew}
\setotherlanguage{english}
%\setmainfont{Frank Ruehl CLM}
\setmainfont{David CLM}
\setmonofont{Miriam Mono CLM}
\setsansfont{Simple CLM}
\newfontface\niceee{Brush Script MT}
\DeclarePairedDelimiter\set\{\}
% Use the following if you only want to change the font for Hebrew
%\newfontfamily\hebrewfont[Script=Hebrew]{David CLM}
%\newfontfamily\hebrewfonttt[Script=Hebrew]{Miriam Mono CLM}
%\newfontfamily\hebrewfontsf[Script=Hebrew]{Simple CLM}
\getenv[\ID]{ID}
\newtheorem{lemma}{טענת עזר}
\title{אלגברה לינארית 2 - ממ"ן 13}
\author{אליחי טורקל \ID}
\date\today

\DeclareFontFamily{OT1}{cmrx}{}
\DeclareFontShape{OT1}{cmrx}{m}{n}{<->cmr10}{}
\let\saveLongrightarrow\Longrightarrow
\makeatletter
\renewcommand*{\Longrightarrow}{%
    \mathrel{\rlap{\fontfamily{cmrx}\fontencoding{OT1}\selectfont=}%
    \hphantom{\saveLongrightarrow}%
    \llap{$\m@th\Rightarrow$}}}
    \makeatother



%\clearpage %Gives us a page break before the next section. Optional.
%\selectlanguage{english}
%Section and subsection automatically number unless you put the asterisk next to them.


\begin{document}
\maketitle %This command prints the title based on information entered above

\section*{שאלה 1}
מצאו בסיס שבו התבנית הריבועית בעל צורה אלכסונית וציין את הצורה האלכסונית.

\subsection*{סעיף א}
\[
q_1: \RR^3 \rightarrow \RR, q_1(x_1, x_2, x_3) = x_1^2 + 4x_2^2 + 3_x3^2 - 8x_1x_2 + 2x_1x_3 - 4x_2x_3
\]
\begin{proof}
    נתחיל בלהשתמש בשיטת לגראנז' ע"מ למצוא תבנית ריבועית:
    \begin{align*}
        q_1(x_1, x_2, x_3)
        &= x_1^2 + 4x_2^2 + 3_x3^2 - 8x_1x_2 + 2x_1x_3 - 4x_2x_3 \\
        &= (x_1 - 4x_2 +2x_3)^2 - (x_3 - 4x_2)^2  + 4x_2^2 + 3_x3^2 - 4x_2x_3 \\
        &= (x_1 - 4x_2 +2x_3)^2 - x_3^2 + 8x_3x_2 - 16x_2^2  + 4x_2^2 + 3_x3^2 - 4x_2x_3 \\
        &= (x_1 - 4x_2 +2x_3)^2 + 2x_3^2 - 12x_2^2 + 4x_2x_3 \\
        &= (x_1 - 4x_2 +2x_3)^2 + 2(x_3+x_2)^2 - 2x_2^2 - 12x_2^2 \\
        &= (x_1 - 4x_2 +2x_3)^2 + 2(x_3+x_2)^2 - 14x_2^2 \\
        &\begin{cases}
            t_1 = x_1 - 4x_2 + x_3 \\
            t_2 = x_3 + x_2 \\
            t_3 = x_2
            \end{cases} \Rightarrow
            \begin{cases}
                x_1 = t_1 - t_2 + 5t_3 \\
                x_2 = t_3 \\
                x_3 = t_2 - t_3
                \end{cases}
                \end{align*}
                וקיבלנו את מטריצת המעבר הבאה:
                \begin{align*}
                    M^{-1} = \begin{pmatrix}
                        1 & -4 & 1 \\
                        0 & 1 & 1 \\
                        0 & 1 & 0
                \end{pmatrix},
                M = \begin{pmatrix}
                    1 & -1 & 5 \\
                    0 & 0 & 1 \\
                    0 & 1 & -1
                    \end{pmatrix}
        \end{align*}
        והבסיס הינו עמודות המטריצה: $\set{(1,0,0), (-1,0,1), (5,1,-1)}$
    \end{proof}

    \pagebreak
    \subsection*{סעיף ב}
    \[
     q_2: \RR^3 \rightarrow \RR, q_2(x_1, x_2, x_3) = x_1 x_2 + x_1 x_3 + x_2 x_3
    \]
    \begin{proof}
        \begin{align*}
            [q_2]_E =
            &\begin{pmatrix}
                0           & \frac{1}{2}   & \frac{1}{2} \\
                \frac{1}{2} & 0             & \frac{1}{2} \\
                \frac{1}{2} & \frac{1}{2}   & 0
            \end{pmatrix}
            \overset{R_1 = R_1 + R_2}\Rightarrow
            \begin{pmatrix}
                \frac{1}{2} & \frac{1}{2}   & 1 \\
                \frac{1}{2} & 0             & \frac{1}{2} \\
                \frac{1}{2} & \frac{1}{2}   & 0
            \end{pmatrix}
            \overset{C_1 = C_1 + C_2}\Rightarrow
            \begin{pmatrix}
                1           & \frac{1}{2}   & 1 \\
                \frac{1}{2} & 0             & \frac{1}{2} \\
                1           & \frac{1}{2}   & 0
            \end{pmatrix} \\
            \overset{R_2 = R_2 - \frac{1}{2}R_1}\Rightarrow
            &\begin{pmatrix}
                1           & \frac{1}{2}   & 1 \\
                0           & -\frac{1}{4}  & 0 \\
                1           & \frac{1}{2}   & 0
            \end{pmatrix}
            \overset{C_2 = C_2 - \frac{1}{2}C_1}\Rightarrow
            \begin{pmatrix}
                1           & 0             & 1 \\
                0           & -\frac{1}{4}  & 0 \\
                1           & 0             & 0
            \end{pmatrix}
            \overset{R_3 = R_3 - R_1}\Rightarrow
            \begin{pmatrix}
                1           & 0             & 1 \\
                0           & -\frac{1}{4}  & 0 \\
                0           & 0             & -1
            \end{pmatrix} \\
            \overset{C_3 = C_3 - C_1}\Rightarrow
            &\begin{pmatrix}
                1           & 0             & 0 \\
                0           & -\frac{1}{4}  & 0 \\
                0           & 0             & -1
            \end{pmatrix}
        \end{align*}
        וקיבלנו שהמטריצה האלכסונית היא:
        \[
        D_{q_2} =             \begin{pmatrix}
            1           & 0             & 0 \\
            0           & -\frac{1}{4}  & 0 \\
            0           & 0             & -1
            \end{pmatrix}
        \]
        כעת נעשה רק את פעולות השורה על מטריצת היחידה ע"מ לקבל את מטריצת המעבר:
        \begin{align*}
            &M^t =
            \begin{pmatrix}
                1 & 0  & 0 \\
                0 & 1  & 0 \\
                0 & 0  & 1
            \end{pmatrix}
            \overset{R_1 = R_1 + R_2}\Rightarrow
            \begin{pmatrix}
                1 & 1  & 0 \\
                0 & 1  & 0 \\
                0 & 0  & 1
            \end{pmatrix}
            \overset{R_2 = R_2 - \frac{1}{2}R_1}\Rightarrow
            \begin{pmatrix}
                1 & 1  & 0 \\
                -\frac{1}{2} & \frac{1}{2}  & 0 \\
                0 & 0  & 1
            \end{pmatrix}
            \overset{R_3 = R_3 - R_1}\Rightarrow
            \begin{pmatrix}
                1 & 1  & 0 \\
                -\frac{1}{2} & \frac{1}{2}  & 0 \\
                -1 & -1  & 1
            \end{pmatrix} \\
            &M =  \begin{pmatrix}
                1 & -\frac{1}{2}  & -1 \\
                1 & \frac{1}{2}   & -1 \\
                0 & 0             & 1
                \end{pmatrix}
        \end{align*}
        ואיברי הבסיס הם עמודות המטריצה המייצגת: $\set{(1,1,0), (-\frac{1}{2}, \frac{1}{2}, 0), (-1, -1, 1)}$
    \end{proof}

    \pagebreak
    \subsection*{סעיף ג}
    יהי $q = q_1 \lor q = q_2$. האם קיים בסיס אורתונורמלי של $\RR^3$ שבו ל $q$ יש צורה אלכסונית?

    \begin{proof}
        מכיוון ששתיהן תבניות לינאריות ריבועיות, אזי קיימת תבנית בילינארית סימטרית המתאימה לה, נסמן את המטריצה המתאימה $A$
        ונקבל שלכל $x \in \RR^3$ מתקיים $q(x) = x^t A x$ וממשפט הליכסון האוניטרי/אוקלידי נקבל כי היא לכסינה אורתוגונלית.
        ז"א קיימת מטריצה:
        \[
        M =  \begin{pmatrix}
            \vdots & \vdots & \vdots\\
            v_1    & v_2    & v_3 \\
            \vdots & \vdots & \vdots
        \end{pmatrix}
        \] כך ש: $M^{-1}AM = M^t A M = D$ כאשר $D$ היא אלכסונית. ז"א $A = M D M^t$
        ממשפט 2.3.6 נובע כי עמודות המטריצה $M$ הן בסיס אורתונורמלי של $\RR^3$, ולכן $B = \set{v_1, v_2, v_3}$
        הינו בסיס אורתונורמלי של $\RR^3$ כעת נסתכל על $[x]_B = M^tx$ וקטור קואורדינטות של $x$ לפי הבסיס $B$
        ונשים לב ש:
        \[
        q(x) = x^t A x = x^t M D M^t x \overset{x^t M = (M^t x)^t}= (M^t x)^t D M^t x = [x]_B^t D [x]_B
         \]
         וקיבלנו ש $q(x) = [x]_B^t D [x]_B$ ז"א קיים בסיס אורתונורמלי שבעבורו ל $q$ יש מטריצה מייצגת אלכסונית
    \end{proof}



    \pagebreak
    \section*{שאלה 2}
    תהי $f: \RR^3 \times \RR^3 \rightarrow \RR$ מוגדרת ע"י:
    \[
    f((x_1, x_2, x_3), (y_1, y_2, y_3)) = 4x_1y_1 + x_2y_2 + x_3y_3 - 2x_1y_2 -2x_2y_1 + 2x_1y_3 + 2x_3y_1 -1.5x_2y_3 - 1.5x_3y_2
    \]

    \subsection*{סעיף א}
    הוכיחו ש $f$ תבנית בילינארית סימטרית. מצאו את התבנית הריבועית $q_f$ המסומכת ל $f$ ומצאו בסיס שבו $f$ מיוצגת ע"י מטריצה אלכסונית.
    מצאו צורה אלכסונית של $q_f$
    \begin{proof}
    נסמן את המטריצה המתאימה:
    \[
    [f]_E = \begin{pmatrix}
    4 & -2 & 2 \\
    -2 & 1 & -\frac{1}{2} \\
    2 & -\frac{1}{2} & 1
    \end{pmatrix}
    \]
    ניתן לראות כי $[f]_E = [f]_E^t$ ז"א המטריצה סימטרית ולכן מ 4.2.2 נובע ש $f$ סימטרית.
    כעת נשתמש בשיטת לגראנז' ע"מ למצוא את התבנית הריבועית:
    \begin{align*}
        &q_f(x_1, x_2, x_3) = f((x_1,x_2,x_3), (x_1,x_2,x_3)) \\
        &= 4x_1^2 + x_2^2 + x_3^2 -2x_1x_2 - 2x_1x_2 + 2x_1x_3 + 2x_3x_1 - 1.5x_2x_3 - 1.5x_3x_2 \\
        &= 4x_1^2 + x_2^2 + x_3^2 - 4x_1x_2 + 4x_1x_3 - 3x_2x_3 \\
        &= (2x_1 - x_2 + x_3)^2 - (x_3 - x_2)^2 + x_2^2 + x_3^2 - 3x_2x_3 \\
        &= (2x_1 - x_2 + x_3)^2 -x_3^2 + 2x_3x_2 -x_2^2 + x_2^2 + x_3^2 - 3x_2x_3 \\
        &= (2x_1 - x_2 + x_3)^2 - x_2x_3 \\
    \end{align*}
    נהפוך את הביטוי האחרון גם הוא לביטוי ריבועי ע"י הצבה:
    \begin{align*}
        \begin{cases}
            x_2 = t_1 - t_2 \\
            x_3 = t_1 + t_2
        \end{cases}
        \Rightarrow
        \begin{cases}
            t_1 = \frac{1}{2}(x_2 + x_3) \\
            x_3 = \frac{1}{2}(x_3 - x_2)
        \end{cases} \\
        x_2x_3 = (t_1 - t_2)(t_1 + t_2) = (t_1^2 - t_2^2) = \frac{1}{4}(x_2 + x_3)^2 - \frac{1}{4}(x_3 - x_2)^2
    \end{align*}
    נציב חזרה ונקבל:
    \[ q_f(x_1, x_2, x_3) = (2x_1 - x_2 + x_3)^2 - \frac{1}{4}(x_2 + x_3)^2 + \frac{1}{4}(x_3 - x_2)^2 \]
    נציב משתנים חדשים:
    \[
    \begin{cases}
        y_1 = 2x_1 - x_2 + x_3 \\
        y_2 = x_2 + x_3 \\
        y_3 = x_3 - x_2
    \end{cases} \Rightarrow
    \begin{cases}
        x_1 = \frac{1}{2}y_1 - \frac{1}{2}y_3 \\
        x_2 = \frac{1}{2}y_2 - \frac{1}{2}y_3 \\
        x_3 = \frac{1}{2}y_2 + \frac{1}{2}y_3
    \end{cases}
    \]
    וקיבלנו את התבנית הריבועית: $q(y_1, y_2, y_3) = y_1^2 - \frac{1}{4}y_2^2 + \frac{1}{4}y_3^2$
    לכל $x_i$ נציב את המקדמים של $y_i$ כשורה במטריצה ונקבל את מטריצת המעבר:
    \[
    M =
    \begin{pmatrix}
        \frac{1}{2} & 0 & - \frac{1}{2} \\
        0 & \frac{1}{2} & - \frac{1}{2} \\
        0 & \frac{1}{2} & \frac{1}{2}
    \end{pmatrix}
    \]
    עמודות המטריצה הם וקטורי הבסיס: $\set{(\frac{1}{2}, 0, 0), (0, \frac{1}{2}, \frac{1}{2}), (-\frac{1}{2}, -\frac{1}{2}, \frac{1}{2})}$
    \end{proof}

    \subsection*{סעיף ב}
    בדקו את נכונות נוסחת המעבר מן הבסיס הסטנדרטי לבסיס שמצאתם בסעיף הקודם
    \begin{proof}
        \begin{align*}
        M^t \cdot A \cdot M &=
        \begin{pmatrix}
            \frac{1}{2} & 0 & 0 \\
            0 & \frac{1}{2} & \frac{1}{2} \\
            -\frac{1}{2} & -\frac{1}{2} & \frac{1}{2}
        \end{pmatrix} \cdot
        \begin{pmatrix}
            4 & -2 & 2 \\
            -2 & 1 & -\frac{1}{2} \\
            2 & -\frac{1}{2} & 1
        \end{pmatrix} \cdot
        \begin{pmatrix}
            \frac{1}{2} & 0 & - \frac{1}{2} \\
            0 & \frac{1}{2} & - \frac{1}{2} \\
            0 & \frac{1}{2} & \frac{1}{2}
        \end{pmatrix} \\
        &=
        \begin{pmatrix}
            2 & -1 & 1 \\
            0 & \frac{1}{4} & \frac{1}{4} \\
            0 & \frac{1}{4} & - \frac{1}{4}
            \end{pmatrix}\cdot
            \begin{pmatrix}
                \frac{1}{2} & 0 & - \frac{1}{2} \\
                0 & \frac{1}{2} & - \frac{1}{2} \\
                0 & \frac{1}{2} & \frac{1}{2}
            \end{pmatrix}
            =
            \begin{pmatrix}
                1 & 0 & 0 \\
                0 &  \frac{1}{4} & 0 \\
                0 & 0 & -\frac{1}{4}
            \end{pmatrix}
        \end{align*}
        וקיבלנו את מקדמי $q_f$ שמצאנו כאיברי האלכסון במטריצה, כנדרש.
    \end{proof}

    \section*{שאלה 3}
    נתון שדה $F$ שמקיים $char F \neq 2$ יהיו $\ell_1, \ell_2$ תבניות לינאריות על מרחב $V$.

    \subsection*{סעיף א}
    נניח כי לכל $v \in V$ מתקיים $\ell_1(v) \ell_2(v) = 0$ הוכיחו ש $\ell_1 = 0 \lor \ell_2 = 0$
    \begin{proof}
    נניח בשלילה כי קיימים $u, w \in V$ כך ש $\ell_1(u) \neq 0 \land \ell_2(w) \neq 0$
    ונתון כי $\ell_1(u)\ell_2(u) = \ell_1(w)\ell_2(u) = 0$ ז"א ש $\ell_1(w) = \ell_2(u) = 0$
    וגם $\ell_1(u+w)\ell_2(u+w) = 0$
    ומלינאריות של $\ell_i$:
    \[
    0 = \ell_1(u+w)\ell_2(u+w)
    = (\ell_1(u) + \ell_1(w))(\ell_2(u) + \ell_2(w))
    = (\ell_1(u) + 0)(0 + \ell_2(w))
    = \ell_1(u)\ell_2(w)
    \]
    וקיבלנו ש $\ell_1(u)\ell_2(w)=0$ ז"א ש $\ell_1(u) = 0 \lor \ell_2(w) = 0$ בסתירה לזה שהגדרנו $\ell_1(u) \neq 0 \land \ell_2(w) \neq 0$.
    ולכן בהכרח $\ell_1 = 0 \lor \ell_2 = 0$.
    \end{proof}

    \subsection*{סעיף ב}
    הוכיחו שאם $f$ תבנית בילינארית אנטי סימטרית ו $f \neq 0$ אז לא יתכן כי $f(u, v) = \ell_1(u) \ell_2(v)$ לכל $u, v \in V$
    \begin{proof}
        נניח בשלילה ש $f(u,v) = \ell_1(u)\ell_2(v)$
    יהי $0 \neq w \in V$ מהאנטי סימטריות אנו מקבלים ש $f(w,w) = -f(w,w) = 0$ ז"א ש $\ell_1(w)\ell_2(w) = 0$
    ומסעיף א נקבל כי $\ell_1 = 0 \lor \ell_2 = 0$ ומכאן ש $f(u,v) = \ell_1(v)\ell_2(u)=0$ בסתירה לנתון ש $f \neq 0$.
    \end{proof}


    \pagebreak
    \section*{שאלה 4}
    תהי $\ell \neq 0$ תבנית ליניארית על מרחב $n$ מימדי $V$.
    הוכיחו כי קיים $B = \set{v_1, \cdots, v_n}$ בסיס של $V$ כך שלכל $v \in V$
    \[
    v = \sum_{i=1}^n \alpha_i v_i \Rightarrow
    \ell(v) = \alpha_1
    \]
    \begin{proof}
    נתון ש $\ell \neq 0 \land Im \ell \subseteq F$ ז"א ש $0 < dim \ell \leq dim F = 1 \Leftrightarrow dim \ell = 1$
    ולכן $dim Ker \ell = n-1$
    יהי $\tilde{v}_1 \in V$ כך ש $\ell(\tilde{v}_1) = \alpha_1 \neq 0$ (קיים כי $dim Im \ell=1$)
    ונסמן $v_1 = \frac{\tilde{v}_1}{\alpha_1}$ מלינאריות נקבל כי:
    \[
    \ell(v_1) = \ell(\frac{\tilde{v}_1}{\alpha_1}) = \frac{1}{\alpha_1} \ell(\tilde{v}_1) = \frac{1}{\alpha_1} \cdot \alpha_1 = 1
    \]
    ובנוסף ניקח $C = \set{v_2, \cdot v_n}$ בסיס כךשהו לקרנל, ונקבל ש: $B = C \cup \set{v_1}$ בסיס ל $V$ וגם:
    \[
    \forall v \in V, \ell(v) = \ell(\sum_{i=1}^n \alpha_i v_i) = \sum_{i=1}^n \alpha_i \ell(v_i) = \alpha_1 \cdot \ell(v1) = \alpha_1 \cdot 1 = \alpha_1
    \]
    $\forall 1 < i \leq n, \ell(v_i) = 0$ מכיוון שהם בקרנל של $\ell$. \\
    וקיבלנו ש $B$ הוא בסיס כנדרש.
    \end{proof}

    \pagebreak
    \section*{שאלה 5}
    תהי $f: \RR^n \times \RR^n \rightarrow \RR$ פונקציה מוגדרת לפי $f(x,y) = n \cdot \sum_{i=1}^n (x_i - m_x)(y_i - m_y)$
    כאשר $x = (x_1, \dotsc, x_n), y = (y_1, \dotsc, y_n), m_x = \frac{x_1 + \dotsc + x_n}{n}, m_y = \frac{y_1 + \dotsc + y_n}{n}$

    \subsection*{סעיף א}
    בדוק שפונקציה זו היא תבנית בילינארית סימטרית
    \begin{proof}
        נשים לב כי $f(x,y) = n(n-1) \cdot cov(x,y)$ כאשר $cov(x,y)$ היא התבנית הבילינארית בשם קווריאנס מעמוד 16.
        באותה דוגמא מראים כי המטריצה המתאימה לתבנית $cov(x,y)$ היא סימטרית, ולכן $cov(x,y) = cov(y,x)$.
        ומכאן ש:
        \[ f(x,y) = n(n-1) \cdot cov(x,y) = n(n-1) \cdot cov(y,x) = f(y,x) \]
        וקיבלנו כי $f$ היא תבנית בילינארית סימטרית.
    \end{proof}

    \subsection*{סעיף ב}
    מצאו את המטריצה המייצגת של $f$ לפי הבסיס הסטנדרטי של $\RR^n$
    \begin{proof}
        לפי אותה דוגמא בספר נתון כי:
    \[
    [cov]_E = \frac{1}{n(n-1)}
    \begin{pmatrix}
        n-1    & -1     & \dotsc & -1 \\
        -1     & n-1    & \dotsc & -1 \\
        \dotsc & \dotsc & \dotsc & \dotsc \\
        -1     & -1     & \dotsc & n-1
    \end{pmatrix}
    \]
    ולכן נקבל ש:
    \[
    [f]_E = n(n-1) \cdot [cov]_E =
    \begin{pmatrix}
        n-1    & -1     & \dotsc & -1 \\
        -1     & n-1    & \dotsc & -1 \\
        \dotsc & \dotsc & \dotsc & \dotsc \\
        -1     & -1     & \dotsc & n-1
    \end{pmatrix}
    \]
    \end{proof}

    \pagebreak
    \subsection*{סעיף ג}
    בדקו ש \( 0 \leq f(x,x) \leq n \norm{x}^2 \) לכל $x \in \RR^n$ האם קיימים $x',x'' \in \RR^n \sub \set{0}$ כך ש
    \(f(x'', x'') = 0, f(x', x') = n \norm{x'}^2\)

    \begin{proof}
        נחשב את הערכים העצמיים של $[f]_E$ בעזרת פולינום אופייני:
            \begin{align*}
                \pip{tI - [f]_E} =
                &\pip{\begin{pmatrix}
                    t + 1 - n  & 1         & \dotsc & 1 \\
                    1          & t + 1 - n & \dotsc & 1 \\
                    \dotsc     & \dotsc    & \dotsc & \dotsc \\
                    1          & 1         & \dotsc & t + 1 - n
                \end{pmatrix}} \\
                \overset{R_1 = \sum_{i=1}^n R_i}\Rightarrow
                &\pip{\begin{pmatrix}
                    t          & t         & \dotsc & t \\
                    1          & t + 1 - n & \dotsc & 1 \\
                    \dotsc     & \dotsc    & \dotsc & \dotsc \\
                    1          & 1         & \dotsc & t + 1 - n
                \end{pmatrix}} =
                t \cdot \pip{\begin{pmatrix}
                    1          & 1         & \dotsc & 1 \\
                    1          & t + 1 - n & \dotsc & 1 \\
                    \dotsc     & \dotsc    & \dotsc & \dotsc \\
                    1          & 1         & \dotsc & t + 1 - n
                \end{pmatrix}} \\
                \overset{\forall 1 < i \leq n, R_i = R_i - R_1}\Rightarrow
                t \cdot &\pip{\begin{pmatrix}
                    1          & 1         & \dotsc & 1 \\
                    0          & t - n & \dotsc & 0 \\
                    \dotsc     & \dotsc    & \dotsc & \dotsc \\
                    0          & 0         & \dotsc & t - n
                \end{pmatrix}}
                \ontop{מטריצה משולשית}= t \cdot 1 \cdot (t-n)^{n-1}
            \end{align*}
            וקיבלנו בעצם כי ישנם 2 ע"ע, $0, n$ כאשר $0$ עם ר"א $1$ ו $n$ עם ר"א $n-1$. \\
            ולכן לפי שאלה 5 בממן הקודם נקבל כי \( \forall x \in \RR^n, \frac{\ang{[f]_E \cdot x, x}}{\ang{x,x}} = \frac{f(x,x)}{\ang{x,x}} \leq n \) \\
            ז"א \( f(x,x) \leq n \ang{x,x} = n \cdot \norm{x}^2 \)
            ובצורה דומה נסתכל על $-f$ ונקבל כי: \\
            \( \forall x \in \RR^n, \frac{-f(x,x)}{\ang{x,x}} \leq 0 \)
            ז"א $-f(x,x) \leq 0 \cdot \ang{x,x} = 0$ ולכן $f(x,x) \geq 0$
            וביחד קיבלנו ש:
            \[
            0 \leq f(x,x) \leq n \cdot \norm{x}^2
            \]
            ויהי $x' \neq 0$ וקטור עצמי של $n$ ו $x'' \neq 0$ וקטור עצמי של $0$ נקבל כי: \\
            \( f(x', x') = \ang{nx', x'} = n \norm{x'}^2 \)
            וגם:
            \( f(x'', x'') = \ang{0 \cdot x'', x''} = 0 \) כנדרש.
    \end{proof}

\end{document}
