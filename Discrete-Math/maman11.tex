% !TEX program = xelatex
\def\NN{\mathbb{N}}
\def\RR{\mathbb{R}}
\def\ZZ{\mathbb{Z}}
\def\QQ{\mathbb{Q}}
\def\PP{\mathcal{P}}
\def\SS{\mathcal{S}}
\def\DD{\mathcal{D}}
\def\sub{\setminus}
\def\bld{\mathbf}
\def\lmf{\lim_{n \to \infty}}
% Make the ref command use parenthesis
\let\oldref\ref
\renewcommand{\ref}[1]{(\oldref{#1})}

\newcommand{\ontop}[1]{\overset{\text{#1}}}
\newcommand{\ang}[1]{\langle #1 \rangle}
\newcommand{\pip}[1]{\left| #1 \right|}



% style
\newcommand{\bm}[1]{\displaystyle{#1}}
\def\nl{$ $ \newline}

\ExplSyntaxOn

\NewDocumentCommand{\getenv}{om}
{
  \sys_get_shell:nnN{ kpsewhich ~ --var-value ~ #2 }{}#1
}

\ExplSyntaxOff

%environments

\documentclass{article}
\usepackage[]{amsthm} %lets us use \begin{proof}
\usepackage{amsmath}
\usepackage{mathtools}
\usepackage{enumerate}
\usepackage{xparse}
\usepackage[makeroom]{cancel}
\usepackage[]{amssymb} %gives us the character \varnothing
\usepackage{polyglossia}
\usepackage{fontspec}
% \usepackage[frak=mma]{mathalfa}
\setdefaultlanguage{hebrew}
\setotherlanguage{english}
%\setmainfont{Frank Ruehl CLM}
\setmainfont{David CLM}
\setmonofont{Miriam Mono CLM}
\setsansfont{Simple CLM}
\newfontface\niceee{Brush Script MT}
\DeclarePairedDelimiter\set\{\}
% Use the following if you only want to change the font for Hebrew
%\newfontfamily\hebrewfont[Script=Hebrew]{David CLM}
%\newfontfamily\hebrewfonttt[Script=Hebrew]{Miriam Mono CLM}
%\newfontfamily\hebrewfontsf[Script=Hebrew]{Simple CLM}
\getenv[\ID]{ID}
\newtheorem{lemma}{טענת עזר}

\title{בדידה - ממ"ן 11}
\author{אליחי טורקל \ID}
\date\today

%\clearpage %Gives us a page break before the next section. Optional.
%\selectlanguage{english}
	%Section and subsection automatically number unless you put the asterisk next to them.

\begin{document}
	\maketitle %This command prints the title based on information entered above

	\section*{שאלה 1}


	\begin{itemize}
		\item[א.]לא נכון.
		\item[ב.]  נכון.
		\item[ג.] נכון.
		\item[ד.]נכון.
		\item[ה.]נכון.
		\item[ו.]לא נכון.
		\item[ז.]לא נכון.
		\item[ח.]לא נכון.
	\end{itemize}

	\pagebreak
	\section*{שאלה 2}
	\subsection*{סעיף א}
	הוכיחו ש $(A \cup B) \sub (C \sub B) = B \cup (A \sub C)$
	\begin{proof}
		נגדיר
		$L = (A \cup B) \sub (C \sub B), \thickspace R = B \cup (A \sub C)$ \\
		ונוכיח $\forall x((x \in L \land x \in R) \lor (x \notin L \land x \notin R))$
		\begin{enumerate}[(I)]
			\item אם $x \in B$ אזי עפ"י הגדרת האיחוד $x \in R$
			וגם $x \in (A \cup B)$ \\
			ואז אם בנוסף $x \in C$ אזי עפ"י הגדרת ההפרש $x \notin (C \sub B)$
			וגם אם $x \notin C$ נקבל ש $x \notin (C \sub B)$
			ולכן $x \in L$.
			\item אם $x \notin B$ אזי עפ"י הגדרת האיחוד $x \notin R$, וגם $x \notin (A \cup B)$
ולכן עפ"י הגדרת ההפרש $x \notin L$.
		\end{enumerate}
	\end{proof}

	\subsection*{סעיף ב}
	הוכיחו ש $\PP(A \sub B) \subseteq (\PP(A) \sub \PP(B)) \cup \set*{\emptyset}$
	\begin{proof}
		נוכיח בדרך השלילה ונניח ש
		\[
			\PP(A \sub B) \not\subseteq (\PP(A) \sub \PP(B)) \cup \set*{\emptyset}
		\]
		אזי עפ"י הגדרת תת קבוצה, מתקבל ש
		\begin{equation} \label{assumption:1}
			\exists x (x \in \PP(A \sub B) \land
					x \notin ((\PP(A) \sub \PP(B)) \cup \set*{\emptyset}))
		\end{equation}
		נסתכל על הצד השמאלי של הנחה
\ref{assumption:1} ונראה שעפ"י הגדרת קבוצת החזקה
		\[
			x \in \PP(A \sub B) \Rightarrow
			x \subseteq (A \sub B)
		\]
		\begin{equation} \label{math:2}
			x \subseteq A \text{ זרות  ו} B  \text{ ו }  x \text{עפ"י הגדרת ההפרש(חיסרנו מ $A$ את כל האיברים של $B$) }
		\end{equation}
עכשיו נסתכל על הצד הימני של הנחה \ref{assumption:1}
		\[
			x \notin ((\PP(A) \sub \PP(B)) \cup \set{\emptyset})
		\]
		עפ"י הגדרת האיחוד $x$  לא נמצא ב $(\PP(A) \sub \PP(B))$ וגם לא ב $\set{\emptyset}$
		ועל כן עפ"י הגדרת ההפרש או ש $x \notin \PP(A)$ או ש $x \in \PP(B)$. \\
 		מכאן נובע עפ"י הגדרת קבוצת החזקה ש $x \not \subseteq A$  או ש $x \subseteq B$
		  וזה בא בסתירה עם \ref{math:2} \\
		   ולכן הזהות חייבת להיות נכונה.
	\end{proof}

	\pagebreak
	\subsection*{סעיף ג}
	הוכיחו שאם $A, B$ קבוצות סופיות אז $|\PP(A)| = |\PP(A \cap B)| \cdot |\PP(A \sub B)|$
	\begin{proof}
		\begin{lemma} \label{lemma:1}
			לכל שתי קבוצות $A, B$ מתקיים $A = (A \cap B) \cup (A \sub B)$ \\
			ובנוסף $(A \cap B)$ ו $(A \sub B)$ זרות זו לזו
			\begin{proof}
		יהי $x \in A$ עפ"י הגדרות החיתוך וההפרש:
				\begin{enumerate}[(I)]
					\item $x \in B \Rightarrow (x \in (A \cap B) \land (x \notin (A \sub B)))$
					\item $x \notin B \Rightarrow (x \notin (A \cap B) \land (x \in (A \sub B)))$
				\end{enumerate}
				ואם $x \notin A$ ברור ש $ x \notin (A \cap B) \land x \notin (A \sub B)$
			\end{proof}
		\end{lemma}
		עפ"י טענת עזר \ref{lemma:1} ומשפט 1.19 נקבל:
		\[ |A| = |A \cap B| + |A \sub B| - |(A \cap B) \cap (A \sub B)| \]
		באותה טענת עזר \ref{lemma:1} הוכחנו גם שהקבוצות זרות, ולכן החיתוך שלהם הוא ריק, אזי
		\[ |A| = |A \cap B| + |A \sub B| \]
			ועל פי משפט 1.6 לכל קבוצה סופית A מתקיים $|\PP(A)| = 2^{|A|}$ \\
		\[
			\PP(A)| = 2^{|A|} = 2^{|(A \cap B)| + |(A \sub B)|}
			= 2^{|(A \cap B)|} \cdot 2^{|(A \sub B)|}
			= |\PP(A \cap B)| \cdot |\PP(A \sub B)|
		\]
	\end{proof}

	\pagebreak
	\section*{שאלה 3}
	יהיו $A, B, C \subseteq U$ לקבוצה אוניברסלית $U$
	\subsection*{סעיף א}
	הוכיחו $|A \Delta B| \geq 2 \Leftarrow (B \cup A^c \neq U \land A \cup B^c \neq U)$
	\begin{proof}
		נראה ש $(|A \sub B| \geq 1) \land (|B \sub A| \geq 1)$  ומאחר שהקבוצות זרות יתקיים $|A \Delta B| \geq 2$
		\begin{enumerate}[(I)]
			\item $ A \cup B^c \neq U \overset{(1)}\Rightarrow
			 (A \cup B^c)^c \neq \emptyset \overset{(2)}\Rightarrow
			 A^c \cap B \neq \emptyset \overset{(3)}\Rightarrow
			 B \sub A \neq \emptyset$

			 \item $ B \cup A^c \neq U \overset{(1)}\Rightarrow
			 (B \cup A^c)^c \neq \emptyset \overset{(2)}\Rightarrow
			   B^c \cap A \neq \emptyset \overset{(3)}\Rightarrow
			   A \sub B \neq \emptyset$
		\end{enumerate}
(1) עפ"י משפט 1.23 \\
(2) עפ"י כללי דה-מורגן (משפט 1.26) \\
(3) עפ"י משפט 1.24
	\end{proof}

	\subsection*{סעיף ב}
	הוכיחו כי
	$A \cap C \subseteq B \subseteq A \cup C \Leftarrow A \Delta B \subseteq A \Delta C$
	\begin{proof}
נחלק ל 2 הוכחות נפרדות:
		\begin{enumerate}[(I)]
			\item נוכיח ש $A \cap C \subseteq B$ בשלילה: \\
			יהי
			$x \notin B \land x \in (A \cap C)$,
			אזי לפי הגדרת החיתוך $x \in A \land x \in C$ \\
			ולכן לפי הגדרת ההפרש הסימטרי $(x \in A \Delta B \Leftarrow x \notin B \land x \in A)$ \\
			וגם $(x \notin A \Delta C \Leftarrow x \in C \land x \in A)$ \\
			אז אנחנו רואים שיש איבר שנמצא ב $A \Delta B$ ולא ב $A \Delta C$ בסתירה עם ההגדרה שלו כתת קבוצה

			\item נוכיח ש $B \subseteq A \cup C$ גם בשלילה: \\
			יהי
			$x \in B \land x \notin (A \cup C)$,
			אזי לפי הגדרת האיחוד $x \notin A \land x \notin C$ \\
			ולכן לפי הגדרת ההפרש הסימטרי $(x \in A \Delta B \Leftarrow x \in B \land x \notin A)$ \\
			וגם $(x \notin A \Delta C \Leftarrow x \notin C \land x \notin A)$ \\
			אז גם כאן אפשר לראות שיש איבר שנמצא ב $A \Delta B$ ולא ב $A \Delta C$ בסתירה עם ההגדרה שלו כתת קבוצה
		\end{enumerate}
	\end{proof}

	\pagebreak
	\subsection*{סעיף ג}
	הוכח כי $A \Delta B = \set{1,3} \Leftarrow A \Delta \set{1,2} = B \Delta \set{2,3}$
	\begin{proof}
		נשתמש במה שהוכחנו בשאלה 23:
		\begin{itemize}
			\item[א.] $A \Delta B = C \Rightarrow A \Delta C = B \land B \Delta C = A$
			\item[ב.] הפרש סימטרי הוא קיבוצי ולכן $(A \Delta B) \Delta C = A \Delta (B \Delta C) $
		\end{itemize}
		ולכן
		\begin{align*}
			A \Delta \set{1,2} = B \Delta \set{2,3}
			&\overset{\text{(א)}}\Rightarrow  A = ( B \Delta \set{1,2})\Delta \set{2,3} \\
			&\overset{\text{(ב)}}\Rightarrow A = B \Delta (\set{1,2} \Delta \set{2,3}) \\
			&\overset{\text{נפשט}}\Rightarrow A = B \Delta \set{1,3} \\
			&\overset{\text{(א)}}\Rightarrow A \Delta B = \set{1,3} \\
		\end{align*}
	\end{proof}

	\pagebreak
	\section*{שאלה 4}
		\[\forall k \in \NN (A_k = \set{2^{nk} | n \in \NN}) \]
		\subsection*{סעיף א}
		מצא $A_?$ ככה ש
		$A_? = \bigcup_{k = 0}^{\infty}A_k$ \\
		תשובה: $A_1$ \\
	נימוק: $A_1 = \set{2^n | n \in \NN}$ וכל איבר בקבוצה $A_?$ ניתן לייצוג ע"י $2^{nk}$, בנוסף $n \in \NN \land k \in \NN$  ולכן גם $nk \in \NN$ מכאן ש $2^{nk} \in A_1$

	\subsection*{סעיף ב}
	מצא $A_?$ ככה ש $A_? = \bigcap_{k=2}^{5}A_k$ \\
תשובה: $A_{60}$ \\
נימוק: החיתוך של הקבוצות $A_2, A_3, A_4, A_5$ הוא קבוצה שבה כל איבר יכול להיות מיוצג גם ע"י $2^{2n}$ וגם $2^{3n}$ וגם $2^{4n}$ וגם $2^{5n}$.
ולכן אנחנו צריכים למצוא את ה $k$ המינמלי שמתחלק בכל המספרים הנ"ל שהוא בעצם המכפלה המשותפת המינימלית(lcm)של הסדרה  .\\
וניתן לראות שאכן: $2^{60kn} = 2^{2(30kn)} = 2^{3(20kn)} = 2^{4(15kn)} = 2^{5(10kn)}$. \\
(ניתן להגיע ל 60 ע"י המשפט היסודי של האריתמטיקה, לראות שכולם ראשוניים חוץ מ 4, ו 4 מתחלק ב 2, אז נוריד את 2 ונכפיל את כל שאר המספרים אחד בשני)

\subsection*{סעיף ג}
מצא $A_?$ ככה ש
$A_? = \bigcap_{k = 1}^{\infty}A_k$ \\
תשובה: $A_0$ \\
נימוק: $A_0 = {1}$ וכל קבוצה $A_k$ כאשר $k > 0$ מכילה גם את 1 וגם עוד איברים.
ולכן האיבר היחיד שקיים בכל הקבוצות $A_k$ הוא 1, ו $A_0$ היא הקבוצה היחידה שמכיליה רק אותו ולכן שווה לאיחוד כולו. \\
בהסתכלות אחרת, שמזכירה את מה שכתבנו בסעיף הקודם, נחפש את הכפולה המשותפת לכל המספרים $2^{kn}$ לכל $k$ טבעי, ונראה שהמספר היחיד שמקיים את זה הוא 0, ולכן $A_? = A_0$.

\subsection*{סעיף ד}
מצא $A_?$ ככה ש $A_? = \set{\frac{x}{8} | x \in (A_1 \sub A_2) \cap A_3}$ \\
תשובה: $A_6$ \\
נימוק: עפ"י הגדרות האיחוד: $x \in (A_1 \sub A_2)$  לפי הגדרות ההפרש אנחנו מקבלים ש $x = 2^d$  כאשר $d$ הוא מספר אי זוגי
בנוסף מתוך הגדרות האיחוד $x \in A_3$ מה שאומר ש $3 | d$ ($3$ מחלק את $k$) \\
בנוסף עקב ההגדרה של $\frac{x}{8}$ אנחנו יכולים לכתוב $x = 2^{k-3}$ \\
כל מספר אי זוגי ניתן לכתיבה בתור $2k + 1$, וכל מספר שמתחלק ב 3 ניתן לכתיבה בתור $3k$, ולכן נגדיר את $d = 3(2k + 1)$ ונקבל את החזקה הבאה: $2^{3(2k+1)-3} = 2^{6k}$
והקבוצה היחידה שכל האיברים שלה ניתנים ליצוג בתור $2^{6k}$ היא $A_6$. \\
\end{document}
