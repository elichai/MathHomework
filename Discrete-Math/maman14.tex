% !TEX program = xelatex
\def\NN{\mathbb{N}}
\def\RR{\mathbb{R}}
\def\ZZ{\mathbb{Z}}
\def\QQ{\mathbb{Q}}
\def\PP{\mathcal{P}}
\def\SS{\mathcal{S}}
\def\DD{\mathcal{D}}
\def\sub{\setminus}
\def\bld{\mathbf}
\def\lmf{\lim_{n \to \infty}}
% Make the ref command use parenthesis
\let\oldref\ref
\renewcommand{\ref}[1]{(\oldref{#1})}

\newcommand{\ontop}[1]{\overset{\text{#1}}}
\newcommand{\ang}[1]{\langle #1 \rangle}
\newcommand{\pip}[1]{\left| #1 \right|}



% style
\newcommand{\bm}[1]{\displaystyle{#1}}
\def\nl{$ $ \newline}

\ExplSyntaxOn

\NewDocumentCommand{\getenv}{om}
{
  \sys_get_shell:nnN{ kpsewhich ~ --var-value ~ #2 }{}#1
}

\ExplSyntaxOff

%environments

\documentclass{article}
\usepackage[]{amsthm} %lets us use \begin{proof}
\usepackage{amsmath}
\usepackage{mathtools}
\usepackage{enumerate}
\usepackage{xparse}
\usepackage[makeroom]{cancel}
\usepackage[]{amssymb} %gives us the chA \mathcal{R} Acter \varnothing
\usepackage{polyglossia}
% \usepackage[frak=mma]{mathalfa}
\setdefaultlanguage{hebrew}
\setotherlanguage{english}
\usepackage{fontspec}
%\setmainfont{Frank Ruehl CLM}
\setmainfont{David CLM}
\setmonofont{Miriam Mono CLM}
\setsansfont{Simple CLM}
\DeclarePairedDelimiter\set\{\}
% Use the following if you only want to change the font for Hebrew
%\newfontfamily\hebrewfont[Script=Hebrew]{David CLM}
%\newfontfamily\hebrewfonttt[Script=Hebrew]{Miriam Mono CLM}
%\newfontfamily\hebrewfontsf[Script=Hebrew]{Simple CLM}
\getenv[\ID]{ID}
\newtheorem{lemma}{טענת עזר}
\newtheorem{remark}{הערה}

\title{בדידה - ממ"ן 14}
\author{אליחי טורקל \ID}
\date\today

%\clearpage %Gives us a page break before the next section. Optional.
%\selectlanguage{english}
	%Section and subsection automatically number unless you put the asterisk next to them.

\begin{document}
	\maketitle %This command prints the title based on information entered above

	\section*{שאלה 1}
	הוכיחו את הזהות:
	\[
	\sum_{k=m}^n {k \choose m} = {n+1 \choose m+1}
	\]
	\subsection*{סעיף א}
	באינדוקציה:
	\begin{proof}
		המשוואה חסרת משמעות כאשר $n<m$ ולכן נגדיר את בסיס האינדוקציה כ $n=m$ ונוכיח לכל $n > m$: \\
		\textbf{בסיס האינדוקציה:} $n=m$
		\begin{align*}
			{m \choose m} &= \frac{\cancel{m!}}{\cancel{m!}(m-m)!} = \frac{1}{1} = 1 \\
			&= {m+1 \choose m+1} = \frac{\cancel{(m+1)!}}{\cancel{(m+1)!}(m+1-(m+1)!)} = \frac{1}{1} = 1
		\end{align*}
		\textbf{הנחת האינדוקציה:} נניח שלכל $n \geq m$ מתקיים $\sum_{k=m}^n {k \choose m} = {n+1 \choose m+1}$ \\
		\textbf{צעד האינדוקציה:} נוכיח לכל $n = n+1$:
		\begin{align*}
			\sum_{k=m}^{n+1} {k \choose m} &=
			{n+1 \choose m} + \sum_{k=m}^{n}{k \choose m} \ontop{הנחת האינדוקציה}=
			{n+1 \choose m} + {n+1 \choose m+1} \\
			&= \frac{(n+1)!}{m!(n-m+1)!} + \frac{(n+1)!}{(m+1)!(n-m)!} \\
			&= \frac{(n+1)!}{m!(n-m+1)(n-m)!} + \frac{(n+1)!}{(m+1)m!(n-m)!} \\
			&= \frac{(n-m+1)(n+1)! + (m+1)(n+1)!}{(m+1)!(n-m+1)!} =
			   \frac{(n-\cancel{m}+1+\cancel{m}+1)(n+1)!}{(m+1)!(n-m+1)!} \\
			&= \frac{(n+2)(n+1)!}{(m+1)!(n-m+1)!} =
			\frac{(n+2)!}{(m+1)!(n+2-(m+1))!} =
			{n+2 \choose m+1}
		\end{align*}
		ובכך הושלם צעד האינדוקציה כנדרש והוכחנו את הזהות לכל $n \geq m$.
	\end{proof}
	\subsection*{סעיף ב}
	קומבינטורית:
	\begin{proof}
		נסתכל על צד ימין של המשוואה, ונפרש אותו בהתאם לבחירת $m+1$ תתי קבוצות מתוך הקבוצה $\set{0,1,\dots, n}$, נסתכל על המספר הגדול ביותר בכל תת קבוצה, נקרא לו $x$,
		בגלל שהקבוצה מתחילה ב $0$ ואנו בוחרים $m+1$ איברים אזי $x \geq m$. \\
		כל תת קבוצה כזו, היא בעצם בחירה של $m$ איברים מתוך הקבוצה $\set{0,1,\dots,x-1}$(מכיוון שבכל תת קבוצה שכזו $x$ נמצא בה עפ"י הגדרה) וככה מתקבלת תת קבוצה של $m+1$ איברים ($m$ האיברים שבחרנו, וגם $x$),
		ולכל $m \leq x \leq n$ כמות תת הקבוצות שבהן $x$ הוא האיבר הגדול ביותר בעצם שקולה ללבחור $m$ איברים מתוך הקבוצה $\set{0,1,\dots,x-1}$(הקבוצה מכילה $x$ איברים) ולהוסיף אליהם את $x$(ע"מ לקבל $m+1$ איברים) ולכן ישנם ${x \choose m}$ אופציות לבחירה לכל $x$.
		כמות ה $x$ שיש הם מ $m$ (כי הראינו ש $x \geq m$) ועד $n$ (כי המספר הגדול ביותר בקבוצה $\set{0,1,..n}$ הינו $n$ ו $x$ מוגדר כמספר הגדול ביותר בתת קבוצה). \\
		ולפיכך הבעיה שקולה ללסכם את כל הבחירות לכל $x$ ואם נציב $k=x$ נקבל בעצם:
		${k \choose m} + {k+1 \choose m} \dots {n \choose m} = \sum_{k=m}^n{k \choose m}$
		שזה בדיוק הצד השמאלי של המשוואה.
	\end{proof}

	\pagebreak

	\section*{שאלה 2}
	\subsection*{סעיף א}
	מיצאו את מספר הפונקציות $f: \set{1,2,3,4,5,6,7,8} \to \set {1,2,3,4}$ כך שכמות המקורות לכל איבר בטווח שווה. \\
	\textbf{תשובה:} בגלל שבפונקציה לכל איבר בתחום יש תמונה, ובגלל שנתון לנו שכמות המקורות של כל האיברים בטווח שווה, אזי בהכרח לכל איבר בטווח יש 2 מקורות.
	ולכן נחשב את מספר הפונקציות, ע"י בחירת תתי קבוצות של 2 מקורות לכל איבר בטווח ע"י להתחיל ב ${8 \choose 2}$ לאיבר הראשון בטווח, ואח"כ ${6 \choose 2}$ לאיבר השני, וכן הלאה. \\
	לפיכך נקבל: ${8 \choose 2} \cdot {6 \choose 2} \cdot {4 \choose 2} \cdot {2 \choose 2} = 2520$

	\subsection*{סעיף ב}
	במקומות המסומנים: 1,2,3,4,5,6,7,8 מסדרים את הסימנים 1,1,2,3,3,4,4.
	מיצאו את מספר הסידורים בהם המספרים 1,2,3,4 לא יושבים במקום המסומן באותו המספר. \\
	\textbf{פתרון:}
	נשתמש בעיקרון ההכלה וההפרדה,
	\begin{enumerate}
		\item נתחיל בלחשב את הסכום הכולל, שזה פשוט לבחור 2 מקומות ל2 הספרות הראשונות, אח"כ עוד 2 לספרות הבאות וכו'
		ולכן $|U| = {8 \choose 2}{6 \choose 2}{4 \choose 2} = 2520$.
		\item לאחר מכן, נחלק ל 4 קבוצות שנסמנן $A_i$ כל קבוצה מייצגת את סידור המספרים שבו הספרה $i$ נמצאת במקום ה $i$, לדוגמא ב $A_1$ הסדרה מתחילה ב 1.
		לכן בכל קבוצה צריך למלא 7 מקומות חסרים, גם כאן ע"י בחירה של מקומות:
		$|A_i| = {7 \choose 1}{6 \choose 2}{4 \choose 2} = 630$
		\item נחשב את החיתוך בין כל 2 קבוצות, $A_i \cap A_j$כאשר $i \neq j$ ע"י בעצם לקבע את הספרות $i$ ו $j$ במקומן, ובכך נשארנו עם 6 מקומות פנויים, ובהם נבחר ספרות:
		$|A_i \cap A_j| = {6 \choose 1}{5 \choose 1}{4 \choose 2} = 180$. (ניתן לראות כי ישנם ${4 \choose 2} = 6$ חיתוכים של 2 קבוצות)
		\item נחשב את כל החיתוכים שבין 3 קבוצות שונות $A_i \cap A_j \cap A_k$ כאשר $i \neq j \neq k$ ע"י בעצם לקבע את שלשת הספרות המשותפות ולבחור ספרות ל5 המקומות הפנויים:
		$|A_i \cap A_j \cap A_k| = {5 \choose 1}{4 \choose 1}{3 \choose 1} = 60$ (ניתן לראות כי ישנם ${4 \choose 3} = 4$ חיתוכים של 3 קבוצות)
		\item נחשב את החיתוך של כל ארבעת הקבוצות, $\bigcap_{i=1}^4 A_i$ ע"י לקבע את כל ארבעת הספרות הראשונות ולחשב את התמורה של ארבעת המקומות שנשארו:
		$|\bigcap_{i=1}^4 A_i| = 4!$ (ישנו רק חיתוך אחד של כל ארבעת הקבוצות).
	\end{enumerate}
	עכשיו נשתמש בעיקרון ההכלה וההפרדה לפי סעיף 4.2 בספר, ונחשב:
	\begin{align*}
		|U| - (4 \cdot |A_i| - 6 \cdot |A_i \cap A_j| + 3 \cdot |A_i \cap A_j \cap A_k| - |\bigcap_{i=1}^4 A_i|) &= \\
		2520 - (4 \cdot 630 - 6 \cdot 180 + 4 \cdot 60 - 4!) &= 864
	\end{align*}

	\pagebreak
	\section*{שאלה 3}
	מפזרים 13 כדורים זהים ב 6 תאים שונים.
	\subsection*{סעיף א}
	מיצאו את מספר הפיזורים שבהם שלושת התאים הראשונים מכילים ביחד לפחות 10 כדורים. \\
	\textbf{תשובה:} נצטרך לפזר לפחות 10 כדורים בשלשת התאים הראשונים, ועד 3 כדורים בשלושת התאים האחרונים.
	בגלל שהכדורים זהים אז בעצם נבחר תאים במקום לבחור כדורים,
	ובגלל שהתאים שונים ואפשר כמה כדורים באותו תא נשתמש בפונקציה $D$ ונבחר תאים בשביל כדורים. \\
	בנוסף בגלל המגבלות על שלשת התאים הראשונים והאחרונים, נפצל לאופציות, \\
	אופציה אחת כאשר בראשונים יש 10 כדורים ובאחרונים יש 3: $D(3, 10)D(3,3)$.
	אופציה שניה בראשונים יש 11 כדורים, ובאחרונים 2: $D(3,11)D(3,2)$.
	שלישית בראשונים יש 12 ובאחרונים 1: $D(3,12)D(3,1)$.
	ורביעית כל הכדורים בראשונים: $D(3,13)D(3,0)$.
	נסכם את כלל האופציות ונקבל:
	\begin{align*}
		&\DD(3, 10)\DD(3,3) + \DD(3,11)\DD(3,2) + \DD(3,12)\DD(3,1) + \DD(3,13)\DD(3,0) = \\
		&{12 \choose 10}{5 \choose 3} + {13 \choose 11}{4 \choose 2} + {14 \choose 12}{3 \choose 1} + {15 \choose 13}{2 \choose 0} =
		1506
	\end{align*}
	\subsection*{סעיף ב}
	מיצאו את מספר הפיזורים שבהם אין אף תא שבו 3 כדורים בדיוק. \\
	\textbf{תשובה:}
	נשתמש בעיקרון ההכלה והפרדה:
	\begin{enumerate}
		\item נתחיל בלחשב את הסכום הכולל, שזה פיזור של 13 כדורים זהים ל 6 תאים שונים, ולפיכך $|U| = \DD(6, 13) = {18 \choose 13} = 8,568$
		\item נסמן כל קבוצה שבה בתא אחד יש בדיוק 3 כדורים כ $A_i$ כאשר ה $i$ מסמן את מספר התא שבו ישנם 3 כדורים.
		מספר הפיזורים שבהם בתא $i$ ישנו 3 כדורים שקול למספר הפיזורים של 10 כדורים זהים ל 5 תאים $|A_i|\DD(5,10) = {14 \choose 10} = 1,001$.
		\item נחשב עתה את מספר החיתוכים בין כל 2 תאים שונים, ז"א כמות הפיזורים שבהם בשני תאים ישנם בדיוק 3 כדורים, אזי נקבע 2 תאים עם 3 כדורים ונחשב פיזור של 7 כדורים ל 4 תאים:
		לכל $i \neq j$ נחשב $|A_i \cap A_j| = \DD(4,7) = {10 \choose 7} = 120$ (ניתן לראות כי ישנם ${6 \choose 2} = 15$ חיתוכים של 2 קבוצות).
		\item נחשב את מספר החיתוכים של שלושה קבוצות שונות, ע"י קביעה של 3 כדורים בכל תא משלשת התאים, ופיזור ארבעת הכדורים הנותרים בשלשת התאים הנותרים:
		לכל $i \neq j \neq k$ נחשב $|A_i \cap A_j \cap A_k| = \DD(3,4) = {6 \choose 4} = 15$ (ניתן לראות כי ישנם ${6 \choose 3} = 20$ חיתוכים בין כל 3 תאים שונים).
		\item נחשב את החיתוך של 4 תאים, ע"י קיבוע 3 כדורים בכל אחד מארבעת התאים, ופיזור הכדור האחרון בין 2 התאים שנותרו, ועל כן:
		לכל $i,j,k,l$ נחשב $|A_i \cap A_j \cap A_k \cap A_l| = \DD(2,1) = {2 \choose 1} = 2$ (ניתן לראות כי ישנם ${6 \choose 4} = 15$ חיתוכים שונים של 4 קבוצות).
		\item החיתוכים של 5 ו 6 תאים הינם ריקים כי אין מספיק כדורים ע"מ למקם 3 כדורים ב5 ויותר תאים.
	\end{enumerate}
	עכשיו נחשב לפי עקרון ההכלה והפרדה:
	\begin{align*}
	|U| &- (6 \cdot |A_i| - 15 \cdot |A_i \cap A_j| + 20 \cdot |A_i \cap A_j \cap A_k| - 15 \cdot |A_i \cap A_j \cap A_k \cap A_l|) = \\
	8568 &- (6 \cdot 1001 - 15 \cdot 120 + 20 \cdot 15 - 15 \cdot 2) = 4092
	\end{align*}

	\section*{שאלה 4}
	\subsection*{סעיף א}
	יהיו $p_1,p_2,\dots,p_n$ מספרים ראשוניים שונים זה מזה, ו $k_1, k_2,\dots,k_n$ מספרים טבעיים.
	כמה מספרים טבעים מחלקים את $p_1^{k_1}p_2^{k_2}\cdots p_n^{k_n}$. \\
	\textbf{תשובה:}
	נקרא למספר הנ"ל $x$, $x$ מתחלק בכל כפולה של $p_i^{k_j}$ ככה ש $0 \leq j \leq k_i$ בגלל שמספר שמתחלק ב $p^k$ וודאי מתחלק גם ב $p^{k-1}$ וודאי מתחלק גם באפס. \\
	ולכן כמות המספרים הטבעיים המחלקים את $x$ זה כל הקומבינציות שאפשר לשים בחזקות של $p_i$ ככה שהחזקה טבעית וקטנה מ $k_i$
	ולכן אנו מקבלים $\bm{\prod_{i=1}^n (k_i + 1)}$ (מוסיפים $1$ בגלל שגם $0$ זה מספר טבעי וחזקה אפשרית לכל $p_i$).

	\subsection*{סעיף ב}
	מיצאו את מספר המספרים הטבעיים המחלקים לפחות את אחד המספרים: $10^{40}, 20^{30}, 40^{20}$. \\
	\textbf{תשובה:} נתחיל בלפרק את המספרים לגורמים הראשוניים שלהם: \\
	$10^{40}$ = $(2 \cdot 5)^{40}$ = $\boxed{2^{40} \cdot 5^{40}}$ \\
	$20^{30}$ = $(2 \cdot 2 \cdot 5)^{30}$ = $2^{30} \cdot 2^{30} \cdot 5^{30}$ = $\boxed{2^{60} \cdot 5^{30}}$ \\
	$40^{20}$ = $(2 \cdot 2 \cdot 2 \cdot 5)^{20}$ = $2^{20} \cdot 2^{20} \cdot 2^{20} \cdot 5^{20}$ = $\boxed{2^{60} \cdot 5^{20}}$ \\
	נשים לב כי החזקות של 2 ו 5 של $20^{30}$ גבוהות יותר משל $40^{20}$ ועל כן כל מה שמחלק את $40^{20}$ מחלק גם את $20^{30}$.
	נשאר לנו לחשב את המחלקים של $10^{40}$ ושל $20^{30}$ ולחסר את הכפילויות.
	אם ניקח את המקסימום שבין החזקות של 2 והמקסימום שבין החזקות של 5 ונוסיף להם 1(כמו בשאלה הקודמת) נקבל $61 \cdot 41 = 2501$ אופציות למספרים שונים. \\
	אך יש טווח של חזקות, ספציפית $2^{41 \dots 60} \cdot 5^{31 \dots 40}$ שלא מחלק אף אחד מהמספרים, ולכן נחסר את כמות המספרים בטווח הנ"ל גם בעזרת כפל החזקות: $20 \cdot 10 = 200$. \\
	ולפיכך נקבל $61 \cdot 41 - (20 \cdot 10) = 2301$

	\pagebreak
	\section*{שאלה 5}
	\subsection*{סעיף א}
	מיצאו את כל השלשות $\ang{j,k,l} \in \NN \times \NN \times \NN$ המקיימות $2j+3k+5l=10$. \\
	\textbf{פתרון:} האי שוויונות הבאים חייבים להתקיים: $l < 3$, $k < 4$, $j < 6$ אחרת המשוואה תהיה גדולה מ10.
	נפריד לאופציות לפי הערך של $l$:
	\begin{enumerate}
		\item כאשר $l=2$ נקבל $2j+3k=0$ ולפיכך בהכרח מתקיים $j=k=0$ והפתרון $\boxed{\ang{0,0,2}}$.
		\item כאשר $l=1$ נקבל $2j+3k=5$ ולכן הפיתרונות הטבעיים היחידים הם $k=1$, $j=1$ שזה $\boxed{\ang{1,1,1}}$, כל מספר אחר ב $j$ או $k$ יתן לנו משוואה ללא פתרון טבעי.
		\item כאשר $l=0$ נקבל $2j+3k=10$:
		\begin{enumerate}
			\item כאשר $k=0$ נקבל $2j = 10$ והפתרון $\boxed{\ang{5,0,0}}$.
			\item כאשר $k=1$ נקבל $2j=7$ ואין פתרון טבעי.
			\item כאשר $k=2$ נקבל $2j=4$ והפתרון $\boxed{\ang{2,2,0}}$.
			\item כאשר $k=3$ נקבל $2j=1$ ואין פתרון טבעי.
		\end{enumerate}
	\end{enumerate}
	לסיכום, למשוואה יש את הפתרונות הבאים: $\set{\ang{0,0,2}, \ang{1,1,1}, \ang{5,0,0}, \ang{2,2,0}}$

	\subsection*{סעיף ב}
	מיצאו את המקדם של $x^{10}$ בביטוי $(1+x^2+x^3+x^5)^{10}$. \\
	\textbf{פתרון:}
	נציב את המולטינום במשוואת הפיתוח ונקבל:
	\begin{align*}
		&\sum_{k_1+k_2+k_3+k_4 = 10}\frac{10!}{k_1!k_2!k_3!k_4!}1^{k_1}(x^{2})^{k_2}(x^{3})^{k_3}(x^{5})^{k_4} \\
		= &\sum_{k_1+k_2+k_3+k_4 = 10}\frac{10!}{k_1!k_2!k_3!k_4!}x^{2k_2 + 3k_3 + 5k_4}
	\end{align*}
	אנחנו רוצים למצוא את קבוצת ה $k_i$ ככה שהחזקה תהיה שווה לעשר, \\
	 ז"א ${2k_2+3k_3+5k_4 = 10}$. בסעיף א מצאנו את הפתרונות בדיוק למשוואה הזו,
	והם כולם עומדים גם במגבלה של $k_1+k_2+k_3+k_4 = 10$ מכיוון ש $k_1$ הוא איבר חופשי שלא נמצא במשוואה השניה ולכן יכול להיות כל מספר טבעי ע"מ להשלים את הסכום לעשר.
	נגדיר את $k_1 = 10 - k_2 - k_3 - k_4$ ונחשב את כל המקדמים:
	\begin{enumerate}
		\item $\ang{0,0,2} \Rightarrow \frac{10!}{8!0!0!2!} = 45$
		\item $\ang{1,1,1} \Rightarrow \frac{10!}{7!1!1!1!} = 720$
		\item $\ang{5,0,0} \Rightarrow \frac{10!}{5!5!0!0!} = 252$
		\item $\ang{2,2,0} \Rightarrow \frac{10!}{6!2!2!0!} = 1260$
	\end{enumerate}
	ולפיכך נקבל $45x^{10} + 720x^{10} + 252x^{10} + 1260x^{10} = 2277x^{10}$
	ולכן $2277$ הינו המקדם של $x^{10}$.

	\subsection*{סעיף ג}
	מיצאו את המקדם ע"י שימוש בפירוק $1+x^2+x^3+x^5 = (1+x^2)(1+x^3)$ ובנוסחת הבינום של ניוטון. \\
	\textbf{פתרון:}
	נציב את הנתון בביטוי:
	\begin{align*}
		(1+x^2+x^3+x^5)^{10} &=
		((1+x^2)(1+x^3))^{10} \\
		&= (1+x^2)^{10}(1+x^3)^{10} \\
		&=
		\left( \sum_{i=0}^{10} {10 \choose i} x^{2i}  \right)
		\left( \sum_{j=0}^{10} {10 \choose j} x^{3j}  \right) \\
		&= \left( \sum_{i=0}^{10} \sum_{j=0}^{10}  {10 \choose i}{10 \choose j} x^{2i+3j} \right)
	\end{align*}
	קיבלנו סדרה של סכומים, אנו מעוניינים רק באלה שבהם החזקה שווה לעשר, ז"א אנו מחפשים את הפתרונות הטבעיים של המשוואה
	$2i + 3j = 10$, שהם: $\set{\ang{2,2}, \ang{5, 0}}$.
	עכשיו נציב במקדמים ונסכם:
	\[
		{10 \choose 2}{10 \choose 2} x^{10} + {10 \choose 5}{10 \choose 0} x^{10} =
		45 \cdot 45 \cdot x^{10} + 252 \cdot 1 \cdot x^{10} = 2277x^{10}
	\]
	וניתן לראות כי התשובה זהה לתשובה בסעיף ב.


	\end{document}
