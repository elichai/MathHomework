% !TEX program = xelatex
\def\NN{\mathbb{N}}
\def\RR{\mathbb{R}}
\def\ZZ{\mathbb{Z}}
\def\QQ{\mathbb{Q}}
\def\PP{\mathcal{P}}
\def\SS{\mathcal{S}}
\def\DD{\mathcal{D}}
\def\sub{\setminus}
\def\bld{\mathbf}
\def\lmf{\lim_{n \to \infty}}
% Make the ref command use parenthesis
\let\oldref\ref
\renewcommand{\ref}[1]{(\oldref{#1})}

\newcommand{\ontop}[1]{\overset{\text{#1}}}
\newcommand{\ang}[1]{\langle #1 \rangle}
\newcommand{\pip}[1]{\left| #1 \right|}



% style
\newcommand{\bm}[1]{\displaystyle{#1}}
\def\nl{$ $ \newline}

\ExplSyntaxOn

\NewDocumentCommand{\getenv}{om}
{
  \sys_get_shell:nnN{ kpsewhich ~ --var-value ~ #2 }{}#1
}

\ExplSyntaxOff

%environments

\documentclass{article}
\usepackage[]{amsthm} %lets us use \begin{proof}
\usepackage{amsmath}
\usepackage{mathtools}
\usepackage{enumerate}
\usepackage{xparse}
\usepackage[makeroom]{cancel}
\usepackage[]{amssymb} %gives us the chA \mathcal{R} Acter \varnothing
\usepackage{polyglossia}
% \usepackage[frak=mma]{mathalfa}
\setdefaultlanguage{hebrew}
\setotherlanguage{english}
\usepackage{fontspec}
%\setmainfont{Frank Ruehl CLM}
\setmainfont{David CLM}
\setmonofont{Miriam Mono CLM}
\setsansfont{Simple CLM}
\DeclarePairedDelimiter\set\{\}
% Use the following if you only want to change the font for Hebrew
%\newfontfamily\hebrewfont[Script=Hebrew]{David CLM}
%\newfontfamily\hebrewfonttt[Script=Hebrew]{Miriam Mono CLM}
%\newfontfamily\hebrewfontsf[Script=Hebrew]{Simple CLM}
\getenv[\ID]{ID}
\newtheorem{lemma}{טענת עזר}

\title{בדידה - ממ"ן 12}
\author{אליחי טורקל \ID}
\date\today

%\clearpage %Gives us a page break before the next section. Optional.
%\selectlanguage{english}
	%Section and subsection automatically number unless you put the asterisk next to them.

\begin{document}
	\maketitle %This command prints the title based on information entered above

	\section*{שאלה 1}
	על הקבוצה $\PP(\set{1,2,3,4})$ נתונים שני יחסים $\mathcal{R}, \mathcal{S}$ המוגדרים כך:
	 $A,B \in \PP(\set{1,2,3,4})$ $A \mathcal{R} B \iff A \cup \set{1,2} = B \cup \set{1,2}$
	ו $A \mathcal{S} B \iff A \cup \set{1,2} \subset B \cup \set{1,2}$ \\
	הערה: כל קבוצה שלא הגדרנו מה היא, מוגדרת כאיבר ב $\PP(\set{1,2,3,4})$
	\subsection*{סעיף א}
	נראה ש $\mathcal{R}$ הוא יחס שקילות:
\begin{enumerate}
	\item רפלקסיביות:
	$A \mathcal{R} A \iff A \cup \set{1,2} = A \cup \set{1,2}$,
	הצד הימני והשמאלי של השוויון זהים, וע"כ תמיד מתקיים עפ"י הגדרת יחס השוויון כרפלקסיבי,
	ולפיכך $R$ רפלקסיבי, ומקיים $A \mathcal{R} A$.
	\item טרנזיטיביות:
	אם מתקיים $A \cup \set{1,2} = B \cup \set{1,2}$
	וגם $B \cup \set{1,2} = C \cup \set{1,2}$
	אז גם פה, עפ"י הגדרת יחס השוויון כיחס טרנזיטיבי נקבל $A \cup \set{1,2} = C \cup \set{1,2}$
ולפיכך מתקבל ש $A \mathcal{R} B \land BRC \to ARC$ ו $R$ יחס טרנזיטיבי.
	\item סימטריה:
	יחס השוויון הוא סימטרי, וע"כ:\\
	$A \cup \set{1,2} = B \cup \set{1,2} \iff
	B \cup \set{1,2} = A \cup \set{1,2}$
	ולכן לכל $A \mathcal{R} B$ מתקיים $B \mathcal{R} A$ ולפיכך $R$ סימטרי.
\end{enumerate}
על מנת למצוא את מחלקות השקילות נסתכל על הקבוצה ועל היחס,
הקבוצות שעליהן היחס פועל, מוכלות ב $\PP(\set{1,2,3,4})$,
והיחס מתקיים במידה ואיחוד של שני תתי קבוצות ביחד עם $\set{1,2}$ שווה. \\
נגדיר $C = A \cup \set{1,2}$, עפ"י הגדרת האיחוד תמיד יתקיים $1,2 \in C$
ועל כן, $C$ ישתנה א"םם $A$ מכיל משהו שהוא לא $1,2$, ובגלל ש $A \in \PP(\set{1,2,3,4})$
אז $A$ עלול להכיל את $3,4$, ואיזה מהם ש $A$ יכיל גם $B$ יצטרך להכיל ע"מ לקיים את יחס השקילות.
ועל כן נחשב את מחלקות השקילות כך שיהיו זרות אחת לשניה, לא ריקות ושאיחודן יהיה קבוצת החזקה:
\begin{enumerate}
	\item $3,4 \in A,B \to
	S_{\set{1,2,3,4}}=\set{\set{3,4}, \set{1,3,4}, \set{2,3,4}, \set{1,2,3,4}}$
	\item $3 \in A,B \land 4 \not\in A,B \to
	S_{\set{1,2,3}}=\set{\set{3}, \set{1,3}, \set{2,3}, \set{1,2}}$
	\item $3 \not\in A,B \land 4 \in A,B \to
	S_{\set{1,2,4}}=\set{\set{4}, \set{1,4}, \set{2,4}, \set{1,2,4}}$
	\item $3,4 \not\in A,B \to
	S_{\set{1,2}}=\set{\emptyset, \set{1}, \set{2}, \set{1,2}}$
\end{enumerate}
	\pagebreak
	\subsection*{סעיף ב}
נראה ש $\mathcal{S}$ הוא יחס סדר חלקי:
\begin{enumerate}
	\item אנטי רפלקסיביות:
	$A \cup \set{1,2} \subset A \cup \set{1,2}$
	ניתן לראות כי שני הצדדים שווים, וע"כ לפי הגדרת תת קבוצה ממש, קבוצה אינה חלקית ממש לעצמה,
	ולפיכך $\mathcal{S}$ אנטי-רפלקסיבי, ולעולם לא מתקיים $A\mathcal{S}A$.
	\item טרנזיטיביות:
	נגדיר $A,B,C \in \PP(\set{1,2,3,4})$,
	אם מתקיים $A \cup \set{1,2} \in B \cup \set{1,2}$
	וגם $B \cup \set{1,2} \in C \cup \set{1,2}$ אזי:
\[
	\forall x \in A \cup \set{1,2} \ontop{(1)}\Rightarrow
	x \in B \cup \set{1,2} \ontop{(2)}\Rightarrow
	x \in C \cup \set{1,2}
\]
ולפיכך מתקיים ש $A \cup \set{1,2} \in C \cup \set{1,2}$, ולכן
$A \mathcal{S} B \land B \mathcal{S} C \to A \mathcal{S} C$, ו $\mathcal{S}$ הוא טרנזיטיבי. \\
(1) -
עפ"י הגדרת ההכלה ממש וש $A \cup \set{1,2} \in B \cup \set{1,2}$ \\
(2) עפ"י הגדרת ההכלה ממש וש $B \cup \set{1,2} \in C \cup \set{1,2}$ \\
	\item $\mathcal{S}$ אינו סדר מלא, כי אינו משווה, נוכיח ע"י דוגמא נגדית:
	נסתכל על הקבוצות $A = \set{3}, B = \set{4}$, נראה ש:
	\[
		A \cup \set{1,2} = \set{1,2,3},  B \cup \set{1,2} = \set{1,2,4} \to \set{1,2,3} \not \subset \set{1,2,4} \land \set{1,2,4} \not \subset \set{1,2,3}
	\]
	\item האיברים המקסימלים של הקבוצה הסדורה $\langle \PP(\set{1,2,3,4}), \mathcal{S} \rangle$:
בהינתן קבוצה $A \in \PP(\set{1,2,3,4})$ המקיימת $A \cup \set{1,2} = \set{1,2,3,4}$
לא קיימת אף קבוצה $X \in \PP(\set{1,2,3,4})$ כך ש $\set{1,2,3,4} \subset X \cup{1,2}$
ולפיכך, כל קבוצה $A$ המקיימת את התנאי היא קבוצה מקסימלית.
כל תת קבוצה שלא מכילה את $3,4$  תהיה בהכרח תת קבוצה של $\set{1,2,3,4}$ ועל כן לא תהיה מקסימלית. \\
הקבוצות המקסימליות לגבי היחס $\set{3,4}, \set{1,3,4}, \set{2,3,4}, \set{1,2,3,4} :\mathcal{S}$.
\end{enumerate}

	\pagebreak
	\section*{שאלה 2}

	\subsection*{סעיף א}
	על הקבוצה $A = \RR \times \RR$ מגדירים יחס $\mathcal{R}$ כך ש
	$\forall \langle x_1, y_1 \rangle, \langle x_2, y_2 \rangle \in A$ מתקיים
	$\langle x_1, y_1 \rangle \mathcal{R} \langle x_2, y_2 \rangle \iff (x_1 + y_2 = x_2 + y_2 = 1 \lor (x_1 + y_1 - 1)(x_2 + y_2  - 1) > 0)$ \\
	הוכיחו ש $\mathcal{R}$ יחס שקילות ומצאו את מספר מחלקות השקילות שלו, ותארו אותן במישור. \\
	פתרון: \\
	נתחיל בלהראות ש $\mathcal{R}$ הינו יחס שקילות:
	\begin{enumerate}
		\item רפלקסיביות -
		יהי $\langle x, y \rangle \in A$ אחת משתי האופציות הבאות מתקיימות:
		\begin{enumerate}
			\item  $x + y = 1 \Rightarrow \Big( \langle x, y \rangle \mathcal{R} \langle x, y \rangle \iff x + y = x + y = 1 \Big)$
			\item $x + y \neq 1 \Rightarrow x + y - 1 \neq 0 \ontop{(a)}\Rightarrow (x+y-1)^2 > 0$ \\
			(a) $\forall x \in \RR,  x^2 \geq 0$
		\end{enumerate}
		ועל כן $\mathcal{R}$ הוא רפלקסיבי.
		\item  טרנזיטיביות -
		יהיו $\langle x_1, y_1 \rangle, \langle x_2, y_2 \rangle, \langle x_3, y_3 \rangle \in A$
		כך ש $\langle x_1, y_1 \rangle \mathcal{R} \langle x_2, y_2 \rangle$ וגם $\langle x_2, y_2 \rangle \mathcal{R} \langle x_3, y_3 \rangle$
		אחת משתי האופציות מתקיימות עפ"י הגדרת היחס:
		\begin{enumerate}
			\item \begin{align*}
			&x_1 + y_1 = 1 \ontop{(a)}\Rightarrow
			x_2 + y_2 = 1 \ontop{(b)}\Rightarrow
			x_3 + y_3 = 1 \Rightarrow \\
			&x_1+y_1 = x_3 + y_3 = 1 \Rightarrow
			\langle x_1, y_1 \rangle \mathcal{R} \langle x_3, y_3 \rangle
			\end{align*}
			\item \begin{align*}
				&x_1 + y_1 \neq 1 \ontop{(a)}\Rightarrow
				(x_1 + y_1 - 1)(x_2 + y_2 - 1) > 0 \ontop{(c)}\Rightarrow
				x_2 + y_2 - 1 \neq 0 \Rightarrow \\
				&x_2 + y_2 \neq 1 \ontop{(b)}\Rightarrow
				(x_2 + y_2 - 1)(x_3 + y_3 - 1) > 0 \ontop{(d)}\Rightarrow \\
				&(x_1 + y_1 - 1)(x_2 + y_2 - 1)(x_3 + y_3 - 1) > 0 \ontop{(e)}\Rightarrow
				(x_1 + y_1 - 1)(x_3 + y_3 - 1) > 0
			\end{align*}
		\end{enumerate}
		לסיכום, הראינו שהיחס $\mathcal{R}$ טרנזיטיבי בשתי הדרכים שבהן הוא יכול להתקיים. \\
		(a) - עפ"י הגדרת היחס $\mathcal{R}$ והנתון $\langle x_1, y_1 \rangle \mathcal{R} \langle x_2, y_2 \rangle$ \\
		(b) - עפ"י הגדרת היחס $\mathcal{R}$ והנתון $\langle x_2, y_2 \rangle \mathcal{R} \langle x_3, y_3 \rangle$ \\
		(c) $x \cdot y > 0 \Rightarrow x \neq 0 \land y \neq 0$ \\
		(d) - מותר להכפיל את שני הצדדים של אי שוויון בביטוי שגדול מאפס\\
		(e) - נחלק את שני הצדדים ב $x_2 + y_2 - 1$, מותר כי הראינו כבר שהביטוי שונה וגדול מאפס

	\item סימטריה -
	יהיו $\langle x_1, y_1 \rangle, \langle x_2, y_2 \rangle \in A,
	\langle x_1, y_1 \rangle \mathcal{R} \langle x_2, y_2 \rangle$
	גם כאן נסתכל על שתי הדרכים שבהן היחס מתקיים:
	\begin{enumerate}
		\item לפי הגדרת יחס השוויון: $x_1 + y_1 = x_2 + y_2 = 1 \iff x_2 + y_2 = x_1 + y_1 = 1$
		\item לפי אקסיומת החילוף בשדה הממשיים:  \\
		$(x_1 + y_1 - 1)(x_2 + y_2 - 1) > 0 \iff (x_2 + y_2 - 1)(x_1 + y_1 - 1) > 0$
	\end{enumerate}
	ועל כן, היחס $\mathcal{R}$ הינו יחס סימטרי.
\end{enumerate}
הראנו ש $\mathcal{R}$ הינו יחס רפלקסיבי, טרנזיטיבי, וסימטרי, כנדרש בהגדרת יחס השקילות.
נמצא את מחלקות השקילות: \\
ליחס $\mathcal{R}$ ישנן 3 מחלקות שקילות, אחת של כל הזוגות המקיימות את האופציה הראשונה $x + y = 1$,
ובאופציה השניה, בגלל שמדובר במכפלת 2 ביטויים ככה שיהיו גדולים מאפס אז זה מתחלק לזוגות הביטויים הגדולים מאפס והקטנים מאפס, ניתן לראות שהמחלקות זרות בגלל שביטוי שלילי כפול ביטוי חיובי יהיה קטן מאפס,
וכל זוג שמקיים $x + y = 1$  לעולם לא יקיים את האופציה השניה מכיוון שיקיים $x + y - 1 = 0$  ולכן יאפס את הביטוי (הטענה עובדת גם בכיוון הנגדי).
בנוסף, קל לראות  שאיחודן יתן לנו את הקבוצה כולה.
נתאר את הקבוצות:
\begin{enumerate}
	\item $S_{\langle x, 1-x \rangle} = \set{\langle x, y \rangle \in A | x + y = 1}$
	 מחלקת שקילות זו מייצגת את כלל הנקודות במישור הנמצאות על הישר $y = -x + 1$ למשל $\langle 1, 0 \rangle, \langle 0, 1 \rangle$.
	 \item $S_{\langle x, >(1-x) \rangle} = \set{\langle x, y \rangle \in A| x + y > 1}$
	 מחלקת שקילות זו מייצגת את כלל הנקודות במישור הנמצאות מעל הישר במחלקה הראשונה, למשל $\langle 2, 2 \rangle$
	 \item $S_{\langle x, <(1-x) \rangle} = \set{\langle x, y \rangle \in A| x + y < 1}$
	 מחלקת שקילות זו מייצגת את כלל הנקודות במישור הנמצאות מתחת הישר במחלקה הראשונה, למשל $\langle 0, 0 \rangle$
\end{enumerate}
ניתן לראות גם כי שלשת מחלקות אלה אכן מגדירות חלוקה של $A$ בגלל שאחת מגדירה את כל הנקודות שבקו ישר, אחת מגדירה את כל הנקודות שמעליו, והשלישית מגדירה את כל הנקודות שמתחתיו.
ועל כן הן זרות זו לזו, ואיחודן הוא המישור כולו.
\subsection*{סעיף ב}
על הקבוצה
$B = (0, \infty) \times (0, \infty)$ מגדירים יחס $\mathcal{S}: \forall \ang{a, b}, \ang{c, d} \in B, \ang{a, b} \mathcal{S} \ang{c,d})$
אם ורק אם $\frac{ab}{a^2 + b^2} < \frac{cd}{c^2 + d^2}$
\begin{enumerate}
	\item \begin{enumerate}
		\item הוכיחו ש $\forall a,b > 0, a \neq b \to \ang{a,b}\mathcal{S}\ang{a,a}$
		\begin{proof}
			נציב את האיברים באי שוויון ונקבל:
			\begin{align*}
				\frac{ab}{a^2 + b^2} < \frac{a^2}{a^2 + a^2} \Rightarrow
				\frac{ab}{a^2 + b^2} < \frac{1}{2} \Rightarrow
				2ab < a^2 + b^2 &\Rightarrow \\
				a^2 + b^2 - 2ab > 0 &\Rightarrow
				(a-b)^2 > 0
			\end{align*}
			כל מספר בריבוע גדול או שווה לאפס ולכן קל לראות שהביטוי $(a-b)^2 > 0$ מתקיים לכל  $a \neq b$  ולפיכך היחס $\ang{a,b} \SS \ang{a,a}$ מתקיים כנדרש.
		\end{proof}
		\item הוכיחו ש $n$ מספר טבעי כך ש $\ang{1, 1/n}\SS\ang{a,b} \Leftarrow \frac{1}{n} < \frac{ab}{a^2 + b^2}$
		\begin{proof}
			נציב בתנאי ונקבל: $\frac{\frac{1}{n}}{1 + \frac{1}{n^2}} < \frac{ab}{a^2 + b^2}$
			בנוסף אנו יודעים ש \\
			$\forall a > 0, b > 1 \Rightarrow a > \frac{a}{b}$, ובגלל שבנתון  $n \in \NN$ ומחלקים ב $n$ אז $n \neq 0$ ולכן גדול מאחד.
			ולפיכך מתקיים $\frac{\frac{1}{n}}{1 + \frac{1}{n^2}} < \frac{1}{n}$ ועל כן עפ"י הנתון מתקיים:
			\[
				\frac{\frac{1}{n}}{1 + \frac{1}{n^2}} < \frac{1}{n} < \frac{ab}{a^2 + b^2} \Rightarrow
				\frac{\frac{1}{n}}{1 + \frac{1}{n^2}} < \frac{ab}{a^2 + b^2}
			\]
			כנדרש לפי הגדרת היחס $\ang{1, 1/n} \SS \ang{a,b}$.
		\end{proof}
	\end{enumerate}
	\item הוכיחו ש $\SS$ הוא יחס חלקי: \\
	גם טרנזיטיביות וגם אנטי רפלקסיביות נובעים ישירות מהגדרת $>$ כיחס סדר.
	\begin{enumerate}
		\item טרנזיטיביות -
		\begin{align*}
			& \forall \ang{a, b}, \ang{c,d}, \ang{e,f} \in A, \ang{a,b}\SS\ang{c,d}, \ang{c,d}\SS\ang{e,f}  \\
			& \Rightarrow	\frac{ab}{a^2 + b^2} < \frac{cd}{c^2 + d^2} \land
			\frac{cd}{c^2 + d^2} < \frac{ef}{e^2 + f^2} \ontop{(a)}\Rightarrow
			\frac{ab}{a^2 + b^2} < \frac{ef}{e^2 + f^2}
		\end{align*}
		(a) -
		הגרירה נובעת ישירות מהגדרת היחס $>$ כיחס סדר. \\
		ועל כן היחס $\SS$ הוא יחס טרנזיטיבי.
		 \item אנטי רפלקסיבי -
		 \[
			\forall \ang{a, b} \in A,
			\ang{a, b}\SS\ang{a, b} \Rightarrow \frac{ab}{a^2 + b^2} < \frac{ab}{a^2 + b^2}
		 \]
		 אי שוויון זה לעולם לא יכול להתקיים מתוך הגדרת $>$ כיחס אנטי רפלקסיבי.
		 ועל כן $\SS$ הוא יחס אנטי רפלקסיבי
		 \item ע"מ להראות שהוא יחס סדר חלקי ולא מלא נראה שהיחס אינו משווה:
		 \[
			 \forall \ang{a,b}, \ang{b,a} \in A,
			 \ang{a,b} \neq \ang{b,a},
			 \ang{a,b}\SS\ang{b,a} \Rightarrow
			 \frac{ab}{a^2+b^2} < \frac{ba}{b^2+a^2}
		 \]
		 אי שוויון זה לעולם לא יתקיים בגלל שפעולות החיבור והכפל הינן חילופיות בשדה הממשיים, והיחס $<$ הוא יחס סדר וע"כ אנטי רפלקסיבי.
	\end{enumerate}
	\item מיצאו את כל האיברים המקסימליים והמינימליים ב $B$ ביחס לסדר $\SS$:
	\begin{enumerate}
		\item איברים מינימליים: \\
		 הראנו קודם שלכל $a, b, a \neq b$ מתקיים $\ang{a,b} \SS, \ang{a,a}$ ולפיכך אף $\ang{a,a}$ לא יכול להיות מינימלי. \\
		בנוסף הראנו ש $\ang{1, 1/n}\SS\ang{a,b}$ וע"כ גם אף $\ang{a,b}$ אינו מינימלי, ולפיכך אין איברים מינימלים.
		\item איברים מקסימלים: \\
		גם כאן הראנו קודם  שלכל $a,b, a \neq b$ מתקיים $\ang{a,b} \SS, \ang{a,a}$ ולפיכך $\ang{a,b}$ לא מקסימלי.
		אפשר להגיד אותו דבר על $\ang{1, 1/n}$, אבל זה מקרה פרטי שכלול בתוך $\ang{a,b}$. \\
		אך כאן אין לנו טענה שמראה ש $\ang{a,a}$ אינו מקסימלי, וע"כ ננסה לראות לכל $\ang{a,a}$ קיימים $\ang{b,c} \in B, b \neq c$ כך ש $\ang{a,a}\SS \ang{b,c}$:
		\begin{align*}
			\frac{a^2}{a^2 + a^2} < \frac{bc}{b^2 + c^2} \Rightarrow
			&\frac{1}{2} < \frac{bc}{b^2 + c^2} \Rightarrow
			\frac{1}{2} < \frac{bc}{b^2 + c^2} \Rightarrow \\
			&b^2 + c^2 < 2bc \Rightarrow
			b^2 + c^2 - 2bc < 0 \Rightarrow
			(b-c)^2 < 0
		\end{align*}
		בסתירה עם הנתון ש $b \neq c$ (כל מספר בריבוע ששונה מאפס גדול מאפס). \\
		ועל כן כל האיברים $\ang{a,a} \in B$ הם איברים מקסימלים ב $B$ לגבי הסדר $\SS$.
	\end{enumerate}
\end{enumerate}

\pagebreak
\section*{שאלה 3}
$f: \NN \to \NN$
\subsection*{סעיף א}
הוכיחו ש $f$ היא חד-חד-ערכית אם ורק אם לכל שתי קבוצות אינסופיות שונות $A,B \subseteq \NN$ מתקיים $f[A] \neq f[B]$.
\begin{proof}
	נוכיח כל כיוון בנפרד:
	\begin{enumerate}
		\item כיוון אחד: נניח ש $f$ חד חד ערכית, \\
		נניח בדרך השלילה ש $f[A] = f[B]$ אזי לכל $x \in A$ קיים $y \in B$ כך ש  $f(x) = f(y)$,
		ובנוסף לכל $y \in B$ קיים $x \in A$ כך ש $f(x) = f(y)$, \\
		ובגלל שהנחנו ש $f$ חד חד ערכית, אזי $x = y$
		ומכאן נובע שלכל $x \in A$ קיים $y \in B$ כך ש $x=y$
		ולכל $y \in B$ קיים $x \in A$ כך ש $x=y$, בסתירה עם הנתון ש $A \neq B$.
		ולפיכך מתקבל ש $f[A] \neq f[B]$.

		\item כיוון שני: נניח ש $f[A] \neq f[B]$,\\
		נניח בדרך השלילה ש $f$ אינה חד-חד-ערכית, קיימים $x, y \in \NN, x \neq y$ כך ש $f(x) \neq f(y)$,
		נראה שהקבוצה $\NN\sub \set{x}$ מכילה את $y$ ועל כן $f[\NN\sub\set{x}]$ מכילה את $f(y)=f(x)$
		מכאן ניתן לראות שכל דמות ב $f[\NN\sub\set{x}]$ נמצאת גם ב $f[\NN]$ ועל כן $f[\NN] = f[\NN\sub\set{x}]$
		בסתירה להנחה ש $f[A] \neq f[B]$ לכל שתי קבוצות אינסופיות שונות. \\
		ולפיכך מתקבל ש $f$ חד-חד ערכית.
	\end{enumerate}
\end{proof}

\subsection*{סעיף ב}
הוכיחו ש $f$ היא פונקציה על אם ורק אם לכל שתי קבוצות אינסופיות שונות $A,B \in \NN$ מתקיים $f^{-1}[A] \neq f^{-1}[B]$.
\begin{proof}
	נוכיח כל כיוון בנפרד:
	\begin{enumerate}
		\item כיוון אחד: נניח ש $f$ היא על, \\
		נניח בדרך השלילה ש $f^{-1}[A] = f^{-1}[B]$, מכאן נקבל ש
		$f(f^{-1}(A)) = f(f^{-1}(B))$ (הדמות של פונקציה מעל מקורות שווים היא שווה),
		ולפי משפט 3.8 בגלל ש $f$ היא על נקבל ש $A=B$ בסתירה עם הנתון. \\
		ולפיכך $f^{-1}[A] \neq f^{-1}[B]$ כנדרש.

		\item כיוון שני: נניח ש $f^{-1}[A] \neq f^{-1}[B]$, \\
		נניח בדרך השלילה ש $f$ אינה על,
		לכן קיים $y \in \NN$ כך שלכל $x \in \NN$ $f(x) \neq y$ (אין לו מקור)
		ועל כן מתקיים: $f^{-1}[\set{y}] \cup f^{-1}[\NN \sub \set{y}] = \emptyset \cup f^{-1}[\NN \set{y}]$
		(בגלל שאין ל $y$ מקור אז מתקיים $f^{-1}[\set{y}]=\emptyset$)
		ולפי שאלה 6,ה מתקבל ש: $f^{-1}[\set{y} \cup \NN \sub \set{y}] = f^{-1}[\NN \sub \set{y}]$
		ועל כן $f^{-1}[\NN] = f^{-1}[\NN \sub \set{y}] \Rightarrow \NN = \NN \sub \set{y}$
		משמע סתירה. \\
		ולפיכך מתקבל ש $f$ היא על כנדרש.
	\end{enumerate}
\end{proof}


\pagebreak
\section*{שאלה 4}
\subsection*{סעיף א}
$\ZZ^* = \ZZ \sub \set{0}$,
נתונה $f: \QQ \times \ZZ^* \to \QQ \times \ZZ^*$ המוגדרת כך: \\
$\forall q \in \QQ, n \in \ZZ^*, f\ang{q,n} = \ang{\frac{q}{n}, n}$.
\begin{enumerate}
	\item הוכיחו ש $f$ היא חד-חד-ערכית ועל:
	\begin{proof} \nl
		\begin{enumerate}
			\item נוכיח שהיא על:
			יהי $p \in \QQ, n \in \ZZ^*$, נראה שלכל $\ang{p, n}$ יש מקור מהצורה: $\ang{pn, n}$ ע"י להציב בפונקציה:
			\[
				f\ang{pn, n} = \ang{n \cdot \frac{pn}{n}, n} = \ang{p, n}
			\]
			ולפיכך, לכל תמונה יש מקור, משמע שהתמונה שווה לטווח משמע שהפונקציה היא על.

			\item נוכיח שהיא חד-חד-ערכית:
			יהי $p,q \in \QQ, n,m \in \ZZ^*$
			נוכיח שאם $f\ang{p,m} = f\ang{q,n}$ אז $\ang{p,m} = \ang{q,n}$
			\[
				f\ang{p,m} = f\ang{q,n} \iff
				\ang{\frac{p}{m}, m} = \ang{\frac{q}{n}, n}
			\]
			לפי הגדרת יחס השוויון על זוגות סדורים, נקבל ש $n = m$ וגם $\frac{q}{m} = \frac{p}{n}$ ולכן
			$p = q \iff \frac{p}{n} = \frac{q}{n} \iff \frac{p}{m} = \frac{q}{n}$
			ועל כן אנו רואים שבהכרח $m = n \land p = q \iff \ang{p,m} = \ang{q,n}$ \\
			ולפיכך לכל תמונה יש מקור אחד ויחיד, ועל כן $f$ היא חד חד ערכית.
		\end{enumerate}
		ובכך הראנו ש $f$ היא חד חד ערכית ועל.
	\end{proof}
	\item מיצאו את $f^{-1}$:  \\
	בעקבות ההוכחה שלנו ש $f$ היא חד-חד-ערכית ועל, מתקבל שיש לה הופכי, נציג את ההופכי ונוכיח את נכונותו:
	$f^{-1}\ang{q,n} = \ang{qn,n}$. \\
	נוכיח באמצעות משפט 3.25:
	\[
		f^{-1}(f\ang{q,n}) = f^{-1}\ang{\frac{q}{n}, n} =
		\ang{n \cdot \frac{q}{n}, n} = \ang{q,n}
	\]
	ולפיכך הראנו ש $f^{-1} \circ f = I_{Q \times \ZZ^*}$ והוכחנו את נכונות ההופכי.
\end{enumerate}
\pagebreak
\subsection*{סעיף ב}
נתונות הפונקציות: $g,h: \ZZ \times \ZZ \to \ZZ \times \ZZ$ המוגדרות כך: $\ang{x,y} \in \ZZ \times \ZZ$, \\
$h\ang{x,y} = \ang{x + 3y, x + 5y}, \ g\ang{x,y} = \ang{2x + 3y, 3x + 5y}$
הוכיחו שרק אחת משתיהן היא הפיכה ומיצאו את ההופכי.
\begin{enumerate}
	\item נתחיל בלהוכיח ש $h$ אינה הפיכה ע"י להראות שהיא לא על:
	\begin{proof}
		נוכיח שלזוג $\ang{1,2}$ מהטווח של $h$ אין מקור בתחום:
		\begin{align*}
			&\begin{cases}
				x + 3y = 1\\
				x + 5y = 2 \\
			\end{cases}
			\Rightarrow
			\begin{array}{l}
				x = 1 - 3y\\
				x = 2 - 5y \\
			\end{array}
			\Rightarrow \\
			&1 - 3y = 2 - 5y
			\Rightarrow
			2y = 1
			\Rightarrow
			y = \frac{1}{2} \not \in \ZZ
		\end{align*}
		ניתן לראות שלזוג $\ang{1,2}$ לא יכול להיות מקור ב $\ZZ \times \ZZ$ כי לא קיים $y \in \ZZ$ שעונה על ההגדרה של $h$.
		ולכן הפונקציה אינה על ומכאן שאינה הפיכה.
	\end{proof}

	\pagebreak
	\item
	\begin{enumerate}
		\item נוכיח ש $g$ היא חד-חד-ערכית:
		\begin{proof}
			יהי $\ang{a,b}, \ang{c,d} \in \ZZ \times \ZZ$ כך ש $g\ang{a,b} = g\ang{c,d}$
			נראה שבהכרח $\ang{a,b} = \ang{c,d}$:
			\begin{align*}
				&g\ang{a,b} = g\ang{c,d} \overset{g}\Rightarrow
				\ang{2a + 3b, 3a + 5b} = \ang{2c + 3d, 3c + 5d} \\
				&\ontop{(a)}\Rightarrow
				\begin{cases}
					2a + 3b = 2c + 3d \\
					3a + 5b = 3c + 5d \\
				\end{cases}
				\ontop{(b)}\Rightarrow
				\begin{cases}
					6a + 9b = 6c + 9d \\
					6a + 10b = 6c + 10d \\
				\end{cases}
				\ontop{(c)}\Rightarrow
				0a + 1b = 0c + 1d \\
				&\Rightarrow
				\boxed{b = d}
				\ontop{(d)}\Rightarrow
				2a + 3b = 2c + 3b \Rightarrow
				2a = 2c \Rightarrow
				\boxed{a = c}
			\end{align*}
			ובעקבות זה שהוכחנו ש $b = d$ וגם $a = c$ מתקבל ש $\ang{a,b} = \ang{c,d}$ כנדרש. \\
			ולפיכך $g$ הינה חד-חד-ערכית. \\
		\end{proof}
		\item נוכיח ש $g$ היא על:
		\begin{proof}
			יהי $\ang{a,b} \in \ZZ\times\ZZ$ זוג סדור מהטווח של $g$, נראה שיש להם מקור $\ang{m,n}$ ונוכיח שהוא נמצא בתחום:
			\begin{align*}
				&g\ang{m,n} = \ang{a,b} \overset{g}\Rightarrow
				\begin{cases}
					2m + 3n = a \\
					3m + 5n = b \\
				\end{cases} \ontop{(b)}\Rightarrow
				\begin{cases}
					6m + 9n = 3a \\
					6m + 10n = 2b \\
				\end{cases} \\
				&\ontop{(c)}\Rightarrow
				0m + 1n = 2b - 3a \Rightarrow
				\boxed{n = 2b - 3a \ontop{(e)}\in \ZZ}
				\ontop{(d)}\Rightarrow
				2m + 3(2b - 3a) = a \\
				&\Rightarrow
				2m + 6b - 9a = a \Rightarrow
				2m = 10a - 6b \Rightarrow
				\boxed{m = 5a - 3b \ontop{(e)}\in \ZZ}
			\end{align*}
			הראנו כאן שגם $m \in \ZZ$ וגם $n \in \ZZ$ מכאן ש $\ang{m,n} \in \ZZ\times\ZZ$ שזה תחום הפונקציה, ועל כן לכל זוג סדור $\ang{a,b}$ בטווח הפונקציה, קיים זוג סדור $\ang{m,n}$ בתחומה כך ש $g\ang{m,n} = \ang{a,b}$. \\
			ולפיכך $g$ הינה פונקציה על.
		\end{proof}
	\end{enumerate}
	הוכחנו ש $g$ היא חד חד ערכית ועל, וע"כ לפי משפט 3.8 $g$ היא הפיכה. \\ \nl
	(a) - עפ"י הגדרת השוויון על אי זוגות סדורים. \\
	(b) - נכפיל את המשוואה הראשונה ב 3 והשניה ב 2. \\
	(c) - נחסר את המשוואות. \\
	(d) - נציב חזרה במשוואה הראשונה. \\
	(e) - התוצאה של חיסור מספרים שלמים הינה מספר שלם. \\
	\item $g^{-1}\ang{x,y} = \ang{5x-3y, 2y-3x}$
	נוכיח את נכונותו באמצעות משפט 3.25:
	\begin{align*}
		g^{-1}(g\ang{x,y}) &= g^{-1}\ang{2x+3y, 3x + 5y} \\
		&=\ang{5(2x+3y)-3(3x + 5y), 2(3x + 5y) - 3(2x+3y)} \\
		&= \ang{10x+15y -9x -15y, yx + 10y -6x -9y} =
		\ang{x, y}
	\end{align*}
	לפיכך הראנו ש $g^-1 \circ g = I_{\ZZ \times \ZZ}$ והוכחנו את נכונות ההופכי.
\end{enumerate}



\end{document}
