% !TEX program = xelatex
\def\NN{\mathbb{N}}
\def\RR{\mathbb{R}}
\def\ZZ{\mathbb{Z}}
\def\QQ{\mathbb{Q}}
\def\PP{\mathcal{P}}
\def\SS{\mathcal{S}}
\def\DD{\mathcal{D}}
\def\sub{\setminus}
\def\bld{\mathbf}
\def\lmf{\lim_{n \to \infty}}
% Make the ref command use parenthesis
\let\oldref\ref
\renewcommand{\ref}[1]{(\oldref{#1})}

\newcommand{\ontop}[1]{\overset{\text{#1}}}
\newcommand{\ang}[1]{\langle #1 \rangle}
\newcommand{\pip}[1]{\left| #1 \right|}



% style
\newcommand{\bm}[1]{\displaystyle{#1}}
\def\nl{$ $ \newline}

\ExplSyntaxOn

\NewDocumentCommand{\getenv}{om}
{
  \sys_get_shell:nnN{ kpsewhich ~ --var-value ~ #2 }{}#1
}

\ExplSyntaxOff

%environments

\documentclass{article}
\usepackage[]{amsthm} %lets us use \begin{proof}
\usepackage{amsmath}
\usepackage{enumerate}
\usepackage{xparse}
\usepackage[makeroom]{cancel}
\usepackage[]{amssymb} %gives us the character \varnothing
\usepackage{polyglossia}
\setdefaultlanguage{hebrew}
\setotherlanguage{english}
\usepackage{fontspec}
%\setmainfont{Frank Ruehl CLM}
\setmainfont{David CLM}
\setmonofont{Miriam Mono CLM}
\setsansfont{Simple CLM}
% Use the following if you only want to change the font for Hebrew
%\newfontfamily\hebrewfont[Script=Hebrew]{David CLM}
%\newfontfamily\hebrewfonttt[Script=Hebrew]{Miriam Mono CLM}
%\newfontfamily\hebrewfontsf[Script=Hebrew]{Simple CLM}
\getenv[\ID]{ID}

\title{אינפי 1 - ממ"ן 11}
\author{אליחי טורקל \ID}
\date\today

%\clearpage %Gives us a page break before the next section. Optional.
%\selectlanguage{english}
	%Section and subsection automatically number unless you put the asterisk next to them.

\begin{document}
	\maketitle %This command prints the title based on information entered above

	\section*{שאלה 1}
	\subsection*{סעיף א}
	הוכיחו כי
	 $\bm{\forall n( (n \in \NN) \to ({2n \choose n} \leq 4^n) )}$

	\begin{proof}
		\[ 4^n = 2^{2^n}=2^{2n}=(1+1)^{2n} \]
		ולפי משפט 1.34 (נוסחת הבינום של ניוטון):
		\begin{align*}
			(1+1)^{2n} &= \sum_{i=0}^{2n}{{2n \choose i}1^{2n-i}1^i} \\[1em]
			&= \sum_{i=0}^{2n}{2n \choose i} \\[1em]
			&= {2n \choose n} + \sum_{\substack{i=0 \\ i \neq n}}^{2n}{2n \choose i} \geq {2n \choose n} \\
		\end{align*}
		ולכן
		$\bm{4^n \geq {2n \choose n}}$
	\end{proof}


	\subsection*{סעיף ב}
	הוכיחו כי
	 $\bm{\forall n( (n \in \NN) \to ( {2n \choose n} \geq \frac{4^n}{2n+1} ) )}$

	 \begin{proof}
		\nl
		נוכיח באינדוקציה על n+1 \\
		בסיס האינדוקציה
		$n = 1$
		\[
			{2 \choose 1} \geq \frac{4^1}{2+1} = 2 \geq 1 \frac{1}{3}
		\]
נניח שהטענה נכונה בעבור n (הנחת האינדוקציה):
		$\bm{{2n \choose n} \geq \frac{4^n}{2n+1}}$ \\
נוכיח בעבור $n + 1$
		\begin{align*}
			{2(n+1) \choose n+1} \geq &\frac{4^{n+1}}{2(n+1)+1} \\[1em]
			{2n + 2 \choose n + 1} \geq &\frac{4^{n+1}}{2n + 3} \\[1em]
			=&\frac{4 \cdot 4^n}{2n+3}
			=4 \cdot \frac{4^n}{2n+3}
		\end{align*}
עפ"י נוסחת המקדמים הבינומים
		\begin{alignat*}{2}
			{2n + 2 \choose n + 1} &=
			\frac{(2n+2)!}{(n+1)! \cdot (2n+2-n-1)!}
			&&=\frac{(2n+2)!}{(n+1)! \cdot (n+1)!} \\[1em]
			&=\frac{(2n+2) \cdot (2n+1) \cdot (2n)!}{(n+1)^2 \cdot n! \cdot n!}
			&&=\frac{(2n+2) \cdot (2n+1)}{(n+1)^2} \cdot \frac{(2n)!}{n! \cdot n!} \\[1em]
			&=\frac{(2n+2) \cdot (2n+1)}{(n+1)^2} \cdot {2n \choose n}
			&& \Leftarrow \bigg( {2n \choose n} = \frac{(2n)!}{n! \cdot n!} \bigg)
		\end{alignat*}

		השבר השמאלי הוא בהכרח חיובי כי נתון לנו ש n טבעי.
ולכן עפ"י הנחת האינדוקציה ואקסיומה 1.35.4 נכפיל את 2 הצדדים של הנחת האינדוקציה בשבר הנ"ל
		\begin{align*}
			{2n \choose n} \cdot \frac{(2n+2) \cdot (2n+1)}{(n+1)^2} \geq
			&\frac{4^n}{2n+1} \cdot \frac{(2n+2) \cdot (2n+1)}{(n+1)^2} \\
			=&\frac{4^n}{\cancel{2n+1}} \cdot \frac{2 \cdot \cancel{(n+1)} \cdot \cancel{(2n+1)}}{(n+1)\cancel{^2}} \\
			=&\frac{4^n \cdot 2}{n+1}
			=\frac{4^n \cdot 2 \cdot 2}{2(n+1)} =
			\frac{4^{n+1}}{2n+2}
		\end{align*}
		ובגלל ש n טבעי ובעזרת טענה 1.36 מתקבל ש
		$\bm{2n + 3 > 2n + 2}$.
		בשבר חיובי,
		במידה ומגדילים את המכנה אז השבר כולו קטן
		$\bm{\frac{4^{n+1}}{2n+2} \geq \frac{4^{n+1}}{2n+3}}$
		ולכן
		$\bm{{2(n+1) \choose n+1} \geq \frac{4^{n+1}}{2(n+1)+1}}$
	\end{proof}

	\pagebreak
	\section*{שאלה 2}
	יהיו
	$\bm{a,b \in \RR \land b \neq 0}$
	\subsection*{סעיף א}
	הוכיחו ש
	$\bm{|a-b| \leq b^2 \Rightarrow |\frac{a}{b}| \leq |b| + 1}$

	\begin{proof}

		כל מספר ממשי בריבוע הוא אי שלילי, ולכן ניתן להפוך את $b^2$ ל $|b^2|$ ואז לפי טענה 1.48 $|b \cdot b| = |b|b|$ ולאחר מכן נחלק ב $|b|$ שהוא מספר חיובי (עפ"י הנתון וטענה 1.48.1) ונקבל
		\[ \frac{|a-b| }{|b|} \leq |b| \]
		ולפי תכונות הערך המוחלט ושאלה 39.ד:
		\begin{align*}
			|b| \geq \Big| \frac{a-b}{b} \Big|
			\geq  \Big| \frac{a}{b}-1 \Big|
			&\overset{(1)}{\geq}
		 	\bigg| \Big| \frac{a}{b} \Big| - |1| \bigg|
			 \overset{(2)}{\geq}
		 	\Big| \frac{a}{b} \Big| - |1| \\
			 &\Downarrow \\
			|b| &\geq \Big| \frac{a}{b} \Big| -1 \\
			|b| + 1 &\geq \Big| \frac{a}{b}  \Big| \Rightarrow
			\Big| \frac{a}{b} \Big| \leq |b| + 1
		\end{align*}
		(1) - עפ"י טענה 1.49(אי שוויון המשולש) \\
		(2) - עפ"י טענה 1.48(תכונות הערך המוחלט)
	\end{proof}

	\subsection*{סעיף ב}
	הוכיחו ש
	$\big( \frac{a + |a|}{2} \big)^2 + \big( \frac{a - |a|}{2} \big)^2 = a^2$
	\begin{proof}
		אם $a > 0$
		\[
			\Big( \frac{a+a}{2} \Big)^2 + \Big( \frac{a-a}{2} \Big) ^2 =
			\Big( \frac{\cancel{2}a}{\cancel{2}} \Big)^2 + \cancel{\Big( \frac{0}{2} \Big)^2} = a^2
		\]
				אם $a \geq 0$
		\[
		\Big( \frac{a-a}{2} \Big)^2 + \Big( \frac{a-(-a)}{2} \Big) ^2 =
		\cancel{\Big( \frac{0}{2} \Big)^2} + \Big( \frac{\cancel{2}a}{\cancel{2}} \Big)^2 = a^2
		\]
	\end{proof}

	\pagebreak
	\section*{שאלה 3}
	\subsection*{סעיף א}
	פתרו את האי שוויון
	$\lfloor |x+1| - |x| \rfloor \geq x^2$
	\begin{proof}
		נפריד את האי שוויון לתחומים
		\begin{enumerate}[I]
			\item אם $x \geq 0$ אז $x + 1 > 0$
			\[
				\lfloor x + 1 - x \rfloor \geq x^2 \Rightarrow
				\lfloor 1 \rfloor \geq x^2 \Rightarrow
				1 \geq x^2 \Rightarrow
				-1 \leq x \leq 1
			\]
				מכיוון שאנחנו בתחום של $x \geq 0$ אז נקבל $0 \leq x \leq 1$
			\item אם $-1 \leq x < 0$ אז $x+1 > 0$
			\[
				\lfloor x + 1 + x \rfloor \geq x^2 \Rightarrow
				\lfloor 2x+1 \rfloor \geq x^2
			\]
			עפ"י טענה 1.64.1 (תכונות החלק השלם)
			\[
				2x + 1 \geq \lfloor 2x + 1 \rfloor \geq x^2 \Rightarrow
				-x^2 + 2x + 1 \geq 0
			\]
			עפ"י נוסחת השורשים:
			$ax^2 + bx + c = 0 \Rightarrow \frac{-b \pm \sqrt{b^2 -4ac}}{2c}$
			\[
			\frac{-2 \pm \sqrt{4+4}}{-2} =
			\frac{\cancel{-2}}{\cancel{-2}} \pm \frac{\cancel{2}\sqrt{2}}{\cancel{2}} =
			1 \pm \sqrt{2}
			 \]
			 \[ 1 - \sqrt{2} \leq x \leq 1 + \sqrt{2} \]
			 מכיוון שאנחנו בתחום של $x > 0$ ניקח את האופציה שתתן לנו את המספר הקטן ביותר $1 - \sqrt{2}$ ונציב
			 \[
			 	\lfloor2 \cdot (1 - \sqrt{2}) + 1 = \lfloor 2 - 2\sqrt{2} \rfloor = \lfloor 1 - \sqrt{2} \rfloor = 0
			 \]
 			 אז המספר הקטן ביותר נותן לנו $x=0$ והתחום הוא $x<0$ ולכן אין פתרון בתחום
 			 \item אם $x \leq -1$ אז $x + 1 \leq 0$
 			 \[
 			 	\lfloor -x -1 +x \rfloor \geq x^2 \Rightarrow
 			 	\lfloor -1 \rfloor \geq x^2 \Rightarrow
 			 	-1 \geq x^2
 			 \]
 			 ול $-1 \geq x^2 $ אין פתרון, ולכן אין פתרון בתחום.
		\end{enumerate}
	אם נאחד את כל התחומים נקבל את הפתרון
	$0 \leq x \leq 1 \iff [0,1]$
	\end{proof}

	\pagebreak
	\subsection*{סעיף ב}
	פתרו את המשוואות הבאות
	\begin{enumerate}[i]
		\item $\lfloor x \rfloor^2 = 16$
		\begin{proof}
			\[
				\lfloor x \rfloor^2 = 16 \Rightarrow
				\lfloor x \rfloor = \sqrt{16} = \pm 4
			\]
			עפ"י טענה 1.64 (תכונת החלק השלם)
			\[4 \leq x < 5 \lor -4 \leq x < -3 \iff [-4, -3) \cup [4,5) \]
		\end{proof}
		\item $\lfloor x^2 \rfloor = 3$
		\begin{proof}
			עפ"י טענה 1.64 (תכונת החלק השלם)
			\[ 3 \leq x^2 < 4 \]
			\begin{gather*}
				\begin{flalign*}
					&(x \geq \sqrt{3} \lor x \leq - \sqrt{3}) \Leftarrow x^2 \geq 3
					&x^2 < 4 \Rightarrow  (-2 < x < 2) &&
				\end{flalign*} \\
				\Downarrow \\
				(-2 < x \leq -\sqrt{3}) \lor (\sqrt{3} \leq x < 2)  \iff
					(-2, -\sqrt{3}] \cup [\sqrt{3}, 2)
			\end{gather*}
		\end{proof}
	\end{enumerate}

	\pagebreak
	\section*{שאלה 4}
	\subsection*{סעיף א}
	תהי קבוצה $A \subseteq \RR$ צפופה בקטע $(1, \infty)$ .הוכיחו שהקבוצה
	 $\bm{B = \bigg\{ \dfrac{a}{n} \bigg| a \in A, n \in \NN \bigg\}}$
	 צפופה בקטע $(0,1)$

	 יהי $x,y \in (0,1)$ כך ש $y > x$ נוכיח שקיים $b \in B$ כך ש $x < b < y$ כלומר, שקיימים $a \in A, n \in \NN$ כך ש $x < \frac{a}{n} < y$
	 \begin{proof}
	 	\[
	 		x < \frac{a}{n} < y \Rightarrow
	 		xn < a < yn
	 	 \]
	 	 עפ"י מסקנה 1.61 לכל $c > 0$ קיים $n \in \NN$ ככה ש
	 	 $1 < cn \Leftarrow \frac{1}{n} < c$
	 	 ולכן $1 < xn < a < yn$ ומכאן $x < \frac{a}{n} < y$.
	 \end{proof}

 	\subsection*{סעיף ב}
 	נסחו את המשפט A אינה צפופה בקטע I בלי להשמש ב "לא". \\
 	תשובה:
 	קיימים $x,y \in I$ המקיימים $x < y$ ככה שלכל $a \in A$ מתקיים $x \geq a$ או $a \geq y$
 	ומוצרן:
 	$\bm{\exists x,y \in I(x< y \rightarrow \forall a \in A (a \geq y \lor a \leq x))}$

 	\subsection*{סעיף ג}
 	תהי $A \subseteq (1, \infty)$ הוכיחו שהקבוצה
 	$C = \bigg\{ \dfrac{a}{n^2(a+1)} \bigg| a \in A, n \in \NN \bigg\}$
 	אינה צפופה בקטע $[0,1]$
 	\begin{proof}
 		נוכיח כי קיימים $x, y \in [0,1]$ ככה שלכל $c \in C$ מתקיים $x \geq c \lor c \geq y$.
 		נסתכל על
 		$n \in \NN, a \in A$  \ $\bm{c = \frac{a}{n^2(a+1)}}$
 		\begin{enumerate}[I]
			\item אם $n = 1$ אז
			\[
				c = \frac{a}{1^2(a+1)}
				  = \frac{a + 1 - 1}{a + 1}
				  = 1 - \frac{1}{a+1}
			\]
			ובגלל ש A מוכלת בתוך $(1, \infty)$ ו $a \in A$ אז $a > 1$ ולכן בגלל שיש לנו את a במכנה של שבר שלילי, ניקח את ה a הקטן ביותר ע"מ לקבל את ה c הקטן ביותר האפשרי. ולכן נציב $a = 1$:
			\[ c > 1 - \frac{1}{1+1} \iff c > \frac{1}{2} \]

			\item  אם $n \geq 2$
			\[ c = \frac{1}{n^2} \cdot \frac{a}{a+1} \]
			נשים לב ש
			$\frac{a}{a+1} < 1 \Leftarrow 1 < a < a + 1$
			ולכן נבחר a ככה ש $\frac{a}{a+1}$ הוא 1, ונבחר $n = 2$ וככה נקבל את המספר המקסימלי ש c יכול להיות

			\[ c = \frac{a}{n^2(a+1)} < \frac{1}{n^2} \cdot 1 \leq \frac{1}{2^2} \leq \frac{1}{4} \]
			ולכן
			$c > \frac{1}{2} \lor c \leq \frac{1}{4}$
			עכשיו ניקח x,y שנמצאים בדיוק בין רבע לחצי, לדוגמא
			$x = \frac{1}{3}, y = \frac{2}{5}$
			שניהם נמצאים ב $[0,1]$ ומקיימים $y > x$. אבל בגלל שלכל $c \in C$ מתקיים
			 $c > \frac{1}{2} \lor c \leq \frac{1}{4}$ ובנוסף גם x וגם y גדולים מחצי וקטנים מרבע אזי לא קיים c שמקיים $x < c < y$
		\end{enumerate}
 	\end{proof}

\end{document}
