% !TEX program = xelatex
\def\NN{\mathbb{N}}
\def\RR{\mathbb{R}}
\def\ZZ{\mathbb{Z}}
\def\QQ{\mathbb{Q}}
\def\PP{\mathcal{P}}
\def\SS{\mathcal{S}}
\def\DD{\mathcal{D}}
\def\sub{\setminus}
\def\bld{\mathbf}
\def\lmf{\lim_{n \to \infty}}
% Make the ref command use parenthesis
\let\oldref\ref
\renewcommand{\ref}[1]{(\oldref{#1})}

\newcommand{\ontop}[1]{\overset{\text{#1}}}
\newcommand{\ang}[1]{\langle #1 \rangle}
\newcommand{\pip}[1]{\left| #1 \right|}



% style
\newcommand{\bm}[1]{\displaystyle{#1}}
\def\nl{$ $ \newline}

\ExplSyntaxOn

\NewDocumentCommand{\getenv}{om}
{
  \sys_get_shell:nnN{ kpsewhich ~ --var-value ~ #2 }{}#1
}

\ExplSyntaxOff

%environments

\documentclass{article}
\usepackage[]{amsthm} %lets us use \begin{proof}
\usepackage{amsmath}
\usepackage{mathtools}
\usepackage{enumerate}
\usepackage{xparse}
\usepackage[makeroom]{cancel}
\usepackage[]{amssymb} %gives us the chA \mathcal{R} Acter \varnothing
\usepackage{polyglossia}
% \usepackage[frak=mma]{mathalfa}
\setdefaultlanguage{hebrew}
\setotherlanguage{english}
\usepackage{fontspec}
%\setmainfont{Frank Ruehl CLM}
\setmainfont{David CLM}
\setmonofont{Miriam Mono CLM}
\setsansfont{Simple CLM}
\DeclarePairedDelimiter\set\{\}
% Use the following if you only want to change the font for Hebrew
%\newfontfamily\hebrewfont[Script=Hebrew]{David CLM}
%\newfontfamily\hebrewfonttt[Script=Hebrew]{Miriam Mono CLM}
%\newfontfamily\hebrewfontsf[Script=Hebrew]{Simple CLM}
\getenv[\ID]{ID}

\title{אינפי 1 - ממ"ן 14}
\author{אליחי טורקל \ID}
\date\today

%\clearpage %Gives us a page break before the next section. Optional.
%\selectlanguage{english}
	%Section and subsection automatically number unless you put the asterisk next to them.

\begin{document}
	\maketitle %This command prints the title based on information entered above

	\section*{שאלה 1}
	יהיו $f$ ו $g$ פונקציות מ $\RR$ ל $\RR$ הוכיחו או הפריכו את הטענות הבאות:
	\subsection*{סעיף א}
	אם $f \circ g$ היא על $\RR$, אזי $f$ היא על $\RR$.
	\begin{proof}
		יהי $b \in \RR$ מכיוון ש $f \circ g$ היא על קיים $a \in \RR$ כך ש $b = f(g(a))$, ובגלל ש $g: \RR \to \RR$ אזי $c = g(a) \in \RR$ ולפיכך מצאנו $c \in \RR$ כך ש $b = f(c)$ ובכך הראינו, שלכל $b \in \RR$ קיים $c$ כך ש $b = f(c)$ ולכן $f$ היא על.
	\end{proof}

	\subsection*{סעיף ב}
	אם $f \circ g$ היא על, אזי $g$ היא על.
	\begin{proof} \nl
		הטענה אינה נכונה, לדוגמא:
		$g(x) = \begin{cases}
			x \text{  if } x > 0 \\
			x-1 \text{  if } x \leq 0
		\end{cases}$ ,
		$f(x) = \begin{cases}
			x \text{  if } x > 0 \\
			x + 1 \text{ if } x \leq 0
		\end{cases}$. \\
	\begin{enumerate}
		\item כאשר $x > 0$ נקבל $g(x) = x$ וכאשר $x \leq 0$ נקבל $g(x) = x-1$ ולכן \\
		$(-1, 0]$ לא נמצא בתמונה של $g$, ולפיכך $g$ אינה על.
		\item כאשר $x > 0$ נקבל $f(g(x)) = x$ וכאשר $x \leq 0$ נקבל $f(g(x)) = (x - 1) + 1 = x$ ולכן לכל $x \in \RR$ מתקיים $(f \circ g)(x) = x$ ולכן $f \circ g$ היא על.
	\end{enumerate}
	\end{proof}

	\subsection*{סעיף ג}
	אם $f \circ g$ היא על ו $f$ היא חח"ע אזי $g$ היא על.

	\begin{proof}
	לפי הנתון וסעיף א נקבל ש $f$ היא חח"ע ועל, ולפיכך קיימת פונקציה $f^{-1}$ שהיא ההופכית של $f$ וגם היא חח"ע ועל. \\
	ובנוסף נתון לנו ש $f \circ g$ היא על, ולכן לכל $b \in \RR$ קיים $c \in \RR$ ככה ש $b = f^{-1}(c)$, ובגלל ש $f \circ g$ היא על אז קיים $a \in \RR$ ככה ש $c = f(g(a))$,
	ולכן $f^{-1}(c) = g(a) = b$ ולכן, לכל $b \in \RR$ קיים $a \in \RR$ המקיים $g(a) = b$ ולפיכך $g$ הינו על.
	\end{proof}

	\subsection*{סעיף ד}
	אם $f \circ g$ היא מונוטונית עולה, אז $g$ היא מונוטונית.
	\begin{proof}
		$f(x) = g(x) = \begin{cases}
			\frac{1}{x} \text{  if } x \neq 0 \\
			0 \text{  if } x = 0
		\end{cases}$
		לפי שאלה 13 בספר, ראינו ש $g(x)$ מונוטונית יורדת בכל הקטע בה היא מוגדרת, ובגלל שהגדרנו את $x=0$ אזי הגדרנו את התחום שלה לכל $\RR$ כנדרש.
		אבל היא כבר לא מונטונית יורדת , מכיוון ש $0 > -1$ ו $0 > \frac{1}{-1}$ $\Rightarrow$ $f(0) > f(-1)$ שזה עולה, למרות שבין איברים שאינם אפס הערך יורד כמו שרואים בשאלה 13. \\
		אבל כאשר $x = 0$ נקבל $f(g(x)) = 0$ וכאשר $x \neq 0$ נקבל $f(g(x)) = \frac{1}{\frac{1}{x}} = x$ ולפיכך $(f \circ g)(x) = x$ לכל $x \in \RR$ וכידוע זו פונקציה מונוטונית עולה.
	\end{proof}

	\subsection*{סעיף ה}
	אם $f \circ g$ היא מונוטונית עולה, ו $f$ היא מונוטונית יורדת, אזי $g$ היא מונוטונית יורדת
	\begin{proof}
		יהי $x_1, x_2 \in \RR$ בלי הגבלת הכלליות, נניח ש $x_1 > x_2$ לפי הנתון מתקבל ש $f(g(x_1)) > f(g(x_2))$,
		יהי $y_1 = g(x_1), y_2 = g(x_2)$, בעקבות הנתון ש $f$ הינה מונוטונית יורדת, ובגלל ש $f(y_1) > f(y_2)$ מתקבל ש $y_1 < y_2$ ובעצם ש $g(x_1) < g(x_2)$ ובגלל ש $x_1 > x_2$ מתקבל ש $g$ מונוטונית יורדת.
	\end{proof}

	\pagebreak
	\section*{שאלה 2}

	\subsection*{סעיף א}
	הוכיחו לפי הגדרת הגבול בנוסח קושי את הגבול: $\limf{x}{4}\sqrt{2x^2 - 7} = 5$
	\begin{proof}
		לכל $\varepsilon > 0$ נבחר $\delta = min\set{1, \frac{5\varepsilon}{18}} > 0$
		כך שלכל $x$ המקיים $0 < |x - 4| < \delta$ נראה שמתקיים $\left| \sqrt{2x^2 - 7} - 5 \right| < \varepsilon$. \\
		נתחיל בלהראות שהפונקציה מוגדרת בסביבה שבחרנו:
		$|x-4| \leq 1 \Rightarrow 3 < x < 5$ ונראה ש $2 \cdot 3^2-7 = 47 > 0$ ולכן השורש מוגדר, והפונקציה מוגדרת בסביבה. \\
		נשחק עם $\left| \sqrt{2x^2 - 7} - 5 \right| < \varepsilon$, נתחיל בלחלק בצמוד ששונה בהכרח מאפס (שורש הוא חיובי + 5).
		\begin{align*}
			\left| \sqrt{2x^2 - 7} - 5  \right| &=
			\left| \frac{(\sqrt{2x^2 - 7} - 5)(\sqrt{2x^2 - 7} + 5)}{\sqrt{2x^2 - 7} + 5}  \right| =
			\left| \frac{2x^2 - 32}{\sqrt{2x^2 - 7} + 5}  \right| \\
			&= \left| \frac{2(x+4)(x-4)}{\sqrt{2x^2 - 7} + 5}  \right| =
			 2|x-4| \frac{|x+4|}{|\sqrt{2x^2 - 7} + 5|} \ontop{(a)} <
			 2|x-4| \frac{x+4}{5}
		\end{align*}
		ידוע לנו כי $|x-4| < \delta$ וגם ש $x < 5$ לכן נציב במה שקיבלנו:
		\[
			2|x-4| \frac{x+4}{5} <
			2\cdot \frac{5\varepsilon}{18} \cdot \frac{5+4}{5} \leq \varepsilon
		\]
		והוכחנו ש $|f(x) - L| < \varepsilon$ כנדרש בהגדרת הגבול. \\
		(a) - הראנו כבר שהמכנה חיובי,ובנוסף $x$ חיובי, ולכן ניתן להוריד את הערך המוחלט, ונקטין את המכנה ע"מ להגדיל את הערך
	\end{proof}
	\pagebreak
	\subsection*{סעיף ב}
	הוכיחו לפי הגדרת הגבול בלשון $\varepsilon, M$ את הגבול: $\bm\lmf{\frac{x+1}{\lfloor x \rfloor}} = 1$
	\begin{proof}
		הפונקציה מוגדרת לכל $x \neq 0$ ונסתכל על $x > 1$
		$\left| \frac{x+1}{\lfloor x \rfloor} - 1 \right| < \varepsilon$,
		מתכונות החלק השלם $x-1 \leq \lfloor x \rfloor$ ובגלל ש $x$ חיובי, אנו רואים כי $x+1>x-1$ ולכן המכנה תמיד גדול מהמונה, ולכן השבר כולו גדול מאחד והערך המוחלט כולו גדול מאפס.
		ולפיכך:
		\begin{align*}
			&\frac{x+1}{x-1}-1 \leq \frac{x+1}{\lfloor x \rfloor} - 1 = \left|\frac{x+1}{\lfloor x \rfloor} - 1 \right| \\
			&\frac{x+1}{x-1}-1 =
			\frac{x-1+1+1}{x-1} -1 =
			\frac{2}{x-1}
		\end{align*}
		ולכן, נגדיר $M = \frac{4}{\varepsilon} + 1$ ולכן לכל $x > M$ מתקיים: \\
		\begin{align*}
			\frac{2}{x-1} \ontop{(a)} < \frac{2}{\frac{4}{\varepsilon} + 1 - 1} \leq \frac{\varepsilon}{2} < \varepsilon
		\end{align*}
		(a) - הגדלת המכנה מקטינה את השבר.

		ולפיכך הראינו שלכל $\varepsilon > 0$ קיים $M = \frac{4}{\varepsilon}+1$ כך שלכל $x > M > 1$ מתקיים $\left| \frac{x+1}{\lfloor x \rfloor} - 1 \right| < \varepsilon$ כנדרש לפי הגדרת הגבול.
	\end{proof}

	\pagebreak
	\section*{שאלה 3}
	\subsection*{סעיף א}
	תהי $f$ פונקציה המוגדרת בקטע $(M_0, \infty)$.
	נסטחו את הטענה "לא קיים ל $f$ גבול סופי כש $x \to \inf$": \\
	\textbf{פתרון:}  לא קיים ל $f$ גבול סופי כש $x \to \infty$
	\begin{enumerate}
		\item בניסוח קושי: \\ אמ"מ קיים $\varepsilon > 0$ כך שלכל $M \in \RR$ קיים $x > M$ המקיים $|f(x)-L| \geq \varepsilon$.
		\item בניסוח היינה: \\ אמ"מ קיימת סדרה: $\bm{(x_n)_{n=1}^\infty}$ המקיימת $\bm{\lmf x_n = \infty}$ כך ש $\bm{\lmf f(x_n) \not= L}$
	\end{enumerate}

	\subsection*{סעיף ב}
	הוכיחו כי $L = \frac{1}{2}$ אינו הגבול של $f(x) = \ang{x}$ כש $x \to \infty$ בלשון קושי.
	\begin{proof}
		יהי $\varepsilon = \frac{1}{4}$, לכל $M \in \RR$ נגדיר $x = \lceil M \rceil + 1 > M$,
		נראה ש $|\ang{x} - \frac{1}{2}| \geq \frac{1}{4}$: \\
		$x$ הינו מספר שלם, ולכן לפי תכונת החלק השברי מתקיים $\ang{x} = 0$ ולכן:
		\[
			\left| \ang{x} - \frac{1}{2} \right| =
			\left| 0 - \frac{1}{2} \right| =
			\frac{1}{2} \geq \frac{1}{4}
		\]
		ובכך הוכחנו שחצי אינו הגבול של $f(x)$ כנדרש.
	\end{proof}

	\pagebreak

	\subsection*{סעיף ג}
	הוכיחו כי לא קיים ל $f(x) = \ang{x}$ גבול סופי כש $x \to \infty$ בשתי דרכים:
	\begin{enumerate}
		\item לפי הגדרת קושי:
		\begin{proof}
			לכל $L \in \RR, L \neq 0$ יהי $\varepsilon = \left| \frac{L}{2} \right|$, לכל $M \in \RR$ נגדיר $x = \lceil M \rceil + 1 > M$ ונראה ש $|\ang{x} - L| \geq \left| \frac{L}{2} \right|$: \\
			$x$ הינו מספר שלם, ולכן לפי תכונות החלק השלם מתקיים $\ang{x} = 0$ ולכן:
			\[
				\left|0 - L \right| = |L| \geq \left| \frac{L}{2} \right|
			\]
			נוכיח בנפרד שאין גבול ב $L = 0$, יהי $\varepsilon = \frac{1}{4}$, לכל $M \in \RR$ נגדיר $x = \rceil M \rceil + \frac{1}{2} > M$ ונראה ש $|\ang{x} - 0| \geq \frac{1}{4}$: \\
			$x$ הינו סכום של מספר שלם ושל חצי ולפיכך $\ang{x} = \frac{1}{2}$, ולכן:
			\[
			\left| \frac{1}{2} - 0 \right| = \frac{1}{2} \geq \frac{1}{4}
			\]
		\end{proof}

		\item לפי הגדרת היינה אם קיימות 2 סדרות $(a_n), (b_n)$ המקיימות $\bm{\lmf a_n = \lmf b_n = \infty}$ אך $\bm{\lmf f(a_n) \neq \lmf f(b_n)}$ אזי אין ל $f$ גבול כאשר $x \to \infty$. \\
		\begin{proof} \nl
			תהי $(a_n) = n$ ולכן $\bm{\lmf a_n = \infty}$, ותהי $(b_n) = n + \frac{1}{3}$ \\
			 ולכן $\bm{\lmf b_n = "\infty + \frac{1}{3}" = \infty}$. \\
			לפי תכונת החלק השברי, לכל מספר שלם $\ang{n} = 0$ ולכן $\bm{\lmf f(a_n) = \lmf 0 = 0}$,
			 ובנוסף לפי תכונת החלק השברי, לכל $n$ שלם מתקיים $\ang{n + \frac{1}{3}} = \frac{1}{3}$ \\
			 ולכן $\bm{\lmf f(b_n) = \lmf \frac{1}{3} = \frac{1}{3}}$. \\
			 אזי אין ל $f$ גבול כאשר $x \to \infty$.
		\end{proof}
	\end{enumerate}

	\pagebreak
	\section*{שאלה 4}
	חשבו את הגבולות הבאים או הוכיחו שאינם קיימים:
	\subsection*{סעיף א}
	\[ \lim_{x \to 0} \frac{\tan 5x}{\sin 3x} \]
	\begin{proof}
		נגדיר סביבה נקובה $N^{*}_{\frac{\pi}{2}}(0)$, ועל כן ניתן לחלק בכפולות של $x$.
		\[
			\frac{\tan 5x}{\sin 3x} =
			\frac{5x \frac{\tan 5x}{5x}}{3x \frac{\sin 3x}{3x}} =
			\frac{5}{3} \cdot \frac{\frac{\tan 5x}{5x}}{\frac{\sin 3x}{3x}} =
		\]

		לפי שאלה 63 הגבול לכל $f(x)$ עם גבול ב $x \to 0$ ולכל $k \neq 0$ מתקיים $\bm{\lim_{x \to 0} f(x) = \lim_{x \to 0} f(kx)}$. \\
		ולכן ביחד עם משפט 4.45 נקבל: $\bm{\lim_{x \to 0}\frac{\sin kx}{kx}} = 1$, וביחד עם שאלה 67 $\bm{\lim_{x \to 0}\frac{\tan kx}{kx}} = 1$ \\
		ולכן בעזרת אריתמטיקה של גבולות וגבול פונקציה לינארית:
		\[
			\lim_{x \to 0}
			\frac{5}{3} \cdot \frac{\frac{\tan 5x}{5x}}{\frac{\sin 3x}{3x}} =
			\frac{5}{3} \cdot \frac{1}{1} = \frac{5}{3}
		\]
		לסיכום $\lim_{x \to 0} \frac{\tan 5x}{\sin 3x} = \frac{5}{3}$
	\end{proof}

	\subsection*{סעיף ב}
	\[ \lim_{x \to 0} \frac{\sqrt{\cos x} - \cos x}{x^2} \]
	\begin{proof}
		נסתכל על הסביבה הנקובה $N^{*}_{\frac{\pi}{2}}(0)$ ונקבל ש $\sqrt{\cos x} + \cos x > 0$(הנקודה שבה קוסינוס חוצה את ציר ה $x$). \\
		ועכשיו נכפיל את המונה והמכנה בצמוד:
		\begin{align*}
			\frac{\sqrt{\cos x} - \cos x}{x^2} &=
			\frac{(\sqrt{\cos x} - \cos x)(\sqrt{\cos x} + \cos x)}{x^2(\sqrt{\cos x} + \cos x)} =
			\frac{(\cos x - \cos^2 x}{x^2(\sqrt{\cos x} + \cos x)} \\
			&= \frac{-\cos x(\cos x - 1)}{x^2(\sqrt{\cos x} + \cos x)} =
			\frac{-cos x}{\sqrt{\cos x} + \cos x} \cdot \frac{cos x - 1}{x^2}
		\end{align*}
		לפי שאלה 67 $\bm{\lim_{x \to 0}} \frac{\cos x -1}{x^2} = - \frac{1}{2}$.
		לפי טענה 4.44 $\bm{\lim_{x \to 0}} \cos x = 1$,
		ולפי שאלה 88 $\bm{\lim_{x \to 1}} \sqrt{x} = 1$
		ובגלל שדיברנו על סביבה נקובה של $N^{*}_{\frac{\pi}{2}}(0)$ אזי $cos \neq 0$
		ולכן מתקיימים שלשת התנאים של גבול פונקציית הרכבה, ולפיכך $\bm{\lim_{x \to 0}} \sqrt{\cos x} = 1$. \\
		עכשיו נציב את הכל ונשתמש באריתמטיקה של גבולות:
		\[
			\bm{\lim_{x \to 0}}
			\frac{-cos x}{\sqrt{\cos x} + \cos x} \cdot \frac{cos x - 1}{x^2} =
			\frac{-1}{1 + 1} \cdot -\frac{1}{2} =
			\frac{-1}{2} \cdot -\frac{1}{2} =
			\frac{1}{4}
		\]
	\end{proof}
	\subsection*{סעיף ג}
	\[ \limf{x}{\infty} \frac{3x^5 - 5x^4 + x \cos x}{3x^2 - 5x^3 + x \sqrt{x}} \]
	\begin{proof}
		נוציא גורם משותף מהמונה והמכנה:
		\begin{align*}
			\frac{3x^5 - 5x^4 + x \cos x}{3x^2 - 5x^3 + x \sqrt{x}} =
			\frac{x^5(3 - \frac{5}{x} + \frac{1}{x^4}\cos x)}{x^3(\frac{3}{x} - 5 + \frac{1}{x}\frac{1}{\sqrt{x}})} =
			x^2\frac{3 - \frac{5}{x} + \frac{1}{x^4}\cos x}{\frac{3}{x} - 5 + \frac{1}{x}\frac{1}{\sqrt{x}}}
		\end{align*}
		הגבול הידוע של $\limf{x}{\infty}\frac{1}{x^4}=0$ ובנוסף $\cos x$ חסומה מלעיל ומלרע. ולכן לפי משפט 2.22 לפונקציות נקבל ש $\limf{x}{\infty}cos x \frac{1}{x^4} = 0$. \\
		נפצל את הגבול ל2, החצי הראשון הינו גבול ידוע $\limf{x}{\infty}x^2 = \infty$ ובגבול השני נציב גבולות ידועים ונשתמש באריתמטיקה של הגבולות:
		\[
			\limf{x}{\infty}
			\frac{3 - \frac{5}{x} + \frac{1}{x^4}\cos x}{\frac{3}{x} - 5 + \frac{1}{x}\frac{1}{\sqrt{x}}}
			\frac{3 - 0 + 0}{0 - 5 + 0} =
			\frac{-3}{5}
		\]
		וביחד נקבל:
		\[
			\limf{x}{\infty} \frac{3x^5 - 5x^4 + x \cos x}{3x^2 - 5x^3 + x \sqrt{x}} =
			"\infty \cdot \frac{-3}{5}" = -\infty
		\]
	\end{proof}

	\subsection*{סעיף ד}
	\[ \limf{x}{0} \frac{x + \sin x}{x^2 + \sin^2 x} \]
	\begin{proof}
		נחלק את המונה והמכנה ב $x$ (סביבה נקובה של אפס ולכן $x \neq 0$)
		\begin{align*}
			\frac{x + \sin x}{x^2 + \sin^2 x} =
			\frac{1 + \frac{\sin x}{x}}{x + \sin x \frac{\sin x}{x}}
		\end{align*}
		נתחיל בלמצוא גבול מימין, ז"א כאשר $x > 0$, נשתמש בגבולות ידועים, ובפרט בגבולות ממשפט 4.45 וטענה 4.44, יש לשים לב שהגבולות הם משני הצדדים, ובפרט מימין(נהפוך $0$ ל $0^+$ משפט 4.48).
		נציב את הגבולות ממשפט 4.45 וטענה 4.44 ונעשה אריתמטיקה ע"מ למצוא גבול מימין,
		\begin{align*}
			\limf{x}{0^+}
			\frac{1 + \frac{\sin x}{x}}{x + \sin x \frac{\sin x}{x}}  =
			\frac{1 + 1}{0^+ + 0^+ \cdot 1}  = \frac{2}{0^+} = \infty
		\end{align*}
		נעשה את אותו הדבר בשביל גבול משמאל (נשתמש בגבולות לשני הצדדים, ונחליף את $0$ ב $0^-$ משפט 4.48):
		\begin{align*}
			\limf{x}{0^-}
			\frac{1 + \frac{\sin x}{x}}{x + \sin x \frac{\sin x}{x}}  =
			\frac{1 + 1}{0^- + 0^- \cdot 1}  = \frac{2}{0^-} = -\infty
		\end{align*}
		ניתן לראות כי הגבולות החד צדדים שונים, ולכן לפי 4.48 אין לפונקציה גבול בנקודה $x=0$
	\end{proof}

	\pagebreak

	\subsection*{סעיף ה}
	\[
		\limf{x}{0} \lfloor \tan x \rfloor \cos x,
		\limf{x}{\frac{\pi}{2}} \lfloor \tan x \rfloor \cos x
	\]
	\begin{proof}
		נתחיל עם הגבול $\limf{x}{0} \lfloor \tan x \rfloor \cos x$.
		נסתכל על הגבולות החד צדדיים, מצד ימין בסביבה $N_{\frac{\pi}{4}}(0^+)$ ולפי הגדרת החלק השלם, נקבל ש $\lfloor \tan x \rfloor = 0$ (בין 0 ל $\pi/4$ טנגנס נע בין 0 ל 1 )
		ובעזרת אריתמטיקה של גבולות נקבל:
		\begin{align*}
			\limf{x}{0^+} 0^+ \cdot 1 = 0^+
		\end{align*}
		ומצד שמאל, בסביבה $N_{\frac{\pi}{4}}(0^-)$ ולפי הגדרת החלק השלם, נקבל ש $\lfloor \tan x \rfloor = -1$(בין $-\pi/4$ ל$0$ טנגנס נע בין $-1$ ל $0$).
		וגם כאן נשתמש באריתמטיקה:
		\begin{align*}
			\limf{x}{0^-} -1 \cdot 1 = -1
		\end{align*}
		וקיבלנו 2 גבולות חד צדדים שונים, ולפי 4.48 מתקבל שאין לפונקציה גבול בנקודה ${x = 0}$.
	\end{proof}
	נוכיח את הגבול $\limf{x}{\frac{\pi}{2}} \lfloor \tan x \rfloor \cos x$. \\
	 נפצל לשני צדדים, ונתחיל משמאל בסביבה הנקובה $N_{\frac{\pi}{2}}$ ולכן $\cos x > 0$.
	מתכונות החלק השלם נקבל:
	\begin{align*}
		(\tan x -1) \cos x &< \lfloor \tan x \rfloor \cos x \leq \tan x \cos x \\
		\cos x \tan x - \cos x &< \lfloor \tan x \rfloor \cos x \leq \tan x \cos x \\
		\sin x - \cos x &< \lfloor \tan x \rfloor \cos x \leq \sin x
	\end{align*}
	נשתמש במשפט הסנדוויץ' ונחשב את גבולות שני הצדדים בעזרת שאלה 77 ואריתמטיקה:
	$\limf{x}{\frac{\pi}{2}^-}\sin x - \cos x = 1 - 0 = 1$. \\
	והצד השני:
	$\limf{x}{\frac{\pi}{2}^-}\sin x = 1$ \\
	ולפיכך לפי משפט הסנדוויץ' נקבל כי $\limf{x}{\frac{\pi}{2}^-} \lfloor \tan x \rfloor \cos x = 1$. \\

	נסתכל על אותה סביבה נקודה אך מצד ימין, ולכן $\cos x < 0$ ומתכונות החלק השלם נקבל:
	\begin{align*}
		(\tan x -1) \cos x &> \lfloor \tan x \rfloor \cos x \geq \tan x \cos x \\
		\cos x \tan x - \cos x &> \lfloor \tan x \rfloor \cos x \geq \tan x \cos x \\
		\sin x - \cos x &> \lfloor \tan x \rfloor \cos x \geq \sin x
	\end{align*}
	וגם כאן נשתמש בסנדוויץ' ונקבל את אותן גבולות מכיוון שהן גבולות מלאים ויש לנו את 4.48.
	ולכן $\limf{x}{\frac{\pi}{2}^+}\sin x - \cos x = 1$, ו- $\limf{x}{\frac{\pi}{2}^+}\sin x = 1$.
	ולכן גם כאן נקבל בעזרת הסנדוויץ' $\limf{x}{\frac{\pi}{2}^+} \lfloor \tan x \rfloor \cos x = 1$. \\
	וממשפט 4.48 נקבל $\limf{x}{\frac{\pi}{2}} \lfloor \tan x \rfloor \cos x = 1$
\end{document}
