% !TEX program = xelatex
\def\NN{\mathbb{N}}
\def\RR{\mathbb{R}}
\def\ZZ{\mathbb{Z}}
\def\QQ{\mathbb{Q}}
\def\PP{\mathcal{P}}
\def\SS{\mathcal{S}}
\def\DD{\mathcal{D}}
\def\sub{\setminus}
\def\bld{\mathbf}
\def\lmf{\lim_{n \to \infty}}
% Make the ref command use parenthesis
\let\oldref\ref
\renewcommand{\ref}[1]{(\oldref{#1})}

\newcommand{\ontop}[1]{\overset{\text{#1}}}
\newcommand{\ang}[1]{\langle #1 \rangle}
\newcommand{\pip}[1]{\left| #1 \right|}



% style
\newcommand{\bm}[1]{\displaystyle{#1}}
\def\nl{$ $ \newline}

\ExplSyntaxOn

\NewDocumentCommand{\getenv}{om}
{
  \sys_get_shell:nnN{ kpsewhich ~ --var-value ~ #2 }{}#1
}

\ExplSyntaxOff

%environments

\documentclass{article}
\usepackage[]{amsthm} %lets us use \begin{proof}
\usepackage{amsmath}
\usepackage{mathtools}
\usepackage{enumerate}
\usepackage{xparse}
\usepackage{centernot}
\usepackage[makeroom]{cancel}
\usepackage[]{amssymb} %gives us the chA \mathcal{R} Acter \varnothing
\usepackage{polyglossia}
% \usepackage[frak=mma]{mathalfa}
\setdefaultlanguage{hebrew}
\setotherlanguage{english}
\usepackage{fontspec}
%\setmainfont{Frank Ruehl CLM}
\setmainfont{David CLM}
\setmonofont{Miriam Mono CLM}
\setsansfont{Simple CLM}
\DeclarePairedDelimiter\set\{\}
% Use the following if you only want to change the font for Hebrew
%\newfontfamily\hebrewfont[Script=Hebrew]{David CLM}
%\newfontfamily\hebrewfonttt[Script=Hebrew]{Miriam Mono CLM}
%\newfontfamily\hebrewfontsf[Script=Hebrew]{Simple CLM}
\getenv[\ID]{ID}

\title{אינפי 1 - ממ"ן 16}
\author{אליחי טורקל \ID}
\date\today

%\clearpage %Gives us a page break before the next section. Optional.
%\selectlanguage{english}
	%Section and subsection automatically number unless you put the asterisk next to them.

\begin{document}
	\maketitle %This command prints the title based on information entered above

	\section*{שאלה 1}
	\subsection*{סעיף א}
	\begin{proof}
		\begin{align*}
			\limf{n}{\infty}\left(1 + \frac{1}{n^2}\right)^{\frac{1}{\sin \frac{1}{n}}} =
			\limf{n}{\infty}\left(\left(1 + \frac{1}{n^2}\right)^{n^2}\right)^{\frac{1}{n^2 \sin \frac{1}{n}}} \\
			(a_n) = (1 + \frac{1}{n^2})^{n^2}, (b_n) = \frac{1}{n^2 \sin \frac{1}{n}} \\
			\text{ (אפסה כפול חסומה) }
			\limf{n}{\infty}(b_n) = 0,
			\limf{n}{\infty}{(a_n)} = e
			\text{לפי 6.16 } \\
			\limf{n}{\infty}(a_n)^{(b_n)} = e^0 = 1
			\text{ לפי 6.15 }
		\end{align*}
	\end{proof}

	\subsection*{סעיף ב}
	\begin{proof}
		\begin{align*}
			\brac{1+e^{-x}}^{x^2} = \brac{1 + \frac{1}{e^x}}^{x^2} =
			\brac{\brac{1+\frac{1}{e^x}}^{e^x}}^{\frac{x^2}{e^x}}
		\end{align*}
		תהי $(x_n)$ סדרה כלשהי המקיימת $x_n \xrightarrow[x \to \infty]{} \infty$
		נסתכל על הגבול $\limf{n}{\infty}{\brac{\brac{1+\frac{1}{e^x}}^{e^x}}^{\frac{x^2}{e^x}}}$, \\
		לפי משפט 6.15 נגדיר: $(a_n) = \brac{1 + \frac{1}{e^{x_n}}}^{e^{x_n}}, (b_n) = \frac{x_n^2}{e^{x_n}}$. \\
		נציב $y = e^x$ ולפי משפט הרכבה, הגדרת הנייה, וטענה 6.16 נקבל $\limf{n}{\infty}(a_n) = e$ \\
		ולפי מסקנה 2.49 נקבל:
		\[ \limf{n}{\infty}(b_n) = \limf{n}{\infty}\frac{n^2}{e^n} = 0 \]
		ולפי 6.15 נקבל:
		\[ \limf{n}{\infty} (a_n)^{(b_n)} = \limf{n}{\infty} (a_n)^{\limf{n}{\infty}(b_n)} = e^0 = 1 \]
		זה נכון לכל סדרה $(x_n)$ המקיימת את התנאים של היינה ולכן נקבל:
		\[ \limf{x}{\infty} (1 + e^{-x})^{x^2} = e^0 = 1 \]
	\end{proof}


	\section*{שאלה 2}
	\begin{proof}
		נתחיל בלחשב גבולות של הפונקציה באינסוף ומינוס אינסוף, \\
		$\cos x$ חסומה, ולכן $(1 - \cos x)$ גם חסומה, $e^x$ אפסה,\\
		 ואפסה כפול חסומה היא אפסה, ולכן $e^x(1 - \cos x)$ היא אפסה, ולכן:
		\begin{align*}
			&\limf{x}{\infty} x + e^x(1- \cos x) = "\infty + 0" = \infty  \\
			&\limf{x}{-\infty} x + e^x(1- \cos x) = "-\infty + 0" = -\infty  \\
		\end{align*}
		יהי $y \in \RR$ כלשהו, נראה שקיים $x$ המקיים $y = f(x)$,
		לפי הגדרת הגבול באינסוף קיים $m_2 > y$ כך שלכל $x_1 > m_2$ מתקיים $f(x_2) > y$. \\
		ומהגבול במינוס אינסוף, קיים $m_2 < y$ כך שלכל $x_2 < m_2$ מתקיים $f(x_2) < y$. \\
		ובעצם קיבלנו שקיימים $x_2 < y < x_1$ כך שקיימים $f(x_2) < y < f(x_1)$. \\
		$f$ היא רציפה מכיוון ש $\cos x$ היא רציפה, וגם $e^x$ היא רציפה, ומכפלה וסכום של פונ' רציפות היא רציפה. \\
		ובפרט היא רציפה ב $[x_2, x_1]$ ולכן ממשפט ערך הביניים קיים $c \in [x_2, x_1]$ כך ש $f(c) = y$. \\
		ולסיכום, מצאנו שיש מקור לכל $y \in \RR$ ולכן $f$ היא על.
	\end{proof}

	\pagebreak

	\section*{שאלה 3}
	\subsection*{סעיף א}
	\textbf{תחום הגדרה:} $\sqrt{\pip{x}}$, כל $x$, מכיוון ש $\pip{x} \geq 0$ ו $\sqrt{x}$ מוגדרת לכל $x \geq 0$,
	$\sin x$ מוגדרת לכל $x$, ולכן $f$ מוגדרת לכל $x \in \RR$. \\
	\textbf{רציפות:} $\pip{x}$ רציפה לכל $x$, $\sin x$ רציפה לכל $x$, ו $\sqrt{x}$ רציפה לכל $x \geq 0$,
	ולכן הרכבה של $\sqrt{\pip{x}}$ רציפה לכל $x$,
	ולכן $f(x)$ רציפה לכל $x \in \RR$ כסכום, הרכבה, ומכפלה של פונ' רציפות. \\

	\textbf{גזירות:} נפצל לפי הערך המוחלט
	\selectlanguage{english}
	\begin{enumerate}
		\item $(0, \infty)$: $f(x) = \sqrt{x} \sin x$, $f'(x) = \frac{\sin x}{2\sqrt{x}} + \sqrt{x}\cos x$
		\item $(-\infty, 0)$: $f(x) = \sqrt{-x} \sin x$, $f'(x) = - \frac{\sin x}{2\sqrt{-x}} - \sqrt{-x} \cos x$
		\item $x = 0$: $\limf{x}{0} \frac{\sqrt{\pip{x}} \sin x - 0}{ x -0} = \frac{\sqrt{\pip{x}} \sin x}{x} = \limf{x}{0}\sqrt{\pip{x}}\limf{x}{0}\frac{\sin x}{x} \overset{(a)}= 0 \cdot 1 = 0$
	\end{enumerate}
	\selectlanguage{hebrew}
	(a) - גבול ידוע ורציפות $\sqrt{\pip{0}}$. \\
	ולכן $f(x)$ גזירה בכל $x$ ונגזרתה:
	\[
	f'(x) = \begin{cases}
		\frac{\sin x}{2\sqrt{x}} + \sqrt{x} + \cos x &\text{ if } x > 0 \\
		-\frac{\sin x}{2\sqrt{x}} + \sqrt{-x} + \cos x &\text{ if } x < 0 \\
		0 &\text{ if } x = 0 \\
	\end{cases}
	\]
	\subsection*{סעיף ב}
	\textbf{תחום הגדרה:} $x^2$ מוגדרת לכל $x$, $\sin \frac{1}{x}$ ו $\cos \frac{1}{x}$
	מוגדרות לכל $x \neq 0$, אבל $g(x)$ סוגרת את ה "חור" ולכן תחום ההגדרה של הוא כל $\RR$. \\
	\textbf{גזירות:} נתחיל ב $x \neq 0$:
	\begin{align*}
		g'(x) &= 2x \sin \frac{1}{x} \cos \frac{1}{x} + x^2 \cos \frac{1}{x} \cdot (- \frac{1}{x^2}) \cos \frac{1}{x} - x^2 \sin \frac{1}{x}\sin \frac{1}{x}(-\frac{1}{x^2}) \\
		&= 2x\sin \frac{1}{x} \cos \frac{1}{x} - \cos^2 \frac{1}{x} + \sin^2 \frac{1}{x} \\
		&= 2x \sin \frac{1}{x} \cos \frac{1}{x} - 1
	\end{align*}
	וב $x = 0 $ נשתמש ישירות בהגדרת הנגזרת:
	\[
	\limf{x}{0} \frac{g(x) - g(0)}{x - 0} =
	\limf{x}{0} \frac{x^2 \sin \frac{1}{x} \cos \frac{1}{x}}{x} \overset{(a)}=
	\limf{x}{0} x \limf{x}{0} \sin \frac{1}{x} \cos \frac{1}{x} \overset{(b)}=
	0 = g(0)
	\]
	(a) - מקומיות הגבול \\
	(b) - לפי שאלה בספר, כפל סדרות חסומות הינה סדרה חסומה, וחסומה כפול אפסה היא חסומה (בהקבלה לפונקציות) \\
	ומכאן ש $g'(x)$ \textbf{רציפה} וגזירה גם ב $x = 0$ ולכן בכל $\RR$, ונגזרתה:
	\[
	g'(x) = \begin{cases}
		2x \sin \frac{1}{x} \cos \frac{1}{x} - 1 &\text{ if } x \neq 0 \\
		0 &\text{ if } x = 0
	\end{cases}
	\]

	\section*{שאלה 4}
	\subsection*{סעיף א}
	$\limf{x}{0} x = 0$
	ומרציפות $e^x$: $\limf{x}{0} e^x - 1 = 1 - 1 = 0$ \\
	לפי לופיטל:
	$(e^x-1)' = e^x$, $(x)' = 1$ ולכן:
	\[ \limf{x}{0}\frac{e^x-1}{x} = \limf{x}{0} \frac{e^x}{1} = 1\]

	\subsection*{סעיף ב}
	מרציפות:
	\[ \limf{x}{0^-} \frac{e^x - 1}{2} = \limf{x}{0} \frac{e^x -1 }{2} = \frac{e^0 - 1}{2} = 0 \]
	עכשיו נחשב גבול ב $0^+$, (נשתמש בזה ש $\sqrt{\cos x}$ רציפה בכל $x$)
	\begin{enumerate}
		\item אם $a = 1$ אז, $\limf{x}{0^+} \frac{1 - \sqrt{\cos x}}{x} = \frac{1 - 1}{0} = "\frac{0}{0}"$
		ולכן נשתמש בלופיטל: $(1 - \sqrt{\cos x})' = \frac{1}{2\sqrt{\cos x}}(- \sin x)$, $(x)' = 1$
		ולכן:
		\[
		\limf{x}{0^+} \frac{1 - \sqrt{\cos x}}{x} = \limf{x}{0^+} - \frac{\frac{\sin x}{2\sqrt{\cos x}}}{1} = \limf{x}{0^+} - \frac{\sin x}{2\sqrt{\cos x}} = - \frac{0}{2 \cdot 1} = 0
		\]
		\item אם $a \neq 1$ אז $\limf{x}{0^+} \frac{a - \sqrt{\cos x}}{x} = "\frac{a - 1}{0^+}$
		ולכן אם $a \neq 1$ אזי הגבול אינסופי (פלוס או מינוס תלוי בערך $a$)
	\end{enumerate}
	ולכן קיבלנו שכאשר $a = 1$ אזי הגבולות החד צדדיים שווים, ושווים לערך הפונקציה בנקודה, ולכן $f(x)$ רציפה ב $x = 0 $\\
	אך כאשר $a \neq 1$ אין גבול סופי מימין ולכן $f(x)$ לא רציפה ב $x = 0$. \\
	\textbf{גזירות:}
	מכיוון שכאשר $a \neq 1$ הפונקציה אינה רציפה, אז היא לא יכולה להיות גזירה, ולכן נציב $a = 1$ ונשתמש בהגדרת הנגזרת:
	\begin{align*}
		&\limf{x}{0^+} \frac{\frac{e^x - 1}{2} - 0}{x - 0}
		= \frac{1}{2}\limf{x}{0^+}\frac{e^x-1}{x}
		\overset{(a)}= \frac{1}{2} \cdot 1
		= \boxed{\frac{1}{2}} \\
		&\limf{x}{0^-} \frac{\frac{1 - \sqrt{\cos x}}{x} - 0}{x  - 0}
		= \limf{x}{0^-} \frac{1 - \sqrt{\cos x}}{x^2}
		\overset{(b)}= \limf{x}{0^-} \frac{(1 - \sqrt{\cos x})(1+\sqrt{\cos x})}{x^2(1 + \sqrt{\cos x})} \\
		= &\limf{x}{0^-} \frac{1 - \cos x}{x^2 + x^2\sqrt{\cos x}}
		\overset{(c)}= \limf{x}{0^-} \frac{(1 - \cos x)(1 + \cos x)}{(x^2 + x^2\sqrt{\cos x})(1+\cos x)}
		= \limf{x}{0^-} \frac{1 - \cos^2 x}{x^2(1 + \sqrt{\cos x})(1 + \cos x)}\\
		= &\brac{\limf{x}{0^-} \frac{\sin}{x}}^2 \limf{x}{0^-} \frac{1}{(1 + \sqrt{\cos x})(1 + \cos x)}
		= 1^2 \cdot \frac{1}{4} = \boxed{\frac{1}{4}}
	\end{align*}
	(a) - הוכחנו בסעיף א \\
	(b) - $1 + \sqrt{\cos x} > 0$ \\
	(c) - $1 + \cos x > 0$ בסביבת $N_\delta(0)$ \\
	ולכן כאשר $a = 1$, $f(0)$ גזירה מימין ו $f_+'(0) = \frac{1}{2}$ ומשמאל ו $f_-'(0) = \frac{1}{4}$
	מכיוון שהנגזרות החד צדדיות שונות, מתקבל ש $f$ אינה גזירה ב $x=0$. \\
	ולסיכום, $f$ אינה גזירה ב $x=0$ לכל $a$.

	\section*{שאלה 5}
	$|f(x)| \leq x^2 \Rightarrow -x^2 \leq f(x) \leq x^2$
	כאשר $x = 0$ נקבל $-0 \leq f(0) \leq 0$ ולכן $f(0) = 0$. \\
	מהגדרת הנגזרת $f'(0) = \limf{x}{0}\frac{f(x) - f(0)}{x-0} = \limf{x}{0}\frac{f(x)}{x}$
	מכיוון ש $-x^2 \leq f(x) \leq x^2$, אז נסתכל על סביבה נקובה של $x=0$:
	\begin{enumerate}
		\item בעבור $x > 0$: $-x \leq \frac{f(x)}{x} \leq x$
		\item בעבור $x < 0$: $-x \geq \frac{f(x)}{x} \geq x$
	\end{enumerate}
	ומרציפות אנו יודעים ש $\limf{x}{0} x = \limf{x}{0} -x = 0$
	ולכן בשני המצבים הפונקציה שלנו כלואה בין שתי פונקציות עם אותו הגבול, ולכן ממשפט הסנדוויץ נקבל:
	\[ \limf{x}{0} \frac{f(x)}{x} = 0 \]
	ולכן $f(x)$ גזירה ב $x = 0$ ונגזרתה היא $f'(0) = 0$.



\end{document}
