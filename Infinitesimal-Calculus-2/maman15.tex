% !TEX program = xelatex
\def\NN{\mathbb{N}}
\def\RR{\mathbb{R}}
\def\ZZ{\mathbb{Z}}
\def\QQ{\mathbb{Q}}
\def\PP{\mathcal{P}}
\def\SS{\mathcal{S}}
\def\DD{\mathcal{D}}
\def\sub{\setminus}
\def\bld{\mathbf}
\def\lmf{\lim_{n \to \infty}}
% Make the ref command use parenthesis
\let\oldref\ref
\renewcommand{\ref}[1]{(\oldref{#1})}

\newcommand{\ontop}[1]{\overset{\text{#1}}}
\newcommand{\ang}[1]{\langle #1 \rangle}
\newcommand{\pip}[1]{\left| #1 \right|}



% style
\newcommand{\bm}[1]{\displaystyle{#1}}
\def\nl{$ $ \newline}

\ExplSyntaxOn

\NewDocumentCommand{\getenv}{om}
{
  \sys_get_shell:nnN{ kpsewhich ~ --var-value ~ #2 }{}#1
}

\ExplSyntaxOff

%environments

\documentclass{article}
\usepackage{amsmath,mathtools,enumerate,xparse,centernot,polyglossia,graphicx}
\usepackage[utf8]{inputenc}
\usepackage[a4paper, margin=1.1in]{geometry}
\usepackage[]{amsthm} %lets us use \begin{proof}
\usepackage[makeroom]{cancel}
\usepackage[]{amssymb} %gives us the chA \mathcal{R} Acter \varnothing
% \usepackage[frak=mma]{mathalfa}
\setdefaultlanguage{hebrew}
\setotherlanguage{english}
\usepackage{fontspec}
%\setmainfont{Frank Ruehl CLM}
\setmainfont{David CLM}
\setmonofont{Miriam Mono CLM}
\setsansfont{Simple CLM}
\DeclarePairedDelimiter\set\{\}
% Use the following if you only want to change the font for Hebrew
%\newfontfamily\hebrewfont[Script=Hebrew]{David CLM}
%\newfontfamily\hebrewfonttt[Script=Hebrew]{Miriam Mono CLM}
%\newfontfamily\hebrewfontsf[Script=Hebrew]{Simple CLM}
\getenv[\ID]{ID}
\graphicspath{ {./} }

\newtheorem{theorem}{Theorem}[section]
\newtheorem{corollary}{Corollary}[theorem]
\newtheorem{lemma}[theorem]{טענת עזר}

\title{אינפי 2 - ממ"ן 15}
\author{אליחי טורקל \ID}
\date\today

%\clearpage %Gives us a page break before the next section. Optional.
%\selectlanguage{english}
	%Section and subsection automatically number unless you put the asterisk next to them.

\begin{document}
	\maketitle %This command prints the title based on information entered above


	\section*{שאלה 1}
	סדרת הפונקציות: $(f_n(x))$ מוגדרת בקטע $[0,1]$ בצורה הבאה:
	\begin{align*}
		n \geq 1,
		f_(x) = \begin{cases}
			1 - x \cdot 2^n &\text{ if } 0 \leq x < \frac{1}{2^n} \\
			0 &\text{ if } \frac{1}{2^n} \leq x \leq 1
		\end{cases}
	\end{align*}

	\subsection*{סעיף א}
	הראו כי $(f_n(x))$ מתכנסת נקודתית בקטע $[0,1]$ ומצאו את פונקציית הגבול:
	\begin{proof}
		נפצל ל 2 מקרים:
		\begin{enumerate}
			\item $x = 0$ ואז נקבל $f_n(0) = 1 - 0\cdot 2^n = 1 \xrightarrow[n \to \infty]{} 1$
			\item $x > 0$ יהי $x > 0$ ויהי $\varepsilon > 0$, נסמן $N=\lceil \log_2(1/x) \rceil$ ולכן לכל $n>N$ מתקיים $x \geq \frac{1}{2^n}$
			ומכאן ש $f_n(x) = 0$ ולכן $0=f(x) < \varepsilon$ או במילים אחרות $f(x) \xrightarrow[n \to \infty]{} 0$ בתחום.
		\end{enumerate}
		וביחד קיבלנו ש:
		\begin{align*}
			\limf{n}{\infty} f_n(x) = f(x) = \begin{cases}
				1 &\text{ if } x = 0 \\
				0 &\text{ if } 0 < x \leq 1
			\end{cases}
		\end{align*}
	\end{proof}
	\subsection*{סעיף ב}
	האם $(f_n(x))$ מתכנסת במ"ש ב $[0,1]$
	\begin{proof}
		נבדוק את הגבולות החד צדדיים של הפונקציות $f_n$:
		\begin{align*}
			&\limf{x}{1/2n^-} f_n(x)
			= \limf{x}{1/2n^-} (1 - x \cdot 2^n) = (1 - 1/2n \cdot 2n) = 1 -1 = 0 \\
			&\limf{x}{1/2n^+} f_n(x)
			= 0
		\end{align*}
		וקיבלנו שהגבול משמאל ומימין שווה, וגם $f_n(1/2n) = 0$, ומכאן שכל $f_n(x)$ רציפות,
		אך פונקציית הגבול $f(x)$ אינה רציפה, ומתוך הקונטרה פוזיטיב של משפט 6.4 נקבל שאם פונקציית הגבול אינה רציפה והסדרה מתכנסת אזי או שהפונקציות בסדרה אינן רציפות או שהן לא מתכנסות במ"ש.
		ומכיוון שהוכחנו שהפונקציות רציפות והסדרה מתכנסת, אזי היא \textbf{לא מתכנסת במ"ש}.
	\end{proof}
	\subsection*{סעיף ג}
	האם מתקיים השוויון $\limf{n}{\infty} \int_0^1 f_n(x) \, dx = \int_0^1 (\limf{n}{\infty} f_n(x)) \, dx$
	\begin{proof}
		נחשב את האינטגרלים:
		\begin{alignat*}{2}
			&\int_0^1 f_n(x) \, dx
			&&= 0 + \int_0^{1/2^n} (1 - x \cdot 2^n) \, dx
			= (x - x^2 \cdot 2^{n-1}) \biggr|_0^{1/2^n} \\
			& &&=  1/2^n - \frac{2^{n-1}}{2^{2n}} - (0 - 0^2 \cdot 2^{n-1})
			= \frac{1}{2^n} - \frac{1}{2^{n+1}}
			\xrightarrow[n \to \infty]{} 0 \\
			&\int_0^1 (\limf{n}{\infty} f_n(x)) \, dx
			&&= \int_0^1 f(x) \, dx = 0 - 0 = 0
		\end{alignat*}
		ולכן השוויון אכן מתקיים.

	\end{proof}

	\section*{שאלה 2}
	תהי $(f_n(x))$ סדרת פונקציות רציפות בקטע $[0, 1]$.
	הוכיחו שאם הטור $\sum_{k=1}^\infty f_k(x)$ מתכנס במ"ש בקטע $[0,1)$ אז הטור $\sum_{k=1}^\infty f_k(1)$ מתכנס.
	\begin{proof}
		נשתמש במבחן קושי 5.4,
		יהי $\varepsilon > 0$ מכיוון ש $\sum f_k(x)$ מתכנס במ"ש אזי לפי מבחן מבחן קושי להתכנסות במ"ש קיים $N$ כך שלכל זוג מספרים $m > n > N$
		ובפרט, לכל $n > N$ ולכל $p$ טבעי, נסמן $m = n+p > n$ ולכל $x \in [0, 1)$ ובפרט ל $x$ בסביבה שמאלית של 1,
		מתקיים: $\pip{\sum_{k=n+1}^m f_k(x)} < \varepsilon$
		ומרציפות $f_k(x)$ מתקיים:
		\[
			\limf{x}{1^-} \pip {\sum_{k=n+1}^m f_k(x)} =
			\pip {\sum_{k=n+1}^m \limf{x}{1^-}  f_k(x)} =
			\pip {\sum_{k=n+1}^m f_k(1)} \ontop{(a)}< \varepsilon
		\]
		ובעצם קיבלנו ש: $\pip{f_{n+1}(1) + f_{n+2}(1) + \cdots + f_{n+p}(1)} = \pip {\sum_{k=n+1}^m f_k(1)} < \varepsilon$
		כנדרש במשפט 5.4, ומכאן שהטור $\sum f_k(1)$ מתכנס. \\
		(a) - זה מתקיים לכל $x$ בסביבה שמאלית של 1, ולכן מתקיים לגבול ומרציפות מתקבל שהגבול שווה לנקודה.

	\end{proof}

	\section*{שאלה 3}
	עבור כל אחד מהטורים הבאים בדקו אם הוא מתכנס במידה שווה בקטע הנתון.
	\subsection*{סעיף א}
	$\sum_{n=1}^\infty a_nx^n$
	בתחום התכנסותו כאשר $a_1 = \frac{1}{2}$, $a_{n+1}=\frac{1}{2}\brac{a_n^2 + a_n}$ לכל $n \geq 1$.
	\begin{proof}
		נתחיל בלהוכיח באינדוקציה כי $0 < a_n < 1$: \\
		\textbf{בסיס האינדוקציה:} $0 < a_1 = \frac{1}{2} < 1$. \\
		\textbf{הנחת האינדוקציה:} $0 < a_n < 1$. \\
		\textbf{צעד האינדוקציה:} \begin{align*}
			0 < a_n < 1
			\Rightarrow 0 < a_n + a_n^2 < 2
			\Rightarrow 0 < \frac{1}{2}\brac{a_n + a_n^2} < 1
			\Rightarrow 0 < a_{n+1} < 1
		\end{align*}
		נוכיח כעת באינדוקציה כי הסדרה יורדת: \\
		\textbf{בסיס האינדוקציה:} $a_1 > a_2 = \frac{1}{2} \brac{\frac{1}{2} + \brac{\frac{1}{2}}^2} = \frac{3}{8}$. \\
		\textbf{הנחת האינדוקציה:} $a_n > a_{n+1}$. \\
		\textbf{צעד האינדוקציה:} \begin{align*}
			a_n > a_{n+1}
			\Rightarrow a_n + a_n^2 > a_{n+1} + a_{n+1}^2
			\Rightarrow \frac{1}{2}\brac{a_n + a_n^2} > \frac{1}{2}\brac{a_{n+1} + a_{n+1}^2}
			\Rightarrow a_{n+1} > a_{n+2}
		\end{align*}
		וקיבלנו ש $a_n$ מונוטונית וחסומה, וע"כ מתכנסת. כעת נחשב את גבולה:
		\begin{align*}
			\limf{n}{\infty} a_n = \limf{n}{\infty} a_{n+1} = L
			\Rightarrow L = \frac{1}{2}\brac{L + L^2}
			\Rightarrow L^2 - L = 0
			\Rightarrow L(L - 1) = 0
			\Rightarrow L = 0 \lor L = 1
		\end{align*}
		אך מכיוון שהוכחנו ש $a_n < 1$ והסדרה מונוטונית יורדת, אזי שהגבול לא יכול להיות 1 ולכן בהכרח $L = 0$.
		כעת נשתמש במשפט 6.11 ע"מ להוכיח את רדיוס ההתכנסות:
		\begin{align*}
			\limf{n}{\infty} \pip{\frac{a_n}{a_{n+1}}}
			= \limf{n}{\infty} \pip{\frac{a_n}{\frac{1}{2}\brac{a_n^2 + a_n}}}
			= \limf{n}{\infty} \pip{\frac{2a_n}{a_n\brac{a_n + 1}}}
			= \limf{n}{\infty} \pip{\frac{2}{\brac{a_n + 1}}}
			= \pip{\frac{2}{\brac{0 + 1}}}
			= 2
		\end{align*}
		ומכאן ש $R=2$ כעת נבדוק האם הטור מתכנס בקצוות בעזרת מבחן דאמבלר:
		\begin{align*}
			&\pip{\frac{a_{n+1 \cdot 2^{n+1}}}{a_n \cdot 2^n}}
			= \pip{2\frac{\frac{1}{2}\brac{a_n^2 + a_n}}{a_n}}
			= \pip{\frac{\brac{a_n(a_n + 1)}}{a_n}}
			= \brac{a_n + 1} > 1 \\
			&\pip{\frac{a_{n+1 \cdot (-2)^{n+1}}}{a_n (-2) 2^n}}
			= \pip{-2\frac{\frac{1}{2}\brac{a_n^2 + a_n}}{a_n}}
			= \pip{-\frac{\brac{a_n(a_n + 1)}}{a_n}}
			= \brac{a_n + 1} > 1
		\end{align*}
		ומכאן שהטורים $\sum_{n=1}^\infty a_n \cdot 2^n$ ו $\sum_{n=1}^\infty a_n \cdot (-2)^n$
		ולכן תחום ההתכנסות של הטור הוא $\boxed{(-2, 2)}$
		ולפי הערה ג בעמוד 210 בכרך ג טור חזקות בעל רדיוס סופי וחיובי מתכנס במ"ש אם"ם הוא מתכנס בקצוות, ולכן הטור אינו מתכנס בקצוות.
	\end{proof}

	\subsection*{סעיף ב}
	$\sum_{n=1}^\infty u_n(x)$ בקטע $[0,1]$ כאשר $u_n(x) = (-1)^n$
	\begin{proof}
		נסמן: $a_n = (1-x)x^n$, כאשר $x \in [0,1]$ מתקיים $0 \leq 1-x \leq 1$ וגם $x^n \xrightarrow[n \to \infty]{} 0$ ולכן כאפסה כפול חסומה $a_n \xrightarrow[n \to \infty]{} 0$ ומונוטונית יורדת. \\
		ולכן לפי מבחן לייבניץ הטור מתכנס. ומכיוון שהיא מתכנסת לאפס לכל $x$ בתחום אזי הוא מתכנס לפונקציה רציפה בקטע סגור ולכן ממשפט דיני הוא מתכנס במ"ש בקטע.
	\end{proof}

	\subsection*{סעיף ג}
	$\sum_{n=1}^\infty |u_n(x)|$ בקטע $[0,1]$ כאשר $u_n(x)$ הוא אותה הגדרה מסעיף ב.
	\begin{proof}
		\[ \sum_{n=1}^\infty |u_n(x)|
		= \sum_{n=1}^\infty (1-x)x^n
		= \sum_{n=1}^\infty (-x^{n+1}) + x^n  \]
		ניתן לראות כי בעצם מדובר בסכום טלסקופי ולכן:
		\[
			\sum_{n=1}^\infty |u_n(x)|
			= \sum_{n=1}^\infty (-x^{n+1}) + x^n
			= \limf{n}{\infty} x - x^n
		\]
		ולכן כאשר $x = 1$ נקבל ש $\limf{n}{\infty} -x^n + x = 1-1 = 0$.
		וכאשר $0 \leq x < 1$ נקבל ש $\limf{n}{\infty} x^n \xrightarrow[n \to \infty]{} 0$ ולכן $\limf{n}{\infty} -x^n + x = x - 0 = x$
		וקיבלנו שפונקציית הגבול אינה רציפה ולכן הטור לא מתכנס במ"ש בקטע(משפט 6.4).
	\end{proof}

	\pagebreak
	\section*{שאלה 4}
	פתחו את הפונקציה $f(x) = \frac{x}{1 + x - 2x^2}$ לטור טיילור מהצורה $\sum a_nx^n$ ומצאו את תחום ההתכנסות של הטור.
	\begin{proof}
		\begin{align*}
			f(x) &= \frac{x}{1 + x - 2x^2}
			= \frac{x}{(1-x)(1+2x)}
			= \frac{A(1+2x) + B(1-x)}{(1-x)(1+2x)} \\
			&\Rightarrow A(1+2x) + B(1-x) = x
			\Rightarrow (2A-B)x + (A+B) = x
			\Rightarrow \begin{cases}
				2A - B = 1 \\
				A + B = 0 \Rightarrow A = -B
			\end{cases} \\
			&\Rightarrow 3A = 1
			\Rightarrow A = \frac{1}{3},
			B = - \frac{1}{3} \\
			&\Rightarrow
			\frac{A(1+2x) + B(1-x)}{(1-x)(1+2x)}
			= \frac{1/3(1+2x) - 1/3(1-x)}{(1-x)(1+2x)}
			= \boxed{\frac{1}{3} \brac{\frac{1}{1-x}- \frac{1}{1+2x}}} \\
			&\ontop{(a)}= \frac{1}{3} \brac{\sum_{n=0}^\infty x^n - \sum_{n=0}^\infty (-2x)^n}
			= \boxed{\sum_{n=0}^\infty \frac{1 - (-2)^n}{3}x^n}
		\end{align*}
		כעת נשתמש במשפט 6.10 ע"מ למצוא את תחום ההתכנסות של הטור:
		\[ C =  \overline{\limf{n}{\infty}} \sqrt[n]{\pip{\frac{1 - (-2)^n}{3}}}  \]
		נפצל לזוגי ואי זוגי ונקבל:
		\begin{align*}
			&\overline{\limf{n}{\infty}} \sqrt[2n]{\pip{\frac{1 - (-2)^{2n}}{3}}}
			= \limf{n}{\infty} \sqrt[2n]{\pip{\frac{1 - (-2)^{2n}}{3}}}
			= \limf{n}{\infty} \sqrt[2n]{\pip{\frac{1 - 4^n}{3}}}
			= \limf{n}{\infty} \frac{\sqrt[2n]{4^n-1}}{\sqrt[2n]{3}}
			= 2 \\
			&\overline{\limf{n}{\infty}} \sqrt[2n-1]{\pip{\frac{1 - (-2)^{2n-1}}{3}}}
			= \limf{n}{\infty} \sqrt[2n-1]{\frac{1 + (2)^{2n-1}}{3}}
			= \limf{n}{\infty} \frac{\sqrt[2n-1]{1 + (2)^{2n-1}}}{\sqrt[2n-1]{3}}
			= 2
		\end{align*}
		ולכן $R = \frac{1}{2}$, כעת נבדוק את הקצוות $|x| = 1/2$:
		\begin{align*}
			&\sum_{n=0}^\infty \frac{1 - (-2)^n}{3} \brac{\frac{1}{2}}^n
			= \sum_{n=0}^\infty \frac{1}{3}\brac{\frac{1}{2^n} - \brac{\frac{-2}{2}}^n}
			= \sum_{n=0}^\infty \frac{1}{3}\brac{\frac{1}{2^n} - (-1)^n} \\
			&\sum_{n=0}^\infty \frac{1 - (-2)^n}{3} \brac{-\frac{1}{2}}^n
			= \sum_{n=0}^\infty \frac{1}{3}\brac{\frac{1}{(-2)^n} - \brac{\frac{-2}{-2}}^n}
			= \sum_{n=0}^\infty \frac{1}{3}\brac{\frac{1}{(-2)^n} - 1}
		\end{align*}
		בשני המצבים אנו רואים כי $a_n \not\xrightarrow[n \to \infty]{} 0$ ומיכוון שהתנאי ההכרחי לא מתקיים אזי הטורים מתבדרים. \\
		ובסה"כ תחום ההתכנסות של הטור הינו $\brac{-\frac{1}{2}, \frac{1}{2}}$. \\
		(a) - מתוך הדיון בעמודים 111-112
	\end{proof}

	\pagebreak
	\section*{שאלה 5}
	הפונקציות $f_n(x)$ מוגדרות בקטע $[0, \infty)$ באינדוקציה ע"י:
	$f_{n+1} = \int_0^x f_n(t) + 1 \, dt$, $f_0(x) = 0$.
	\subsection*{סעיף א}
	רשמו את הפונקציות בצורה מפורשת ומצאו את פונקציית הגבול של הסדרה $f_n(x)$.
	\begin{proof}
		נתחיל בלפתח את האיברים הראשונים בסדרה:
		\begin{align*}
			&f_0(x) = 0 \\
			&f_1(x) \int_0^x f_0(t) + 1 = \int_0^x 1 \, dt = x \\
			&f_2(x) = \int_0^x f_1(t) + 1 \, dt = \int_0^x t + 1 \, dt = \frac{x^2}{2} + x \\
			&f_3(x) = \int_0^x f_2(t) + 1 \, dt = \int_0^x \frac{t^2}{2} + t + 1 \, dt = \frac{x^3}{6} + \frac{x^2}{2} + x
		\end{align*}
		ובעצם ניתן לראות כי כל איבר בסדרה, שווה לאיבר הקודם פלוס חזקה של x במיקום שלו חלקי $n!$
		שזה בעצם אומר ש: $f_n = \sum_{i=1}^n \frac{x^i}{i!}$
		כעת נוכיח זאת באינדוקציה, כאשר את הבסיס הראינו כבר למעלה: \\
		\textbf{צעד האינדוקציה:}
		\begin{align*}
			f_{n+1}(x)
			&= \int_0^x f_n(t) + 1 \, dt
			= \int_0^x \sum_{i=1}^n \frac{t^i}{i!} + 1 \, dt
			=  \sum_{i=1}^n \int_0^x \frac{t^i}{i!} \, dt + \int_0^x 1 \, dt \\
			&=  \sum_{i=1}^n \frac{t^{i + 1}}{i! \cdot (i+1)} + t
			=  \sum_{i=1}^n \frac{t^{i + 1}}{(i+1)!} + t
			=  \sum_{i=2}^{n + 1} \frac{t^i}{i!} + t
			=  \sum_{i=1}^{n + 1} \frac{t^i}{i!}
		\end{align*}
		ע"מ למצוא את פונקציית הגבול נשים לב כי: $\sum_{n=0}^\infty \frac{x^n}{n!} = e^x$
		ולכן:
		\begin{align*}
			\limf{n}{\infty} \sum_{i=1}^n \frac{x^i}{i!}
			= \limf{n}{\infty} \sum_{i=0}^n \frac{x^i}{i!} - \frac{x^0}{0!}
			= e^x - 1
		\end{align*}
	\end{proof}

	\subsection*{סעיף ב}
	בדקו כי הסדרה $f_n(x)$ מתכנסת במ"ש בקטע $[0, b]$ לכל $b > 0$ אך לא מתכנסת במ"ש ב $[0, \infty)$
	\begin{proof}
		בפרק 6, שאלה 12 סעיף ג הוכחנו כי $\sum_{n=0}^\infty \frac{x^n}{n!}$ מתכנסת במ"ש בקטע $[-A, A]$ לכל $A \in \RR$ ולכן גם מתכנסת במ"ש ב $[0, A]$ החלקי לו.
		ומההוכחה של סעיף ד באותה שאלה נובע כי הטור לא מתכנס במ"ש בקטע $(0, \infty)$. \\
		ולכן גם לטור שלנו $\sum_{n=0}^\infty \frac{x^n}{n!} - 1$ מתכנס במ"ש ב $[0, b]$ לכל $b > 0$ ולא מתכנס במ"ש ב $[0, \infty)$.
	\end{proof}

\end{document}
