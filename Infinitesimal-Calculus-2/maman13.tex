% !TEX program = xelatex
\def\NN{\mathbb{N}}
\def\RR{\mathbb{R}}
\def\PP{\mathcal{P}}
\def\sub{\setminus}

% Make the ref command use parenthesis
\let\oldref\ref
\renewcommand{\ref}[1]{(\oldref{#1})}



% style
\newcommand{\bm}[1]{\displaystyle{#1}}
\def\nl{$ $ \newline}

\ExplSyntaxOn

\NewDocumentCommand{\getenv}{om}
{
  \sys_get_shell:nnN{ kpsewhich ~ --var-value ~ #2 }{}#1
}

\ExplSyntaxOff

%environments

\documentclass{article}
\usepackage{amsmath,mathtools,enumerate,xparse,centernot,polyglossia,graphicx}
\usepackage[utf8]{inputenc}
\usepackage[a4paper, margin=1.1in]{geometry}
\usepackage[]{amsthm} %lets us use \begin{proof}
\usepackage[makeroom]{cancel}
\usepackage[]{amssymb} %gives us the chA \mathcal{R} Acter \varnothing
% \usepackage[frak=mma]{mathalfa}
\setdefaultlanguage{hebrew}
\setotherlanguage{english}
\usepackage{fontspec}
%\setmainfont{Frank Ruehl CLM}
\setmainfont{David CLM}
\setmonofont{Miriam Mono CLM}
\setsansfont{Simple CLM}
\DeclarePairedDelimiter\set\{\}
% Use the following if you only want to change the font for Hebrew
%\newfontfamily\hebrewfont[Script=Hebrew]{David CLM}
%\newfontfamily\hebrewfonttt[Script=Hebrew]{Miriam Mono CLM}
%\newfontfamily\hebrewfontsf[Script=Hebrew]{Simple CLM}
\getenv[\ID]{ID}
\graphicspath{ {./} }

\newtheorem{theorem}{Theorem}[section]
\newtheorem{corollary}{Corollary}[theorem]
\newtheorem{lemma}[theorem]{טענת עזר}

\title{אינפי 2 - ממ"ן 13}
\author{אליחי טורקל \ID}
\date\today

%\clearpage %Gives us a page break before the next section. Optional.
%\selectlanguage{english}
	%Section and subsection automatically number unless you put the asterisk next to them.

\begin{document}
	\maketitle %This command prints the title based on information entered above


	\section*{שאלה 1}
	הראו כי $1/4$ הוא קירוב של ערך האינטגרל $\int_0^{1/4} \cos x^2 \, dx$ בדיוק של 3 ספרות אחרי הנקודה.

	\begin{proof}
		נגדיר $f(x) = \int_0^x \cos t^2 \, dt$ ונעריך את $f(1/4)$:
		\begin{align*}
			f'(x) \overset{(1.33)}= \cos x^2 \\
			f''(x) = -2x \sin x^2 \\
			f^{(3)}(x) = -2\sin x^2 -4x^2 \cos x^2 \\
			f^{(4)}(x) = -12 x \cos x^2 + 8x^3 \sin x^2
		\end{align*}
		נשתמש בפיתוח מקלורן כאשר $n=3$
		\begin{align*}
			f(1/4) = \sum^3_{k=0} \frac{f^{(k)(1/4)}}{4!} \cdot \brac{\frac{1}{4}}^4 + R_3(1/4)
		\end{align*}
		נתחיל בלבדוק את שגיאת הקירוב של הפולינום בעזרת שארית לגראנז':
		\begin{align*}
			\pip{R_3(1/4)} = \pip{\frac{f^{(4)}(c)}{4!} \cdot \brac{\frac{1}{4}}^4}
			= \pip{\frac{8c^3 \sin c^2 - 12c \cos c^2}{24 \cdot 2^8}}
			= \pip{\frac{8c^3 \sin c^2 - 12c \cos c^2}{3 \cdot 2^{11}}}
		\end{align*}
		ידוע כי $\pip{\sin x} \leq \pip{x}$ וגם נתון ש $0 < c < 1/4$ ומכיוון שסינוס עולה בקטע $[0, \pi/2]$
		אזי מתקבל ש $0 < \sin c^2 \leq c^2$.
		ובאותו התחום גם $0 \leq \cos c \leq 1$ ומכיוון שמדובר בערך מוחלט עם חיסור שני איברים אזי מטענת עזר~\ref{lemma:1}:
		\begin{align*}
			\pip{\frac{8c^3 \sin c^2 - 12 c \cos c^2}{3 \cdot 2^{11}}} \leq
			max \left\{
				 \frac{0 - 12c}{3 \cdot 2^{11}},
				 \frac{8c^3 \cdot c^2 - 0}{3 \cdot 2^{11}}
				 \right\}
		\end{align*}
		האיבר השמאלי קטן מאפס ולכן קטן גם מ $0.5 \cdot 10^{-3}$ נראה שגם הימני קטן מ $0.5 \cdot 10^{-3}$:
		\begin{align*}
			\pip{\frac{8c^3 \cdot c^2 - 0}{3 \cdot 2^{11}}}
			= \frac{8c^3 \cdot c^2}{3 \cdot 2^{11}}
			\leq \frac{8 \frac{1}{4^5}}{3 \cdot 2^{11}}
			= \frac{1}{3 \cdot 2^{18}}
			= \frac{1}{2048 \cdot 2^7 \cdot 3}
			< 0.5 \cdot 10^{-3}
		\end{align*}

		ולכן בשני המקרים קיבלנו ש $\pip{R_3{1/4}} < 0.5 \cdot 10^{-3}$.

		כעת נחשב את ערך הפולינום: (האיבר הראשון הוא האינטגרל מאפס עד אפס, ומהגדרה 1.22 הוא אפס)
		\begin{align*}
			P(1/4)
			= \frac{f^{(0)}(0)}{1} + \frac{\cos 0^2}{1}\cdot\frac{1}{4} - \frac{2\sin 0}{2!} \cdot\frac{1}{4}^2 + \frac{-2 \sin 0^2 -4 \cdot 0^2 \cos 0^2}{3!} \cdot\frac{1}{4}^3
			= 0 + 1 \cdot \frac{1}{4} - 0 + 0 = \frac{1}{4}
		\end{align*}
		כנדרש.
	\end{proof}

	\begin{lemma}\label{lemma:1}
		נוכיח טענה חזקה יותר.
		\begin{proof}
			יהי $x \in [a_1, b_1]$, $y \in [a_2, b_2]$
			מטענה 1.48(2) באינפי 1 נובע כי:
			\begin{align*}
			   \pip{x-y} = max \set{x-y, y-x} \\
			   x - y \leq x - a_2 \leq b_1 - a_2 \\
			   y - x \leq y - a_1 \leq b_2 - a_1 \\
			   \pip{x-y} \leq max \set {b_1 - a_2, b_2 - a_1}
			\end{align*}
		\end{proof}
	\end{lemma}

	\pagebreak
	\section*{שאלה 2}
	הוכיחו בעזרת פיתוח מתאים כי: $1 + \frac{x}{2} - \frac{x^2}{8} < \sqrt{x+1} < 1 + \frac{x}{2} - \frac{x^2}{8} + \frac{x^3}{16}$ לכל $x > 0$
	\begin{proof}
		נגדיר $f(x) = \sqrt{x + 1}$
		ונחשב את פיתוח מקלורן כאשר $n=2$ ונסתכל על שארית לגראנז' (כמו שעשינו באינפי 1 עם משפט הערך הממוצע):
		\begin{align*}
			f'(x) = \frac{1}{2\sqrt{x+1}} \\
			f''(x) = -\frac{1}{4\sqrt{(x+1)^3}}
			f'''(x) = \frac{3}{8\sqrt{(x+1)^5}}
		\end{align*}
		ולפי פיתוח מקלורן ושארית לגראנז' נקבל ש:
		\begin{align*}
			\sqrt{x+1} = \sum^2_{k=0} \frac{f^{(k)(0)}}{k!}x^k + \frac{f^{3}(c)}{3!}x^3
			= 1 + \frac{1}{2}x + \frac{1}{8}x^2 + \frac{1}{16\sqrt{(1+c)^5}}x^3
		\end{align*}
		ומכיוון ש $0 < c < x$ נקבל ש:
		\begin{align*}
			1 + \frac{1}{2}x - \frac{1}{8}x^2 < 1 + \frac{1}{2}x + \frac{1}{8}x^2 + \frac{1}{16\sqrt{(1+c)^5}}x^3 < 1 + \frac{1}{2}x + \frac{1}{8}x^2 + \frac{1}{16}x^3
		\end{align*}
		שזה בעצם שקול ל:
		\begin{align*}
			1 + \frac{x}{2} - \frac{x^2}{8} < \sqrt{x + 1} < 1 + \frac{x}{2} + \frac{x^2}{8} + \frac{x^3}{16}
		\end{align*}
		כנדרש.
	\end{proof}

	\subsection*{שאלה 3}
	פונקציה $f(x)$ גזירה בקטע $(-\delta, \delta)$ ומקיימת $f'(x) = 2 \sin (f(x))$ לכל $- \delta < x < \delta$.
	הראו כי $f$ גזירה 3 פעמים ב 0.
	מצאו את פולינום מקלורן ממעלה 3 של $f(x)$ אם ידוע בנוסף ש $f(0) = \pi/2$.

	\begin{proof}
		נשתמש בכלל השרשרת ונסמן $\varphi(x) = (2\sin x) \circ f(x)$, נתון לנו כי $f(x)$ גזירה בקטע $(-\delta, \delta)$ ו $2 \sin x$ גזירה לכל $x \in \RR$ ולכן $\varphi(x)$ גזירה ונגזרת היא הנגזרת השניה של $f$:
		\[ f''(x) = 2 \cos(f(x)) \cdot f'(x) \]
		לפי כלל נגזרת המכפלה וכלל השרשרת גם פונקציה זו גזירה בקטע $(-\delta, \delta)$
		מכיוון ש $f(x)$ גזירה בקטע והראינו כבר ש $f'(x)$ גזירה בקטע ובנוסף הפונקציות הטריגונומטריות גזירות בקטע, ונגזרתה היא:
		\[
			f^{(3)} = -2 \sin(f(x)) \cdot f'(x) \cdot f'(x) + 2 \cos(f(x)) \cdot f''(x)
			= \boxed{2(\cos(f(x)) f''(x) - \sin(f(x)) f'(x)^2)}
		 \]
		 כעת נמצא את פולינום מקלון ממעלה 3:
		 \begin{align*}
			f(x)
			= \sum^3_{k=0} \frac{f^{(k)(0)}}{k!}x^k + R_3
			= f(0) + f'(0)x + \frac{f''(0)x^2}{2} + \frac{f^{(3)}(0)x^3}{6} + R_3
		 \end{align*}
		 נתון לנו כי $f(0) = \pi/2$ נציב $f'(0) = 2\sin(\pi/2) = 2$ נציב גם $f''(0) = 2\cos(\pi/2) \cdot 2 = 0$
		 וגם \\
		 $f^{(3)} = 2(\cos(\pi/2) \cdot 0 - \sin(\pi/2) 2^2) = 2(-4) = -8$
		 ולכן נציב בפולינום ונקבל:
		 \[
		 	P_3(x) = \pi/2 + 2x + \frac{0 \cdot x^2}{2} + \frac{-8x^3}{6}
			= \boxed{\frac{\pi}{2} + 2x - \frac{4x^3}{3}}
		 \]
	\end{proof}

	\section*{שאלה 4}
	תהי $f(x)$ גזירה פעמיים ב $\RR$, $|f(x)| \leq 1$, $|f''(x)| \leq 1$ לכל $x$.
	רשמו פיתוחי טיילור עבור $f(x+\sqrt{x})$ ו $f(x - \sqrt{2})$ בנקודה $x$ שרירותית
	והוכיחו בעזרת פיתוחים אלה כי $|f'(x)| \leq \sqrt{2}$ לכל $x$.

	\begin{proof}
		נשתמש בפיתוח טיילור ממעלה 1 בנקודה כלשהי $a$ על הפונקציה $f(x + \sqrt{2})$ עם שארית בצורת לגראנז'
		\begin{align*}
			f(x + \sqrt{2})
			&= \sum^1_{k=0} \frac{f^{(k)}(a)}{k!}(x + \sqrt{2} - a)^k + \frac{f''(c_1)}{2!}(x + \sqrt{2} - a)^2 \\
			&= f(a) + f'(a)(x + \sqrt{2} - a) + \frac{f''(c_1)}{2}(x + \sqrt{2} - a)^2
		\end{align*}
		ובצורה דומה:
		\begin{align*}
			f(x - \sqrt{2})
			&= \sum^1_{k=0} \frac{f^{(k)}(a)}{k!}(x - \sqrt{2} - a)^k + \frac{f''(c_2)}{2!}(x - \sqrt{2} - a)^2 \\
			&= f(a) + f'(a)(x - \sqrt{2} - a) + \frac{f''(c_2)}{2}(x - \sqrt{2} - a)^2
		\end{align*}
		נחסר את שתי המשוואות ונקבל:
		\begin{align*}
			&f(x + \sqrt{2}) - f(x - \sqrt{2}) \\
			&= f(a) + f'(a)(x + \sqrt{2} - a) + \frac{f''(c_1)}{2}(x + \sqrt{2} - a)^2
			- f(a) - f'(a)(x - \sqrt{2} - a) - \frac{f''(c_2)}{2}(x - \sqrt{2} - a)^2 \\
			&= f'(a)(x + \sqrt{2} - a - x + \sqrt{2} + a) + \frac{f''(c_1)}{2}(x + \sqrt{2} - a)^2 - \frac{f''(c)}{2}(x - \sqrt{2} - a)^2 \\
			&= f'(a)(2\sqrt{2}) + \frac{f''(c_1)}{2}(x + \sqrt{2} - a)^2 - \frac{f''(c_2)}{2}(x - \sqrt{2} - a)^2
		\end{align*}
		כעת נציב $a=x$ ונקבל:
		\begin{align*}
			f(x + \sqrt{2}) - f(x - \sqrt{2})
			&= 2\sqrt{2}f'(x) + \frac{f''(c_1)}{2}(x + \sqrt{2} - x)^2 - \frac{f''(c_2)}{2}(x - \sqrt{2} - x)^2 \\
			&= 2\sqrt{2}f'(x) + \frac{f''(c_1)}{2}(\sqrt{2})^2 - \frac{f''(c_2)}{2}(\sqrt{2})^2 \\
			&= 2\sqrt{2}f'(x) + f''(c_1) - f''(c_2)
		\end{align*}
		מתוך $|f(x)| \leq 1 \Rightarrow -1 \leq f(x) \leq 1$ נקבל ש:
		\begin{align*}
			-2 \leq
			f(x + \sqrt{2}) - f(x - \sqrt{2})
			= &2\sqrt{2}f'(x) + f''(c_1) - f''(c_2)
			\leq 2 \\
			f''(c_2) - f''(c_1) - 2
			\leq &2\sqrt{2}f'(x)
			\leq 2 + f''(c_2) - f''(c_1)
		\end{align*}
		ומתוך $|f''(x)| \leq 1$ נקבל ש $|f''(c_2) - f''(c_1)| \leq 2$ ולכן:
		\begin{align*}
			|f'(x)| \leq \sqrt{2}
			\Leftarrow
			-\sqrt{2} \leq f'(x) \leq \sqrt{2}
			\Leftarrow
			-2 \leq \sqrt{2}f'(x) \leq 2
			\Leftarrow
			-2 - 2
			\leq 2\sqrt{2}f'(x)
			\leq 2 + 2
		\end{align*}
		כנדרש.
	\end{proof}

	\pagebreak
	\section*{שאלה 5}
	היעזרו בפיתוח מקלורן מסדר מתאים ע"מ לחשב את הגבולות הבאים:
	\subsection*{סעיף א}
	\[	\limf{x}{0} \frac{(e^x-1)^2 - x \ln (1+x)}{\tan x - x}	\]
	\begin{proof}
		נשתמש בפיתוחי מקלורן:
		\begin{align*}
			e^x = 1 + x + \frac{x^2}{2} + R_2(x) \\
			\ln(x+1) = x - \frac{x^2}{2} + T_2(x) \\
			\tan x = x + \frac{x^3}{3} + S_3(x)
		\end{align*}
		נציב בגבול ונקבל:
		\begin{align*}
			&\limf{x}{0} \frac{(1 + x + \frac{x^2}{2} + R_2(x)-1)^2 - x (x - \frac{x^2}{2} + T_2(x))}{x + \frac{x^3}{3} + S_3(x) - x} \\
			&=\limf{x}{0} \frac{(x + \frac{x^2}{2} + R_2(x))^2 - x^2 + \frac{x^3}{2} - xT_2(x) }{\frac{x^3}{3} + S_3(x)} \\
			&=\limf{x}{0} \frac{x^2 + x^3 + 2xR_2(x) + \frac{x^4}{4} + x^2R_2(x) + R_2(x)^2 - x^2 + \frac{x^3}{2} - xT_2(x) }{\frac{x^3}{3} + S_3(x)} \\
			&=\limf{x}{0} \frac{\frac{x^4}{4} + \frac{3x^3}{2} + 2xR_2(x)  + x^2R_2(x) + R_2(x)^2 - xT_2(x) }{\frac{x^3}{3} + S_3(x)} \\
			&=\limf{x}{0} \frac{\frac{x}{4} + \frac{3}{2} + 2\frac{R_2(x)}{x^2}  + x\frac{R_2(x)}{x^2} + R_2(x)\frac{R_2(x)}{x^3} - \frac{T_2(x)}{x^2} }{\frac{1}{3} + \frac{S_3(x)}{x^3}} \\
		\end{align*}
		ממשפט 4.7 נקבל ש:
		\[
			\limf{x}{0} \frac{R_2(x)}{x^2} = \limf{x}{0} \frac{T_2(x)}{x^2} = \limf{x}{0} \frac{S_3(x)}{x^3} = 0
		\]
		ולכן מאריתמטיקה של גבולות ורציפות חלק מהאיברים:
		\begin{align*}
			&=\limf{x}{0} \frac{\frac{x}{4} + \frac{4}{3} + 2\frac{R_2(x)}{x^2}  + x\frac{R_2(x)}{x^2} + R_2(x)\frac{R_2(x)}{x^3} - \frac{T_2(x)}{x^2} }{\frac{1}{3} + \frac{S_3(x)}{x^3}} \\
			&=\frac{0 + \frac{4}{3} + 2 \cdot 0  + 0 \cdot 0 + R_2(x) \cdot 0 - 0 }{\frac{1}{3} + 0}
			= \boxed{\frac{9}{2}}
		\end{align*}
	\end{proof}

	\pagebreak
	\subsection*{סעיף ב}
	\[
		\limf{n}{\infty} \frac{n^2}{\cos \frac{1}{n}} - \frac{n}{\sin \frac{1}{n}}
		\]
	\begin{proof}
		נסמן $X_n = \frac{1}{n}$ ומתקיים: $(X_n) \xrightarrow[n \to \infty]{}0$ ולכן מהגדרת היינה נקבל:
		\begin{align*}
			\limf{n}{\infty} \frac{n^2}{\cos \frac{1}{n}} - \frac{n}{\sin \frac{1}{n}}
			&= \limf{x}{0} \frac{1}{x^2\cos x} - \frac{1}{x\sin x}
			= \limf{x}{0} \frac{x\sin x - x^2\cos x}{x^2\cos x \cdot x\sin x} \\
			&= \limf{x}{0} \frac{\sin x - x^2\cos x}{x^2\cos x \cdot \sin x}
			= 2\limf{x}{0} \frac{\sin x - x\cos x}{x^2 \sin 2x}
		\end{align*}
		כעת נשתמש בפיתוחי מקלורן ע"מ לחשב את הגבול, ע"מ לחשב את $\sin 2x$ נשתמש בשאלה 93 ונציב $q(x) = 2x$
		\begin{align*}
			\sin x = x - \frac{x^3}{6} + R_3(x) \\
			\sin 2x = 2x + S_1(x) \\
			\cos x = 1 - \frac{x^2}{2} + T_2(x)
		\end{align*}
		כעת נציב חזרה בגבול ונקבל:
		\begin{align*}
			2\limf{x}{0} \frac{\sin x - x\cos x}{x^2 \sin 2x}
			&= 2\limf{x}{0} \frac{x - \frac{x^3}{6} + R_3(x) - x + \frac{x^3}{2} - xT_2(x)}{2x^3 + x^2S_1(x)} \\
			&= 2\limf{x}{0} \frac{\frac{1}{x^2} - \frac{1}{6} + \frac{R_3(x)}{x^3} - \frac{1}{x^2} + \frac{1}{2} - \frac{T_2(x)}{x^2}}{2 + \frac{S_1(x)}{x}}
			= 2\limf{x}{0} \frac{\frac{1}{3} + \frac{R_3(x)}{x^3} - \frac{T_2(x)}{x^2}}{2 + \frac{S_1(x)}{x}}
		\end{align*}
		ממשפט 4.7 נקבל ש:
		\[
			\limf{x}{0} \frac{R_3(x)}{x^3} = \limf{x}{0} \frac{T_2(x)}{x^2} = \limf{x}{0} \frac{S_1(x)}{x} = 0
		\]
		ומאריתמטיקה של גבולות:
		\begin{align*}
			\limf{n}{\infty} \frac{n^2}{\cos \frac{1}{n}} - \frac{n}{\sin \frac{1}{n}}
			= 2\limf{x}{0} \frac{\frac{1}{3} + \frac{R_3(x)}{x^3} - \frac{T_2(x)}{x^2}}{2 + \frac{S_1(x)}{x}}
			= 2 \frac{\frac{1}{3} + 0 - 0}{2 + 0}
			= \boxed{\frac{1}{3}}
		\end{align*}
	\end{proof}
\end{document}
