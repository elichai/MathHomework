% !TEX program = xelatex
\def\NN{\mathbb{N}}
\def\RR{\mathbb{R}}
\def\ZZ{\mathbb{Z}}
\def\QQ{\mathbb{Q}}
\def\PP{\mathcal{P}}
\def\SS{\mathcal{S}}
\def\DD{\mathcal{D}}
\def\sub{\setminus}
\def\bld{\mathbf}
\def\lmf{\lim_{n \to \infty}}
% Make the ref command use parenthesis
\let\oldref\ref
\renewcommand{\ref}[1]{(\oldref{#1})}

\newcommand{\ontop}[1]{\overset{\text{#1}}}
\newcommand{\ang}[1]{\langle #1 \rangle}
\newcommand{\pip}[1]{\left| #1 \right|}



% style
\newcommand{\bm}[1]{\displaystyle{#1}}
\def\nl{$ $ \newline}

\ExplSyntaxOn

\NewDocumentCommand{\getenv}{om}
{
  \sys_get_shell:nnN{ kpsewhich ~ --var-value ~ #2 }{}#1
}

\ExplSyntaxOff

%environments

\documentclass{article}
\usepackage{amsmath,mathtools,enumerate,xparse,centernot,polyglossia,graphicx}
\usepackage[utf8]{inputenc}
\usepackage[a4paper, margin=1.1in]{geometry}
\usepackage[]{amsthm} %lets us use \begin{proof}
\usepackage[makeroom]{cancel}
\usepackage[]{amssymb} %gives us the chA \mathcal{R} Acter \varnothing
% \usepackage[frak=mma]{mathalfa}
\setdefaultlanguage{hebrew}
\setotherlanguage{english}
\usepackage{fontspec}
%\setmainfont{Frank Ruehl CLM}
\setmainfont{David CLM}
\setmonofont{Miriam Mono CLM}
\setsansfont{Simple CLM}
\DeclarePairedDelimiter\set\{\}
% Use the following if you only want to change the font for Hebrew
%\newfontfamily\hebrewfont[Script=Hebrew]{David CLM}
%\newfontfamily\hebrewfonttt[Script=Hebrew]{Miriam Mono CLM}
%\newfontfamily\hebrewfontsf[Script=Hebrew]{Simple CLM}
\getenv[\ID]{ID}
\graphicspath{ {./} }

\title{אינפי 2 - ממ"ן 11}
\author{אליחי טורקל \ID}
\date\today

%\clearpage %Gives us a page break before the next section. Optional.
%\selectlanguage{english}
	%Section and subsection automatically number unless you put the asterisk next to them.

\begin{document}
	\maketitle %This command prints the title based on information entered above


	\section*{שאלה 1}
	חשבו את האינטגרלים הבאים

	\subsection*{סעיף א}
	\[
		\int \frac{dx}{(2-x)\sqrt{1-x}}
	\]

	\begin{proof}
		\begin{align*}
			\int \frac{dx}{(2-x)\sqrt{1-x}} &=
			\begin{bmatrix}
				1 - x = u^2 \Rightarrow u = \sqrt{1-x} \\
				dx = -2u \, du
			\end{bmatrix}
			= \int \frac{-2u}{(1+u^2)u} \, du \\
			&= -2 \int \frac{du}{1+u^2} = -2\arctan u + C
			= -2\arctan(\sqrt{1-x}) + C
		\end{align*}
	\end{proof}

	\subsection*{סעיף ב}
	\[
		\int \frac{x+4}{(x^2 + x + 1)^2} \, dx
	\]

	\begin{proof}
		נשתמש בשיטה המפורטת בעמוד 142:
		\begin{align*}
			\int \frac{x+4}{(x^2 + x + 1)^2} \, dx
			= \frac{1}{2} \int \frac{2x + 1}{(x^2 + x + 1)^2} \, dx
			+ \frac{7}{2} \int \frac{dx}{(x^2 + x + 1)^2}
		\end{align*}
		נתחיל בלחשב את האינטגרל הראשון:
		\begin{align*}
			\int \frac{2x + 1}{(x^2 + x + 1)^2} \, dx
			= \begin{bmatrix}
				u = x^2 + x + 1 \\
				du = 2x + 1 \, dx
			\end{bmatrix}
			= \int \frac{du}{u^2} = - \frac{1}{2u} + C
		\end{align*}
		נציב חזרה ונקבל ש:
		\begin{align*}
			\frac{1}{2} \int \frac{2x + 1}{(x^2 + x + 1)^2} \, dx
			= - \frac{1}{2x^2 + 2x + 2} + C
		\end{align*}
		כעת נחשב את האינטגרל השני, ונסמן: $d^2 = 1 - \frac{1}{4} = \frac{3}{4}$
		\begin{align*}
			\int \frac{dx}{(x^2 + x + 1)^2}
			= \int \frac{dx}{((x+\frac{1}{2})+ \frac{3}{4})^2}
			= \begin{bmatrix}
				x + \frac{1}{2} = t \\
				dx = dt
			\end{bmatrix}
			= \int \frac{dt}{(t^2 + \frac{3}{4})^2}
			\overset{I_2}= \frac{4}{6} \cdot \frac{t}{t^2 + \frac{3}{4}}
			 + \frac{4}{6} \cdot \frac{2}{\sqrt{3}} \cdot \arctan \frac{2t}{\sqrt{3}} + C
		\end{align*}
		נציב חזרה ונקבל ש:
		\begin{align*}
			\frac{7}{2} \int \frac{dx}{(x^2 + x + 1)^2}
			= \frac{7 \cdot 4}{2 \cdot 6} \frac{x + \frac{1}{2}}{(x + \frac{1}{2}) + \frac{3}{4}}
			+ \frac{14}{3\sqrt{3}} \arctan{\frac{2x + 1}{\sqrt{3}}} + C
		\end{align*}
		כעת נסכם את שניהם יחד ונקבל:
		\begin{align*}
			\int \frac{x+4}{(x^2 + x + 1)^2} \, dx
			= - \frac{1}{2x^2 + 2x + 2}
			+ \frac{7x + \frac{7}{2}}{3x^2 + 3x + 3}
			+ \frac{14}{3\sqrt{3}} \arctan{\frac{2x + 1}{\sqrt{3}}} + C
		\end{align*}
	\end{proof}

	\subsection*{סעיף ג}
	\[
		\int x(\arctan^2 x) \, dx
	\]
	\begin{proof}
	נשתמש באינטגרציה בחלקים:
		\begin{align*}
			\int x(\arctan^2 x) \, dx
			= \begin{bmatrix}
				u = \arctan^2 x & v' = x \\
				u' = \frac{2\arctan x}{x^2 + 1} & v = \frac{x^2}{2}
			\end{bmatrix}
			= \frac{x^2 \arctan^2 x}{2}
			- \int \frac{\cancel{2}x^2 \arctan x}{\cancel{2}(x^2 + 1)} \, dx
		\end{align*}
		כעת נחשב את האינטגרל שנשאר:
		\begin{align*}
			\int \frac{x^2}{x^2 + 1} \cdot \arctan x \, dx
			&= \begin{bmatrix}
				u = \arctan x & v' = \frac{x^2}{x + 1} \\
				u' = \frac{1}{1 + x^2} & v = x - \arctan x
			\end{bmatrix} \\
			&= \arctan x (x - \arctan x)
			- \int \frac{x - \arctan x}{1+x^2} \, dx \\
			&= \arctan x (x - \arctan x)
			- \int \frac{x}{1+x^2} \, dx
			+ \int \frac{\arctan x}{1+x^2} \, dx
		\end{align*}

		נחשב את האינטגרל הראשון:
		\begin{align*}
			\int \frac{x}{1+x^2} \, dx
			= \begin{bmatrix}
				t = 1 + x^2 \Rightarrow x = \sqrt{t - 1} \\
				dt = 2x \, dx
			\end{bmatrix}
			= \frac{1}{2} \int \frac{dt}{t}
			= \frac{1}{2} \ln |t| + C
			= \frac{1}{2} \ln (x^2 + 1) + C
		\end{align*}
		ואת האינטגרל השני:
		\begin{align*}
			\int \frac{\arctan x}{1+x^2} \, dx
			= \begin{bmatrix}
				t = \arctan x \\
				dt = \frac{1}{1+x^2} \, dx
			\end{bmatrix}
			= \int t \, dt = \frac{t^2}{2} + C
			= \frac{1}{2} \arctan^2 x + C
		\end{align*}

		כעת נחבר הכל חזרה ונקבל:
		\begin{align*}
			\int x(\arctan^2 x) \, dx
			&= \frac{x^2 \arctan^2 x}{2} -\brac{\arctan x (x - \arctan x) - \frac{1}{2} \ln (x^2 + 1) + \frac{1}{2}\arctan^2 x} \\
			&= \frac{1}{2}\brac{x^2 \arctan^2 x - 2x \arctan x + \arctan^2 x + \ln(x^2+1)}
		\end{align*}
	\end{proof}

	\pagebreak
	\subsection*{סעיף ד}
	\[
		\int_0^{\frac{\pi}{4}}	\tan^4 x \, dx
	\]
	\begin{proof}
		\begin{align*}
			\int_0^{\frac{\pi}{4}}	\tan^4 x \, dx
			&= [t = \tan x]
			= \int_0^1 \frac{t^4}{1+t^2} \, dt
			= \int_0^1 t^2 \cdot \frac{t^2}{1+t^2} \, dt \\
			&= \begin{bmatrix}
				u = t^2 & v' = \frac{t^2}{1+t^2} \\
				u' = 2t & v = t - \arctan t
			\end{bmatrix}
			= t^3 - t^2 \arctan t \biggr|_0^1
			- \int_0^1 2t(t - \arctan t) \, dt \\
			&= t^3 - t^2 \arctan t \biggr|_0^1
			- \int_0^1 2t^2 \, dt + \int_0^1 2t\arctan t \, dt
		\end{align*}
		האינטגרל השמאלי הוא טריוויאלי: $\int_0^1 2t^2 \, dt = \frac{2t^3}{3} \biggr|_0^1$
		נחשב את הימני בנפרד:
		\begin{align*}
			\int_0^1 2t\arctan t \, dt
			= \begin{bmatrix}
				u = \arctan t & v' = 2t \\
				u' = \frac{1}{1+t^2} & v = t^2
			\end{bmatrix}
			= t^2 \arctan t \biggr|_0^1
			- \int_0^1 \frac{t^2}{1+t^2}
			= t^2 \arctan t - t + \arctan t \biggr|_0^1
		\end{align*}
	\end{proof}
	נציב חזרה את התוצאות ונקבל:
	\begin{align*}
		t^3 \cancel{- t^2 \arctan t} - \frac{2t^3}{3} + \cancel{t^2 \arctan t} - t + \arctan t \biggr|_0^1
		&= \frac{t^3}{3} - t + \arctan t \biggr|_0^1 \\
		&= 1 - 1 + \frac{\pi}{4} - 0 + 0 - 0
		= \boxed{\frac{\pi}{4}}
	\end{align*}


	\pagebreak
	\section*{שאלה 2}
	לכל אחד מהאינטגרלים קבעו אם הוא מתכנס בהחלט, מתכנס בתנאי או מתבדר
	\subsection*{סעיף א}
	\[
	\int_0^\infty x \sin x^3 \, dx
	\]
	\begin{proof}
		באינטגרל ישנה נקודה בעייתית אחת והיא האינסוף, אזי נפצל ל 0 עד 1 ו 1 עד אינסוף:
		\begin{align*}
			\int_0^\infty x \sin x^3 \, dx = \int_0^1 x \sin x^3 \, dx + \int_1^\infty x \sin x^3 \, dx
		\end{align*}
		האינטגרל $\int_0^1 x \sin x^3 \, dx$ הוא אינטגרל מסויים של פונקציה רציפה ב $\RR$ ובפרט ב $[0,1]$ ולכן הוא מתכנס. \\
		בעבור האינטגרל השני נסמן:
		\begin{align*}
		\begin{bmatrix}
			t = x^3 \\
			x = \sqrt[3]{t} \\
			dx = \frac{1}{3 \sqrt[3]{t^2}} \, dt \\
			x = \infty \Rightarrow t = \infty \\
			x = 1 \Rightarrow t = 1
		\end{bmatrix}
	\end{align*}
	ונקבל $\int_1^\infty x \sin x^3 \, dx =
	\frac{1}{3} \int_1^\infty \frac{\sin t}{\sqrt[3]{t}} \, dt +
	\int_1^\infty \frac{\sin t}{t^{\frac{1}{3}}}\, dx$
		ולכן משאלה 32 בעמוד 37 בכרך ב נקבל שהאינטגרל מתכנס. \\
		וביחד קיבלנו שהאינטגרל $\int_0^\infty x \sin x^3 \, dx$ כולו מתכנס. \\
		כעת נבדוק האם הוא מתכנס בהחלט: \\
		נניח בשלילה שהוא מתכנס בהחלט, ז"א ש $\int_0^\infty x \cos 2x^3 \, dx$ מתכנס. \\
		נשתמש בזהות $\sin^2 x = \frac{1 - \cos 2x}{2}$ ונקבל: $x \sin^2 x^3 = \frac{x(1 - \cos 2x^3)}{2}= \frac{x}{2} - \frac{x \cos 2x^3}{2}$
		האינטגרל $\int_0^\infty x \cos 2x^3 \, dx$ מתכנס בדיוק בצורה דומה למה שעשינו פה למעלה. \\
		מכיוון ש $-1 \leq \sin x \leq 1$ אזי $\sin^2 x \leq 1$ ומכיוון שהכפלה במספר הקטן מאחד מקטינה את המספר אזי נקבל ש $\pip{x \sin x} \geq x \sin^2 x$. \\
		ולפי מבחן ההשוואה נקבל ש $\int_0^\infty x \sin^2 x^3 \, dx$ נחבר עם הזהות שהראינו למעלה ונקבל ש:
		\[
		\int_0^\infty \brac{x \sin^2 x^3 + \frac{x \cos 2x^3}{2}} =
		\int_0^\infty \brac{\frac{x}{2} - \frac{x \cos 2x^3}{2} + \frac{x \cos 2x^3}{2}} =
		\int_0^\infty \frac{x}{2}
		\]
		אך ידוע כי $\int_0^\infty \frac{x}{2} dx = \frac{1}{4} x^2 \big|^\infty_0 = \infty$ והגענו לסתירה.
		ומכאן שהאינטגרל $\int_0^\infty x \sin x^3 \, dx$ אינו מתכנס בהחלט אלא מתכנס בתנאי.
	\end{proof}

	\pagebreak
	\subsection*{סעיף ב}
	\[
		\int_0^\infty f(x) \, dx = \int_0^\infty \frac{\arctan \frac{1}{x}}{\sqrt{x}} \, dx
	\]
	\begin{proof}
		באינטגרל זה יש 2 נקודות בעייתיות: אפס ואינסוף.
		\begin{enumerate}
			\item נתחיל עם 0 עד 1: \\
			נסמן $g(x) = \frac{1}{\sqrt{x}}$ ונשתמש בגירסא הגבולית של מבחן ההשוואה:
			\[
			\limf{x}{0} \frac{f(x)}{g(x)} = \limf{x}{0} \arctan \frac{1}{x} = \limf{t}{\infty} \arctan t = \frac{\pi}{2}
			\]
			ולכן לפי סעיף א מכיוון ש $\int_0^1 \frac{1}{\sqrt{x}} \, dx$ מתכנס אזי גם $\int_0^1 f(x) \, dx$ מתכנס.

			\item נבדוק את 1 עד אינסוף: \\
			הפעם נסמן $g(x) = \frac{1}{\sqrt[3]{x^2}}$ ונשתמש בגירסאת האינסוף של אותו מבחן ההשוואה:
			\[
				\limf{x}{\infty} \frac{f(x)}{g(x)} = \limf{x}{\infty} x \arctan x = \limf{t}{0} \frac{\arctan t}{t} = 1
			\]
			ולכן מכיוון ש $\int_1^\infty \frac{1}{\sqrt[3]{x^2}} \, dx$ מתכנס אזי גם $\int_1^\infty f(x) \, dx$ מתכנס.
		\end{enumerate}
		ביחד קיבלנו ש $\int_0^\infty f(x) \, dx = \int_0^1 f(x) \, dx + \int_1^\infty f(x) \, dx$ מתכנס. \\
		ע"מ לבדוק האם הוא מתכנס בהחלט נשים לב כי לכל $x \in (0, \infty)$ מתקבל ש $f(x) > 0$ ולכן $f(x) = |f(x)|$ בתחום, ולכן האינטגר מתכנס בהחלט.
	\end{proof}

	\subsection*{סעיף ג}
	\[
		\int_0^1 \frac{\cos \ln x}{x} \, dx
	\]
	\begin{proof}
		הנקודה הבעייתית היא אפס, ולכן נשתמש בהגדרה 3.1 ונבדוק את הגבול:
		\[
		\limf{t}{0+} \int_t^1 \frac{\cos \ln x}{x} \, dx = [u = \ln x] =
		\limf{t}{0+} \int^0_{\ln t} \cos u \, du = \sin u \biggr|^0_{\ln t} =
		\limf{t}{0+} -\sin(\ln t)
		\]
		$\limf{t}{0+} \ln t = -\infty$ ופונקציית הסינוס לא מתכנסת במינוס אינסוף, וע"כ האינטגרל מתבדר.
	\end{proof}

	\pagebreak
	\section*{שאלה 3}
	הוכיחו/הפריכו את הטענות הבאות:
	\subsection*{סעיף א}
	אם $f(x)$ רציפה ב $[0, \infty)$ והאינטגרל $\int_0^\infty f(x) \, dx$ מתכנס אז $\limf{x}{\infty} \int_{\ln x}^x f(t) \, dt = 0$.

	\begin{proof}
		הטענה הכונה. \\
		הנקודה הבעייתית היחידה היא האינסוף (מכיוון ש $f$ רציפה בכל $[0, \infty)$). \\
		נסמן $F(x) = \int_0^x f(t) \, dt$ ומהנוסחא היסודית נקבל ש $\int_{\ln x}^x f(t) \, dt = F(x) - F(\ln x)$. \\
		מכיוון שנתון שהאינטגרל המוכלל $\int_0^\infty f(x)$ מתכנס, אזי:
		\begin{align*}
			\limf{x}{\infty} F(x) = \limf{x}{\infty} F(\ln x) = \int_0^\infty f(t) \, dt \\
			\limf{x}{\infty} f(t) \, dt = F(x) - F(\ln x) = 0
		\end{align*}
		ובכך הוכחנו את הטענה.
	\end{proof}

	\subsection*{סעיף ב}
	אם $\int_a^\infty f(x) \, dx$ מתכנס ו $\int_a^\infty g(x) \, dx$ מתכנס, אזי $\int_a^\infty f(x) g(x) \, dx$ מתכנס.
	\begin{proof}
		הטענה לא נכונה. \\
		נסתכל על דוגמא 3.12 בעמוד 37 בכרך ב, משאלה 32 מתקבל ש $\int_1^\infty \frac{\cos x}{x^{0.5}} \, dx$ מתכנס, אך הכפל של הפונקציה בעצמה $\int_1^\infty \frac{\cos^2 x}{x}$ מתבדר כמו שמוכח בדוגמא 3.12
	\end{proof}

	\subsection*{סעיף ג}
	אם $f(x)$, $g(x)$ פונקציות רציפות אי שליליות בקטע $[0, \infty)$ ו
	$\int_0^\infty g(x) \, dx$, $\int_0^\infty f(x) \, dx$ מתכנסים, אז $\int_0^\infty M(x) \, dx$ מתכנס כאשר $M(x) = max \set{f(x), g(x)}$ לכל $x \geq 0$.

	\begin{proof}
		הטענה נכונה. \\
		$M(x) = max \set{f(x), g(x)} = \frac{f(x) + g(x)}{2} + \frac{\pip{f(x) - g(x)}}{2}$ אזי $M(x)$ רציפה כמנה, סכום, והרכבה של פונקציות רציפות.
		ולכן גם אינטגרבילית. \\
		ומכיוון ששתי הפונקציות אי שליליות אזי מתקיים: $0 \leq M(x) \leq f(x) + g(x)$ ומתוך האדיטיביות של האינטגרל אנו מקבלים ש $\int_0^\infty f(x) + g(x) \, dx$ מתכנס.
		ולכן לפי מבחן ההשוואה באינסוף נקבל ש $\int_0^\infty M(x) \, dx$ מתכנס.
	\end{proof}


	\pagebreak
	\section*{שאלה 4}
	תהיינה $f(x)$, $g(x)$ פונקציות חסומות ב $[a, \infty)$. כמו כן, נניח ש $g(x)$ רציפה ב $[a, \infty)$ ו $f(x)$ מונוטונית ובעלת נגזרת רציפה ב $[a, \infty)$.
	הוכיחו כי האינטגרל $\int_a^\infty f'(x)g(x) \, dx$ מתכנס.

	\begin{proof}
		נניח ש $f$ יורדת, וההוכחה תעבוד גם בעבור $f$ עולה. \\
		מחסימות $g(x)$ בקטע קיים $M \in \RR$ כך ש $|g(x)| \leq M$ לכל $x \in [a, \infty)$
		נכפיל את שני הצדדים ב $|f'(x)| \geq 0$ ונקבל ש $|f'(x)g(x)| \leq |f'(x)M|$, כעת נשתמש במבחן ההשוואה באינסוף כדי לבדוק את ההתכנסות,
		מכיוון ש $f(x)$ יורדת אזי $f'(x) \leq 0$ ומכאן ש $|f'(x)| = -f'(x)$. ולכן:
		\[
		\int_a^\infty |f'(x)M| \, dx = \int_a^\infty -f'(x)M \, dx = \limf{t}{\infty} M\int_a^t f'(x) \, dx = M \limf{t}{\infty}(f(t) - f(a))
		\]
		מכיוון ש $f$ מונוטונית וחסומה אזי שקיים $L = \limf{t}{\infty} f(t)$ גבול סופי. ומאריתמטיקה של גבולות נקבל: \\
		$M \cdot \limf{t}{\infty}(f(t) - f(a)) = M \cdot L - M \cdot f(a)$.
		וע"כ $\int_a^\infty |f'(x)M| \, dx$ מתכנס ולכן גם $\int_a^\infty |f'(x)g(x)| \, dx$ מתכנס. ז"א $\int_a^\infty f'(x)g(x) \, dx$ מתכנס בהחלט ובפרט מתכנס.
	\end{proof}

	\pagebreak
	\section*{שאלת רשות (במקום שאלה 5)}
	הוכיחו כי:
	$\int_0^\infty \frac{\sin^2 x}{x^2} \, dx = \int_0^\infty \frac{\sin x}{x} \, dx$
	\begin{proof}
		נתחיל בלהראות ששני הצדדים אכן מתכנסים:
		\begin{enumerate}
			\item $0 \leq \sin^ x \leq 1 \Leftarrow \sin x \leq 1$,  $0 < x^2$
			ביחד נקבל ש $0 < \frac{\sin^2 x}{x^2} < \frac{1}{x^2}$.
			ומכיוון שידוע ש $\int_0^\infty \frac{1}{x^2}$ מתכנס אזי ממבחן ההשוואה באינסוף $\int_0^\infty \frac{\sin^2 x}{x^2}$ מתכנס.

			\item נפצל ל $\int_1^\infty \frac{\sin x}{x} + \int_0^1 \frac{\sin x}{x}$ החצי הראשון מתכנס לפי שאלה 32 בעמוד 37 בכרך ב
			החצי השני נשתמש במבחן ההשוואה: \\
			מכיוון ש $\sin x \leq 1$ ומכיוון שמדובר בקטע $[0,1]$ נקבל: $0 \leq \frac{\sin x}{x} \leq \frac{1}{x}$.
			ידוע כי $\int_0^1 \frac{1}{x}$ מתכנס, ולכן גם $\int_0^1 \frac{\sin x}{x}$ מתכנס. \\
			וביחד $\int_0^\infty \frac{\sin x}{x}$ מתכנס.
		\end{enumerate}
		כעת נשתמש באינטגרציה בחלקים ע"מ להוכיח את השוויון:
		\begin{align*}
			\int_0^\infty \frac{\sin^2 x}{x^2} \, dx &= \begin{bmatrix}
				u = \sin^2 x & v' = \frac{1}{x^2} \\
				u' = 2\sin x \cdot \cos x = \sin 2x & v = - \frac{1}{x}
			\end{bmatrix} \\
			&= \frac{sin^2 x}{x} \biggr|_0^\infty - \int_0^\infty - \frac{\sin 2x}{x} \, dx =
			\limf{x}{\infty} \frac{\sin^2 x}{x} - \limf{x}{0} \frac{\sin^2 x}{x} + \int_0^\infty \frac{\sin 2x}{x} \, dx \\
			&= 0 - 0 + \int_0^\infty \frac{\sin 2x}{x} \, dx = [t = 2x \Rightarrow dt = 2dx] =
			\int_0^\infty \frac{2 \sin t}{t} \cdot \frac{1}{2} \, dt = \int_0^\infty \frac{\sin t}{t} \, dt
		\end{align*}
		כנדרש.
	\end{proof}
\end{document}
