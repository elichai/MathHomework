% !TEX program = xelatex
\def\NN{\mathbb{N}}
\def\RR{\mathbb{R}}
\def\ZZ{\mathbb{Z}}
\def\QQ{\mathbb{Q}}
\def\PP{\mathcal{P}}
\def\SS{\mathcal{S}}
\def\DD{\mathcal{D}}
\def\sub{\setminus}
\def\bld{\mathbf}
\def\lmf{\lim_{n \to \infty}}
% Make the ref command use parenthesis
\let\oldref\ref
\renewcommand{\ref}[1]{(\oldref{#1})}

\newcommand{\ontop}[1]{\overset{\text{#1}}}
\newcommand{\ang}[1]{\langle #1 \rangle}
\newcommand{\pip}[1]{\left| #1 \right|}



% style
\newcommand{\bm}[1]{\displaystyle{#1}}
\def\nl{$ $ \newline}

\ExplSyntaxOn

\NewDocumentCommand{\getenv}{om}
{
  \sys_get_shell:nnN{ kpsewhich ~ --var-value ~ #2 }{}#1
}

\ExplSyntaxOff

%environments

\documentclass{article}
\usepackage[]{amsthm} %lets us use \begin{proof}
\usepackage{amsmath}
\usepackage{mathtools}
\usepackage{enumerate}
\usepackage{xparse}
\usepackage[makeroom]{cancel}
\usepackage[]{amssymb} %gives us the character \varnothing
\usepackage{polyglossia}
\usepackage{fontspec}
% \usepackage[frak=mma]{mathalfa}
\setdefaultlanguage{hebrew}
\setotherlanguage{english}
%\setmainfont{Frank Ruehl CLM}
\setmainfont{David CLM}
\setmonofont{Miriam Mono CLM}
\setsansfont{Simple CLM}
\newfontface\niceee{Brush Script MT}
\DeclarePairedDelimiter\set\{\}
% Use the following if you only want to change the font for Hebrew
%\newfontfamily\hebrewfont[Script=Hebrew]{David CLM}
%\newfontfamily\hebrewfonttt[Script=Hebrew]{Miriam Mono CLM}
%\newfontfamily\hebrewfontsf[Script=Hebrew]{Simple CLM}
\getenv[\ID]{ID}
\newtheorem{lemma}{טענת עזר}
\title{אלגברה לינארית - ממ"ן 11}
\author{אליחי טורקל \ID}
\date\today

\DeclareFontFamily{OT1}{cmrx}{}
\DeclareFontShape{OT1}{cmrx}{m}{n}{<->cmr10}{}
\let\saveLongrightarrow\Longrightarrow
\makeatletter
\renewcommand*{\Longrightarrow}{%
    \mathrel{\rlap{\fontfamily{cmrx}\fontencoding{OT1}\selectfont=}%
    \hphantom{\saveLongrightarrow}%
    \llap{$\m@th\Rightarrow$}}}
\makeatother


%\clearpage %Gives us a page break before the next section. Optional.
%\selectlanguage{english}
	%Section and subsection automatically number unless you put the asterisk next to them.


\begin{document}
	\maketitle %This command prints the title based on information entered above

	\section*{שאלה 1}

	\subsection*{סעיף א}
	פתרו את $3x + 4 = \frac{1}{2}$ מעל $\ZZ_5$
	פתרון:
	\[
		3x + 4 \equiv \frac{1}{2} \overset{(1)}\equiv
		3 \Rightarrow
		3x \equiv -1 \overset{(2)}\Rightarrow
		3x \equiv 4 \Rightarrow
		x \equiv 4 \cdot 3^{-1} \overset{(3)}\Rightarrow
		x \equiv 4 \cdot 2 \equiv 3
	\]
(1) $ 2 \cdot 3 \equiv 1 \pmod 5 \Rightarrow 2^{-1} \equiv 3 \pmod 5$ \\
(2) $ 5 - 1 = 4 \Rightarrow -1 \equiv 4 \pmod 5$ \\
(3) $ 3 \cdot 2 \equiv 1 \pmod 5 \Rightarrow 3^{-1} \equiv 2 \pmod 5$ \\ \\
למשוואה יש פתרון יחיד $x = 3$

\subsection*{סעיף ב}
פתרו את $3x^2 - 2x + 1 = 0$ מעל $\ZZ_{11}$ \\
פתרון: \\
עפ"י נוסחת השורשים:
\begin{align*}
	3x^2 - 2x + 1 \overset{(1)}\equiv
	3x^2 + 9x + 1 \Rightarrow
	\frac{-9 \pm \sqrt{9^2-4 \cdot 3}}{2 \cdot 3}
	\overset{(2)}\equiv \frac{2 \pm \sqrt{3}}{6} \\[1em]
	\frac{2 + \sqrt{3}}{6} \overset{(3)}\equiv
	\frac{7}{6} \overset{(4)}\equiv 3
	\quad \lor \quad
	\frac{2 - \sqrt{3}}{6} \overset{(3)}\equiv
	-\frac{3}{6} \overset{(4)}\equiv -6
	\overset{(5)}\equiv 5
\end{align*}
למשוואה יש 2 פתרונות $x_1 = 3$ , $x_2 = 5$ \\

\noindent
(1) $ 11 - 2 = 9 \Rightarrow -2 \equiv 9 \pmod{11}$ \\
(2) $ 11 - 9 = 2 \Rightarrow -9 \equiv 2 \pmod{11}$ \\
(3) $ 5^2 = 25 \equiv 3 \pmod{11} \Rightarrow \sqrt{3} \equiv 5 \pmod{11}$ \\
(4) $ 6 \cdot 2 \equiv 1 \pmod{11} \Rightarrow 6^{-1} \equiv 2 \pmod{11}$ \\
(5) $ 11 - 6 = 5 \Rightarrow -6 \equiv 5 \pmod{11}$

	\pagebreak
	\section*{שאלה 2}
	\subsection*{סעיף א}
	פתרו את המערכת הלינארית הבאה
	\[
		\ZZ_7 \text{מעל }
		\begin{cases}
			2x - y + 4z = 1 \\
			x + 2y - 3z = 6 \\
			x - y + z = 3
		\end{cases}
	\]
	פתרון:
 \begin{align*}
	&\begin{pmatrix}
		2 & -1 & 4 & 1 \\
		1 & 2 & -3 & 6 \\
		1 & -1 & 1 & 3
	\end{pmatrix}
	\overset{R_1 \leftrightarrow R2}\Longrightarrow
	\begin{pmatrix}
		1 & 2 & -3 & 6 \\
		2 & -1 & 4 & 1 \\
		1 & -1 & 1 & 3
	\end{pmatrix}
	\overset{R_2 \rightarrow R_2 - 2R_1}{\underset{R_3 \rightarrow R_3 - R_1}\Longrightarrow}
	\begin{pmatrix}
		1 & 2 & -3 & 6 \\
		0 & 2 & 3 & 3 \\
		0 & 4 & 4 & 4
	\end{pmatrix}
	\overset{R_1 \rightarrow R_1 + 3R_3}{\underset{R_3 \rightarrow R_3 -2R_2}\Longrightarrow} \\
	&\begin{pmatrix}
		1 & 0 & 2 & 4 \\
		0 & 2 & 3 & 3 \\
		0 & 0 & 5 & 5
	\end{pmatrix}
	\overset{R_2 \rightarrow 4R_2}{\underset{R_3 \rightarrow 3R_3}\Longrightarrow}
	\begin{pmatrix}
		1 & 0 & 2 & 4 \\
		0 & 1 & 5 & 5 \\
		0 & 0 & 1 & 1
	\end{pmatrix}
	\overset{R_1 \rightarrow R_1 - 2R_3}{\underset{R_2 \rightarrow R_2 - 5R_3}\Longrightarrow}
	\begin{pmatrix}
		1 & 0 & 0 & 2 \\
		0 & 1 & 0 & 0 \\
		0 & 0 & 1 & 1
	\end{pmatrix}
 \end{align*}
לאחר הדירוג ניתן לראות כי כל המשתנים קשורים, ולכן למטריצה יש פתרון יחיד, שהוא:
$x = 2, y = 0, z = 1$.
	\subsection*{סעיף ב}
	פתרו את המערכת הלינארית הבאה
	\[
		\RR \text{מעל }
		\begin{cases}
			2x + 2y + 8z + w + 18t = 0 \\
			3x + 3y + 3z + 13w + 5t = 0\\
			2x + 2y + 4z + 3w + 2t = 0
		\end{cases}
	\]
	פתרון:
	ניתן לראות כי המערכת הינה הומוגנית ומספר המשתנים גדול ממספר המשוואות ועל כן ע"פ משפט 1.13.1 יש לה פתרון לא טריוויאלי.
	נראה איזה מהמשתנים חופשיים ואיזה קשורים בעזרת מטריצת מדרגות:
	\begin{alignat*}{4}
		&\begin{pmatrix}
			2 & 2 & 8 & 1 & 18 \\
			3 & 3 & 3 & 13 & 5 \\
			2 & 2 & 4 & 3 & 2
		\end{pmatrix}
		&\overset{R_1 \rightarrow \frac{R_1}{2}}\Longrightarrow
		&\begin{pmatrix}
			1 & 1 & 4 & 0.5 & 9 \\
			3 & 3 & 3 & 13 & 5 \\
			2 & 2 & 4 & 3 & 2
		\end{pmatrix}
		&\overset{R_2 \rightarrow R_2 - 3R_1}{\underset{R_3 \rightarrow R_3 - 2R_1}\Longrightarrow} \\
		&\begin{pmatrix}
			1 & 1 & 4 & 0.5 & 9 \\
			0 & 0 & -9 & 11.5 & -22 \\
			0 & 0 & -4 & 2 & -16
		\end{pmatrix}
		&\overset{R_2 \rightarrow - \frac{R_2}{9}}{\Longrightarrow}
		&\begin{pmatrix}
			1 & 1 & 4 & 0.5 & 9 \\
			0 & 0 & 1 & -\frac{23}{18} & \frac{22}{9} \\
			0 & 0 & -4 & 2 & -16
		\end{pmatrix}
		&\overset{R_3 \rightarrow R_3 + 4R_2}{\Longrightarrow}\\
		&\begin{pmatrix}
			1 & 1 & 4 & 0.5 & 9 \\
			0 & 0 & 1 & -\frac{23}{18} & \frac{22}{9} \\
			0 & 0 & 0 & -\frac{28}{9} & -\frac{56}{9}
		\end{pmatrix}
		&\overset{R_3 \rightarrow -\frac{9}{28}R_3}{\Longrightarrow}
		&\begin{pmatrix}
			1 & 1 & 4 & 0.5 & 9 \\
			0 & 0 & 1 & -\frac{23}{18} & \frac{22}{9} \\
			0 & 0 & 0 & 1 & 2
		\end{pmatrix}
		&\overset{R_1 \rightarrow R_1 - \frac{1}{2}R_3}{\underset{R_2 \rightarrow R_2 + \frac{23}{18}R_3}\Longrightarrow} \\
		&\begin{pmatrix}
			1 & 1 & 4 & 0 & 8 \\
			0 & 0 & 1 & 0 & 5 \\
			0 & 0 & 0 & 1 & 2
		\end{pmatrix}
		&\overset{R_1 \rightarrow -4R_2}{\Longrightarrow}
		&\begin{pmatrix}
			1 & 1 & 0 & 0 & -12 \\
			0 & 0 & 1 & 0 & 5 \\
			0 & 0 & 0 & 1 & 2
		\end{pmatrix}
	 \end{alignat*}
	 לאחר הדירוג ניתן לראות כי יש 3 משתנים קשורים $x, z, w$  ו2 משתנים חופשיים: $y, t$
	 ובגלל שהמערכת היא מעל $\RR$ אז ישנם אינסוף פתרונות.

	\pagebreak
	\section*{שאלה 3}
	\[
		\text{ פרמטרים ממשיים} a,t \text{ כאשר } \RR \text{מעל }
		\begin{cases}
			x + ay + z = 1 \\
			ax + a^2y + z = 2 + a \\
			ax + 3ay + z = 2 - t
		\end{cases}
	\]
	נתחיל בלדרג את המטריצה \\
	\begin{align*}
		&\begin{pmatrix}
			1 & a & 1 & 1 \\
			a & a^2 & 1 & 2+a \\
			a & 3a & 1 & 2-t
		\end{pmatrix}
		\overset{R_2 \rightarrow R_2 - aR_1}{\underset{R_3 \rightarrow R_3 - aR_1}\Longrightarrow}
		\begin{pmatrix}
			1 & a & 1 & 1 \\
			0 & 0 & 1-a & 2 \\
			0 & 3a-a^2 & 1-a & 2-t-a
		\end{pmatrix}
		\overset{R_2 \leftrightarrow R_3}\Longrightarrow \\
		&\begin{pmatrix}
			1 & a & 1 & 1 \\
			0 & 3a-a^2 & 1-a & 2-t-a \\
			0 & 0 & 1-a & 2
		\end{pmatrix}
	 \end{align*}
	למערכת יש פתרון יחיד אם ורק אם כל המשתנים הינם קשורים, \\
	ניתן לראות שזה קורה במטריצה הנ"ל במידה ו $1-a \neq 0 \land 3a - a^2 \neq 0$
	אם נפתור את האי שוויונות האלה נראה שזה קורה כאשר $a \neq 0 \land a \neq 1 \land a \neq 3$ \\
ע"מ לראות מתי יש אינסוף פתרונות ומתי אין פתרונות כלל נבחן כל מקרה בנפרד.
\begin{enumerate}[(I)]
	\item 	 אם $a = 0$ אז
	\[ \begin{pmatrix}
		1 & 0 & 1 & 1 \\
		0 & 0 & 1 & 2-t \\
		0 & 0 & 1 & 2
	\end{pmatrix}
	\overset{R_2 \rightarrow R_2 - R_3}\Longrightarrow
	\begin{pmatrix}
		1 & 0 & 1 & 1 \\
		0 & 0 & 0 & -t \\
		0 & 0 & 1 & 2
	\end{pmatrix}
	\overset{R_2 \leftrightarrow R_3}\Longrightarrow
	\begin{pmatrix}
		1 & 0 & 1 & 1 \\
		0 & 0 & 1 & 2 \\
		0 & 0 & 0 & t
	\end{pmatrix}
	\]
	ניתן לראות בשורה האחרונה במטריצה המדורגת שאם $t \neq 0$ אז יש סתירה כי נקבל מספר השונה מאפס אבל שווה לאפס ולכן אין פתרונות כלל.
	אבל אם $t = 0$ אז המשתנה $y$ הינו משתנה חופשי ו $x, y$ הם משתנים קשורים,
	 ובגלל שיש משתנה חופשי ואנחנו מעל $\RR$ אז ישנם אינסוף פתרונות. \\
 נסמן $y = r$  ונקבל את הפתרון הכללי $(-1, r, 2) \Leftarrow x = -1 \land y = r \land z = 2$
	\item אם $a = 1$ אז
	$\begin{pmatrix}
		1 & 1 & 1 & 1 \\
		0 & 2 & 0 & 1-t \\
		0 & 0 & 0 & 2
	\end{pmatrix}$
	ניתן לראות כי השורה האחרונה היא סתירה(0=2) ולכן אין פתרונות למערכת.
	\pagebreak
	\item אם $a = 3$ אז

	\begin{align*}
		&\begin{pmatrix}
			1 & 3 & 1 & 1 \\
			0 & 0 & -2 & -t-1 \\
			0 & 0 & -2 & 2
		\end{pmatrix}
		\overset{R_2 \leftrightarrow R_3}\Longrightarrow
		\begin{pmatrix}
			1 & 3 & 1 & 1 \\
			0 & 0 & -2 & 2 \\
			0 & 0 & -2 & -t-1
		\end{pmatrix}
		\overset{R_3 \rightarrow R_3 + R_2}\Longrightarrow \\
		&\begin{pmatrix}
			1 & 3 & 1 & 1 \\
			0 & 0 & -2 & 2 \\
			0 & 0 & 0 & -t-3
		\end{pmatrix}
		\overset{R_2 \rightarrow \frac{R_2}{2}}\Longrightarrow
		\begin{pmatrix}
			1 & 3 & 1 & 1 \\
			0 & 0 & -1 & 1 \\
			0 & 0 & 0 & -t-3
		\end{pmatrix}
		\overset{R_1 \rightarrow R_1 + R_3}\Longrightarrow \\
		&\begin{pmatrix}
			1 & 3 & 0 & 2 \\
			0 & 0 & -1 & 1 \\
			0 & 0 & 0 & -t-3
		\end{pmatrix}
		\overset{R_2 \rightarrow -R_2}{\underset{R_3 \rightarrow -R_3}\Longrightarrow}
		\begin{pmatrix}
			1 & 3 & 0 & 2 \\
			0 & 0 & 1 & -1 \\
			0 & 0 & 0 & t+3
		\end{pmatrix}
		\end{align*}
		ניתן לראות כי אם $t \neq -3$
		אז השורה השלישית היא סתירה כי נקבל מספר ששונה מאפס אך שווה לאפס. \\
		ובמידה ו $t = 3$
		אז גם כאן נקבל מערכת  שבה $y$ הוא משתנה חופשי ו $x,z$ הינם משתנים קשורים
		ולכן גם כאן ישנם אינסוף פתרונות. \\
	 את הפתרון הכללי ניתן לקבל ע"י הסתכלות המטריצה המדורגת, נסמן $y = r$
	 ונקבל: $(2-3r, r, -1) \Leftarrow x = 2 - 3r, y = r, z= -1$
\end{enumerate}
לסיכום, \\
 אם $a \neq 3 \land a \neq 1 \land a \neq 0$ אזי יש פתרון יחיד למערכת. \\
ואם $t \neq 0 \land a = 0$ או $a = 1$ או $a = 3 \land t \neq 3 $ אז אין פתרונות למערכת. \\
ואם $t = 0 \land a = 0 $ אז יש אינסוף, והפתרון הכללי הוא $(-1, r, 2)$ \\
ואם $\land t = -3 \land a = 3$ אז יש אינסוף פתרונות והפתרון הכללי הוא $(2-3r, r, -1)$

	\pagebreak
	\section*{שאלה 4}
	נתונה התת קבוצה $A = \set{(1,1,\lambda), (\lambda, -1, \lambda^2), (1, -\lambda, 0)} \subseteq \RR^3$
	\subsection*{סעיף א}
מצאו את כל הערכים של $\lambda$ עבורים הקבוצה $A$ היא בלתי תלויה לינארית \\
נסמן $A = \set{\bld{u}, \bld{v}, \bld{w}}$.
עפ"י הגדרה 2.6.1 הקבוצה היא בלתי תלויה לינארית אם ורק אם למשוואה הבאה יש רק את הפתרון הטריוויאלי:
$\alpha_1\bld{u} + \alpha_2\bld{v} + \alpha_3\bld{w} = 0$.
עפ"י משפט 2.5.3 נציב את הוקטורים בתור עמודות במטריצה ונדרג אותה ע"מ שנראה האם היא תלויה לינארית
\begin{align*}
	&\begin{pmatrix}
		1 & \lambda & 1 \\
		1 & -1 & -\lambda \\
		\lambda & \lambda^2 & 0
	\end{pmatrix}
	\overset{R_2 \rightarrow R_2 - R_1}{\underset{R_3 \rightarrow R_3 - \lambda R_1}\Longrightarrow}
	\begin{pmatrix}
		1 & \lambda & 1 \\
		0 & -1-\lambda & -1-\lambda \\
		0 & 0 & - \lambda
	\end{pmatrix}
\end{align*}
עכשיו כשהמטריצה מדורגת אפשר לראות שבגלל שבשורות 2 ו 3 כל המקדמים הם תלויים בלמדא אז יש לנו 3 מצבים,
במידה ו $\lambda = 0$ שורה 3 מתאפסת,
במידה ו $\lambda = -1$ שורה 2 מתאפסת,
ובמידה ו $\lambda \neq 0 \land \lambda \neq -1$  אף שורה לא מתאפסת,
 נסתכל על כל מצב בנפרד:
\begin{enumerate}[(I)]
	\item אם $\lambda = -1$ אזי
	$
		\begin{pmatrix}
			1 & -1 & 1 \\
			0 & 0 & 1 \\
			0 & 0 & 0 \\
		\end{pmatrix}
		\overset{R_2 \leftrightarrow R_3}{\Longleftarrow}
		\begin{pmatrix}
			1 & -1 & 1 \\
			0 & 0 & 0 \\
			0 & 0 & 1
		\end{pmatrix}
	$
ישנם 2 מקדמים קשורים, ומקדם אחד חופשי(וממשי)  ועל כן יש אינסוף פתרונות והקבוצה A תלויה לינארית.
	\item אם $\lambda = 0$ אזי
	$\begin{pmatrix}
		1 & 0 & 1 \\
		0 & -1 & -1 \\
		0 & 0 & 0
	\end{pmatrix}$
	וגם כאן יש 2 משתנים קשורים ומשתנה אחד חופשי(וממשי) ועל כן גם כאן יהיו אינסוף פתרונות והקבוצה A תלויה לינארית.

	\item אם $\lambda \neq 0 \land \lambda \neq -1$ אזי המטריצה כבר מדורגת וכל המשתנים קשורים, ועל כן יש רק את הפתרון הטרוויאלי והקבוצה A היא בלתי תלויה לינארית.
\end{enumerate}

\subsection*{סעיף ב}
בכל מקרה בו הקבוצה תלויה לינארית, השלימו אותה לקבוצה פורשת של $\RR^3$

\begin{enumerate}[(I)]
	\item אם $\lambda = -1$ אז $A = \set{(1,1, -1), (-1, -1, 1), (1, 1, 0)}$
	 הוקטור הראשון והשני הם תלויים לינארית (הסכום שלהם הוא 0) ועל כן נוסיף את $(1,0,0)$. \\
	ניתן לראות שהקבוצה המתקבלת פורסת את $\RR^3$ ע"י ביצוע כמה פעולות לינאריות פשוטות $(0,0,1)=(1,1,0) - (1,1,-1)$ ו $(0,1,0) = (1,1,0)-(1,0,0)$ \\
לפיכך לאחר ההוספה A שקול ל $\set{(0,0,1), (0,1,0), (1,0,0)}$ שהוא הבסיס הסטנדרטי של המרחב $\RR^3$ ולכן גם הקבוצה A לאחר ההוספה פורסת את המרחב.

\item אם $\lambda = 0$ אז $A = \set{(1,1, 0), (0, -1, 0), (1, 0, 0)}$
בשלשת הוקטורים האיבר האחרון הוא 0 ועל נוסיף את הוקטור $(0,0,1)$. \\
נראה שהקבוצה פורסת את המרחב $\RR^3$
באותה שיטה כמו מקודם, אך כאן $(1,0,0)$ כבר מוכל ב A, ואת $(0,0,1)$ הוספנו עכשיו, לכן נחשב רק את הוקטור האחרון:
 $(0, 1, 0)=-(0, -1, 0)$ \\
לפיכך לאחר ההוספה A שקול ל $\set{(0,0,1), (0,1,0), (1,0,0)}$ שהוא הבסיס הסטנדרטי של המרחב $\RR^3$ ולכן גם הקבוצה A לאחר ההוספה פורסת את המרחב.
\end{enumerate}

\subsection*{סעיף ג}
עבור אילו ערכי $\lambda$ הווקטור $(\lambda, -\lambda^2, -1)$ הוא צירוף לינארי של $A$?
\begin{enumerate}
	\item כאשר $\lambda = -1$ נשתמש במשפט 2.5.3 ע"מ לבדוק האם הוקטור הוא צירוף לינארי של הקבוצה $A$:
	\begin{align*}
		\begin{pmatrix}
			1 & -1 & 1 & -1 \\
			1 & -1 & 1 & -1 \\
			-1 & 1 & 0 & 0 \\
		\end{pmatrix}
			\overset{R_2 \rightarrow R_2 - R_1}{\underset{R_3 \rightarrow R_3+R_1}\Longrightarrow}
		\begin{pmatrix}
			1 & -1 & 1 & -1 \\
			0 & 0 & 0 & 0 \\
			0 & 0 & 1 & -1 \\
		\end{pmatrix}
		\overset{R_2 \leftrightarrow R_3}\Longrightarrow
		\begin{pmatrix}
			1 & -1 & 1 & -1 \\
			0 & 0 & 1 & -1 \\
			0 & 0 & 0 & 0 \\
		\end{pmatrix}
		\end{align*}
		אין שורת סתירה, ולכן יש פתרון, ועל כן הוקטור הינו צירוף לינארי של וקטורי $A$.
	\item כאשר $\lambda = 0$ נעשה בדיוק כמו בסעיף א ונקבל:
	$\begin{pmatrix}
		1 & 0 & 1 & 0 \\
		1 & -1 & 0 & 0 \\
		0 & 0 & 0 & -1
	\end{pmatrix}$ \\
	השורה האחרונה הינה שורת סתירה ולכן אין פתרון, והוקטור אינו צירוף לינארי של וקטורי $A$.
	\item כאשר $\lambda \neq 0 \land \lambda \neq -1$ הראנו בסעיף א שהקבוצה A היא בלתי תלויה לינארית, ולכן עפ"י משפט 2.7.8 הקבוצה פורסת את כל המרחב
	$\RR^3$ ולכן כל וקטור ב $\RR^3$ הוא צירוף לינארי של A.
\end{enumerate}
סיכום: הוקטור הנתון הינו צירוף לינארי של וקטורי $A$ כאשר $\lambda \neq 0$.

\pagebreak
\section*{שאלה 5}
יהיו $\bld{u_1}, \dotsc, \bld{u_k}, \bld{v} \in \RR^n$
נניח שניתן להציג את הווקטור $\bld{v}$ כצירוף לינארי של הוקטורים $\bld{u_1}, \dotsc, \bld{u_k}$ ושההצגה הזאת יחידה. \\
ונניח וקיים וקטור $\bld{w} \in \RR^n$ כך שלמשוואה $x_1\bld{u_1}, \dotsc, x_k\bld{u_k} = \bld{w}$
אין פתרון.
\subsection*{סעיף א}
הוכיחו כי הוקטורים $\bld{u_1}, \dotsc, \bld{u_k}$ בלתי תלויים לינארית

\begin{proof}
	ע"מ להוכיח שהוקטורים אינם תלויים לינארים, נשתמש בהגדרה 2.6.1 ונוכיח שלמערכת ההומוגנית
	$\sum_{i=1}^{k}{\alpha_i\bld{u_i}} = \bld{0}$
	קיים אך ורק הפתרון הטריוויאלי שהוא $\alpha_1=\dotsc=\alpha_k=0$. \\
	עפ"י הנתון ש $\bld{v}$ הינו צירוף לינארי של הוקטורים, ושהוא ההצגה היחידה, ולפי משפט 2.5.3 מתקבל שקיימת קבוצה יחידה של
	$\set{\beta_1, \dotsc, \beta_k}$ ככה ש
	$\sum_{i=1}^{k}{\beta_i\bld{u_i}} = \bld{v}$. \\
	אם נחבר את המשוואה
	$\sum_{i=1}^{k}{\beta_i\bld{u_i}} = \bld{v}$
	 עם המשוואה
	$\sum_{i=1}^{k}{\alpha_i\bld{u_i}} = \bld{0}$
	\begin{align*}
		\hspace{178pt minus 1fil}
		\sum_{i=1}^{k}{(\alpha_i + \beta_i)\bld{u_i}} = \bld{v} + \bld{0} = \bld{v}
		\hspace{25pt minus 1fil} \text{נקבל: }
	\end{align*}
	ניתן לראות שאם קיים איזשהו  $\beta_i \neq 0$ זה אומר שיהיה סקלר $\alpha_i + \beta_i \neq \beta_i$
	שנמצא בפתרון המשוואה,	מה שבא בסתירה עם הנתון של $\bld{v}$ קיימת הצגה יחידה.
	ועל כן למערכת $\sum_{i=1}^{k}{\alpha_i\bld{u_i}} = \bld{0}$ קיים אך ורק הפתרון הטרוויאלי
	$\alpha_1 = \cdots \alpha_k = 0$
	ועל כן עפ"י הגדרה 2.6.1 הוקטורים $\bld{u_1}, \cdots, \bld{u_k}$ הינם בלתי תלוים לינארית
\end{proof}

\subsection*{סעיף ב}
הוכיחו כי $k < n$
\begin{proof}
	עפ"י משפט 2.7.8 כל קבוצה של $n$ וקטורים בלתי תלויים לינארים פורסת את המרחב $F^n$
ועל כן אם קיים וקטור במרחב שלא נפרס ע"י קבוצת הוקטורים שלנו, ואזי שאנו יודעים שהקבוצה בלתי תלויה לינארית, מתקבל ש
$n \neq k$. \\
בנוסף, לפי משפט 2.6.6 כל קבוצת $k$ וקטורים ב $F^n$ אם $k>n$ אז הקבוצה תלויה לינארית.
ובגלל שהקבוצה שלנו בלתי תלויה לינארית מתקבל ש $k \not> n$ \\
אם נחבר את שתי טענות אלו מתקבל שבהכרח $k < n$.
\end{proof}

\subsection*{סעיף ג}
הוכיחו הי הקבוצה $\set{\bld{u_1}, \cdots, \bld{u_k}, \bld{w}}$ בלתי תלויה לינארית.
\begin{proof}
	נוכיח בדרך השלילה,
	ונניח שהקבוצה תלויה לינארית, ועל עפ"י הגדרה 2.6.2 קיימים
	$\sum_{i=1}^{k}{\alpha_i\bld{u_k}} + \alpha_{k+1}\bld{w} = \bld{0}$
	ככה שלא כל ה $\alpha_i$ אפסים. \\
	ולכן עפ"י תכונות החיבור של וקטורים מתקבל $\sum_{i=1}^{k}{\alpha_i\bld{u_k}} = -\alpha_{k+1}\bld{w}$
	ולפי תכונות הכפל בסקלר מתקבל
	$\sum_{i=1}^{k}{x_i\bld{u_k}} = \bld{w} \Leftarrow \sum_{i=1}^{k}{\frac{\alpha_i}{-\alpha_{k+1}}\bld{u_k}} = \bld{w}$
	מה שבא בסתירה עם הנתון שלמשוואה $x_1\bld{u_1} + \cdots + x_k\bld{u_k}=\bld{w}$ אין פתרון.
\end{proof}


\end{document}
