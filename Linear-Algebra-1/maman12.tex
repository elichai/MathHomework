% !TEX program = xelatex
\def\NN{\mathbb{N}}
\def\RR{\mathbb{R}}
\def\PP{\mathcal{P}}
\def\sub{\setminus}

% Make the ref command use parenthesis
\let\oldref\ref
\renewcommand{\ref}[1]{(\oldref{#1})}



% style
\newcommand{\bm}[1]{\displaystyle{#1}}
\def\nl{$ $ \newline}

\ExplSyntaxOn

\NewDocumentCommand{\getenv}{om}
{
  \sys_get_shell:nnN{ kpsewhich ~ --var-value ~ #2 }{}#1
}

\ExplSyntaxOff

%environments

\documentclass{article}
\usepackage[]{amsthm} %lets us use \begin{proof}
\usepackage{amsmath}
\usepackage{mathtools}
\usepackage{enumerate}
\usepackage{xparse}
\usepackage[makeroom]{cancel}
\usepackage[]{amssymb} %gives us the character \varnothing
\usepackage{polyglossia}
\usepackage{fontspec}
% \usepackage[frak=mma]{mathalfa}
\setdefaultlanguage{hebrew}
\setotherlanguage{english}
%\setmainfont{Frank Ruehl CLM}
\setmainfont{David CLM}
\setmonofont{Miriam Mono CLM}
\setsansfont{Simple CLM}
\newfontface\niceee{Brush Script MT}
\DeclarePairedDelimiter\set\{\}
% Use the following if you only want to change the font for Hebrew
%\newfontfamily\hebrewfont[Script=Hebrew]{David CLM}
%\newfontfamily\hebrewfonttt[Script=Hebrew]{Miriam Mono CLM}
%\newfontfamily\hebrewfontsf[Script=Hebrew]{Simple CLM}
\getenv[\ID]{ID}
\newtheorem{lemma}{טענת עזר}
\title{אלגברה לינארית - ממ"ן 12}
\author{אליחי טורקל \ID}
\date\today

\DeclareFontFamily{OT1}{cmrx}{}
\DeclareFontShape{OT1}{cmrx}{m}{n}{<->cmr10}{}
\let\saveLongrightarrow\Longrightarrow
\makeatletter
\renewcommand*{\Longrightarrow}{%
    \mathrel{\rlap{\fontfamily{cmrx}\fontencoding{OT1}\selectfont=}%
    \hphantom{\saveLongrightarrow}%
    \llap{$\m@th\Rightarrow$}}}
\makeatother


%\clearpage %Gives us a page break before the next section. Optional.
%\selectlanguage{english}
	%Section and subsection automatically number unless you put the asterisk next to them.


\begin{document}
	\maketitle %This command prints the title based on information entered above

	\section*{שאלה 1}
	נתונה המטריצה הבאה מעל $\RR$:
	$A = \begin{pmatrix}
		1 & 0 & 2 & 3 \\
		2 & -1 & 3 & 6 \\
		1 & 4 & 4 & 0 \\
	\end{pmatrix}$. \\
	מצאו מטריצה $R$ קנונית ומטריצה $P$ הפיכה כך ש $R = PA$. \\
	פתרון: \\
	ע"מ למצאו מטריצה $R$ קנונית ומטריצה $P$ הפיכה שמקיימות $R = PA$, נשתמש בהכללה של טענה 3.9.9 הנתונה. \\
	נדרג את $A$ ובמקביל נפעיל את אותן פעולות שורה על מטריצת היחידה, ולכן המטריצה שתתקבל מהפעולות על מטריצת היחידה תהיה הפיכה עפ"י מסקנה 3.9.5.\\
	ובנוסף המטריצה שתתקבל מדירוג מטריצת $A$ היא בהכרח שקולת שורה ל $A$ והכפל של $A$ עם $P$ יתן לנו אותה.
	\begin{align*}
		\left( A | I \right) =
		&\left( \begin{array}{cccc|cccc}
			1 & 0 & 2 & 3   & 1 & 0 & 0 \\
			2 & -1 & 3 & 6  & 0 & 1 & 0 \\
			1 & 4 & 4 & 0   & 0 & 0 & 1 \\
		\end{array} \right)
		\overset{R_2 \rightarrow R_2 - 2R_1}{\underset{R_3 \rightarrow R_3 - R_1}\Longrightarrow}
		\left( \begin{array}{cccc|cccc}
			1 & 0 & 2 & 3    & 1 & 0 & 0 \\
			0 & -1 & -1 & 0  & -2 & 1 & 0 \\
			0 & 4 & 2 & -3   & -1 & 0 & 1 \\
		\end{array} \right) \\
		\overset{R_3 \rightarrow R_3 + 4R_2}{\Longrightarrow}
		&\left( \begin{array}{cccc|cccc}
			1 & 0 & 2 & 3    & 1 & 0 & 0 \\
			0 & -1 & -1 & 0  & -2 & 1 & 0 \\
			0 & 0 & -2 & -3  & -9 & 4 & 1 \\
		\end{array} \right)
		\overset{R_1 \rightarrow R_1 + R_3}{\underset{R_2 \rightarrow R_2 - 0.5R_3}\Longrightarrow}
		\left( \begin{array}{cccc|cccc}
			1 & 0 & 0 & 0     & -8 & 4 & 1 \\
			0 & -1 & 0 & 1.5  & 2.5 & 1 & -0.5 \\
			0 & 0 & -2 & -3   & -9 & 4 & 1 \\
		\end{array} \right) \\
		\overset{R_2 \rightarrow -R_2}{\underset{R_3 \rightarrow -0.5R_3}\Longrightarrow}
		&\left( \begin{array}{cccc|cccc}
			1 & 0 & 0 & 0     & -8 & 4 & 1  \\
			0 & 1 & 0 & -1.5  & -2.5 & 1 & 0.5  \\
			0 & 0 & 1 & 1.5   & 4.5 & -2 & -0.5  \\
		\end{array} \right)
	\end{align*}
$P=\begin{pmatrix}
	-8 & 4 & 1 \\
	-2.5 & 1 & 0.5 \\
	4.5 & -2 & -0.5
\end{pmatrix}$ והיא הפיכה בגלל שהיא התקבלה כתוצאה מהפעלת פעולות שורה אלמנטריות על מטריצת היחידה, ועל כן היא מטריצה אלמנטרית. \\
$R = \begin{pmatrix}
	1 & 0 & 0 & 0 \\
	0 & 1 & 0 & -1.5 \\
	0 & 0 & 1 & 1.5
\end{pmatrix}$ ניתן לראות ש $R$ מדורגת וקנונית, ו $R = PA$ בגלל ש \\
$P = \varphi(I), R = \varphi(I)A$, אבל ניתן לבדוק זאת גם בעזרת כפל מטריצות.

	\pagebreak
	\section*{שאלה 2}
	תהיינה $A,B,C,D$ מטריצות מסדר $n\times n$ המקיימות $ABCD = I$.
	הוכחו שמתקיים $ABCD=DABC=CDAB=BCDA=I$
	\begin{proof}
		נוכיח כל שוויון בנפרד:
		\begin{enumerate}
			\item עפ"י משפט 4.5.2 $(ABC)\cdot(D) = I = (D)\cdot ABC = DABC$
			\item עפ"י משפט 4.5.2 $(AB)\cdot(CD) = I = (CD)\cdot AB = CDAB$
			\item עפ"י משפט 4.5.2 $(A)\cdot(BCD) = I = (A)\cdot BCD = ABCD$
		\end{enumerate}
		ולפיכך $ABCD = DABC = CDAB = ABCD = I$
	\end{proof}

	\pagebreak
	\section*{שאלה 3}
תהיינה $A_n,B_n$ מטריצות ריבועיות מאותו סדר. נתון ש $A$ לא סימטרית ו $B$ סימטרית.
הוכיחו שלא קיימת מטריצה $P$ הפיכה המקיימת $P^t = P^{-1}$ וגם $B = P^{-1}AP$

\begin{proof}
	נניח בשלילה שקיימת מטריצה $P$ הפיכה המקיימת את התכונות, ונגיע לסתירה:
	\begin{align*}
		B = P^{-1}AP \ontop{(a)}\Rightarrow
		B^t = P^t A^t P^{{-1}^t} \ontop{(b)}=
		P^{-1} A^t {P^t}^t \ontop{(c)}=
		P^{-1} A^t P
	\end{align*}
	עפ"י ההגדרה ש $B$ סימטרית נובע כי $B = B^t$ ועל כן $P^{-1}A P=P^{-1} A^t P$
	נפתח את השוויון הנ"ל עוד:
	\begin{alignat*}{2}
		&P^{-1}A P &&= P^{-1} A^t P \\
		\ontop{(d)}\Rightarrow
		&P P^{-1}A P &&= P P^{-1} A^t P \\
		\ontop{(e)}\Rightarrow
		&P P^{-1}A P &&= PP^{-1} A^t P P^{-1} \\
		\ontop{(f)}\Rightarrow
		&IAI &&= IA^tI \\
		\ontop{(g)}\Rightarrow
		&A &&= A^t
	\end{alignat*}
	קיבלנו ש $A = A^t$ בסתירה עם הנתון ש $A$ אינה סימטרית, \\
	ולפיכך, לא קיימת מטריצה $P$ הפיכה המקיימת $B = P^{-1}AP \land P^t = P^{-1}$.
\end{proof}
\nl
(a) - נשחלף את המטריצה. \\
(b) - נתון ש $P^t = P^{-1}$ .\\
(c) - עפ"י טענה 3.2.4. \\
(d) - נכפיל ב $P$ משמאל. \\
(e) - נכפיל ב $P^{-1}$ מימין. \\
(f) - עפ"י טענה 3.8.2 נקבל ש $P P^{-1} = I$. \\
(g) - עפ"י מסקנה 3.5.4.

	\pagebreak
	\section*{שאלה 4}
יהיו $a_1, \dots, a_n \in \RR$ שונים מ 0. נתונה המטריצה מסדר $n \times n$:
\[ A = \begin{pmatrix}
	 0 & a_1 & 0 & \dots & 0 \\
	 0 & 0 & a_2 & \dots & 0 \\
	 \vdots & \vdots & \ddots & \ddots & \vdots \\
	 0 & 0 & 0 & \dots & a_{n-1} \\
	 a_n & 0 & \dots & 0 & 0 \\
\end{pmatrix} \]
הוכיחו, ללא חישוב של $A^{-1}$ ש $A$ הפיכה, ולאחר מכן חשבון את $A^{-1}$.
\begin{proof}
	ניתן לראות שאם נתחיל בשורה האחרונה, ונחליף אותה עם השורה מעליה, ואז נחליף את האחת לפני האחרונה עם זאת שמעליה,
	ונעשה ככה עד השורה הראשונה, נקבל מטריצה אלכסונית עם $a_n$ בשורה הראשונה $a_1$ בשניה וכן הלאה עד ש $a_{n-1}$ באחרונה.
	מטריצה זו הינה שקולת שורה למטריצה המקורית, כי החלפת שורות זו פעולת שורה אלמנטרית, ובנוסף נתון לנו ש $a_i \neq 0$
	ולפיכך עפ"י שאלה 3.8.5 מתקבל ש $A$ הפיכה. \\
	עפ"י אותה שאלה מתקבל גם שההופכי של $A$ היא:
	\[ A^{-1} = \begin{pmatrix}
		\frac{1}{a_n} & 0 & \dots & 0 & 0 \\
		0 & \frac{1}{a_1} & 0 & \dots & 0 \\
		0 & 0 & \frac{1}{a_2} & \dots & 0 \\
		\vdots & \vdots & \ddots & \ddots & \vdots \\
		0 & 0 & 0 & \dots & \frac{1}{a_{n-1}} \\
   \end{pmatrix} \]
\end{proof}


	\pagebreak
	\section*{שאלה 5}
	תהיינה $A,B$ מטריצות ממשיות מסדר $n \times n$ כך ש $B^2 A = -2B^3$ וגם $B^3 + AB^2 = 3I$.
	הוכיחו ש $A,B$ הפיכות ובטאו את $A^{-1}, B^{-1}$ באמצעות $B$.
	\begin{proof}
		נתחיל בלהראות ש $B$ הפיכה, אם ניקח את המשוואה השניה: \\
		$B^3 + AB^2 = 3I \Rightarrow (\frac{1}{3}(B+A))B^2 = I$
		נקבל לפי מסקנה 4.5.2 ש $B^2$ הפיכה ולכן גם $B$ הפיכה (כי $(B^2)^{-1}B^2 = I = ((B^2)^{-1}B)B = I$). \\
		עכשיו נראה ש $A$ הפיכה, ע"י להכפיל את המשוואה הראשונה ב $(B^2)^{-1}$, שזה מתאפשר כי הוכחנו ש $B^2$ הפיכה, ונקבל:
		$A = -2B$, ולפי משפט 3.8.4 מכפלה של מטריצה הפיכה בסקאלר השונה מאפס (2-) היא גם הפיכה, ולפיכך $A$ הפיכה. \\
		עכשיו נציב חזרה את $A$ ע"מ לבטא את ההופכי של המטריצות באמצעות $B$: \\
		נציב $A=-2B$ ב $(\frac{1}{3}(B+A))B^2 = I$ ונקבל:
		\[
			\frac{1}{3}(B-2B)B^2 =
			\frac{1}{3}B^{3} = I \ontop{(נכפיל ב $B^{-1}$)}\Rightarrow
			\boxed{B^{-1} = \frac{1}{3}B^2}
		\]
		ונחשב את ההופכי של $A$ באמצעות ההופכי של $B$ ובעזרת משפט 3.8.4:
		\[
			A = -2B \ontop{3.8.4}\Rightarrow
			A^{-1} = -2^{-1}B^{-1} =
			-0.5\frac{1}{3}B^2 \Rightarrow
			\boxed{A^{-1} = -\frac{1}{6}B^2}
		\]
	\end{proof}

	\pagebreak
	\section*{שאלה 6}
	\subsection*{סעיף א}
	חשבו את הדטרמיננטה הבאה:
	$D_n = \left|\begin{array}{ccccccc}
		1 & 1 & . & . & . & 1 & 1 \\
		1 & 1 & . & . & . & 1 & 2 \\
		. & . & . & . & 1 & 3 & 1 \\
		. & . & . & . & 4 & 1 & 1 \\
		. & . & . & . & . & . & . \\
		1 & 1 & n-1 & 1 & . & . & 1 \\
		1 & n & 1 & . & . & . & 1 \\
	\end{array}\right|$

	פתרון: \\
	נתחיל בלחסר מכל שורה במטריצה אחרי הראשונה את השורה הראשונה, פעולה זו לא משנה את הדטרמיננטה לפי משפט 4.3.6:
	\[
		\left|\begin{array}{ccccccc}
			1 & 1 & . & . & . & 1 & 1 \\
			1 & 1 & . & . & . & 1 & 2 \\
			. & . & . & . & 1 & 3 & 1 \\
			. & . & . & . & 4 & 1 & 1 \\
			. & . & . & . & . & . & . \\
			1 & 1 & n-1 & 1 & . & . & 1 \\
			1 & n & 1 & . & . & . & 1 \\
		\end{array}\right|
		\overset{R_i \rightarrow R_i - R_1}{\underset{\forall i, i = 2, i < n}\Longrightarrow}
		\left|\begin{array}{ccccccc}
			1 & 1 & . & . & . & 1 & 1 \\
			0 & 0 & . & . & . & 0 & 1 \\
			. & . & . & . & 0 & 2 & 0 \\
			. & . & . & . & 3 & 0 & 0 \\
			. & . & . & . & . & . & . \\
			0 & 0 & n-2 & 0 & . & . & 0 \\
			0 & n-1 & 0 & . & . & . & 0 \\
		\end{array}\right|
	\]
	אם נשתמש במשפט הפיתוח לפי עמודה ונבחר את עמודה אחד, נקבל: \\
	$\left| D_n \right| = \bm{\sum_{i=1}^n} (-1)^{i+1}a_{i,1} \left|D_{i,1}^M \right|$.
	ניתן לראות, כי לכל $i > 1$ מתקיים $a_{i,1} = 0$, ועל כן כל האיברים בסכום מתאפסים חוץ מהראשון, ואנו מקבלים:
	$\left| D_n \right| = \left| D_{i,1}^M \right|$. \\
	עכשיו במטריצה $D_{i,1}^M$ נחליף בין השורה האחרונה והראשונה, אחת לפני האחרונה והשניה וכן הלאה עד שנקבל את החצי התחתון של האלכסון בחצי העליון של המטריצה:
	\[
		D_{i,1}^M =
		\left|\begin{array}{ccccccc}
			0   & .   & . & . & 0 & 1 \\
			0   & .   & . & 0 & 2 & 0 \\
			.   & .   & . & 3 & 0 & 0 \\
			.   & .   & . & . & . & . \\
			0   & n-2 & 0 & . & . & 0 \\
			n-1 & 0 & .   & . & . & 0 \\
		\end{array}\right|
		\overset{R_i \leftrightarrow R_{n-i}}{\underset{\forall i, i = 1, i \leq \lfloor n/2 \rfloor}\Longrightarrow}
		c \cdot
		\left|\begin{array}{ccccccc}
			n-1 & 0 & .   & . & . & 0 \\
			0   & n-2 & 0 & . & . & 0 \\
			.   & .   & . & . & . & . \\
			.   & .   & . & 3 & 0 & 0 \\
			0   & .   & . & 0 & 2 & 0 \\
			0   & .   & . & . & 0 & 1 \\
		\end{array}\right|
	\]
	לפי משפט 4.3.2 אנחנו צריכים להכפיל את הדטרמיננטה ב $-1$ בכל פעם שמחליפים 2 שורות אחת בשניה,
	נסתכל על הלולאה שכתבנו ע"מ למצוא את $c$,
	במידה וכמות השורות היא זוגית, נחליף $n/2$ שורות, אך ההחלפה האחרונה תהיה של שורה בעצמה, ועל כן ברור שהמטריצה לא תשתנה, אז נספור זאת כ $n/2-1$ החלפות שורה. \\
	אך כאשר מספר השורות הוא אי זוגי, נחליף $\lfloor n / 2 \rfloor = (n-1)/2$ שורות.  \\
	לסיכום: $
	c =
	\begin{cases}
		(-1)^{n/2 - 1} & \text{ if } n-1 \text{ even is } \\
		(-1)^{(n-1)/2} & \text{ if } n-1 \text{ odd is } \\
	\end{cases}
	$. \\
	למטריצה שקיבלנו אנחנו יכולים לחשב דטרמיננטה בקלות בעזרת משפט 4.3.8 בגלל שהיא משולשית עילית, ובגלל שהאלכסון מכיל את כל המספרים שבין $1$ ל $n-1$ אזי: \\
	\boxed{\left| D_n \right| = \left| D_{i,1}^M \right| = c \cdot (n-1)!}


	\subsection*{סעיף ב}
	יהיו $a_1, \dots, a_n$ מספרים ממשיים. נגדיר:
	\[
	\Delta_1 = \left| \begin{array}{ccccc}
		a_1 & a_1^2 & \dots & a_1^{n-1} & 1 \\
		a_2 & a_2^2 & \dots & a_2^{n-1} & 1 \\
		\vdots & \vdots & \ddots & \vdots & \vdots \\
		a_{n-1} & a_{n-1}^2 & \dots & a_{n-1}^{n-1} & 1 \\
		a_n & a_n^2 & \dots & a_n^{n-1} & 1 \\
	\end{array}	\right|
	\text{ ו- }
	\Delta = \left| \begin{array}{ccccc}
		a_1 & a_1^2 & \dots & a_1^{n-1} & 1 + a_1^n \\
		a_2 & a_2^2 & \dots & a_2^{n-1} & 1 +  a_2^n \\
		\vdots & \vdots & \ddots & \vdots & \vdots \\
		a_{n-1} & a_{n-1}^2 & \dots & a_{n-1}^{n-1} & 1 + a_{n-1}^n \\
		a_n & a_n^2 & \dots & a_n^{n-1} & 1 + a_n^n \\
	\end{array}	\right|
	\]
	הוכיחו שאם $\Delta = 0$ ו- $\Delta_1 \neq 0$ אז המכפלה $\prod_{i=1}^{i<n}a_i$ שווה ל $1$ או $-1$.
	\begin{proof}
		עפ"י משפט 4.3.4 מתקבל:
		\[
			\Delta
			 =
			 \Delta_1
			 +
			 \left| \begin{array}{ccccc}
				a_1 & a_1^2 & \dots & a_1^{n-1} & a_1^n \\
				a_2 & a_2^2 & \dots & a_2^{n-1} & a_2^n \\
				\vdots & \vdots & \ddots & \vdots & \vdots \\
				a_{n-1} & a_{n-1}^2 & \dots & a_{n-1}^{n-1} & a_{n-1}^n \\
				a_n & a_n^2 & \dots & a_n^{n-1} & a_n^n \\
			\end{array}	\right|
		\]
		על המטריצה הימנית נעשה כמה פעולות: \\
		א. בעזרת משפט 4.3.3 נחלק כל שורה ב $a_i$ המתאים. \\
		ב. בעזרת משפט 4.3.2 נחליף בין העמודה הראשונה לשניה, השניה השלישית וכן הלאה.
		\begin{align*}
			 \left| \begin{array}{ccccc}
				a_1 & a_1^2 & \dots & a_1^{n-1} & a_1^n \\
				a_2 & a_2^2 & \dots & a_2^{n-1} & a_2^n \\
				\vdots & \vdots & \ddots & \vdots & \vdots \\
				a_{n-1} & a_{n-1}^2 & \dots & a_{n-1}^{n-1} & a_{n-1}^n \\
				a_n & a_n^2 & \dots & a_n^{n-1} & a_n^n \\
			\end{array}	\right|
			\overset{R_i \leftrightarrow \frac{1}{a_i}R_i}{\underset{\forall i, i = 1, i < n}\Longrightarrow}
			\prod_{i=1}^{i<n}a_i
			&\left| \begin{array}{ccccc}
				1 & a_1 & \dots & a_1^{n-2} & a_1^{n-1} \\
				1 & a_2 & \dots & a_2^{n-1} & a_2^{n-1} \\
				\vdots & \vdots & \ddots & \vdots & \vdots \\
				1 & a_{n-1} & \dots & a_{n-1}^{n-2} & a_{n-1}^{n-1} \\
				1 & a_n & \dots & a_n^{n-2} & a_n^{n-1} \\
			\end{array}	\right| \\
			\overset{C_{n-i+1} \leftrightarrow C_{n-i}}{\underset{\forall i, i = 1, i < n-1}\Longrightarrow}
			(-1)^{n-1}
			\prod_{i=1}^{i<n}a_i
			&\left| \begin{array}{ccccc}
				a_1 & \dots & a_1^{n-2} & a_1^{n-1} & 1 \\
				a_2 & \dots & a_2^{n-1} & a_2^{n-1} & 1 \\
				\vdots & \vdots & \ddots & \vdots & \vdots \\
				a_{n-1} & \dots & a_{n-1}^{n-2} & a_{n-1}^{n-1} & 1 \\
				a_n & \dots & a_n^{n-2} & a_n^{n-1} & 1 \\
			\end{array}	\right| =
			(-1)^{n-1}
			\prod_{i=1}^{i<n}a_i
			\Delta_1
		\end{align*}
		לפיכך מתקבל ש
		\[
			\Delta = 0 =
			\Delta_1 + (-1)^{n-1} \prod_{i=1}^{i<n}a_i \Delta_1  =
			\Delta_1(1 + (-1)^{n-1} \prod_{i=1}^{i<n}a_i) \Rightarrow
			(-1)^{n-1} \prod_{i=1}^{i<n}a_i = -1
		\]
		ניתן לראות שכאשר $n-1$ הוא אי זוגי אזי $(-1)^{n-1}=-1$ ולכן $\prod_{i=1}^{i<n}a_i = 1$
		וכאשר $n-1$ זוגי אזי $(-1)^{n-1}=1$ ולכן $\prod_{i=1}^{i<n}a_i = -1$. \\
		ובכך הוכחנו ש $\prod_{i=1}^{i<n}a_i$ בהכרח שווה ל $1$ או $-1$ כנדרש.
	\end{proof}

	\pagebreak
	\section*{שאלה 7}
	תהיינה $A,B \in \mathbf{M}_3(\RR)$ כך ש $A^t = -A$.

	\subsection*{סעיף א}
	הוכיחו שקיים פתרון לא טריוויאלי למערכת $(A^2B)\mathbf{x} = \mathbf{0}$
	\begin{proof}
		לפי הנתון ש $A^t = -A$ נסתכל על הדטרמיננטה של $A$: \\
		לפי משפט 4.3.1
		$\pip{A} = \pip{A^t}$ ולפי הנתון מתקבל ש
		$\pip{A} = \pip{-A}$ לפי משפט 4.3.3 נקבל ש
		$\pip{A} = (-1)^3 \cdot \pip{A} = -\pip{A}$
		וכידוע, המספר הממשי היחיד ששוה לנגדי שלו הוא $0$, ועל כן $\pip{A} = 0$. \\
		עכשיו נחשב את הדטרמיננטה של המטריצה $A^2B$: \\
		לפי משפט 4.5.1
		$\pip{A^2B} = \pip{A}^2 \cdot \pip{B}$
		ולפי מה שהוכחנו עכשיו נקבל ש:
		$\pip{A^2B} = 0^2 \cdot \pip{B} = 0$ \\
		ולפיכך לפי משפט 3.10.6 מתקבל ש $A^2B$ אינה הפיכה ושלמשוואה $(A^2B)\mathbf{x} = \mathbf{0}$ קיים פתרון לא טריוויאלי(סעיף ה).
	\end{proof}

	\subsection*{סעיף ב}
	נניח ש $B^t = B$ וששתי המטריצות $B,A \neq [0]$, האם המטריצה $(A+B)^2$ בהכרח סימטרית? \\

	פתרון: \\
	המטריצה $(A+B)^2$ לא בהכרח סימטרית נראה באמצעות דוגמא:
	\[
		A = \begin{pmatrix}
			0 & 0 & 1 \\
			0 & 0 & 1 \\
			-1 & -1 & 0 \\
		\end{pmatrix}
		\text{ ו- }
		B = B^t = \begin{pmatrix}
			1 & 0 & 1 \\
			0 & 1 & 0 \\
			1 & 0 & 1 \\
		\end{pmatrix}
		\]
		ניתן לראות ש $B = B^t$ כנדרש, ובנוסף $A^t = -A$:
		$A^t = -A = \begin{pmatrix}
				0 & 0 & -1 \\
				0 & 0 & -1 \\
				1 & 1 & 0 \\
			\end{pmatrix}$
		כנדרש. \\
		עכשיו נסתכל על $(A+B)^2$:
		\[
			(A+B)^2 = \begin{pmatrix}
				1 & 0 & 2 \\
				0 & 1 & 1 \\
				0 & -1 & 1\\
			\end{pmatrix}^2 =
			\begin{pmatrix}
				1 & -2 & 4 \\
				0 & 0 & 2 \\
				0 & -2 & 0 \\
			\end{pmatrix}
			\text{ ובנוסף }
			((A+B)^2)^t = \begin{pmatrix}
				1 & 0 & 0 \\
				-2 & 0 & -2 \\
				4 & 2 & 0 \\
			\end{pmatrix}
		\]
		ולפיכך הצגנו מטריצות $A,B$ אשר מקיימות $B^t = B \land A^t = -A$  אך $(A+B)^2$ אינו סמטרי.
\end{document}
