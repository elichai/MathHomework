% !TEX program = xelatex
\def\NN{\mathbb{N}}
\def\RR{\mathbb{R}}
\def\ZZ{\mathbb{Z}}
\def\QQ{\mathbb{Q}}
\def\PP{\mathcal{P}}
\def\SS{\mathcal{S}}
\def\DD{\mathcal{D}}
\def\sub{\setminus}
\def\bld{\mathbf}
\def\lmf{\lim_{n \to \infty}}
% Make the ref command use parenthesis
\let\oldref\ref
\renewcommand{\ref}[1]{(\oldref{#1})}

\newcommand{\ontop}[1]{\overset{\text{#1}}}
\newcommand{\ang}[1]{\langle #1 \rangle}
\newcommand{\pip}[1]{\left| #1 \right|}



% style
\newcommand{\bm}[1]{\displaystyle{#1}}
\def\nl{$ $ \newline}

\ExplSyntaxOn

\NewDocumentCommand{\getenv}{om}
{
  \sys_get_shell:nnN{ kpsewhich ~ --var-value ~ #2 }{}#1
}

\ExplSyntaxOff

%environments

\documentclass{article}
\usepackage[]{amsthm} %lets us use \begin{proof}
\usepackage{amsmath}
\usepackage{mathtools}
\usepackage{enumerate}
\usepackage{xparse}
\usepackage[makeroom]{cancel}
\usepackage[]{amssymb} %gives us the chA \mathcal{R} Acter \varnothing
\usepackage{polyglossia}
% \usepackage[frak=mma]{mathalfa}
\setdefaultlanguage{hebrew}
\setotherlanguage{english}
\usepackage{fontspec}
%\setmainfont{Frank Ruehl CLM}
\setmainfont{David CLM}
\setmonofont{Miriam Mono CLM}
\setsansfont{Simple CLM}
\DeclarePairedDelimiter\set\{\}
% Use the following if you only want to change the font for Hebrew
%\newfontfamily\hebrewfont[Script=Hebrew]{David CLM}
%\newfontfamily\hebrewfonttt[Script=Hebrew]{Miriam Mono CLM}
%\newfontfamily\hebrewfontsf[Script=Hebrew]{Simple CLM}
\getenv[\ID]{ID}
\newtheorem{lemma}{טענת עזר}
\newtheorem{remark}{הערה}

\title{בדידה - ממ"ן 15}
\author{אליחי טורקל \ID}
\date\today

%\clearpage %Gives us a page break before the next section. Optional.
%\selectlanguage{english}
	%Section and subsection automatically number unless you put the asterisk next to them.

\begin{document}
	\maketitle %This command prints the title based on information entered above

	\section*{שאלה 1}
	\subsection*{סעיף א}


	\subsection*{סעיף ב}
	נתחיל בלראות כמה זוגות שונים של מספרים ישנם מקבוצה של 20 מספרים ללא חשיבות לסדר: ${20 \choose 2} = 190$. \\
	עכשיו נבדוק לכמה מהזוגות האלה יש את אותו ההפרש. נסמן ב $d$ את מספר הקבוצות בנות 2 איברים שיש להם את אותו ההפרש, ומיכוון שכל איבר ב $A$ נמצא ב $1 \leq k \in A \leq 63$ אזי ההפרש המקסימלי הוא $63-1=62$.
	ולכן נחלק את מספר הזוגות בכמות ההפרשים הקיימים ונקבל $d = \frac{190}{62} = 3\frac{2}{31}$ מכיוון שהחלוקה גדולה מ 3 בוודאי קיים לפחות $d$ שיש 4 זוגות מספרים שונים שהפרשם שווה לאותו ה $d$ (מתוך עיקרון שובך היונים).


	\section*{שאלה 2}
	\subsection*{סעיף א}
	\begin{enumerate}
		\item $a_1 = 2$ מכיוון שמייצג רק ספרה אחת אזי האופציות הם $\set{1,2,3}$ ו ב $\set{1,3}$ הספרה $2$ מופיעה אפס פעמים שזה מספר זוגי.
		\item $b_1 = 1$ מכיוון שמייצג רק ספרה אחת אזי האופציות הם $\set{1,2,3}$ ורק ב $2$ ישנם מספר אי זוגי של הספרה $2$.
		\item $a_2 = 5$ מכיוון שלמספרים עם 2 ספרות ישנם 3 אופציות לספרה הראשונה ו 3 אופציות לספרה השניה אזי 9 אופציות, מתוכן $11, 13, 31, 33, 22$ מכילים את הספרה $2$ מספר זוגי של פעמים.
		\item $b_2 = 4$ גם כאן יש 9 אופציות למספרים בעלי 2 ספרות ומתוכם רק $12, 21, 32, 23$ מכילים את הספרה $2$ מספר אי זוגי של פעמים
		\item $a_3 = 14$ כאן ישנם $3^3 = 27$ אופציות, מתוכם 2 אופציות המכילות את $2$ פעמיים מהצורה $2\_2$, עוד 2 אופציות מהצורה $22\_$ ועוד 2 אופציות מהצורה $\_22$ ובנוסף יש אופציות המכילות את $2$ אפס פעמים מהצורה: $\_\_\_$ שישנם $2^3 = 8$ כאלה, בסה"כ קיבלנו: $2 + 2 + 2 + 8 = 14$.
		\item $b_3 = 13$ גם כאן ישנם $27$ אופציות, ומתוכן $4$ אופציות המכילות את $2$ פעם אחת מהצורה $2\_\_$ עוד $4$ מהצורה $\_2\_$ ועוד $4$ מהצורה $\_\_2$ ובנוסף עוד אחת שמכילה את $2$ שלוש פעמים: $222$ ובסה"כ: $4 + 4 + 4 + 1 = 13$.
	\end{enumerate}


	\subsection*{סעיף ב}
	את הבסיס סיפקנו, ולכן נניח ש $n > 1$.
	\begin{enumerate}
		\item נתחיל ב $a_n$ ונפצל גם אותו ל 3 אופציות:
		\begin{enumerate}
			\item אם הספרה הראשונה היא $2$ אזי אחריה צריכים להגיע $n-1$ ספרות המכילות מספר אי זוגי של $2$ שזה בעצם $b_{n-1}$.
			\item אם הספרה הראשונה היא אחת מ $\set{1,3}$ אזי אחריה צריכים להגיע $n-1$ ספרות המכילות מספר זוגי של $2$ שזה בעצם $a_{n-1}$.
		\end{enumerate}
		אם נחבר את שלשת האופציות נקבל: $a_n = b_{n-1} + 2a_{n-1}$.

		\item נחשב את $b_n$ וגם אותו נפצל ל 3 אופציות:
		\begin{enumerate}
			\item אם הספרה הראשונה היא $2$ אזי אחריה צריכים להגיע $n-1$ ספרות המכילות מספר זוגי של $2$ שזה בעצם $a_{n-1}$.
			\item אם הספרה הראשונה היא אחת מ $\set{1,3}$ אזי אחריה צריכים להגיע $n-1$ ספרות המכילות מספר אי זוגי של $2$ שזה בעצם $b_{n-1}$
		\end{enumerate}
		אם נחבר את שלשת האופציות נקבל: $b_n = a_{n-1} + 2b_{n-1}$.
	\end{enumerate}

	\subsection*{סעיף ג}
	\begin{enumerate}
		\item נעביר אגפים במשוואה הראשונה ונקבל: $b_{n-1} = a_n - 2a_{n-1}$, נציב $n+1$ במקום ה $n$ ונקבל $b_n = a_{n+1} - 2a_n$ נשווה את זה למשוואה השניה ונקבל \\
		$a_{n-1} + 2b_{n-1} = a_{n+1} - 2a_n$
		 נציב בזה את השוויון של $b_{n-1}$ ונקבל \\
		 $a_{n-1} + 2(a_n - 2a_{n-1}) = a_{n+1} - 2a_n$ נפתח ונצמצם ונקבל $a_{n+1} = 4a_n - 3a_{n-1}$ ונציב $n-1$ כדי לקבל \\
		 $\boxed{a_n = 4a_{n-1} - 3a_{n-2}}$

		 \item נעביר אגפים במשוואה השניה ונקבל $a_{n-1} = b_n - 2b_{n-1}$ ונציב $n+1$ ונקבל $a_n = b_{n+1} - 2b_n$ נשווה את זה למשוואה הראשונה ונציב את המשוואה הקודמת ב $b_{n-1}$
		 ונקבל $b_{n+1} - 2b_n = b_{n-1} + 2(b_n - 2 b_{n-1})$ נפתח ונצמצם: $b_{n+1} = 4b_n - 3b_{n-1}$ ונציב $n-1$ בשביל $\boxed{b_n = 4b_{n-1} - 3b_{n-2}}$.
	\end{enumerate}

	\subsection*{סעיף ד}
	\begin{enumerate}
		\item 	אנו מקבלים לשתי הסדרות את אותה משוואה אופיינית: $x^2 - 4x + 3 = 0$ נציב בנוסחאת השורשים ונקבל: $x_1 = 1, x_2 = 3$.
		נציב בתבנית ונקבל: \\
		$a_n = A1^n + B3^n = A + B3^n$ נציב את 2 האיברים הראשונים:
		\begin{align}
			a_1 = 2 = A + 3B \Rightarrow A = 2 - 3B \\
			a_2 = 5 = A + 3^2B \Rightarrow A = 5 - 3^2B
		\end{align}
		נשווה בין המשוואות $2 - 3B = 5 - 5^2B$ מכאן $B(3^2 - 3) = 3$ ולכן $B = \frac{1}{2}$ נציב חזרה במשוואה הראשונה ונקבל $A = \frac{1}{2}$
		ולכן קיבלנו: $\boxed{a_n = 0.5 + 0.5 \cdot 3^n}$ \\
		נבדוק את $a_3 = 0.5 + 0.5 \cdot 3^3 = 14$ כמו שחישבנו מקודם.
		\item
		נעשה את אותה הפעולה ל $b_n$ ונציב את שני האיברים הראשונים:
		\begin{align}
			b_1 = 1 = A + 3B \Rightarrow A = 1 - 3B \\
			b_2 = 4 = A + 3^2B \Rightarrow A = 4 - 3^2B
		\end{align}
		נשווה בין המשוואות ונקבל: $1 - 3B = 4 - 3^2B$ מכאן $B = \frac{1}{2}$ ונציב חזרה במשוואה הראשונה ונקבל $A = - \frac{1}{2}$.
		ולכן קיבלנו: $\boxed{b_n = -0.5 + 0.5 \cdot 3^n}$ \\
		נבדוק את $b_3 = -0.5 + 0.5 \cdot 3^3 = 13$ כמו שחישבנו מקודם.
	\end{enumerate}

	\subsection*{סעיף ה}
	$A$ מכיל 3 ספרות, ולכן מספר האיברים של $A$ במספר באורך $n$ ספרות הוא $3^n$. \\
	נחבר את המשוואות: $a_n + b_n = \cancel{0.5} + 0.5 \cdot 3^n \cancel{-0.5} + 0.5 \cdot 3^n = (0.5 + 0.5)3^n = 3^n$.


	\section*{שאלה 3}
	\subsection*{סעיף א}
	נפתח את המשוואה ונקבל $f(x) = 1 + 7xf(x) + 8x^2f(x)$ נציב את $f(x)$ ונקבל \\
	$\sum_{i=0}a_ix^i = 1 + 7x(\sum_{i=0}a_ix^i) + 8x^2(\sum_{i=0}a_ix^i)$ \\
	נקבץ מקדמים ונקבל $1 + 7a_0x + \sum_{i=0}(7a_i + 8a_{i+1})x^{i+1} = \sum_{i=0}a_ix^i$ \\
	נבנה שוויון של מקדמי $x$ משני הצדדים $a_n = 7a_{n-1} + 8a_{n-2}$ וגם $a_0 = 1$, $a_1 = 7a_0 = 7$

	\subsection*{סעיף ב}
	נציב במשוואה האופיינית ונקבל $x^2 - 7x - 8 = 0$ נציב בנוסחאת השורשים ונקבל $x_1 = -1$, $x_2 = 8$
	נציב בתבנית ונקבל $a_n = A \cdot (-1)^n + B \cdot 8^n$ נציב $a_0 = 1$ ו $a_1 = 7$ ונקבל את המשוואות:
	\begin{align}
		a_0 = 1 = A \cdot (-1)^0 + B \cdot 8^0 = A + B \Rightarrow A = 1 - B \\
		a_1 = 7 = A \cdot (-1)^1 + B \cdot 8^1 = 8B - A \Rightarrow A = 8B - 7 \\
	\end{align}
	נשווה את המשוואות $1 - B = 8B - 7$ ונקבל $B=\frac{8}{9}$ נציב חזרה במשוואה הראשונה ונקבל $A=\frac{1}{9}$. \\
	נציב בתבנית ונקבל: $a_n = \frac{1}{9} \cdot (-1)^n + \frac{8}{9} \cdot 8^n$

	\pagebreak
	\section*{שאלה 4}
	נפרק לגורמים:
	\[
	\frac{1}{(1-x^2-x^3+x^5)^n} = \frac{1}{(1(1-x^2)-x^3(1-x^2))^n} = \frac{1}{(1-x^3)^n(1-x^2)^n}
	\]
	נציב במשוואת טור אינסופי ונחליף את $x$ עם $x^2$ ונקבל $\frac{1}{(1-x^3)^n} = \sum_{i=0}^\infty D(n, i)x^{2i}$ \\
	אותו הדבר רק נחליף $x$ עם $x^2$ ונקבל $\frac{1}{(1-x^2)^n} = \sum_{j=0}^\infty D(n, i)x^{2j}$ \\
	ומכפלת השברים שווה למכפלת הטורים:
	\[
		\frac{1}{(1-x^3)^n(1-x^2)^n} = \sum_{i=0}^\infty (D(n, i)x^{3i}) \cdot \sum_{j=0}^\infty (D(n, j)x^{2j})
	\]
	מכאן נקבל שהמקדם של $x^{13}$ הינו $D(n, i)D(n, j)x^{3i + 2j}$ והמשוואה $13 = 3i + 2j$ מתקיימת רק בעבור: $i=1, j = 5$, $i=3, j = 2$. \\
	נציב את 2 האופציות ונסכם (כי הוצאנו את המשוואה מתוך סכום) ונקבל
	\[ D(n, 1)D(n, 5) + D(n, 3)D(n, 2) = {n \choose 1}{n + 4 \choose 5} + {n + 2 \choose 3}{n + 1 \choose 2} \]

	\subsection*{סעיף ב}
	מכיוון שערכי $x$ חייבים להיות זוגיים(כולל אפס) אז נקבל את הפונקציה היוצרת הבאה: $(1 + x^2 + x^4 + x^6 + \cdots)^n = \frac{1}{(1-x^2)^n}$ \\
	ומכיוון שערכי $y$ הם כפולות של 3 (גם כולל אפס) נקבל את הפונקציה היוצרת הבאה: $(1 + x^3 + x^6 + x^9 + \cdots)^n = \frac{1}{(1-x^3)^n}$ \\
	ניתן לראות שאלו אותן שתי פונקציות שעליהן הסתכלנו בסעיף הקודם, ובעצם זה אותו פתרון שחיפשנו בסעיף הקודם, ובעצם מספר הפתרונות הוא המקדם של $x^{13}$ ולכן הפתרון גם כאן הוא:
	\[ D(n, 1)D(n, 5) + D(n, 3)D(n, 2) = {n \choose 1}{n + 4 \choose 5} + {n + 2 \choose 3}{n + 1 \choose 2} \]

	\pagebreak
	\section*{שאלה 5}
	\subsection*{סעיף א}
	ל-2 הנעלמים הראשונים אין הגבלות ולכן הפונקציה היוצרת שלהם היא: \\ $(1+x+x^2 + \cdots)^2 = \frac{1}{(1-x)^2}$.
	חמשת הנעלמים הבאים חייבים להיות גדולים מאחד ולכן הפונקציה היוצרת שלהן היא: $(1+x+x^2)^5$ ולפי הנתון זה שווה ל $\frac{(1-x^3)^5}{(1-x)^5}$. \\
	שלשת הנעלמים האחרונים בעצם אומרים שזה כפולות של 3 ולכן נקבל את הפונקציה היוצרת: $(1 + x^3 + x^6 + \cdots)^3 = \frac{1}{(1-x^3)^3}$. \\
	נכפיל את שלשת הפונקציות ונקבל:
	\[
		f(x) = \frac{1}{(1-x)^2}\frac{(1-x^3)^5}{(1-x)^5}\frac{1}{(1-x^3)^3} = \frac{(1-x^3)^2}{(1-x)^7}
	\]

	\subsection*{סעיף ב}
	ע"מ למצוא את מספר הפתרונות נחפש את המקדם של $x^n$ בפונקציה היוצרת $f(x)$,
	אנו יודעים כי: $\frac{1}{(1-x)^7} = \sum_{k=0}^\infty D(7, k)x^k$ ונקבל:
	\[
		(1-x^3)^2\sum_{k=0}^\infty D(7, k)x^k = (1 - 2x^3 + x^6)\sum_{k=0}^\infty {k + 6 \choose k}x^k
	\]
	ולכן המקדם של $x^n$ הוא: $1{n + 6 \choose n} - 2{n - 3 \choose n - 3} + 1{n \choose n - 6}$

	\end{document}
