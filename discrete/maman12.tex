% !TEX program = xelatex
\def\NN{\mathbb{N}}
\def\RR{\mathbb{R}}
\def\PP{\mathcal{P}}
\def\sub{\setminus}

% Make the ref command use parenthesis
\let\oldref\ref
\renewcommand{\ref}[1]{(\oldref{#1})}



% style
\newcommand{\bm}[1]{\displaystyle{#1}}
\def\nl{$ $ \newline}

\ExplSyntaxOn

\NewDocumentCommand{\getenv}{om}
{
  \sys_get_shell:nnN{ kpsewhich ~ --var-value ~ #2 }{}#1
}

\ExplSyntaxOff

%environments

\documentclass{article}
\usepackage[]{amsthm} %lets us use \begin{proof}
\usepackage{amsmath}
\usepackage{mathtools}
\usepackage{enumerate}
\usepackage{xparse}
\usepackage[makeroom]{cancel}
\usepackage[]{amssymb} %gives us the chA \mathcal{R} Acter \varnothing
\usepackage{polyglossia}
% \usepackage[frak=mma]{mathalfa}
\setdefaultlanguage{hebrew}
\setotherlanguage{english}
\usepackage{fontspec}
%\setmainfont{Frank Ruehl CLM}
\setmainfont{David CLM}
\setmonofont{Miriam Mono CLM}
\setsansfont{Simple CLM}
\DeclarePairedDelimiter\set\{\}
% Use the following if you only want to change the font for Hebrew
%\newfontfamily\hebrewfont[Script=Hebrew]{David CLM}
%\newfontfamily\hebrewfonttt[Script=Hebrew]{Miriam Mono CLM}
%\newfontfamily\hebrewfontsf[Script=Hebrew]{Simple CLM}
\getenv[\ID]{ID}
\newtheorem{lemma}{טענת עזר}

\title{בדידה - ממ"ן 11}
\author{אליחי טורקל \ID}
\date\today

%\clearpage %Gives us a page break before the next section. Optional.
%\selectlanguage{english}
	%Section and subsection automatically number unless you put the asterisk next to them.

\begin{document}
	\maketitle %This command prints the title based on information entered above

	\section*{שאלה 1}
	על הקבוצה $\PP(\set{1,2,3,4})$ נתונים שני יחסים $\mathcal{R}, \mathcal{S}$ המוגדרים כך:
	 $A,B \in \PP(\set{1,2,3,4})$ $A \mathcal{R} B \iff A \cup \set{1,2} = B \cup \set{1,2}$
	ו $A \mathcal{S} B \iff A \cup \set{1,2} \subset B \cup \set{1,2}$ \\
	הערה: כל קבוצה שלא הגדרנו מה היא, מוגדרת כאיבר ב $\PP(\set{1,2,3,4})$
	\subsection*{סעיף א}
	נראה ש $\mathcal{R}$ הוא יחס שקילות:
\begin{enumerate}
	\item רפלקסיביות:
	$A \mathcal{R} A \iff A \cup \set{1,2} = A \cup \set{1,2}$,
	הצד הימני והשמאלי של השוויון זהים, וע"כ תמיד מתקיים עפ"י הגדרת יחס השוויון כרפלקסיבי,
	ולפיכך $R$ רפלקסיבי, ומקיים $A \mathcal{R} A$.
	\item טרנזיטיביות:
	אם מתקיים $A \cup \set{1,2} = B \cup \set{1,2}$
	וגם $B \cup \set{1,2} = C \cup \set{1,2}$
	אז גם פה, עפ"י הגדרת יחס השוויון כיחס טרנזיטיבי נקבל $A \cup \set{1,2} = C \cup \set{1,2}$
ולפיכך מתקבל ש $A \mathcal{R} B \land BRC \to ARC$ ו $R$ יחס טרנזיטיבי.
	\item סימטריה:
	יחס השוויון הוא סימטרי, וע"כ:\\
	$A \cup \set{1,2} = B \cup \set{1,2} \iff
	B \cup \set{1,2} = A \cup \set{1,2}$
	ולכן לכל $A \mathcal{R} B$ מתקיים $B \mathcal{R} A$ ולפיכך $R$ סימטרי.
\end{enumerate}
על מנת למצוא את מחלקות השקילות נסתכל על הקבוצה ועל היחס,
הקבוצות שעליהן היחס פועל, מוכלות ב $\PP(\set{1,2,3,4})$,
והיחס מתקיים במידה ואיחוד של שני תתי קבוצות ביחד עם $\set{1,2}$ שווה. \\
נגדיר $C = A \cup \set{1,2}$, עפ"י הגדרת האיחוד תמיד יתקיים $1,2 \in C$
ועל כן, $C$ ישתנה א"םם $A$ מכיל משהו שהוא לא $1,2$, ובגלל ש $A \in \PP(\set{1,2,3,4})$
אז $A$ עלול להכיל את $3,4$, ואיזה מהם ש $A$ יכיל גם $B$ יצטרך להכיל ע"מ לקיים את יחס השקילות.
ועל כן נחשב את מחלקות השקילות כך שיהיו זרות אחת לשניה, לא ריקות ושאיחודן יהיה קבוצת החזקה:
\begin{enumerate}
	\item $3,4 \in A,B \to
	S_{\set{1,2,3,4}}=\set{\set{3,4}, \set{1,3,4}, \set{2,3,4}, \set{1,2,3,4}}$
	\item $3 \in A,B \land 4 \not\in A,B \to
	S_{\set{1,2,3}}=\set{\set{3}, \set{1,3}, \set{2,3}, \set{1,2}}$
	\item $3 \not\in A,B \land 4 \in A,B \to
	S_{\set{1,2,4}}=\set{\set{4}, \set{1,4}, \set{2,4}, \set{1,2,4}}$
	\item $3,4 \not\in A,B \to
	S_{\set{1,2}}=\set{\emptyset, \set{1}, \set{2}, \set{1,2}}$
\end{enumerate}
	\pagebreak
	\subsection*{סעיף ב}
נראה ש $\mathcal{S}$ הוא יחס סדר חלקי:
\begin{enumerate}
	\item אנטי רפלקסיביות:
	$A \cup \set{1,2} \subset A \cup \set{1,2}$
	ניתן לראות כי שני הצדדים שווים, וע"כ לפי הגדרת תת קבוצה ממש, קבוצה אינה חלקית ממש לעצמה,
	ולפיכך $\mathcal{S}$ אנטי-רפלקסיבי, ולעולם לא מתקיים $A\mathcal{S}A$.
	\item טרנזיטיביות:
	נגדיר $A,B,C \in \PP(\set{1,2,3,4})$,
	אם מתקיים $A \cup \set{1,2} \in B \cup \set{1,2}$
	וגם $B \cup \set{1,2} \in C \cup \set{1,2}$ אזי:
\[
	\forall x \in A \cup \set{1,2} \ontop{(1)}\Rightarrow
	x \in B \cup \set{1,2} \ontop{(2)}\Rightarrow
	x \in C \cup \set{1,2}
\]
ולפיכך מתקיים ש $A \cup \set{1,2} \in C \cup \set{1,2}$, ולכן
$A \mathcal{S} B \land B \mathcal{S} C \to A \mathcal{S} C$, ו $\mathcal{S}$ הוא טרנזיטיבי. \\
(1) -
עפ"י הגדרת ההכלה ממש וש $A \cup \set{1,2} \in B \cup \set{1,2}$ \\
(2) עפ"י הגדרת ההכלה ממש וש $B \cup \set{1,2} \in C \cup \set{1,2}$ \\
	\item $\mathcal{S}$ אינו סדר מלא, כי אינו משווה, נוכיח ע"י דוגמא נגדית:
	נסתכל על הקבוצות $A = \set{3}, B = \set{4}$, נראה ש:
	\[
		A \cup \set{1,2} = \set{1,2,3},  B \cup \set{1,2} = \set{1,2,4} \to \set{1,2,3} \not \subset \set{1,2,4} \land \set{1,2,4} \not \subset \set{1,2,3}
	\]
	\item האיברים המקסימלים של הקבוצה הסדורה $\langle \PP(\set{1,2,3,4}), \mathcal{S} \rangle$:
בהינתן קבוצה $A \in \PP(\set{1,2,3,4})$ המקיימת $A \cup \set{1,2} = \set{1,2,3,4}$
לא קיימת אף קבוצה $X \in \PP(\set{1,2,3,4})$ כך ש $\set{1,2,3,4} \subset X \cup{1,2}$
ולפיכך, כל קבוצה $A$ המקיימת את התנאי היא קבוצה מקסימלית.
כל תת קבוצה שלא מכילה את $3,4$  תהיה בהכרח תת קבוצה של $\set{1,2,3,4}$ ועל כן לא תהיה מקסימלית. \\
הקבוצות המקסימליות לגבי היחס $\set{3,4}, \set{1,3,4}, \set{2,3,4}, \set{1,2,3,4} :\mathcal{S}$.
\end{enumerate}

	\section*{שאלה 2}

	\subsection*{סעיף א}
	על הקבוצה $A = \RR \times \RR$ מגדירים יחס $\mathcal{R}$ כך ש
	$\forall \langle x_1, y_1 \rangle, \langle x_2, y_2 \rangle \in A$ מתקיים
	$\langle x_1, y_1 \rangle \mathcal{R} \langle x_2, y_2 \rangle \iff (x_1 + y_2 = x_2 + y_2 = 1 \lor (x_1 + y_1 - 1)(x_2 + y_2  - 1) > 0)$ \\
	הוכיחו ש $\mathcal{R}$ יחס שקילות ומצאו את מספר מחלקות השקילות שלו, ותארו אותן במישור. \\
	פתרון: \\
	נתחיל בלהראות ש $\mathcal{R}$ הינו יחס שקילות:
	\begin{enumerate}
		\item רפלקסיביות -
		יהי $\langle x, y \rangle \in A$ אחת משתי האופציות הבאות מתקיימות:
		\begin{enumerate}
			\item  $x + y = 1 \Rightarrow \Big( \langle x, y \rangle \mathcal{R} \langle x, y \rangle \iff x + y = x + y = 1 \Big)$
			\item $x + y \neq 1 \Rightarrow x + y - 1 \neq 0 \ontop{(a)}\Rightarrow (x+y-1)^2 > 0$ \\
			(a) $\forall x \in \RR,  x^2 \geq 0$
		\end{enumerate}
		ועל כן $\mathcal{R}$ הוא רפלקסיבי.
		\item  טרנזיטיביות -
		יהיו $\langle x_1, y_1 \rangle, \langle x_2, y_2 \rangle, \langle x_3, y_3 \rangle \in A$
		כך ש $\langle x_1, y_1 \rangle \mathcal{R} \langle x_2, y_2 \rangle$ וגם $\langle x_2, y_2 \rangle \mathcal{R} \langle x_3, y_3 \rangle$
		אחת משתי האופציות מתקיימות עפ"י הגדרת היחס:
		\begin{enumerate}
			\item \begin{align*}
			&x_1 + y_1 = 1 \ontop{(a)}\Rightarrow
			x_2 + y_2 = 1 \ontop{(b)}\Rightarrow
			x_3 + y_3 = 1 \Rightarrow \\
			&x_1+y_1 = x_3 + y_3 = 1 \Rightarrow
			\langle x_1, y_1 \rangle \mathcal{R} \langle x_3, y_3 \rangle
			\end{align*}
			\item \begin{align*}
				&x_1 + y_1 \neq 1 \ontop{(a)}\Rightarrow
				(x_1 + y_1 - 1)(x_2 + y_2 - 1) > 0 \ontop{(c)}\Rightarrow
				x_2 + y_2 - 1 \neq 0 \Rightarrow \\
				&x_2 + y_2 \neq 1 \ontop{(b)}\Rightarrow
				(x_2 + y_2 - 1)(x_3 + y_3 - 1) > 0 \ontop{(d)}\Rightarrow \\
				&(x_1 + y_1 - 1)(x_2 + y_2 - 1)(x_3 + y_3 - 1) > 0 \ontop{(e)}\Rightarrow
				(x_1 + y_1 - 1)(x_3 + y_3 - 1) > 0
			\end{align*}
		\end{enumerate}
		לסיכום, הראינו שהיחס $\mathcal{R}$ טרנזיטיבי בשתי הדרכים שבהן הוא יכול להתקיים. \\
		(a) - עפ"י הגדרת היחס $\mathcal{R}$ והנתון $\langle x_1, y_1 \rangle \mathcal{R} \langle x_2, y_2 \rangle$ \\
		(b) - עפ"י הגדרת היחס $\mathcal{R}$ והנתון $\langle x_2, y_2 \rangle \mathcal{R} \langle x_3, y_3 \rangle$ \\
		(c) $x \cdot y > 0 \Rightarrow x \neq 0 \land y \neq 0$ \\
		(d) - מותר להכפיל את שני הצדדים של אי שוויון בביטוי שגדול מאפס\\
		(e) - נחלק את שני הצדדים ב $x_2 + y_2 - 1$, מותר כי הראינו כבר שהביטוי שונה וגדול מאפס

	\item סימטריה -
	יהיו $\langle x_1, y_1 \rangle, \langle x_2, y_2 \rangle \in A,
	\langle x_1, y_1 \rangle \mathcal{R} \langle x_2, y_2 \rangle$
	גם כאן נסתכל על שתי הדרכים שבהן היחס מתקיים:
	\begin{enumerate}
		\item לפי הגדרת יחס השוויון: $x_1 + y_1 = x_2 + y_2 = 1 \iff x_2 + y_2 = x_1 + y_1 = 1$
		\item לפי אקסיומת החילוף בשדה הממשיים:  \\
		$(x_1 + y_1 - 1)(x_2 + y_2 - 1) > 0 \iff (x_2 + y_2 - 1)(x_1 + y_1 - 1) > 0$
	\end{enumerate}
	ועל כן, היחס $\mathcal{R}$ הינו יחס סימטרי.
\end{enumerate}
הראנו ש $\mathcal{R}$ הינו יחס רפלקסיבי, טרנזיטיבי, וסימטרי, כנדרש בהגדרת יחס השקילות.
נמצא את מחלקות השקילות: \\
ליחס $\mathcal{R}$ ישנן 3 מחלקות שקילות, אחת של כל הזוגות המקיימות את האופציה הראשונה $x + y = 1$,
ובאופציה השניה, בגלל שמדובר במכפלת 2 ביטויים ככה שיהיו גדולים מאפס אז זה מתחלק לזוגות הביטויים הגדולים מאפס והקטנים מאפס, ניתן לראות שהמחלקות זרות בגלל שביטוי שלילי כפול ביטוי חיובי יהיה קטן מאפס,
וכל זוג שמקיים $x + y = 1$  לעולם לא יקיים את האופציה השניה מכיוון שיקיים $x + y - 1 = 0$  ולכן יאפס את הביטוי (הטענה עובדת גם בכיוון הנגדי).
בנוסף, קל לראות  שאיחודן יתן לנו את הקבוצה כולה.
נתאר את הקבוצות:
\begin{enumerate}
	\item $S_{\langle x, 1-x \rangle} = \set{\langle x, y \rangle \in A | x + y = 1}$
	 מחלקת שקילות זו מייצגת את כלל הנקודות במישור הנמצאות על הישר $y = -x + 1$ למשל $\langle 1, 0 \rangle, \langle 0, 1 \rangle$.
	 \item $S_{\langle x, >(1-x) \rangle} = \set{\langle x, y \rangle \in A| x + y > 1}$
	 מחלקת שקילות זו מייצגת את כלל הנקודות במישור הנמצאות מעל הישר במחלקה הראשונה, למשל $\langle 2, 2 \rangle$
	 \item $S_{\langle x, <(1-x) \rangle} = \set{\langle x, y \rangle \in A| x + y < 1}$
	 מחלקת שקילות זו מייצגת את כלל הנקודות במישור הנמצאות מתחת הישר במחלקה הראשונה, למשל $\langle 0, 0 \rangle$
\end{enumerate}
ניתן לראות גם כי שלשת מחלקות אלה אכן מגדירות חלוקה של $A$ בגלל שאחת מגדירה את כל הנקודות שבקו ישר, אחת מגדירה את כל הנקודות שמעליו, והשלישית מגדירה את כל הנקודות שמתחתיו.
ועל כן הן זרות זו לזו, ואיחודן הוא המישור כולו.
\subsection*{סעיף ב}

\end{document}
