% !TEX program = xelatex
\def\NN{\mathbb{N}}
\def\RR{\mathbb{R}}
\def\PP{\mathcal{P}}
\def\sub{\setminus}

% Make the ref command use parenthesis
\let\oldref\ref
\renewcommand{\ref}[1]{(\oldref{#1})}



% style
\newcommand{\bm}[1]{\displaystyle{#1}}
\def\nl{$ $ \newline}

\ExplSyntaxOn

\NewDocumentCommand{\getenv}{om}
{
  \sys_get_shell:nnN{ kpsewhich ~ --var-value ~ #2 }{}#1
}

\ExplSyntaxOff

%environments

\documentclass{article}
\usepackage[]{amsthm} %lets us use \begin{proof}
\usepackage{amsmath}
\usepackage{mathtools}
\usepackage{enumerate}
\usepackage{xparse}
\usepackage[makeroom]{cancel}
\usepackage[]{amssymb} %gives us the character \varnothing
\usepackage{polyglossia}
\usepackage{fontspec}
% \usepackage[frak=mma]{mathalfa}
\setdefaultlanguage{hebrew}
\setotherlanguage{english}
%\setmainfont{Frank Ruehl CLM}
\setmainfont{David CLM}
\setmonofont{Miriam Mono CLM}
\setsansfont{Simple CLM}
\newfontface\niceee{Brush Script MT}
\DeclarePairedDelimiter\set\{\}
% Use the following if you only want to change the font for Hebrew
%\newfontfamily\hebrewfont[Script=Hebrew]{David CLM}
%\newfontfamily\hebrewfonttt[Script=Hebrew]{Miriam Mono CLM}
%\newfontfamily\hebrewfontsf[Script=Hebrew]{Simple CLM}
\getenv[\ID]{ID}
\newtheorem{lemma}{טענת עזר}
\title{אלגברה לינארית - ממ"ן 13}
\author{אליחי טורקל \ID}
\date\today

\DeclareFontFamily{OT1}{cmrx}{}
\DeclareFontShape{OT1}{cmrx}{m}{n}{<->cmr10}{}
\let\saveLongrightarrow\Longrightarrow
\makeatletter
\renewcommand*{\Longrightarrow}{%
    \mathrel{\rlap{\fontfamily{cmrx}\fontencoding{OT1}\selectfont=}%
    \hphantom{\saveLongrightarrow}%
    \llap{$\m@th\Rightarrow$}}}
\makeatother


%\clearpage %Gives us a page break before the next section. Optional.
%\selectlanguage{english}
	%Section and subsection automatically number unless you put the asterisk next to them.


\begin{document}
	\maketitle %This command prints the title based on information entered above

	\section*{שאלה 1}
	\subsection*{סעיף א}
	מצאו את כל הפתרונות של המשוואה $z^3+3i\overline{z} = 0$. \\
	\textbf{פתרון:}
	פתרון ראשון הינו קל והוא $z=0$ נציב ונראה: $0^3+3\cdot0 = 0$. \\
	נניח ש $z \neq 0$ ונפתור בהתאם:
	\begin{alignat*}{2}
		&z^3+3i\overline{z} = 0
		&&\big\backslash \cdot z \\
		&z^4 + 3i \overline{z}z = z^4 + 3i |z|^2 = 0 \\
		&z^4 = -3i|z|^2
		&&\big\backslash |z| = r, z = r(\cos \theta + i\sin \theta), z \neq 0 \\
		&(r(\cos \theta + i\sin \theta))^4 = -3ir^2
		&&\big\backslash \text{Moivre De} \\
		&r^4(\cos 4\theta +isin 4\theta) = -3ir^2
		&&\big\backslash \div r^2, r \neq 0 \\
		&r^2(\cos 4\theta +i\sin 4\theta) = -3i
		&&\big\backslash 0 + 1i = 1(\cos(\frac{\pi}{2}) + i\sin(\frac{\pi}{2})) \\
		&r^2(\cos 4\theta +i\sin 4\theta) = -3(\cos(\frac{\pi}{2}) + i\sin(\frac{\pi}{2}))
		&&\big\backslash \cos\alpha=\cos-\alpha, -\sin\alpha = \sin-\alpha \\
		&r^2(\cos 4\theta +i\sin 4\theta) = 3(\cos(-\frac{\pi}{2}) + i\sin(-\frac{\pi}{2}))
		&&\big\backslash \text{הוספת $2\pi$ לזווית} \\
		&r^2(\cos 4\theta +i\sin 4\theta) = 3(\cos(\frac{3\pi}{2}) + i\sin(\frac{3\pi}{2}))
	\end{alignat*}
	ולכן, הערך המוחלט של שאר הפתרונות הוא הוא: $r^2 = 3 \Rightarrow r = \sqrt{3}$
	והארגומנט של שאר הפתרונות כאשר $k \in \ZZ$ הוא $4\theta = \frac{3\pi}{2} + 2k\pi \Rightarrow \theta = \frac{3\pi}{8} + \frac{k\pi}{2}$
	ולכן נקבל את 5 הפתרונות הבאים:
	\begin{enumerate}
		\item $z_1 = 0$
		\item $z_2 = \sqrt{3}(\cos \frac{3\pi}{8} \sin \frac{3\pi}{8})$
		\item $z_3 = \sqrt{3}(\cos \frac{7\pi}{8} \sin \frac{7\pi}{8})$
		\item $z_4 = \sqrt{3}(\cos \frac{11\pi}{8} \sin \frac{11\pi}{8})$
		\item $z_5 = \sqrt{3}(\cos \frac{15\pi}{8} \sin \frac{15\pi}{8})$
	\end{enumerate}
	אם נמשיך להוסיף $\frac{k\pi}{2}$ לזווית נחזור חזרה במעגל ל $z_2$ בגלל שנעבור את $2\pi$.

	\subsection*{סעיך ב}
	$z_1, z_2 \in \CC$ הוכיחו שאם $z_1z_2 \neq -1$ ו $|z_1| = |z_2| = 1$ אזי $\frac{z_1+z_2}{1+z_1z_2} \in \RR$
	\begin{proof}
		\begin{align*}
			\frac{z_1+z_2}{1+z_1z_2} &=
			\frac{(z_1+z_2)\overline{(1+z_1z_2)}}{(1+z_1z_2)\overline{(1+z_1z_2)}} \ontop{(a)}=
			\frac{z_1 + z_1\overline{z_1z_2} + z_2 + z_2\overline{z_1z_2}}{1 + \overline{z_1z_2} + z_1z_2 + z_1z_2\overline{z_1z_2}} \\
			&\ontop{(b)}=	\frac{z_1 + |z_1|^2\overline{z_2} + z_2 + |z_2|^2\overline{z_1}}{1 + \overline{z_1z_2} + z_1z_2 + |z_1|^2|z_2|^2} \ontop{(c)}=
			\frac{z_1 + \overline{z_2} + z_2 + \overline{z_1}}{1 + \overline{z_1z_2} + z_1z_2} \ontop{(d)}=
			\frac{2Re(z_1) + 2Re(z_2)}{1 + 2Re(z_1z_2)}
		\end{align*}
		(a) - נכפיל את המונה והמכנה ב $\overline{(1+z_1z_2)}$ ששונה מאפס כי נתון $z_1z_2 \neq -1$. \\
		(b) - לפי משפט 6.44 $z\overline{z} = |z|^2$. \\
		(c) - נתון ש $|z_1| = |z_2| = 1$. \\
		(d) - לפי משפט 6.4.2 $z + \overline{z} = 2Re z$. \\
		לבסוף קיבלנו שבר שמכיל רק מספרים ממשיים ולפיכך המספר הינו מספר ממשי.
	\end{proof}

	\section*{שאלה 2}
	בדקו האם הקבוצה $V$ מהווה מרחב לינארי מעל השדה $F$.
	\subsection*{סעיף א}
	$F = \CC, V = M^{\CC}_{n \times n}$, עם החיבור הרגיל והכפל בסקלר מסומן $*$ ומוגדר: $\lambda * A = |\lambda|A$.
	\begin{proof}
		$V$ אינו מרחב לינארי מעל $F$ בגלל שאינו מקיים את תכונת פילוג הכפל בסקלר, נניח בשלילה ש $V$ הינו מרחב לינארי מעל $F$, תהי $I \in V$ מטריצת היחידה. \\
		$|1-1| \cdot I = |1|I + |-1|I \Rightarrow 0I = 2I$ בסתירה עם העובדה שכפל מטריצת היחידה בסקלר ששונה מאפס אינו אפס.
		ולכן פילוג הכפל בסקלר לא עובד ולכן $V$ אינו מרחב לינארי מעל $F$.
	\end{proof}

	\subsection*{סעיף ב}
	$F = \RR, V = \set{\ang{a,b} | a,b \in \RR}$ עם הכפל בסקלר הרגיל והחיבור מסומן $\bigoplus$ ומוגדר ע"י $(a,b)\bigoplus(c,d) = (a+c+1, b+d)$
	\begin{proof}
		$V$ אינו מהווה מרחב לינארי מעל $F$ בגלל שאינו מקיים את תכונות הפילוג בכפל בסקלר, נניח בשלילה ש $V$ הינו מרחב לינארי מעל $F$ ונגדיר $v = \ang{1,0} \in V$: \\
		לפי תכונות הפילוג: $2 \cdot (v + v) = 2v + 2v \Rightarrow 2 \cdot \ang{3,0 } = \ang{2,0} + \ang{2,0} \Rightarrow \ang{6,0} = \ang{4,0}$
		קיבלנו סתירה. ולפיכך $V$ אינו מרחב לינארי מעל $F$.
	\end{proof}

	\pagebreak

	\section*{שאלה 3}
	יהי מרחב לינארי $V$ ויהיו $v_1,v_2,v_3 \in V$ וקטורים שונים. \\
	\subsection*{סעיף א}
		אם $Sp\set{v_1, v_2} = Sp\set{v_1, v_3}$ אז הוקטורים $v_2, v_3$ תלויים לינארית.
		\begin{proof}
			לא נכון. \\
			 מכיוון ש $v_3 = \lambda_1 v_1 + \lambda_2 v_2$ לכל $\lambda_1, \lambda_2$ ולכן לא בהכרח ש $v_3$ ו $v_2$ תלויים לינארים אחד בשני.
			כדוגמא ניתן את $Sp\set{v_1(\ang{1,0}), v_2(\ang{1,0})}=Sp\set{v_1(\ang{1,0}), v_3(\ang{1,1})} = \RR^2$
			הצד השמאלי הוא הבסיס של $\RR^2$ ולכן פורס אותו. והצד הימני הוא צירוף לינארי שלו, אותו וקטור ראשון והשני הוא $v_3 = v_1 + v_2$.
			ומהכיוון השני $v_2 = v_3 - v_1$. ולכן לפי משפט 7.5.4 הם פורסים את אותו המרחב שהוא $\RR^2$. \\
			ובנוסף ניתן לראות כי סקלר שפותר את המשוואה $\ang{1,0} = \lambda \ang{1,1}$ ולכן $v_2$ ו $v_3$ אינם תלויים לינארית.
		\end{proof}

		\subsection*{סעיף ב}
		אם $v_1 - 2v_2 + v_3 = 0$ אז $Sp\set{v_1, v_2} = Sp\set{v_1, v_3}$.
		\begin{proof}
			נכון.
			נסמן $A = \set{v_1, v_2}, B = \set{v_1, v_3}$. ושניהם תתי קבוצות של המרחב הלינארי $V$. \\
			נראה שהוקטורים של $B$ הם צירוף לינארים של $A$ $v_1$ נמצא גם ב $A$ ולכן הוא צירוף לינארי של $A$. אם נשחק קצת עם המשוואה הנתונה נקבל $v_3 = 2v_2 - v_1$ ולכן גם $v_3$ הוא צירוף לינארי של $A$.  \\
			נראה עכשיו שהוקטורים של $A$ הם צירוף לינארי של $B$, גם כאן $v_1$ נמצא בשניהם. ואם נשחק עם המשוואה הנתונה קצת נקבל $v_2 = 0.5v_1 + 0.5v_3$ ולכן גם $v_2$ הוא צירוף לינארי של $B$. \\
			לפיכך 2 התנאים במשפט 7.5.4 מתקיימים ולכן $Sp(A) = Sp(B)$.

		\end{proof}

		\subsection*{סעיף ג}
		אם הקבוצה $\set{v_1, v_2, v_3}$ תלויה לינארי אז $Sp\set{v_1,v_2} = Sp\set{v_1 + v_2, v_2 + v_3}$
		\begin{proof}
			לא נכון.
			נראה לדוגמא: $v_3 = \ang{0,1,0}, v_2 = \ang{2,0,0}, v_1 = \ang{1,0,0}$
			ולכן נקבל את הקבוצה: $A = Sp\set{\ang{1,0,0}, \ang{2,0,0}}$ ואת $B = Sp\set{\ang{1,1,0}, \ang{2,1,0}}$
			וניתן  לראות כי $\ang{2,1,0} \in B$ אבל $\ang{2,1,0} \not\in A$ ולכן $A \neq B$.
		\end{proof}


		\pagebreak
		\section*{שאלה 4}
		בדקו אלו מהקבוצות הבאות הן מרחבים לינארים
		\subsection*{סעיף א}
		$K \set{\ang{x,y,z,t} \in (\ZZ\ / 5\ZZ)^4 | x+y-z+t = a \land 2x+y+z-3t=0}$
		\begin{proof}
			$K$ .הינו מרחב לינארי, בגלל שהוא בעצם מייצג מערכת משוואות הומוגנית עם 2 משוואות ו 4 נעלמים, ולפי דוגמא ד בעמוד 158 הוא מהווה מרחב לינארי
		\end{proof}

		\subsection*{סעיף ב}
		$L= \set{\ang{x,y,z} \in \RR^3 | y+z = |x|}$
		\begin{proof}
			$L$ אינו מרחב לינארי מכיוון שאינו סגור בחיבור. ניקח לדוגמא את שני הוקטורים הבאים: $a = \ang{1,-3,2}$, $b = \ang{2,-2,4}$ ניתן לראות כי $-3 + 2 = |1|$ ו- $-2 + 4 = |2|$
		אבל סכומם: $a + b = \ang{3, -5, 6}$ אינו נמצא בקבוצה, מכיוון ש $-5 + 6 \neq 3$, ולכן $L$ אינו מרחב לינארי.
		\end{proof}

		\subsection*{סעיף ג}
		$M = \set{p(x)} \in \RR_4[x] | p(-3) = 0$
		\begin{proof}
			$M$ הינו מרחב לינארי. נוכיח באמצעות משפט 7.3.2:
		\begin{enumerate}
			\item $M \neq \emptyset$ לדוגמא $p(x) = x + 3 \in M$.
			\item לפי הגדרת סכום פולינומים, סכום של פולינומים שקול לפולינום המייצג את סכום ערכי הפולינומים. ולכן לכל $w_1(-3) = 0$,$w_2(-3) = 0$ מתקבל ש $(w_1+w_2)(-3) = 0 + 0 = 0$ ולכן $M$ סגור בחיבור.
			\item מכפלת סקלר בפולינום שקולה למכפלת ערך הפולינום בסקלר ולכן לכל $\lambda \in \RR$ ולכל $w(-3) = 0$ נקבל $(\lambda \cdot w)(-3) = \lambda \cdot 0 = 0$.
		\end{enumerate}
		$M$ הינו תת קבוצה של המרחב הלינארי $\RR_4[x]$(לפי דוגמא ה בעמוד 159) והראינו כי $M$ מקיים את כל שלשת התנאים של משפט 7.3.2 ולפיכך $M$ הינו מרחב לינארי מעל $\RR$.
		\end{proof}

		\subsection*{סעיף ד}
		$S = \set{\ang{x,y} \in \RR^2 | x^2 - y^2 = 0}$
		\begin{proof}
			$S$ אינו מרחב לינארי, כי אינו סגור בחיבור, לדוגמא: $a = \ang{1, -1}$ ניתן לראות ש $1^2 - (-1)^2 = 0$ ו- $b = \ang{1,1}$ ניתן לראות ש $1^2 - 1^2 = 0$
		אבל $a + b = \ang{2, 0}$ אך $2^2 - 0^2 \neq 0$, ולפיכך $S$ אינו מרחב לינארי.
		\end{proof}

		\subsection*{סעיף ה}
		$T = \set{ax^3 + bx^2 + cx + d \in \RR_4[x] | (a+b-c)^2 + (b + 2d)^2 = 0}$
		\begin{proof}
			נוכיח ש $T$ הינו מרחב לינארי, ע"י הסתכלות בתנאים.
			$w^2 + z^2 = 0$ מתקיים רק בעבור $z^2 = 0 = z$, $w^2 = 0 = w$ ולפיכך נקבל:
			$a+b-c = 0$ ו- $b+2d = 0$, נשחק עם המשוואות ונקבל $c = a + b$ ו- $b = -2d$ ולכן $c = a - 2d$, עכשיו נציב את אלה בפולינום:
			\begin{align*}
			ax^3 + bx^2 + cx + d &= ax^3 -2dx^2 + (a-2d)x + d \\
			&= ax^3 - 2dx^2 + ax - 2dx + d = a(x^3 + x) + d(2x^2 - 2x + 1)
		\end{align*}
			ולכן קיבלנו שהפולינום שקול לצירוף הלינארי של הוקטורים $\set{x^3 + x, 2x^2, -2x, 1}$.
			שהם תת קבוצה של המרחב הלינארי $\RR_4[x]$ ולכן לפי משפט 7.5.1 הם מהווים קבוצה פורסת למרחב לינארי. ולפיכך גם $T$ הינו מרחב לינארי מעל השדה $\RR$.
		\end{proof}

		\section*{שאלה 5}
		יהי $p$ מספר ראשוני.
		\subsection*{סעיף א}
		יהי $v \in (\ZZ / p\ZZ)^2$ כאשר $v \neq \ang{0,0}$ מהו מספר הוקטורים התלויים לינארית ב $v$? \\
		\textbf{פתרון:} הוקטורים התלויים לינארית ב $v$ הם כל הוקטורים מהצורה $\lambda \cdot v$ כאשר $\lambda \in \ZZ / p\ZZ$
		ולכן בגלל שישנם $p$ לאמדות כאלה, לכן ישנם $p$ וקטורים שונים התלויים לינארית ב $v$.

		\subsection*{סעיף ב}
		מצאו את מספר המטריצות ההפיכות ב $M_{2\times 2}(\ZZ / p\ZZ)$
		\begin{proof}
			כל מטריצה ריבועית הפיכה הינה בלתי תלויה לינארית, ולכן נבחר את וקטור השורה הראשון להיות כל וקטור שאינו וקטור האפס, מכיוון שלכל איבר יש $p$ אפשרויות נקבל $p^2-1$ אפשרויות (פחות אחד ע"מ להוריד את וקטור האפס). \\
			לוקטור השורה השניה, נבחר כל וקטור שאינו צירוף לינארי של הוקטור הקודם, ובסעיף הקודם הוכחנו שישנם $p$ צירופים לינארים לכל וקטור ואחד הצירופים הינו מכפלה באפס. ולפיכך נוריד את הצירופים הנ"ל מהאופציות האפשריות ונקבל $p^2-p$
			לפי עקרון הכפל בקומבינטוריקה נקבל בסה"כ $(p^2-1)(p^2-p)$ מטריצות שונות שאינן תלויות לינארית, ולכן הפיכות.
		\end{proof}

		\pagebreak
		\section*{שאלה 6}
		נתונים התת מרחבים הבאים של $\RR^3$: \\
		$W = Sp\set{\ang{1,3,4}, \ang{2,5,1}}$ ו- $U = Sp\set{\ang{1,1,2}, \ang{2,2,1}}$
		מצאו קבוצה פורשת סופית בעבור $U \cap W$. \\
		\textbf{פתרון:} נתחיל בלמצוא את וקטורי הבסיס של הקבוצות ע"י לדרג את המטריצה המתקבלת מהצבת הוקטורים כוקטורי שורה:
		\begin{align*}
		W = \begin{pmatrix}
			1 & 3 & 4 \\
			2 & 5 & 1
		\end{pmatrix}
		&\overset{R_2 = 0.5R_2}\Rightarrow
		\begin{pmatrix}
			1 & 3 & 4 \\
			1 & 2.5 & 0.5
		\end{pmatrix}
		\overset{R_1 - R_2}\Rightarrow
		\begin{pmatrix}
			0 & 0.5 & 3.5 \\
			1 & 2.5 & 0.5
		\end{pmatrix} \\
		&\overset{R_2 - 5R_1}\Rightarrow
		\begin{pmatrix}
			0 & 0.5 & 3.5 \\
			1 & 0 & -17
		\end{pmatrix}
		\overset{R_2 \Leftrightarrow 2R_1}\Rightarrow
		\boxed{\begin{pmatrix}
			1 & 0 & -17 \\
			0 & 1 & 7
		\end{pmatrix}} \\
		U = \begin{pmatrix}
			1 & 1 & 2 \\
			2 & 2 & 1
		\end{pmatrix}
		&\overset{R_2 - 2R_1}\Rightarrow
		\begin{pmatrix}
			1 & 1 & 2 \\
			0 & 0 & -1
		\end{pmatrix}
		\overset{R_2 = -R_2}\Rightarrow
		\boxed{\begin{pmatrix}
			1 & 1 & 2 \\
			0 & 0 & 1
		\end{pmatrix}}
	\end{align*}
	נסתכל על הוקטור מהצורה $\ang{a,b,c}$ ניתן לראות כי ב $U$ האיברים $a,b$ מופיעים רק בשורה הראשונה ושניהם שווים ל1 ולכן כל צירוף לינארי של הוקטורים יתן לנו $a=b$,
	אבל האיבר $c$ הוא האיבר הפותח בוקטור השורה השני ולכן הוא יכול להיות כל ערך בעזרת צירוף לינארי של הוקטור. \\
	עכשיו נסתכל על $W$ וננסה לראות האם אפשר לקבל קבוצת וקטורים המכילה את המגבלות של $W$ וגם את של $U$.
	ניתן לראות שאם נחבר את הוקטור שורה הראשון והשני ב $W$ נקבל: $\ang{1,0-17} + \ang{0,1,7} = \ang{1,1,-10}$. \\
	בוקטור זה $a=b$ ובנוסף בגלל שהוא סכום הוקטורים במטריצה המדורגת של $W$ הוא מכיל את כל מגבלות הפריסה של $W$, ולפיכך $W \cap U = \set{\ang{1,1,-10}}$.
\end{document}
