% !TEX program = xelatex
\def\NN{\mathbb{N}}
\def\RR{\mathbb{R}}
\def\ZZ{\mathbb{Z}}
\def\QQ{\mathbb{Q}}
\def\PP{\mathcal{P}}
\def\SS{\mathcal{S}}
\def\DD{\mathcal{D}}
\def\sub{\setminus}
\def\bld{\mathbf}
\def\lmf{\lim_{n \to \infty}}
% Make the ref command use parenthesis
\let\oldref\ref
\renewcommand{\ref}[1]{(\oldref{#1})}

\newcommand{\ontop}[1]{\overset{\text{#1}}}
\newcommand{\ang}[1]{\langle #1 \rangle}
\newcommand{\pip}[1]{\left| #1 \right|}



% style
\newcommand{\bm}[1]{\displaystyle{#1}}
\def\nl{$ $ \newline}

\ExplSyntaxOn

\NewDocumentCommand{\getenv}{om}
{
  \sys_get_shell:nnN{ kpsewhich ~ --var-value ~ #2 }{}#1
}

\ExplSyntaxOff

%environments

\documentclass{article}
\usepackage[]{amsthm} %lets us use \begin{proof}
\usepackage{amsmath}
\usepackage{mathtools}
\usepackage{enumerate}
\usepackage{xparse}
\usepackage[makeroom]{cancel}
\usepackage[]{amssymb} %gives us the character \varnothing
\usepackage{polyglossia}
\usepackage{fontspec}
% \usepackage[frak=mma]{mathalfa}
\setdefaultlanguage{hebrew}
\setotherlanguage{english}
%\setmainfont{Frank Ruehl CLM}
\setmainfont{David CLM}
\setmonofont{Miriam Mono CLM}
\setsansfont{Simple CLM}
\newfontface\niceee{Brush Script MT}
\DeclarePairedDelimiter\set\{\}
% Use the following if you only want to change the font for Hebrew
%\newfontfamily\hebrewfont[Script=Hebrew]{David CLM}
%\newfontfamily\hebrewfonttt[Script=Hebrew]{Miriam Mono CLM}
%\newfontfamily\hebrewfontsf[Script=Hebrew]{Simple CLM}
\getenv[\ID]{ID}
\newtheorem{lemma}{טענת עזר}
\title{אלגברה לינארית - ממ"ן 15}
\author{אליחי טורקל \ID}
\date\today

\DeclareFontFamily{OT1}{cmrx}{}
\DeclareFontShape{OT1}{cmrx}{m}{n}{<->cmr10}{}
\let\saveLongrightarrow\Longrightarrow
\makeatletter
\renewcommand*{\Longrightarrow}{%
    \mathrel{\rlap{\fontfamily{cmrx}\fontencoding{OT1}\selectfont=}%
    \hphantom{\saveLongrightarrow}%
    \llap{$\m@th\Rightarrow$}}}
\makeatother



%\clearpage %Gives us a page break before the next section. Optional.
%\selectlanguage{english}
	%Section and subsection automatically number unless you put the asterisk next to them.


\begin{document}
	\maketitle %This command prints the title based on information entered above

	\section*{שאלה 1}
	\subsection*{סעיף א}
	לכל $\lambda \in F$ וגם $p(x), q(x) \in F_3[x]$
	\begin{enumerate}
		\item נוכיח ש $S$ היא העתקה לינארית:
		\begin{align*}
			&S((p+q)(x))
			= (p+q)(x+1)
			= p(x+1) + q(x+1)
			= S(p(x)) + S(q(x)) \\
			&S((\lambda \cdot p)(x))
			= (\lambda \cdot p)(x+1)
			= \lambda \cdot p(x+1)
			= \lambda \cdot S(p(x))
		\end{align*}
		ולכן לפי הגדרה 9.1.1 מתקבל ש $S$ היא העתקה לינארית.
		\item נוכיח ש $T$ היא העתקה לינארית:
		\begin{align*}
			&T((p+q)(x))
			= (x-1)(p+q)(x)
			= (x-1)p(x) + (x-1)q(x)
			= T(p(x)) + T(q(x)) \\
			&T((\lambda \cdot p)(x))
			= (x-1)(\lambda \cdot p)(x)
			= \lambda(x-1)p(x)
			= \lambda \cdot T(p(x))
		\end{align*}
		ולכן לפי 9.1.1 גם $T$ היא העתקה לינארית.
	\end{enumerate}

	\subsection*{סעיך ב}
	חשבו את $T \circ S(ax^2 + bx + c)$:
	\begin{align*}
		T \circ S(ax^2 + bx + c) &= \\
		&= T(S(ax^2 + bx + c)) \\
		&= T(a(x+1)^2 + b(x+1) + c) \\
		&= T(a(x^2 + 2x + 1) + bx + b + c) \\
		&= T(ax^2 + 2ax + bx + a + b + c) \\
		&= (x-1)(ax^2 + 2ax + bx + a + b + c) \\
		&= ax^3 + 2ax^2 + bx^2 + ax + bx + cx - ax^2 - 2ax - bx - a - b - c \\
		&= ax^3 + (a + b)x^2 + (c-a)x - (a+b+c)
	\end{align*}

	\subsection*{סעיף ג}
	אנו בעצם צריכים לבדוק האם קיים פתרון למשוואה: \\
	$ax^3 + (a + b)x^2 + (c-a)x - (a+b+c) = x^3 - 1$. \\
	ובעצם נקבל את המערכת משוואות הבאה:
	\[
	\begin{cases}
		a = 1 \\
		a + b = 0 \\
		c - a = 0 \\
		a + b + c = 1
	\end{cases}
	\]
	נציב במשוואה השניה את הראשונה ונקבל: $b = -1$ ומהאחרונה נקבל: $1 - 1 + c = 1$ ומכאן $c = 1$ וזה אכן מסתדר גם עם המשוואה השלישית $1-1=0$
	וקיבלנו $r(x) = x^2 - x + 1$ כך ש $(S \circ T)(r(x)) = x^3 - 1$. \\
	בנוסף נבחין כי ההעתקה אינה על, מכיוון שאנו יכולים להנדס משוואה כך שלא יהיה לה פתרון, לדוגמא: \\
	$v(x) = x^3 + 2x^2 + 1$, אם נשווה את הפולינום הנ"ל לצורה הכללית נקבל את מערכת המשוואות הבאה:
	\[
	\begin{cases}
		a = 1 \\
		a + b = 2 \\
		c - a = 0 \\
		a + b + c = 1 \\
	\end{cases}
	\]
	נציב את המשוואה הראשונה בשניה ונקבל: $b = 1$ ואת הראשונה בשלישית ונקבל $c = 1$ ואז במשוואה האחרונה נקבל $1 + 1 + 1 = 1$ שזו סתירה. ולכן הפולינום לא נמצא בתמונה של ההעתקה, ומכאן היא לא על $\RR_3[x]$.

	\pagebreak

	\section*{שאלה 2}
	\subsection*{סעיף א}
	נגדיר את הפונקציה הבאה: $S(1) = (1,0,0)$, $S(x)=(-1,1,1)$, $S(x^2) = (3,0,-1)$ ונוכיח שהיא אכן איזומורפיזם $S: \RR_3[x] \to \RR^3$ ומקיימת את התנאים של $T$:
	\begin{enumerate}
		\item מכיוון ש $(1, x, x^2)$ הינו הבסיס הסטנדרטי של $\RR_3[x]$ אזי לפי 9.4.2 מתקבל ש $S$ הינו העתקה לינארית.
		\item לפי מסקנה 9.6.2 במידה ו $Im S = R^3$ אזי $S$ היא איזומורפיזם, ולכן נסתכל על $Sp\set{(1,0,0), (-1,1,1), (3,0,-1)}$
		 ונשים לב ש: \\
		 $(0,0,1) = -1 \cdot ((3,0,-1)-3(1,0,0))$ \\
		 וגם: $(0,1,0) = (-1,1,1) + (1,0,0) - (0,0,1)$ \\
		  ולכן הקבוצה הפורסת שלנו הינה צירוף לינארי של הבסיס הסטנדרטי של $\RR^3$ ולכן היא גם בסיס ל $\RR^3$ ולפי 9.6.2 $S$ היא איזומורפיזם מ $\RR_3[x]$ על $\RR^3$.
		 \item נראה ש
		 \begin{align*}
			T(x^2 + 2x) = S(x^2) + S(2x) = (3,0,-1) + 2(-1,1,1) = (1, 2, 1) \\
			T(x+1) = S(x) + S(1) = (-1,1,1) + (1,0,0) = (0,1,1) \\
			T(x^2 - 2) = S(x^2) -2S(1) = (3,0,-1) - 2(1,0,0) = (1,0,-1)
		 \end{align*}
		 ובעצם הראינו, ש $S$ הינו איזומורפיזם המקיים את התנאים של $T$ כנדרש.
	\end{enumerate}

	\subsection*{סעיף ב}
		מכיוון ש $Ker \ S = \set{0}$ וגם $S$ הוא העתקה לינארית, אזי ממסקנה 9.6.2 $S$ הוא איזומורפיזם על $V$ ולכן היא חח"ע ועל, ועל כן $Im \ S = V$. \\
		ולכן $Im \ TS = Im \ (T(S(v))) = Im \ T(V) = Im \ T$.

	\pagebreak
	\section*{שאלה 3}
	\subsection*{סעיף א}
	יהי $v \ in \RR^5$, ויהי $\lambda \neq 0$ מכיוון ש $T^2(\lambda v) = 0$ ומכיוון ש $T$ היא טרנסופרמציה לינארית אזי קיים $u \in \RR^5$
	כך ש $u = \lambda T(v)$, ולכן יש שתי אופציות, או ש $v = u$ ואז בהכרח $v = u/\lambda = u = 0$
	או ש $v \neq u$ ואז נוכל לשנות את $u$ ע"י שינוי $\lambda$ ונקבל $T(u) = 0$ ולכן כל מה שנקבל מהתמונה של $T$ יהיה בקרנל של $T$ מכיוון שיחזיר אפס, או במילים אחרות $Img \ T \subseteq Ker \ T$.

	\subsection*{סעיף ב}
	לפי משפט 8.3.4 נקבל ש $dim(Im \ T) \leq dim(Ker \ T)$ ולכן \\
	$dim(Ker \ T) + dim(Im \ T) \leq 2 \cdot dim(Ker \ T)$ וממשפט 9.6.1 \\
	$dim(Ker \ T) + dim(Im \ T) = 5$ וביחד $5 \leq 2 \cdot dim(Ker \ T)$ נחלק בשתיים ונעגל למעלה (כי אין דבר כזה חצי מימד) ונקבל ש $dim(Ker \ T) \geq 3$.
	בנוסף מכיוון ש $Ker \ T$ הוא תת מרחב של $\RR^5$ אזי $dim(Ker \ T) \leq 5$. \\
	וביחד: $3 \leq dim(Ker \ T) \leq 5$.

	\subsection*{סעיף ג}
	יהי $W \subset \RR^5$ תת מרחב של $\RR^5$ וגם $dim(W) = 3$, ממשפט 9.6.1 נקבל ש:
	ממשפט 8.3.6 נובע כי $dim(W \cap Ker \ T) = dim(W) + dim(Ker \ T) - dim(W + Ker \ T)$, \\
	נשתמש באי שוויון מהמשפט הקודם ונקבל $dim(W \cap Ker \ T) \geq 3 + 3 - dim(W + Ker \ T)$
	ובנוסף אנו יודעים כי $dim(W + Ker \ T) \leq 5$ כי הוא תת מרחב של $\RR^5$ ואם נכניס גם את זה לאי שוויון נקבל $dim(W \cap Ker \ T) \geq 6 -5 \geq 1$
	ומכיוון ש $dim(\set{0}) = 0 \not \geq 1$ אזי $W \cap Ket \ T \neq \set{0}$ כנדרש.

	\pagebreak
	\section*{שאלה 4}
	\subsection*{סעיף א}
	נתון כי $(0,4,2) \not \in Im \ T$ ז"א שאין פתרון למערכת $[T]_B[v]_B = [(0,4,2)]_B$, נבדוק בעבור אילו ערכי $a$ אין למערכת הנ"ל אף פתרון.
	בעבור זאת נחשב את הקאוררדינטות של $v$ לפי הבסיס $B$ בעזרת מטריצת המקדמים:
	\begin{align*}
		\brac{\begin{array}{ccc|c}
			1 & 1 & 0 & 0 \\
			1 & 0 & 1 & 4 \\
			0 & 1 & 1 & 2
		\end{array}}
	\end{align*}
	ולכן $[(0,4,2)]_B = \begin{pmatrix}
		1 \\
		-1 \\
		3
	\end{pmatrix}$
	כעת נציב את זה כפתרון למערכת $[T]_B$ ע"מ לראות בעבור אילו ערכי $a$ אין למערכת פתרון:
	\begin{align*}
		\brac{\begin{array}{ccc|c}
			1 & 0 & 1 & 1 \\
			0 & 1 & a & -1 \\
			5 & a & 13-2a & 3
		\end{array}} \overset{R_3 = R_3 - 5R_1 -aR_2}\Rightarrow
		\brac{\begin{array}{ccc|c}
			1 & 0 & 1 & 1 \\
			0 & 1 & a & -1 \\
			0 & 0 & 8-2a-a^2 & a-2
		\end{array}}
	\end{align*}
	ונראה, שאם יש $a$ שבעבורו נקבל בשורה האחרונה שורת אפסים ששווה למספר ששונה מאפס אזי אין פתרון.
	ולכן נראה מתי $8-2a-a^2$ מתאפס בלי ש $a-2$ מתאפס, $8-2a-a^2=-(a+4)(a-2)$ אך $a=2$ יאפס גם את התוצאה, ועל כן $a=-4$ זה הערך היחיד שעבורו נקבל שורת סתירה.
	ולכן $a=-4$.

	\subsection*{סעיף ב}
	נציב את $a=-4$ במטריצה ונקבל:
	$[T]_B = \begin{pmatrix}
		1 & 0 & 1 \\
		0 & 1 & -4 \\
		5 & -4 & 21 \\
	\end{pmatrix}$.
	לפי למה 3.9.6 נקבל ש $Im \ T = Sp\set{T(1,1,0), T(1,0,1), T(0,1,1)}$ שהוא בעצם מרחב העמודות של $[T]_B$ ועל כן נשחלף אותה ונדרג ע"מ למצוא בסיס ל $Im \ T$:
	\begin{align*}
		[T]_B^t = \begin{pmatrix}
			1 & 0 & 5 \\
			0 & 1 & -4 \\
			1 & -4 & 21 \\
		\end{pmatrix} \overset{R_3 = R_3 - R_1 + 4R_2}\Rightarrow
		\begin{pmatrix}
			1 & 0 & 5 \\
			0 & 1 & -4 \\
			0 & 0 & 0 \\
		\end{pmatrix}
	\end{align*}
	וקיבלנו שהבסיס של $[Im \ T]_B$ הוא $\set{(1,0,5), (0,1,-4)}$ נמיר מקאורדינטות חזרה לוקטורים ונקבל: $v_1 = 1 \cdot (1,1,0) + 0 \cdot (1,0,1) + 5 \cdot (0,1,1) = (1,6,5)$. \\
	ובנוסף $v_2 = 0 \cdot (1,1,0) + 1 \cdot (1,0,1) -4 \cdot (0,1,1) = (1,-4, -3)$
	ובסה"כ קיבלנו שהבסיס של $Im \ T$ הוא $\set{(1,6,5), (1,-4,-3)}$. \\
	ע"מ למצוא בסיס ל $Ker \ T$ נסתכל על המשוואה $[T]_B[v]_B = 0$ ונפתור אותה בעזרת מטריצת המקדמים של $[T]_B$ (שמהווה לנו מערכת משוואות הומוגנית):
	\begin{align*}
		\begin{pmatrix}
			1 & 0 & 1 \\
			0 & 1 & -4 \\
			5 & -4 & 21 \\
		\end{pmatrix}
		\overset{R_3 - 5R_1 + 4R_2}\Rightarrow
		\begin{pmatrix}
			1 & 0 & 1 \\
			0 & 1 & -4 \\
			0 & 0 & 0 \\
		\end{pmatrix}
	\end{align*}
	ובעצם קיבלנו את הפתרון: $\begin{cases}
		x + z = 0 \Rightarrow x = -z\\
		y - 4z = 0 \Rightarrow y = 4z\\
		0 = 0
	\end{cases}$
	כאשר $z$ הוא משתנה חופשי, ועל כן קיבלנו שהבסיס של $[Ker \ T]_B$ הוא $\set{(-1,4,1)}$ נמיר חזרה לקואורדינטות, ונקבל:
	$v = -1 \cdot (1,1,0) + 4 \cdot (1,0,1) + 1 \cdot (0,1,1) = (3, 0, 5)$ ועל כן $\set{(3,0,5)}$ הוא הבסיס של $Ker \ T$.


	\subsection*{סעיף ג}
	מתוך משפט 10.6.1 נקבל ש $[T]_E = M^{-1}[T]_BM$ ועל כן $M^{-1}$ היא מטריצת המעבר מ $B$ ל $E$ ומכיוון ש$E$ זה הבסיס הסטנדרטי, אז זה פשוט הוקטורים של $B$ כוקטורי עמודה:
	$M^{-1} = \begin{pmatrix}
		1 & 1 & 0 \\
		1 & 0 & 1 \\
		0 & 1 & 1
	\end{pmatrix}$
	עכשיו נדרג את המטריצה ביחד עם מטריצת היחידה ע"מ למצוא את $M$:
	\begin{align*}
		&\brac{\begin{array}{ccc|ccc}
			1 & 1 & 0 & 1 & 0 & 0\\
			1 & 0 & 1 & 0 & 1 & 0\\
			0 & 1 & 1 & 0 & 0 & 1 \\
		\end{array}} \overset{R_2 = R_2 - R_1}\Rightarrow
		\brac{\begin{array}{ccc|ccc}
			1 & 1 & 0 & 1 & 0 & 0\\
			0 & -1 & 1 & -1 & 1 & 0\\
			0 & 1 & 1 & 0 & 0 & 1 \\
		\end{array}} \overset{R_3 = R_3 + R_2}{\underset{R_1 = R_1 + R_2}\Rightarrow} \\
		&\brac{\begin{array}{ccc|ccc}
			1 & 0 & 1 & 0 & 1 & 0\\
			0 & -1 & 1 & -1 & 1 & 0\\
			0 & 0 & 2 & -1 & 1 & 1 \\
		\end{array}} \overset{R_1 = R_1 - 0.5R_3}{\underset{R_2 = R_2 + 0.5R_3}\Rightarrow}
		\brac{\begin{array}{ccc|ccc}
			1 & 0 & 0 & 0.5 & 0.5 & -0.5\\
			0 & -1 & 0 & -0.5 & 0.5 & -0.5 \\
			0 & 0 & 2 & -1 & 1 & 1 \\
		\end{array}} \overset{R_2 = -R_2}{\underset{R_3 = 0.5R_3}\Rightarrow} \\
		&\brac{\begin{array}{ccc|ccc}
			1 & 0 & 0 & 0.5 & 0.5 & -0.5\\
			0 & 1 & 0 & 0.5 & -0.5 & 0.5 \\
			0 & 0 & 1 & -0.5 & 0.5 & 0.5 \\
		\end{array}}
	\end{align*}
	וקיבלנו ש: $M = \begin{pmatrix}
		0.5 & 0.5 & -0.5\\
		0.5 & -0.5 & 0.5 \\
		-0.5 & 0.5 & 0.5 \\
	\end{pmatrix}$
	נשתמש במשפט 10.6.1 ונחשב:
	\begin{align*}
		[T]_E &=
		\begin{pmatrix}
			1 & 1 & 0 \\
			1 & 0 & 1 \\
			0 & 1 & 1
		\end{pmatrix} \cdot
		\begin{pmatrix}
			1 & 0 & 1 \\
			0 & 1 & -4 \\
			5 & -4 & 21 \\
		\end{pmatrix} \cdot
		\begin{pmatrix}
			0.5 & 0.5 & -0.5\\
			0.5 & -0.5 & 0.5 \\
			-0.5 & 0.5 & 0.5 \\
		\end{pmatrix} \\
		&= \begin{pmatrix}
			1 & 1 & -3 \\
			6 & -4 & 22 \\
			5 & -3 & 17
		\end{pmatrix} \cdot
		\begin{pmatrix}
			0.5 & 0.5 & -0.5\\
			0.5 & -0.5 & 0.5 \\
			-0.5 & 0.5 & 0.5 \\
		\end{pmatrix} =
		\begin{pmatrix}
			\frac{5}{2} & -\frac{3}{2} & -\frac{3}{2} \\
			-10 & 16 & 6 \\
			-\frac{15}{2} & \frac{25}{2} & \frac{9}{2}
		\end{pmatrix}
	\end{align*}

	נכפול את $\begin{pmatrix}
		x \\
		y \\
		z
	\end{pmatrix}$ ב $M$ ע"מ לקבל את $[T(x,y,z)]_E$:
	\begin{align*}
		[T(x,y,z)]_E =
		\begin{pmatrix}
			\frac{5}{2} & -\frac{3}{2} & -\frac{3}{2} \\
			-10 & 16 & 6 \\
			-\frac{15}{2} & \frac{25}{2} & \frac{9}{2}
		\end{pmatrix} \cdot
		\begin{pmatrix}
			x \\
			y \\
			z
		\end{pmatrix} =
		\begin{pmatrix}
			\frac{5}{2}x - \frac{3}{2}y - \frac{3}{2}z \\
			-10x + 16y + 6z \\
			-\frac{15}{2}x + \frac{25}{2}y + \frac{9}{2}z
		\end{pmatrix}
	\end{align*}
	וקיבלנו ש $T(x,y,z) = (\frac{5}{2}x - \frac{3}{2}y - \frac{3}{2}z, -10x + 16y + 6z, -\frac{15}{2}x + \frac{25}{2}y + \frac{9}{2}z)$

	\section*{שאלה 5}
	לפי 10.5.2 נקבל ש $T$ היא איזומורפיזם אמ"מ $[T]_E$ הפיכה. ולכן נחשב את ההופכי שלה בעזרת דירוג $[T]_E$:
	מתוך ההגדרה של $T$ נבנה מטריצת מקדמים ונדרגה ביחד עם מטריצת היחידה:
	\begin{align*}
		&\brac{\begin{array}{ccc|ccc}
			1 & 0 & -2 & 1 & 0 & 0 \\
			2 & -1 & 3 & 0 & 1 & 0 \\
			4 & 1 & 8 & 0 & 0 & 1 \\
		\end{array}} \overset{R_2 = R_2 -2R_1}{\underset{R_4 = R_4 - 4R_1}\Rightarrow}
		\brac{\begin{array}{ccc|ccc}
			1 & 0 & -2 & 1 & 0 & 0 \\
			0 & -1 & 7 & -2 & 1 & 0 \\
			0 & 1 & 16 & -4 & 0 & 1 \\
		\end{array}} \overset{R_3 = R_3 + R_2}{\Rightarrow} \\
		&\brac{\begin{array}{ccc|ccc}
			1 & 0 & -2 & 1 & 0 & 0 \\
			0 & -1 & 7 & -2 & 1 & 0 \\
			0 & 0 & 23 & -6 & 1 & 1 \\
		\end{array}} \overset{R_2 = -R_2}{\underset{R_3 = R_3/23}\Rightarrow}
		\brac{\begin{array}{ccc|ccc}
			1 & 0 & -2 & 1 & 0 & 0 \\
			0 & 1 & -7 & 2 & -1 & 0 \\
			0 & 0 & 1 & -\frac{6}{23} & \frac{1}{23} & \frac{1}{23} \\
		\end{array}} \overset{R_1 = R_1 + 2R_3}{\underset{R_2 = R_2 + 7R_3}\Rightarrow} \\
		&\brac{\begin{array}{ccc|ccc}
			1 & 0 & 0 & \frac{11}{23} & \frac{2}{23} & \frac{2}{23} \\
			0 & 1 & 0 & \frac{4}{23} & -\frac{16}{23} & \frac{7}{23} \\
			0 & 0 & 1 & -\frac{6}{23} & \frac{1}{23} & \frac{1}{23} \\
		\end{array}}
	\end{align*}

	ובעצם גם הוכחנו ש $T$ זה איזומורפיזם, וגם מצאנו את $([T]_E)^{-1} = \begin{pmatrix}
		\frac{11}{23} & \frac{2}{23} & \frac{2}{23} \\
		\frac{4}{23} & -\frac{16}{23} & \frac{7}{23} \\
		-\frac{6}{23} & \frac{1}{23} & \frac{1}{23} \\
	\end{pmatrix}$
	נציב את הקואורדינטות ונקבל: $T^-1 = (\frac{11}{23}x +  \frac{2}{23}y + \frac{2}{23}z),
	(\frac{4}{23}x + -\frac{16}{23}y +  \frac{7}{23}z),
	(-\frac{6}{23}x + \frac{1}{23}y + \frac{1}{23}z)$

\end{document}
