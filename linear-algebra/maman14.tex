% !TEX program = xelatex
\def\NN{\mathbb{N}}
\def\RR{\mathbb{R}}
\def\PP{\mathcal{P}}
\def\sub{\setminus}

% Make the ref command use parenthesis
\let\oldref\ref
\renewcommand{\ref}[1]{(\oldref{#1})}



% style
\newcommand{\bm}[1]{\displaystyle{#1}}
\def\nl{$ $ \newline}

\ExplSyntaxOn

\NewDocumentCommand{\getenv}{om}
{
  \sys_get_shell:nnN{ kpsewhich ~ --var-value ~ #2 }{}#1
}

\ExplSyntaxOff

%environments

\documentclass{article}
\usepackage[]{amsthm} %lets us use \begin{proof}
\usepackage{amsmath}
\usepackage{mathtools}
\usepackage{enumerate}
\usepackage{xparse}
\usepackage[makeroom]{cancel}
\usepackage[]{amssymb} %gives us the character \varnothing
\usepackage{polyglossia}
\usepackage{fontspec}
% \usepackage[frak=mma]{mathalfa}
\setdefaultlanguage{hebrew}
\setotherlanguage{english}
%\setmainfont{Frank Ruehl CLM}
\setmainfont{David CLM}
\setmonofont{Miriam Mono CLM}
\setsansfont{Simple CLM}
\newfontface\niceee{Brush Script MT}
\DeclarePairedDelimiter\set\{\}
% Use the following if you only want to change the font for Hebrew
%\newfontfamily\hebrewfont[Script=Hebrew]{David CLM}
%\newfontfamily\hebrewfonttt[Script=Hebrew]{Miriam Mono CLM}
%\newfontfamily\hebrewfontsf[Script=Hebrew]{Simple CLM}
\getenv[\ID]{ID}
\newtheorem{lemma}{טענת עזר}
\title{אלגברה לינארית - ממ"ן 14}
\author{אליחי טורקל \ID}
\date\today

\DeclareFontFamily{OT1}{cmrx}{}
\DeclareFontShape{OT1}{cmrx}{m}{n}{<->cmr10}{}
\let\saveLongrightarrow\Longrightarrow
\makeatletter
\renewcommand*{\Longrightarrow}{%
    \mathrel{\rlap{\fontfamily{cmrx}\fontencoding{OT1}\selectfont=}%
    \hphantom{\saveLongrightarrow}%
    \llap{$\m@th\Rightarrow$}}}
\makeatother


%\clearpage %Gives us a page break before the next section. Optional.
%\selectlanguage{english}
	%Section and subsection automatically number unless you put the asterisk next to them.


\begin{document}
	\maketitle %This command prints the title based on information entered above

	\section*{שאלה 1}
	\subsection*{סעיף א}
	\begin{proof}
		נבדוק האם קיים צירוף לינארי לא טריוויאלי המקיים: \\
		 $\alpha_1 x \cos x + \alpha_2 \cos x + \alpha_3 \sin x = 0$, כאשר $\alpha_i \in \RR$. \\
		צירוף זה צריך להתקיים לכל $x \in \RR$ ובפרט בעבור:
		\begin{enumerate}
			\item $x = 0$, ונציב במשוואה ונקבל: $\alpha_2 = 0$
			\item $x = \pi$ נציב במשוואה ונקבל: $-\alpha_1\pi - \alpha_2$ נציב $\alpha_2 = 0$, נחלק ב $-\pi$ ונקבל: $\alpha_1 = 0$.
			\item $x = \frac{\pi}{2}$ נציב ונקבל: $\alpha_3 = 0$.
		\end{enumerate}
		ולפיכך בהכרח כל המקדמים שווים ל0 ולכן אין צירוף לינארי לא טרוויאלי השווה לאפס, ולכן הקבוצה היא בת"ל.
	\end{proof}
	\subsection*{סעיך ב}
	נבדוק האם קיים צירוף לינארי לא טריוויאלי המקיים: \\
	 $\alpha_1 (x-1)(x-2) + \alpha_2 x(x-2) + \alpha_3 x(x-1) = 0$, כאשר $\alpha_i \in \RR$. \\
	נקבץ מקדמים:
	\begin{align*}
		\alpha_1 (x-1)(x-2) + \alpha_2 x(x-2) + \alpha_3 x(x-1) &= \\
		&= \alpha_1 (x^2-3x+2) + \alpha_2 (x^2-2x) + \alpha_3 (x^2-x) \\
		&=(\alpha_1 + \alpha_2 + \alpha_3)x^2 - (3\alpha_1 + 2\alpha_2 + \alpha_3)x + 2\alpha_1
	\end{align*}
	נשווה את המקדמים למקדמי וקטור האפס ונקבל:
	\begin{alignat}{2}
		\alpha_1 + \alpha_2 + \alpha_3 &= 0 \label{eq:1} \\
		3\alpha_1 + 2\alpha_2 + \alpha_3 &= 0 \label{eq:2} \\
		2\alpha_1 &= 0 \Rightarrow \alpha_1 = 0 \label{eq:3} \\
		\alpha_2 = -\alpha_3
		&&\big\backslash \ref{eq:1} \text{ ב } \ref{eq:3} \text{ נציב את } \label{eq:4} \\
		0 - 2\alpha_3 + \alpha_3 = 0 \Rightarrow \alpha_3 = 0
		&&\big\backslash \ref{eq:2} \text{ ב }  \ref{eq:4}, \ref{eq:3} \text{ נציב את }  \label{eq:5} \\
		0 + \alpha_2 + 0 = 0 \Rightarrow \alpha_2 = 0
		&&\big\backslash \ref{eq:1} \text{ ב }  \ref{eq:5}, \ref{eq:3} \text{ נציב את }
	\end{alignat}{2}
	ולפיכך אנו רואים כי שלשת המקדמים בהכרח שווים לאפס, ולכן אין פתרון לא טריוויאלי, ולכן הקבוצה היא בת"ל.

	\subsection*{סעיף ג}
	נבדוק האם קיים צירוף לינארי לא טריוויאלי המקיים: \\
	$\alpha_1 \cdot 3 + \alpha_2 \cdot \cos^2 x + \alpha_3 \sin^2 x = 0$ כאשר $\alpha_i \in \RR$. \\
	מתוך הזהות הידועה: $\cos^2 x + \sin^2 x = 1$ נראה שמתקבל: $3 = 3(\cos^2 x + \sin^2 x) \Rightarrow h(x) = 3(g(x) + f(x)) = 3g(c) + 3f(x)$. \\
	מכאן נובע שהקבוצה תלויה לינארית.


	\section*{שאלה 2}
	\subsection*{סעיף א}
	נוכיח ש $W$ היא תת מרחב של $M_{2 \times 2}(\ZZ/5\ZZ)$ בעזרת משפט 7.3.2:
	\begin{enumerate}
		\item $W \neq \emptyset$ למשל מטריצת האפס מקיימת $2a+3b=c$ ולכן נמצאת ב $W$.
		\item לכל זוג וקטורים במרחב: $\begin{pmatrix}
			a_1 & b_1 \\
			c_1 & d_1
		\end{pmatrix}$, $\begin{pmatrix}
			a_2 & b_2 \\
			c_2 & d_2
		\end{pmatrix}$
		סכומם יהיה $\begin{pmatrix}
			a_1 + a_2 & b_1 + b_2 \\
			c_1 + c_2 & d_1 + d_2
		\end{pmatrix}$. \\
		מכיוון ששניהם בקבוצה מתקיים: ככה ש $2a_1+3b_1=c_1$ וגם $2a_2 + 3b_2 = c_2$.
		אם נסכם את שתי המשוואות נקבל שמתקיים: $2(a_1+a_2) + 3(b_1+b_2) = c_1 + c_2$
		שזה בדיוק התנאי הנדרש ע"מ שמטריצת הסכום תהיה במרחב. \\
		ולכן סכום של כל 2 וקטורים בקבוצה נמצא בקבוצה. \\
		\item יהי וקטור מהמרחב: $\begin{pmatrix}
			a_1 & b_1 \\
			c_1 & d_1
		\end{pmatrix}$
		אם נכפיל אותו בסקלאר $\alpha$ נקבל: $\begin{pmatrix}
			\alpha a_1 & \alpha b_1 \\
			\alpha c_1 & \alpha d_1
		\end{pmatrix}$
		ומכיוון שהוא בקבוצה מתקיים $2a_1+3b_1=c_1$ אם נכפיל את המשוואה גם ב $\alpha$ נראה שמתקיים
		$2 \alpha a_1 + 3 \alpha b_1= \alpha c_1$ שזה בדיוק התנאי הנדרש מהמטריצה שקיבלנו לאחר הכפל בסקלאר. \\
		ולכן כפל וקטור בסקלאר מתקיים בקבוצה.
	\end{enumerate}
	ולפיכך מתקבל ש $W$ הינו תת מרחב לינארי. \\
	בקבוצה $U$ נראה ששלשת הוקטורים בקבוצה הפורסת נמצאות כולן ב $M_{2 \times 2}(\ZZ/5\ZZ)$ ולכן לפי משפט 7.5.1 מתקבל שהקבוצה הנפרסת ע"י הוקטורים הנ"ל היא תת מרחב של $M_{2 \times 2}(\ZZ/5\ZZ)$.

	\pagebreak
	\subsection*{סעיף ב}

	\begin{enumerate}
		\item \textbf{בסיס ל U}: 	נתחיל בלסמן את המטריצות: \\
		$M_1 = \begin{pmatrix}
			1 & 0 \\
			3 & 0
		\end{pmatrix}$,
		$M_2 = \begin{pmatrix}
			2 & 1 \\
			3 & 1
		\end{pmatrix}$,
		$M_3 = \begin{pmatrix}
			1 & 4 \\
			1 & 4
		\end{pmatrix}$
		נחשב את הקאורדינטות של המטריצות מעל הבסיס הסטנדרטי $B$: \\
		$[M_1]_B^t = (1,0,3,0)$, $[M_2]_B^t = (2,1,3,1)$, $[M_3]_B^t = (1,4,1,4)$
		עכשיו נציב את הוקטורים האלה כוקטורי שורה במטריצה, ונדרג אותה:
		\begin{align*}
			\begin{pmatrix}
				1 & 0 & 3 & 0 \\
				2 & 1 & 3 & 1 \\
				1 & 4 & 1 & 4
			\end{pmatrix} \overset{R_3=R_3-R_1}{\underset{R_2=R_2-2R_1}\Rightarrow}
			\begin{pmatrix}
				1 & 0 & 3 & 0 \\
				0 & 1 & 2 & 1 \\
				0 & 4 & 3 & 4
			\end{pmatrix} \overset{R_3 = R_3 + R_2}\Rightarrow
			\begin{pmatrix}
				1 & 0 & 3 & 0 \\
				0 & 1 & 2 & 1 \\
				0 & 0 & 0 & 0
			\end{pmatrix}
		\end{align*}
		לפי שורת האפסים אנו רואים שקבוצת הקואורדינטות תלויה לינארית, ולפי משפט 8.4.4 מתקבל שהקבוצה הפורסת תלויה לינארית
		ולפי למה 8.5.1 מתקבל שהשורות שאינן שורות האפס מהוות בסיס למרחב הנפרס, ולכן אנו מקבלים את הבסיס הבא:
		$\set{\begin{pmatrix}
			1 & 0 \\
			3 & 0
		\end{pmatrix}
		\begin{pmatrix}
			0 & 1 \\
			2 & 1
		\end{pmatrix}}$
		ומכיוון שיש 2 וקטורים בבסיס אזי המימד של המרחב הינו 2.

		\item \textbf{בסיס ל W:}
		לפי הגדרת הקבוצה אפשר להציג אותה כצירוף לינארי מהצורה:
		\[ a\begin{pmatrix}
			1 & 0 \\
			2 & 0 \\
		\end{pmatrix} + b \begin{pmatrix}
			0 & 1 \\
			3 & 0
		\end{pmatrix} + d \begin{pmatrix}
			0 & 0 \\
			0 & 1
		\end{pmatrix} =
		\begin{pmatrix}
			a & b \\
			2a + 3b & d
		\end{pmatrix} \]
		ובעצם קיבלנו קבוצה פורסת: $Span(\set{\begin{pmatrix}
			1 & 0 \\
			2 & 0 \\
		\end{pmatrix}, \begin{pmatrix}
			0 & 1 \\
			3 & 0
		\end{pmatrix}, \begin{pmatrix}
			0 & 0 \\
			0 & 1
		\end{pmatrix}})$ \\
		נחשב את הקאורדינטות של הוקטורים בקבוצה הפורסת מעל הבסיס הסטנדרטי $B$ ונציב אותם כוקטורי שורות במטריצה ונקבל:
		$\begin{pmatrix}
			1 & 0 & 2 & 0 \\
			0 & 1 & 3 & 0 \\
			0 & 0 & 0 & 1
		\end{pmatrix}$
		וקיבלנו מטריצת מדרגות ללא שורות אפסים. ולכן היא בת"ל, ולפי משפט 8.4.4 מתקבל שהקבוצה הפורסת היא בת"ל ולפי למה 8.5.1 היא מהווה בסיס למרחב. \\
		ומכיוון שיש בבסיס 3 וקטורים, מתקבל שהמרחב הוא במימד 3.
	\end{enumerate}

	\subsection*{סעיף ג}
	ע"מ למצוא בסיס ל $U \cap W$ נבדוק מתי הצירופים הלינארים של הבסיס של $U$ עומד בתנאים של $W$
	לשם כך אנו צריכים שהמשוואה הבאה תתקיים: \\
	$\alpha_1 \begin{pmatrix}
		1 & 0 \\
		3 & 0
	\end{pmatrix} + \alpha_2 \begin{pmatrix}
		0 & 1 \\
		2 & 1
	\end{pmatrix} =
	\begin{pmatrix}
		a & b \\
		2a + 3b & d
	\end{pmatrix}$
	מכאן לפי חיבור מטריצות אנו מקבלים 4 משוואות:
	\begin{align} \setcounter{equation}{0}
		\alpha_1 = a \label{eq2:1}\\
		\alpha_2 = b \label{eq2:2}\\
		3\alpha_1 + 2\alpha_2 = 2a + 3b \label{eq2:3}\\
		\alpha_2 = d
	\end{align}
	אם נציב ב \ref{eq2:3} את השוויונות שקיבלנו ב \ref{eq2:1}, \ref{eq2:2} נקבל ש $a=b$ ולכן $\alpha_1= \alpha_2$ מכאן ש $c=5a=0$ בגלל שאנחנו מעל $\ZZ/5\ZZ$. \\
	ולכן אם נציב את התנאים חזרה במשוואה נקבל את המטריצה $\begin{pmatrix}
		a & a \\
		0 & a
	\end{pmatrix}$ או בעצם את הצירוף הלינארי $a \begin{pmatrix}
		1 & 1 \\
		0 & 1
	\end{pmatrix}$ ובכך מצאנו קבוצה פורסת למרחב $U \cap W$ שהיא: $\set{\begin{pmatrix}
		1 & 1 \\
		0 & 1
	\end{pmatrix}}$ והיא בת"ל (מכיוון שזה וקטור יחיד ושונה מאפס) ולכן היא בסיס ל $U \cap W$ והמימד הוא 1.

	\subsection*{סעיף ד}
	כן. מכיוון שלפי משפט 8.3.6 מתקיים: \\
	$dim(U + W) = dim(U) + dim(W) - dim(U \cap W) = 2+3-1=4$ \\
	ובנוסף $dim(M_{2\times2}(\ZZ/5\ZZ))=4$ ולפי משפט 8.3.6 מכיוון ש $U,W$ הם תת מרחבים של $M_{2\times2}(\ZZ/5\ZZ)$
	ובנוסף $dim(M_{2\times2}(\ZZ/5\ZZ)) = dim(U) + dim(W)$ אזי מתקיים $M_{2\times2}(\ZZ/5\ZZ) = U \oplus W$ ולפי הגדרה 7.7.1 מתקיים $M_{2\times2}(\ZZ/5\ZZ) = U+W$ כנדרש.


	\subsection*{סעיף ה}
	מכיוון ש $dim(U) = 2$ ו $dim(M_{2\times2}(\ZZ/5\ZZ)) = 4$ חסרים 2 וקטורים בבסיס,
	נשים לב כי בוקטורי הבסיס של $U$ שתי האיברים בעמודה השניה תלויים זה בזה, וגם האיבר בעמודה הראשונה, שורה שניה תלוי תמיד באיברים אחרים.
	ולכן נגדיר  \\
	$T = Span(\set{\begin{pmatrix}
		0 & 1 \\
		0 & 0
	\end{pmatrix}, \begin{pmatrix}
		0 & 0 \\
		1 & 0
	\end{pmatrix}})$.
	נסתכל על איחוד הקבוצות הפורסות: \\
	$U \cup T = \set{\begin{pmatrix}
		1 & 0 \\
		3 & 0
	\end{pmatrix}, \begin{pmatrix}
		0 & 1 \\
		2 & 1
	\end{pmatrix}, \begin{pmatrix}
		0 & 1 \\
		0 & 0
	\end{pmatrix}, \begin{pmatrix}
		0 & 0 \\
		1 & 0
	\end{pmatrix}}$ נציב את הקאורדינטות שלהם מעל הבסיס הסטנדרטי כוקטורי שורה במטריצה:
	\[
	\begin{pmatrix}
		1 & 0 & 3 & 0 \\
		0 & 1 & 2 & 1 \\
		0 & 1 & 0 & 0 \\
		0 & 0 & 1 & 0
	\end{pmatrix} \overset{R_1 = R_1 - 3R_4}{\underset{R_2 = R_2 - 2R_4 - R_3}\Rightarrow}
	\begin{pmatrix}
		1 & 0 & 0 & 0 \\
		0 & 0 & 0 & 1 \\
		0 & 1 & 0 & 0 \\
		0 & 0 & 1 & 0
	\end{pmatrix} R_2 \Leftrightarrow R_3. \begin{pmatrix}
		\cdots
	\end{pmatrix} R_3 \Leftrightarrow R_4
	\begin{pmatrix}
		1 & 0 & 0 & 0 \\
		0 & 1 & 0 & 0 \\
		0 & 0 & 1 & 0 \\
		0 & 0 & 0 & 1
	\end{pmatrix}
	\]
	ומכאן אנו רואים כמה דברים: הקבוצה הינה בת"ל ולכן היא בסיס, ומכיוון שיש לה 4 מימדים אזי היא שווה ל $M_{2\times2}(\ZZ/5\ZZ)$, ובנוסף אם נפרק את מטריצת הקואורדינטות שקיבלנו חזרה לוקטורים נקבל את הבסיס הסטנדרטי, ולכן הקבוצה הפורסת שלנו שקולה לבסיס הסנדרטי ומכאן ש $T \oplus U = M_{2\times2}(\ZZ/5\ZZ)$


	\pagebreak
	\section*{שאלה 3}
	נניח בשלילה כי $v_1 \in Span(\set{v_1+w, \dots, v_k + w})$ ולכן ישנו צירוף לינארי המקיים:
	\begin{align*}
		&v_1 = \sum_{i=1}^k \alpha_i (v_i + w) =
		\sum_{i=1}^k (\alpha_i v_i) + \sum_{i=1}^k (\alpha_i)w \Rightarrow \\
		\Rightarrow
		&\sum_{i=1}^k (\alpha_i)w  = \sum_{i=1}^k (\alpha_i v_i) - v_1 =
		(\alpha_1 - 1)v_1 + \sum_{i=2}^k (\alpha_i v_i)
	\end{align*}
	אם $\sum_{i=1}^k \alpha_i = 0$ אזי האגף הימני שווה לאפס ומיכוון שנתון לנו ש $\set{v_i}$ בת"ל אזי הפתרון היחיד לצירוף לינארי ששווה לאפס הוא הטריוויאלי ולכן $\alpha_1 = 1$ ו $\forall \alpha_i=0$ אבל אז נראה ש $\sum_{i=1}^k \alpha_i =1$ בסתירה. ולכן $\sum_{i=1}^k \alpha_i \neq 0$ ונוכל לחלק בו:
	\begin{align*}
		w = \frac{1}{\sum_{i=2}^k \alpha_i}((\alpha_1 - 1)v_1 + \sum_{i=2}^k (\alpha_i v_i)) =
		\frac{\alpha_1 - 1}{\sum_{i=2}^k \alpha_i}v_1 + \dots + \frac{\alpha_k}{\sum_{i=2}^k \alpha_i}v_k
	\end{align*}
	ובעצם קיבלנו צירוף לינארי של $\set{v_i}$ שנותן לנו את $w$ בסתירה עם הנתון ש  \\ $w \not\in Span(\set{v_1, v_2 \dots, v_k})$.
	ולכן בהכרח מתקיים: \\ $v_1 \not\in Span(\set{v_1 + w, v_2 + w, \dots, v_k + w})$.


	\pagebreak
	\section*{שאלה 4}
	\subsection*{סעיף א}
	מכיוון שאותם מקדמי $a_{ij}$ נמצאים בצד השמאלי והימני של המשוואות אזי מתקבל פתרון מהספרות שנתונות, ובשלושת המשוואות נקבל את אותו הוקטור $(1, -2, 0)$. \\
	ומכיוון שמצאנו יותר מפתרון אחד, אזי שיש לפחות משתנה אחד חופשי ולכן ישנם אינסוף פתרונות. \\
	ומכיוון שיש לנו משתנה חופשי אזי $A$ אינה הפיכה ולכן לפי 3.10.2 יש פתרונות ל $Ax=0$ מעבר לפתרון הטריוויאלי,
	ולכן משאלה 3.7.1 מתקבל שחיסור 2 הפתרונות שהוא: $(0,2,1)$ הוא פתרון למערכת $Ax=0$ ולכן מימד מרחב הפתרונות(שאותו נסמן $P(A)$) הינו לפחות 1 (מכיוון שהמרחב מכיל וקטור שאינו וקטור האפס). \\
	ולפי 8.3.6 מתקבל ש $dim(P(A)) = 3 - \rho(A)$ ובגלל ש $dim(P(A)) > 0$ נקבל ש $\rho(A) < 3$ ומכיוון שנתון ש $\rho(A) \neq 1$ ובנוסף מכיוון שיש פתרונות לא טריוויאלים אזי יש לפחות שורה אחת שאינה אפס ולכן לפי למה 8.5.1 נקבל $\rho(A) > 0$
	אם נצרף את 3 התנאים נקבל שבהכרח $\rho(A)=2$.

	\subsection*{סעיף ב}
	נציב את הדרגה של $A$ במשפט 8.6.1 ונקבל $dim(P(A)) = 3 - 2 = 1$ ולכן הבסיס מכיל וקטור יחיד ומכיוון שיש לנו כבר וקטור פתרון, ווקטור יחיד השונה מאפס הוא בהכרח בת"ל אזי הוא הבסיס. ולכן \\
	$\set{(0,2,1)}$ פורס את מרחב הפתרונות של המערכת.

	\subsection*{סעיף ג}
	לפי שאלה 3.7.1, מכיוון שהמערכת היא מעל $\RR$ נגדיר $c_0 = (1,0,1)$, $d \in Span(\set{0,2,1})$ ונקבל שהפתרון הכללי הוא כל הקומבינציות של $c_0 + d$ או במילים אחרות: \\
	$\set{(1,0,1) + \alpha (0,2,1) | \alpha \in \RR}$

	\pagebreak
	\section*{שאלה 5}
	נשתמש בלמה 3.4.3 ע"מ לקבל את מערכת משוואות:
	\begin{align*}
		BA[B]^c_1 = 0 \\
		BA[B]^c_2 = 0 \\
		\cdots \\
		BA[B]^c_n = 0 \\
	\end{align*}
	נסמן את מרחב הפתרון של המערכת $BA \cdot x = 0$ ב $P(BA)$. \\
	ולפי משפט 8.6.1 $dim(P(BA)) = n - \rho(BA)$ \\
	ולפי שאלה 8.5.7 מכיוון ש $A$ הפיכה נקבל $\rho(BA) = \rho(B)$ \\
	 ולפי משפט 7.5.1 נקבל ש $Sp \set{[B]^c_i} \subseteq P(BA)$. \\
	ולפי משפט 8.3.4 נקבל ש $dim(Sp \set{[B]^c_i}) \leq dim(P(BA))$ \\
	 ולפי הגדרת הדרגה $dim(Sp \set{[B]^c_i}) = \rho(B)$. \\
	 נחבר ביחד את כל השוויונות לאי שוויון אחד גדול \\
	 $\rho(B) \leq dim(Sp \set{[B]^c_i}) \leq dim(P(BA)) \leq n - \rho(BA) \leq n - \rho(B)$ \\
	 ניקח את הקצוות שלו: $\rho(B) \leq n - \rho(B)$ ומכאן $\rho(B) \leq \frac{n}{2}$.


	 \section*{שאלה 6}
	 \subsection*{סעיף א}
	 נפתח את $B$ ע"מ לקבל את הוקטורים בקבוצה: \\
	 $B = (1, 1 + x + x^2 + x^3, 1 - x + x^2 - x^3, 1 + 2x + 4x^2 + 8x^3)$.
	 נציב את הקואורדינטות של הוקטורים ביחס לבסיס הסטנדרטי של $\RR_4[x]$ כוקטורי שורה במטריצה:
	 \begin{align*}
		&\begin{pmatrix}
			 1 & 0 & 0 & 0 \\
			 1 & 1 & 1 & 1 \\
			 1 & -1 & 1 & -1 \\
			 1 & 2 & 4 & 8
		 \end{pmatrix} \overset{R_2 = R_2 - R_1}{\underset{\substack{R_3 = R_3 - R_1 \\ R_4 = R_4 - R_1}}\Rightarrow}
		 \begin{pmatrix}
			1 & 0 & 0 & 0 \\
			0 & 1 & 1 & 1 \\
			0 & -1 & 1 & -1 \\
			0 & 2 & 4 & 8
		\end{pmatrix} \overset{R_3 = R_3 + R_2}{\underset{R_4 = R_4 - 2R_2}\Rightarrow}
		\begin{pmatrix}
			1 & 0 & 0 & 0 \\
			0 & 1 & 1 & 1 \\
			0 & 0 & 2 & 0 \\
			0 & 0 & 2 & 6
		\end{pmatrix} \\ \overset{R_3 = R_3 / 2}{\Rightarrow}
		&\begin{pmatrix}
			1 & 0 & 0 & 0 \\
			0 & 1 & 1 & 1 \\
			0 & 0 & 1 & 0 \\
			0 & 0 & 2 & 6
		\end{pmatrix} \overset{R_4 = (R_4 - 2R_3)/6}{\Rightarrow}
		\begin{pmatrix}
			1 & 0 & 0 & 0 \\
			0 & 1 & 1 & 1 \\
			0 & 0 & 1 & 0 \\
			0 & 0 & 0 & 1
		\end{pmatrix} \overset{R_2 = R_2 - R_3 - R_4}{\Rightarrow}
		\begin{pmatrix}
			1 & 0 & 0 & 0 \\
			0 & 1 & 0 & 0 \\
			0 & 0 & 1 & 0 \\
			0 & 0 & 0 & 1
		\end{pmatrix}
	\end{align*}
	ומכאן קיבלנו שהקבוצה בת"ל והמימד שלה הוא 4 והיא שקולה לבסיס הסטנדרטי של $\RR_4[x]$.

	\subsection*{סעיף ב}
	מתוך הנתון נפתח את הצירוף הלינארי:
	\begin{align*}
		q(x) &= 1 \cdot (1) + (-1) \cdot (1 + x + x^2 + x^3) + 1 \cdot (1 - x + x^2 - x^3) + (-2) \cdot (1 + 2x + 4x + 8x^3)  \\
		&= 1 - 1 - x - x^2 - x^3 + 1 - x + x^2 - x^3 - 2 - 4x -8x^2 - 16x^3 \\
		&= - 1 - 6x - 8x^2 - 18x^3
	\end{align*}

	\subsection*{סעיף ג}
	אנו בעצם מחפשים פתרון למשוואה הבאה:
	\[
	\alpha_1 \cdot 1 + \alpha_2 (1 + x + x^2 + x^3) + \alpha_3 (1 - x + x^2 - x^3) + \alpha_4(1 + 2x + 4x^2 + 8x^3) =
	2 - x^2 + x^3
	\]
	נקבץ מקדמים ונקבל:
	\[
	(\alpha_1 + \alpha_2 + \alpha_3 + \alpha_4)1 + (\alpha_2 - \alpha_3 + 2\alpha_4)x
	+ (\alpha_2 + \alpha_3 + 4\alpha_4)x^2 + (\alpha_2 - \alpha_3 + 8\alpha_4)x^3
	= 2 - x^2 + x^3
	\]
	ובעצם קיבלנו 4 משוואות עם 4 נעלמים:
	\begin{alignat}{7}
		\alpha_1 & + & \alpha_2 & + & \alpha_3 & + & \alpha_4 &= 2 \\
		&& \alpha_2 & - & \alpha_3 & + & 2\alpha_4 &= 0 \\
		&& \alpha_2 & + & \alpha_3 & + & 4\alpha_4 &= -1 \\
		&& \alpha_2 & - & \alpha_3 & + & 8\alpha_4 &= 1
	\end{alignat}
	נציב את המקדמים כשורות במטריצה:
	\begin{align*}
		\left( \begin{array}{cccc|c}
			1 & 1 & 1 & 1 & 2 \\
			0 & 1 & -1 & 2 & 0 \\
			0 & 1 & 1 & 4 & -1 \\
			0 & 1 & -1 & 8 & 1
		\end{array} \right) \overset{R_2 = R_2 + R_3}{\underset{R_4 = R_4 + R_3}\Rightarrow}
		&\left( \begin{array}{cccc|c}
			1 & 1 & 1 & 1 & 2 \\
			0 & 2 & 0 & 6 & -1 \\
			0 & 1 & 1 & 4 & -1 \\
			0 & 2 & 0 & 12 & 0
		\end{array} \right) \overset{R_4 = R_4 - R_2}{\underset{R_3 = R_3 - 0.5R_2}\Rightarrow}
		\left( \begin{array}{cccc|c}
			1 & 1 & 1 & 1 & 2 \\
			0 & 2 & 0 & 6 & -1 \\
			0 & 0 & 1 & 1 & -0.5 \\
			0 & 0 & 0 & 6 & 1
		\end{array} \right) \\ \overset{R_2 = 0.5R_2}{\underset{R_4 = R_4/6}\Rightarrow}
		&\left( \begin{array}{cccc|c}
			1 & 1 & 1 & 1 & 2 \\
			0 & 1 & 0 & 3 & -0.5 \\
			0 & 0 & 1 & 1 & -0.5 \\
			0 & 0 & 0 & 1 & \frac{1}{6}
		\end{array} \right) \overset{R_2 = R_2 - 3R_4}{\underset{R_3 = R_3 - R_4}\Rightarrow}
		\left( \begin{array}{cccc|c}
			1 & 1 & 1 & 1 & 2 \\
			0 & 1 & 0 & 0 & -1 \\
			0 & 0 & 1 & 0 & -\frac{2}{3} \\
			0 & 0 & 0 & 1 & \frac{1}{6}
		\end{array} \right) \\ \overset{R_1 = R_1 - R_2 - R_3 - R_4}{\Rightarrow}
		&\left( \begin{array}{cccc|c}
			1 & 0 & 0 & 0 & \frac{7}{2} \\
			0 & 1 & 0 & 0 & -1 \\
			0 & 0 & 1 & 0 & -\frac{2}{3} \\
			0 & 0 & 0 & 1 & \frac{1}{6}
		\end{array} \right)
	\end{align*}
	ולכן קיבלנו: $\alpha_1 = \frac{7}{3}, \alpha_2 = -1, \alpha_3 = -\frac{2}{3}, \alpha_4 = \frac{1}{6}$ \\
	ומכיוון שאלו המקדמים של הצירוף נובע ש: $[r(x)]_B = \begin{pmatrix}
		\frac{7}{2} \\
		-1 \\
		 -\frac{2}{3} \\
		 \frac{1}{6}
	\end{pmatrix}$.
\end{document}
