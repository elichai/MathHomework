% !TEX program = xelatex
\def\NN{\mathbb{N}}
\def\RR{\mathbb{R}}
\def\ZZ{\mathbb{Z}}
\def\QQ{\mathbb{Q}}
\def\PP{\mathcal{P}}
\def\SS{\mathcal{S}}
\def\DD{\mathcal{D}}
\def\sub{\setminus}
\def\bld{\mathbf}
\def\lmf{\lim_{n \to \infty}}
% Make the ref command use parenthesis
\let\oldref\ref
\renewcommand{\ref}[1]{(\oldref{#1})}

\newcommand{\ontop}[1]{\overset{\text{#1}}}
\newcommand{\ang}[1]{\langle #1 \rangle}
\newcommand{\pip}[1]{\left| #1 \right|}



% style
\newcommand{\bm}[1]{\displaystyle{#1}}
\def\nl{$ $ \newline}

\ExplSyntaxOn

\NewDocumentCommand{\getenv}{om}
{
  \sys_get_shell:nnN{ kpsewhich ~ --var-value ~ #2 }{}#1
}

\ExplSyntaxOff

%environments

\documentclass{article}
\usepackage{amsmath,mathtools,enumerate,xparse,centernot,polyglossia,graphicx, aligned-overset}
\usepackage[utf8]{inputenc}
\usepackage[a4paper, margin=1.1in]{geometry}
\usepackage[]{amsthm} %lets us use \begin{proof}
\usepackage[makeroom]{cancel}
\usepackage[]{amssymb} %gives us the chA \mathcal{R} Acter \varnothing
\usepackage[normalem]{ulem} % [normalem] prevents the package from changing the default behavior of `\emph` to underline.
% \usepackage[frak=mma]{mathalfa}
\setdefaultlanguage{hebrew}
\setotherlanguage{english}
\usepackage{fontspec}
%\setmainfont{Frank Ruehl CLM}
\setmainfont{David CLM}
\setmonofont{Miriam Mono CLM}
\setsansfont{Simple CLM}
\DeclarePairedDelimiter\set\{\}
% Use the following if you only want to change the font for Hebrew
%\newfontfamily\hebrewfont[Script=Hebrew]{David CLM}
%\newfontfamily\hebrewfonttt[Script=Hebrew]{Miriam Mono CLM}
%\newfontfamily\hebrewfontsf[Script=Hebrew]{Simple CLM}
\getenv[\ID]{ID}
\graphicspath{ {./} }

\newtheorem{theorem}{Theorem}[section]
\newtheorem{corollary}{Corollary}[theorem]
\newtheorem{lemma}[theorem]{טענת עזר}

\DeclareMathOperator{\Img}{Im}
\title{תורת הקבוצות - ממ"ן 14}
\author{אליחי טורקל \ID}
\date\today

%\clearpage %Gives us a page break before the next section. Optional.
%\selectlanguage{english}
	%Section and subsection automatically number unless you put the asterisk next to them.

\begin{document}
	\maketitle %This command prints the title based on information entered above


	\section*{שאלה 1}
    תהי $\QQ^{+}$ קבוצת המספרים הרציונליים החיוביים עם יחס הסדר $\prec$ המוגדר כך:
    אם $a,b,c,d \in \NN \sub \set{0}$ והשברים $\frac{a}{b}, \frac{c}{d}$ רשומים בצורה מצומצמת,
    אז $\frac{a}{b} \prec \frac{c}{d}$ אם"ם $a < c$ או $a = c \land b < d$.

    \subsection*{סעיף ב}
    נתחיל דווקא מסעיף ב,
    מהו האיבר הראשון בקבוצה? מהם האיברים הגבוליים?

    \begin{proof}
        יהי $\frac{a}{b} \in \QQ^{+}$
        \begin{enumerate}
            \item אם $b > 1$ אזי $\frac{a}{1} \prec \frac{a}{b}$, מכיוון ש $\frac{a}{1}$ הינו שבר מצומצם, ו $a=a \land b >1$
            \item אם $b=1$ אזי כל $c,d \in \NN\set{0}, c < a$ מביא לנו ש $\frac{c}{d} \prec \frac{a}{b}$
        \end{enumerate}
        ובסה"כ קיבלנו שהאיבר היחיד שאין איברים הקטנים ממנו הוא $\frac{1}{1}$ מכיוון שלא קיים מספר הקטן מהמונה או שווה למונה ומספר הקטן מהמכנה. \\
        נשים לב גם כי כאשר $b > 1$ אז האיבר הקודם המיידי ל $\frac{a}{b}$ הוא $\frac{a}{d}$ כאשר $d < b$ כך ש $gcd(d,a) = 1$(שהוא זר למונה, ע"מ שיהיה שבר מצומצם)
        אך כאשר $b=1 \land a > 1$ אזי לכל $d \in \NN \sub \set{0}$ מתקיים $\frac{a-1}{d} \prec \frac{a}{b}$ ולכן הקודם המיידי צריך להיות ה $d$ המקסימלי, אך זו קבוצה אינסופית, ולכן זה איבר גבולי. \\
        ולסיכום קיבלנו כי $\frac{1}{1}$ הוא האיבר המינימלי, ו $\set{\frac{a}{1} \in \QQ^{+} | a > 1}$ היא קבוצת כל האיברים הגבוליים.
    \end{proof}


    \subsection*{סעיף א}
    הוכיחו כי $\ang{\QQ^{+}, \prec}$ היא קבוצה סדורה היטב בעלת הסודר $\omega \cdot \omega$
    \begin{proof}
        במילים אחרות, אם נסתכל על $\frac{a}{b} \in \QQ^{+}$ שבר מצומצם, אזי מדובר כמעט בסדר המילוני על $\ang{\NN,\NN}$. \\
        בסעיף ב הראינו כי יש איבר מינימום וכי יש $\aleph_0$ איברים גבוליים (כי לכל $n \in \NN, n > 1$ קיים איבר גבולי).
        ולכן אנו יודעים כי $\overline{\ang{\QQ^{+}, \prec}} \geq \omega \cdot \omega$.
        כעת נציג פונקציה חח"ע $f: \QQ^{+} \rightarrow \NN \times \NN$ השומרת סדר (כאשר הסדר בטווח הוא הסדר המילוני)
        \[ f\brac{\frac{a}{b}} = \ang{a, b} \]
        פונקציה זו היא חח"ע מכיוון שהיא ממפה את המונה לאיבר הימני והמכנה לאיבר השמאלי, ולכן ע"מ לקבל 2 זוגות סדורים שווים, המונה והמכנה של המקורות צריכים להיות שווים, וזה בעצם שוויון בין המקורות. \\
        פונקציה זו אינה על מכיוון שהתמונה שלה מכילה רק זוגות סדורים שהאיבר הימני והשמאלי זרים זה לזה. אך היא בוודאי על התמונה שלה. \\
        הגדרת הסדר $\prec$ על המקור הוא בדיוק אותה הגדרה כמו הסדר המילוני על $\NN \times \NN$ (אם הראשון גדול הוא קובע, אם שווה אז השני קובע).
        ולכן היא משמרת סדר מהמקור לתמונה שלה. ומכאן ש $\ang{\QQ^{+}, \prec}$ דומה ל $\ang{Im f, <}$(כאשר $<$ הוא הסדר המילוני).
        בנוסף $\ang{Im f, >}$ היא תת קבוצה סדורה של הקבוצה הסדורה $\ang{\NN \times \NN, <}$ וגם $\ang{\NN \times \NN, <} = \omega \cdot \omega$
        וקיבלנו כי $\overline{\ang{\QQ^{+}, \prec}} \geq \omega \cdot \omega$
        אזי לפי המסקנה בעמוד 55 קיבלנו כי $\overline{\ang{\QQ^{+}, \prec}} = \omega \cdot \omega$.
    \end{proof}

    \subsection*{סעיף ג}
    מצאו תת קבוצה סדורה של $\QQ^{+}$ ש $\prec$ מסדר אתה בסודר $\omega \cdot 2$
    \begin{proof}
        ניקח את תת הקבוצה שנוצרת מהרישא של האיבר הגבולי השני, ז"א $seg(3)$ קבוצה זו הינה סדורה היטב ונראית כך: $\set{\frac{a}{b} \in \QQ^{+} | b \neq 1 \lor a < 3}$
        היא מכילה איבר מינימלי, אחריו $\aleph_0$ איברים, איבר גבולי (2) ואחריו עוד $\aleph_0$ איברים, ולכן הסודר שלה הוא $\omega + \omega = \omega \cdot 2$
    \end{proof}


    \section*{שאלה 2}
    תהי $\ang{A, \prec}$ קבוצה סדורה היטב, ותהי $B \subseteq \NN \sub \set{0}$ תת קבוצה של המספרים הטבעיים החיוביים.
    תהי $f: A \rightarrow \NN$ בעלת התכונות הבאות:
    \begin{enumerate}
        \item אם $a$ העוקב המיידי של $b \in A$ אז $f(a) = 2 + f(b)$
        \item אחרת $f(a)$ הוא המספר הראשון בקבוצה $B \sub f[seg(a)]$, אם היא אינה ריקה.
        \item אחרת $f(a) = 0$
    \end{enumerate}

    \subsection*{סעיף א}
    הסבירו בקצרה, ע"י הדגמת הפונקציה $g$ שבמשפט 7.15 מדוע יש פונקציה כזו
    \begin{proof}
        תהי $h$ סדרה חלקית של איברים מ $A$ שמוגדרת כך: $h: seg(a) \rightarrow \NN, a \in A$.
        הפונקציה $g$ ממשפט 7.15 מוגדרת כך:
        \selectlanguage{english}
        \begin{align*}
            g(h) =
            \begin{cases}
                1 &\text{ if $a$ is the least element in $A$} \\
                2 + h(b) &\text{ if $a$ is the immediate successor of $b \in A$} \\
                \text{the least element in} B \sub f[seg(a)] &\text{ if $B \sub f[seg(a)] \neq \emptyset$ and $a$ is a limit element in $A$} \\
                0 &\text{otherwise}
            \end{cases}
        \end{align*}
        \selectlanguage{hebrew}
    \end{proof}





	\section*{שאלה 2}
	נגדיר את היחס $\prec$ על הקבוצה $\NN \times \NN$ כדלהלן:
	$\ang{m_1, n_1} \prec \ang{m_2, n_2}$ אם"א $m_1 < m_2$ או $n_1 > n_2 \land m_1 = m_2$

	\subsection*{סעיף א}
	הוכיחו כי $\prec$ הוא יחס סדר מלא על $\NN \times \NN$
	\begin{proof}
		\begin{enumerate}
			\item \textbf{טרנזיטיביות:} יהי $\ang{a,b}, \ang{c,d}, \ang{e,f} \in \NN \times \NN$ כך ש: $\ang{a,b} \prec \ang{c,d}$ וגם $\ang{c,d} \prec \ang{e,f}$, \\
			אם $c > a$ אזי $e \geq c$ ולכן $e > a$ ומכאן ש $\ang{a,b} \prec \ang{e,f}$, \\
			ואם $c=a$ אזי $b > d$ ואם $e > c$ אזי $e > a = c$ ולכן $\ang{a,b} \prec \ang{e,f}$
			אך אם $e=c=a$ אזי $b > d > f$ וקיבלנו ש $e=a \land b > f$ ולכן $\ang{a,b} \prec \ang{e,f}$.

			\item \textbf{אי רפלקסיביות} יהי $\ang{a,b} \in \NN \times \NN$ את התנאי הראשון הוא לא מקיים על עצמו מכיוון ש $a=a$ והתנאי השני גם לא מתקיים מכיוון ש $b \not> b$ ולכן $\ang{a,b} \not\prec \ang{a,b}$

			\item \textbf{משווה} יהי $\ang{a,b}, \ang{c,d} \in \NN \times \NN$ אם $c > a$ אזי $\ang{a,b} \prec \ang{c,d}$
			ואם $a > c$ אזי $\ang{c,d} \prec \ang{a,b}$ ואם $a=c$ וגם $b>d$ אזי $\ang{a,b} \prec \ang{c,d}$ ואם $b>d$ אזי $\ang{c,d} \prec \ang{a,b}$
			ואם $b=d$ וגם $a=c$ אזי $\ang{a,b} = \ang{c,d}$
		\end{enumerate}
	\end{proof}

	\subsection*{סעיף ב}
	הראו שבקבוצה הסדורה  $\ang{\NN \times \NN, \prec}$ אין איבר ראשון או אחרון, ולכל איבר בה יש קודם מיידי.
	\begin{proof}
		יהי $\ang{a,b} \in \NN \times \NN$ אזי $\ang{a,b} \prec \ang{a+1, b}$ (מכיוון ש $a < a+1$)
		ולכן $\ang{a,b}$ אינו אחרון. בנוסף $\ang{a, b+1} \prec \ang{a,b}$ (מכיוון ש $a=a \land b+1 > b$) ולכן $\ang{a,b}$ אינו ראשון.
		וקיבלנו שלא קיים איבר ראשון או אחרון בקבוצה הסדורה. \\
		נניח בשלילה ש $\ang{a, b+1}$ אינו הקודם המיידי ל $\ang{a,b}$ ז"א שקיים $\ang{x,y} \in \NN \times \NN$
		כך ש $\ang{a, b+1} \prec \ang{x,y} \prec \ang{a,b}$
		נשים לב שלפי הגדרת היחס או ש $m_1 < m_2$ או ש $m_1 = m_2$, ביחד בהכרח מתקיים $m_1 \leq m_2$
		ולכן בהכרח מתקיים $a \leq x \leq a$ ומכאן ש $x=a$.
		ולכן צריך להתקיים $b+1 > y > b$ אך מכיוון שמדובר במספרים טבעיים זה לא אפשרי, כי אין מספר טבעי בין $b+1$ ל $b$.
		ולכן בהכרח $\ang{a, b+1}$ הוא הקודם המיידי ל $\ang{a,b}$ ומכיוון שזה זוג סדור כלשהו, אזי לכל איבר יש קודם מיידי.
	\end{proof}

	\subsection*{סעיף ג}
	בדקו לאילו איברים של $\NN \times \NN$ יש עוקב מיידי, ולאילו אין.
	\begin{proof}
		מהסעיף הקודם קל לראות כי לכל $\ang{a,b} \in \NN \times \NN$ כאשר $b > 0$ יש עוקב מיידי שהוא $\ang{a, b-1}$
		נניח בשלילה של $\ang{a,0}$ יש עוקב מיידי שהוא $\ang{x,y} \in \NN \times \NN$
		ז"א $\ang{a,0} \prec \ang{x,y}$ ולכן או ש $a < x$ או ש $a=x \land 0 > y$ שלא יכול להתקיים מכיוון שלא קיים מספר טבעי הקטן מאפס.
		ולכן $a < x$ אך מכאן אנו מסיקים ש $\ang{a,0} \prec \ang{x,y+1} \prec \ang{x,y}$ ובעצם ישנם אינסוף איברים בין כל עוקב של $\ang{a,0}$ אליו, ולכן לא קיים לו עוקב מיידי.
	\end{proof}

	\subsection*{סעיף ד}
	מצאו תת-קבוצה סדורה של $\ang{\NN \times \NN, \prec}$ הדומה ל $\ang{\ZZ, <}$.
	\begin{proof}
		נסתכל על התת קבוצה הבאה: $A = \set{\ang{a,b} \in \NN \times \NN | a = 0 \lor b = 0}$,
		קל לראות כי $\ang{A, \prec}$ הינה קבוצה סדורה כי כל הטיעונים בסעיף א עובדים גם כאן.
		נוכיח שהיא דומה ל $\ZZ$ ע"י הצגת פונקציה על $f: \ZZ \rightarrow A$ המשמרת סדר:
		\[
		f(a) = \begin{cases}
			\ang{a, 0} \text{ if } x \geq 0 \\
			\ang{0, -a} \text{ if } x < 0 \\
		\end{cases}
		\]
		\begin{enumerate}
			\item יהיו $a, b \in \ZZ$ כך ש $0 \leq b < a$ ולכן $f(b) = \ang{b,0} \prec \ang{a,0} = f(a)$ מכיוון ש $b<a$
			\item יהיו $a, b \in \ZZ$ כך ש $b < a < 0$ ולכן $f(b) = \ang{0, -b} \prec \ang{0, -a} = f(a)$ מכיוון ש $0=0 \land -a < -b$.
			\item יהיו $a, b \in \ZZ$ כך ש $b < 0 < a$ ולכן $f(b) = \ang{0, -b} \prec \ang{a,0} = f(a)$ מכיוון ש $0<a$
		\end{enumerate}
		וקיבלנו שלפי שאלה 18 $f$ היא פונקציית דמיון מ $\ang{\ZZ, <}$ על $\ang{A, \prec}$.
	\end{proof}

	\subsection*{סעיף ה}
	האם $\ang{\NN \times \NN, \prec}$ ניחנה בתכונת החסם עליון?
	\begin{proof}
		נוכיח שהיא ניחנה בתכונה, יהי $A$ תת קבוצה חסומה מלעיל ולא ריקה של $\ang{\NN \times \NN, \prec}$
		ויהי $\ang{a,b} \in \NN \times \NN$ החסם העליון. נסתכל על האיבר הראשון בכל זוג סדור בקבוצה, ונקבל קבוצת מספרים טבעיים חסומה, ולכן יש לה איבר אחרון נסמן אותו $c \in \NN$
		וניווכח ש $c \leq a$ מכיוון ש $\ang{a,b}$ חסם עליון. \\
		בנוסף, נסתכל על הקבוצה $\set{x | \ang{x,c} \in A}$ גם זו קבוצה חסומה של מספרים טבעיים ולכן יש לה איבר ראשון, נסמן אותו $d \in \NN$
		וניווכח שבמידה ו $c=a$ אזי $b \leq d$.
		וקיבלנו ש $\ang{c,d} \in A$ והוא האיבר האחרון, כי לא קיים $\ang{x,y} \in A$ כך ש $x > c$ או $y < d \land x=c$.
	\end{proof}

	\subsection*{סעיף ו}
	מצאו פונקציית דמיון מ $\ang{\NN \times \NN, \prec}$ לתת קבוצה סדורה של $\ang{\QQ, <}$
	\begin{proof}
		נסתכל על התת קבוצה הבאה: $A = \set{a + \frac{1}{b+2} | a,b \in \NN}$ פשוט לראות כי $\ang{A, <}$ היא תת קבוצה סדורה של $\ang{\QQ, <}$,
		נציג פונקציה על $f: \NN \times \NN \rightarrow A$ שמשמרת את יחסי הסדר המתאימים:
		\[
		f(\ang{a,b}) = a + \frac{1}{b+2}
		\]
		יהיו $\ang{a,b}, \ang{c,d} \in \NN \times \NN$ כך ש $\ang{a,b} \prec \ang{c,d}$
		\begin{enumerate}
			\item אם $a < c$ אזי $f(\ang{a, b}) = a + \frac{1}{b+2} < a + 1 \leq c < c + \frac{1}{d+2} = f(\ang{c,d})$.
			\item אם $b > d \land a=c$ אזי $f(\ang{a,b}) = a + \frac{1}{b+2} = c + \frac{1}{b+2} < c + \frac{1}{d+2} = f(\ang{c,d})$.
		\end{enumerate}
		ובסה"כ קיבלנו ש $f(\ang{a,b}) < f(\ang{c,d})$ ולכן לפי שאלה 18 הפונקציה $f$ היא פונקציית דמיון מ $\ang{\NN \times \NN, \prec}$ על $\ang{A, <}$
	\end{proof}

	\pagebreak
	\section*{שאלה 3}
	תהי $\ang{A, \prec}$ קבוצה סדורה. הוכיחו כי $\ang{A, \prec}$ דומה לתת קבוצה סדורה של $\ang{\ZZ, <}$
	אם"ם כל תת קבוצה של $A$ החסומה ב $\ang{A, \prec}$ היא קבוצה סופית.
	\begin{proof}
		\begin{enumerate}
			\item כיוון אחד: תהי $\ang{A, \prec}$ קבוצה סדורה הדומה לתת קבוצה של $\ang{\ZZ, \prec}$ כל תת קבוצה חסומה של $\ZZ$ היא סופית, ולכן גם כל תת קבוצה סדורה חסומה של $A$ היא סופית.
			\item כיוון שני: תהי $\ang{A, \prec}$ קבוצה סדורה וכל תת קבוצה של היא סופית
			\begin{enumerate}
				\item \textbf{אם A ריקה:} אזי היא תת קבוצה של $\ZZ$ ודומה לעצמה.
				\item \textbf{A אינה ריקה אך אין לה ראשון/אחרון:} מההנחה שכל תת קבוצה חסומה שלה היא סופית מתקיים שלכל $a,b \in A$
				קבוצת האיברים החסומים ביניהם סופית, ולכן משאלה 25 נובע כי $\overline{\ang{A, \prec}} = \omega^{\downarrow} + \omega$, ז"א דומה ל $\ang{\ZZ, <}$

				\item \textbf{אם יש ב A ראשון ואין אחרון:} נסמן את הראשון ב $c \in A$ ולכל $x \in A$ מתקיים ש $\set{a \in A | a \prec x}$ חסומה ($c \leq a \prec x$) ולכן סופית, ולכן לפי משפט 6.17 נקבל ש: $\overline{\ang{A, \prec}} = \omega$ שנבלע בטיפוס הסדר של $\ZZ$ ולכן דומה לתת קבוצה של $\ZZ$
				\item \textbf{אם יש ב A אחרון ואין ראשון:} נסמן את האחרון ב $c \in A$ ולכל $x \in A$ מתקיים ש $\set{a \in A | x < a}$ חסומה ($x < a < c$) ולכן סופית וכל תנאי שאלה 24 מתקיימים ולכן: $\overline{\ang{A, \prec}} = \omega^{\downarrow}$ שנלע בטיפוס הסדר של $\ZZ$ ולכן דומה לתת קבוצה שלה.
				\item \textbf{יש ב A ראשון ואחרון:} אזי $A$ חסומה, וכמובן ש $A \subseteq A$ ולכן היא סופית ולכן בוודאי דומה לתת קבוצה של $\ZZ$ שבאותה העוצמה.
			\end{enumerate}
		\end{enumerate}
	\end{proof}

	\section*{שאלה 4}
	תהי $\ang{A, \prec}$ קבוצה סדורה בצפיפות שהיא בת מניה עם איבר אחרון וללא איבר ראשון.
	הוכיחו כי $\ang{A, \prec}$ דומה לקבוצה הסדורה $\ang{\QQ \cap [0, \infty), >}$
	\begin{proof}
		 יהי $a \in A$ האיבר האחרון ב $A$, נסמן:
		 $B = \QQ \cap [0, \infty)$, $\widehat{B} = B \sub \set{0} = \QQ \cap (0, \infty)$, $\widehat{A} = A \sub \set{a}$ \\
		 נסתכל על $\ang{\widehat{B}, >}$ לקבוצה זו אין ראשון ואחרון מכיוון שמדובר בקטע פתוח, והיא צפופה מכיוון שהרציונלים צפופים. בנוסף היא בת מניה כי מדובר בקבוצה אינסופית של רציונלים, ולכן ממשפט קנטור 6.18 נקבל כי $\overline{\ang{\widehat{B}, >}} = \eta$
		 בצורה דומה גם $\overline{\ang{\widehat{A}, \prec}}=\eta$ מכיוון שנתון לנו ש $A$ צפופה, ולכן גם $\widehat{A}$ צפופה, ובת מניה, וב $\widehat{A}$ אין איבר ראשון או אחרון, ולכן עומד בתנאי משפט קנטור. \\
		 מכאן נסיק שקיימת פונקציית דמיון $f: \widehat{A} \rightarrow \widehat{B}$ שהיא חח"ע ועל ונגדיר $g: A \rightarrow B$ פונקציית דמיון חח"ע בצורה הבאה:
		 $g(x) = \begin{cases}
			f(x) \text{ if } x \neq a \\
			0 \text{ if } x = a
		 \end{cases}$
		  ניתן לראות כי היא גם חח"ע ועל כי הוספנו איבר יחיד למקור ולתמונה,
		  והוא שומר סדר כי $f$ שומרת סדר ו $g$ שולחת את האיבר האחרון האחרון של $A$ לאיבר האחרון של $B$.

		  \section*{שאלה 5}
		  תהי $A$ תת קבוצה בת מניה ולא ריקה של קבוצת המספרים הממשיים $\RR$.
		  הוכיחו שהקבוצה הסדורה $\ang{\RR \sub A, <}$ אינה דומה לקבוצה הסדורה $\ang{\RR, <}$

		  \begin{proof}
			יהי $a \in A$ נסמן $B = (-\infty, a) \sub A \subset \RR$, $C = \RR \sub A$ קבוצה זו חסומה מלעיל אך נראה כי אין לה חסם עליון ב $C$: \\
			יהי $I = (x, y)$ מקטע ממשי כלשהו, נתחיל בלהראות ש $I \sub A \neq \emptyset^{\dagger}$ אם $I \sub A = \emptyset$ אזי $I \subseteq A$ אך $\pip{A} = \aleph_0 < \aleph = \pip{I}$ ולכן זה לא אפשרי. \\
			כעת נניח בשלילה כי $x$ הוא החסם העליון והוא נמצא ב $C$ אזי $x \neq a$ מכיוון ש $a \not\in C$
			אם $x < a$, נסתכל על $(x, a) \sub A$ קבוצה זו לא ריקה מתוך$^{\dagger}$ ולכן קיים $c \in (x, a) \sub A \subseteq C$ כך ש $x < c < a$ ולכן $x$ אינו חסם מלעיל כלל,
			לכן $x > a$, ומכיוון ש $x \neq a$ אזי קיים $c \in C \cap (a, x) \neq^{\dagger} \emptyset$ ומכאן ש $a < c < x$ ולכן גם $c$ הוא חסם מלעיל של $C$ אך הוא קטן מ $x$ בסתירה להיותו חסם עליון.
			ובכל האפשרויות הגענו לסתירה ולכן אין ל $C$ חסם עליון. \\
			ולכן היא לא עומדת בתנאי ד של משפט קנטור 6.19 ומכאן ש $\overline{\ang{C, <}} \neq \rho$ ומאידך $\overline{\ang{\RR, <}} = \rho$ ולכן הן אינן דומות.
		  \end{proof}

	\end{proof}
\end{document}
