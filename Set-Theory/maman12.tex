% !TEX program = xelatex
\def\NN{\mathbb{N}}
\def\RR{\mathbb{R}}
\def\PP{\mathcal{P}}
\def\sub{\setminus}

% Make the ref command use parenthesis
\let\oldref\ref
\renewcommand{\ref}[1]{(\oldref{#1})}



% style
\newcommand{\bm}[1]{\displaystyle{#1}}
\def\nl{$ $ \newline}

\ExplSyntaxOn

\NewDocumentCommand{\getenv}{om}
{
  \sys_get_shell:nnN{ kpsewhich ~ --var-value ~ #2 }{}#1
}

\ExplSyntaxOff

%environments

\documentclass{article}
\usepackage{amsmath,mathtools,enumerate,xparse,centernot,polyglossia,graphicx, aligned-overset}
\usepackage[utf8]{inputenc}
\usepackage[a4paper, margin=1.1in]{geometry}
\usepackage[]{amsthm} %lets us use \begin{proof}
\usepackage[makeroom]{cancel}
\usepackage[]{amssymb} %gives us the chA \mathcal{R} Acter \varnothing
\usepackage[normalem]{ulem} % [normalem] prevents the package from changing the default behavior of `\emph` to underline.
% \usepackage[frak=mma]{mathalfa}
\setdefaultlanguage{hebrew}
\setotherlanguage{english}
\usepackage{fontspec}
%\setmainfont{Frank Ruehl CLM}
\setmainfont{David CLM}
\setmonofont{Miriam Mono CLM}
\setsansfont{Simple CLM}
\DeclarePairedDelimiter\set\{\}
% Use the following if you only want to change the font for Hebrew
%\newfontfamily\hebrewfont[Script=Hebrew]{David CLM}
%\newfontfamily\hebrewfonttt[Script=Hebrew]{Miriam Mono CLM}
%\newfontfamily\hebrewfontsf[Script=Hebrew]{Simple CLM}
\getenv[\ID]{ID}
\graphicspath{ {./} }

\newtheorem{theorem}{Theorem}[section]
\newtheorem{corollary}{Corollary}[theorem]
\newtheorem{lemma}[theorem]{טענת עזר}

\DeclareMathOperator{\Img}{Im}
\title{תורת הקבוצות - ממ"ן 12}
\author{אליחי טורקל \ID}
\date\today

%\clearpage %Gives us a page break before the next section. Optional.
%\selectlanguage{english}
	%Section and subsection automatically number unless you put the asterisk next to them.

\begin{document}
	\maketitle %This command prints the title based on information entered above


	\section*{שאלה 1}
	$\set{A,B,C}$
	הראו שאם $A \neq \emptyset$ ויש תת קבוצה $B \subseteq A$ כך ש $\pip{A} = \pip{B} = \pip{A \sub B}$
	\begin{proof}
		אם $A = B$ אזי $A \sub B = \emptyset$ אך נתון ש $\pip{A} = \pip{A \sub B}$ ולכן גם $A = \emptyset$ בסתירה לנתון ש $A \neq \emptyset$.
		ולכן בהכרח $A \neq B$ ש $B \subset A$ ובעצם קיבלנו קבוצה ששקולה לתת קבוצה ממש שלה, ולכן לפי משפט נ.6 נקבל ש $A$ אינסופית.
	\end{proof}

	\section*{שאלה 2}
	תהיינה $A$ ו $B$ קבוצות אינסופיות. נסמן: ${\kappa_1} = \pip{A}$ ו ${\kappa_2} = \pip{B}$, הוכיחו או הפריכו:

	\subsection*{סעיף א}
	קבוצת הפונקציות מ $A$ ל $\PP(B)$ שקולה לקבוצת הפונקציות מ $B$ ל $\PP(A)$.
	\begin{proof}
		קבוצת הפונקציות מ $A$ ל $\PP(B)$ היא בעצם: $\PP(B)^A$ ועוצמתה היא $\pip{\PP(B)}^{\pip{A}}$. \\
		בצורה דומה קבוצת הפונקציות מ $B$ ל $\PP(A)$ היא $\PP(A)^B$ ועוצמתה היא $\pip{\PP(A)}^{\pip{B}}$.
		\begin{align*}
			&\pip{\PP(B)}^{\pip{A}} = (2^{\kappa_2})^{\kappa_1} \overset{5.15}= 2^{{\kappa_2} \cdot {\kappa_1}} \\
			&\pip{\PP(A)}^{\pip{B}} = (2^{\kappa_1})^{\kappa_2} \overset{5.15}= 2^{{\kappa_1} \cdot {\kappa_2}} \overset{5.11}= 2^{{\kappa_2} \cdot {\kappa_1}}
		\end{align*}
		וקיבלנו שלשתיהן יש את אותה העוצמה, ולכן הן שקולות זו לזו.
	\end{proof}

	\subsection*{סעיף ב}
	קבוצת הפונקציות מ $\PP(A)$ ל $B$ שקולה לקבוצת הפונקציות מ $\PP(B)$ ל $A$.
	\begin{proof}
	קבוצת הפונקציות מ $\PP(A)$ ל $B$ היא בעצם $B^{\PP(A)}$ וקבוצת הפונקציות מ $\PP(B)$ ל $A$ היא בעצם $A^{\PP(B)}$. \\
	נוכיח שהטענה לא נכונה ע"י דוגמא נגדית, נסמן: $A = \NN$ ו $B = \RR$:
	\begin{align*}
		&\pip{B^{\PP(A)}} = \pip{B}^{\pip{\PP(A)}} = {\pip{\RR}}^{\pip{\PP(\NN)}} = \aleph^{2^{\aleph_0}} = \aleph^{\aleph} = (2^{\aleph_0})^\aleph = 2^{\aleph_0 \cdot \aleph} = 2^\aleph = \aleph' \\
		&\pip{A^{\PP(B)}} = \pip{A}^{\pip{\PP(B)}} = \pip{\NN}^{\pip{\PP(\RR)}} = {\aleph_0}^{2^\aleph} = {\aleph_0}^{\aleph'} > \aleph'
	\end{align*}
	ולסיכום קיבלנו כי: $\pip{A^{\PP(B)}} > \aleph' = \pip{B^{\PP(A)}}$ ומכאן ש: $\pip{A^{\PP(B)}} > \pip{B^{\PP(A)}}$
	בסתירה למשפט.
	\end{proof}

	\pagebreak
	\section*{שאלה 3}
	תהי $f: \NN^{\NN} \rightarrow \NN^{\NN}$ פונקציה מקבוצת הסדרות של מספרים טבעיים אל קבוצת הסדרות של מספרים טבעיים המוגדרת כך:
	לכל סדרה $\ang{A_n | n \in \NN}$ נגדיר: $f(\ang{a_n | n \in \NN}) = \ang{b_n | n \in \NN}$
	כאשר $b_n = max\set{a_n, a_{n+1}}$ לכל $n \in \NN$. הוכיחו כי $\pip{Im  f} = \pip{\NN^{\NN}}$
	\begin{proof}
		התמונה של פונקציה תמיד מוכלת בטווח ולכן $Im  f \subseteq \NN^{\NN}$ ומכאן שגם $\pip{Im  f} \leq \pip{\NN^{\NN}}$, כעת נוכיח את הכיוון השני ע"י הצגת פונקציה חח"ע: \\
		יהי $g: \NN^{\NN} \rightarrow Im  f$ פונקציה המוגדרת כך: $g(\ang{a_n | n \in \NN}) = \ang{c_n | c_n = \sum_{k=0}^{n}}$
		ז"א הפונקציה מקבלת סדרת מספרים טבעיים, ומחזירה סדרה שהאיבר ה $k$ בה הוא סכום כל ה $k$ איברים הראשונים בסדרת המקור. \\
		ובכך זה מייצר סדרה עולה, ולכן סדרה זו בהכרח תהיה בתמונה של $f$. ז"א $Im  g \subseteq Im  f$. כעת נוכיח שהיא חח"ע: \\
		נניח בשלילה ש $g$ לא חח"ע, ז"א שקיימים $x_n, y_n \in \NN^{\NN}$ כך ש $x_n \neq y_n$ וגם $g(x_n) = g(y_n)$,
		ז"א ${\text{שלכל}^\dagger}$ $k > n$ מתקיים: $\sum_{i=0}^k x_i = \sum_{i=0}^k y_i$ נוכיח באינדוקציה שמכאן נובע שהסדרות שוות זו לזו: \\
		\textbf{בסיס האינדוקציה:} $k=0$: $\sum_{i=0}^0 x_i = x_0 = \sum_{i=0}^0 y_i = y_0$. \\
		\textbf{הנחת האינדוקציה:} נניח כי $\sum_{i=0}^k x_i = \sum_{i=0}^k y_i \rightarrow \forall i, x_i = y_i$ \\
		\textbf{צעד האינדוקציה:} נוכיח בעבור
		\begin{align*}
			\sum_{i=0}^{k+1} x_i = \sum_{i=0}^{k+1} y_i
			\Rightarrow \sum_{i=0}^{k+1} x_i + x_{k+1} = \sum_{i=0}^{k} y_i + y_{k+1}
			\overset{\dagger}\Rightarrow x_{k+1} = y_{k+1}
		\end{align*}
		וביחד קיבלנו כי הסדרות הללו בהכרח שוות בסתירה להנחה ש $x_n \neq y_n$ ולכן $g$ היא חח"ע. \\
		ומכאן ש $\pip{\NN^{\NN}} \geq \pip{Im f}$ וביחד קיבלנו כי: $\pip{Im  f} \leq \pip{\NN^{\NN}} \leq \pip{Im  f}$ ז"א $\pip{Im f} = \pip{\NN^{\NN}}$ כנדרש.
	\end{proof}


	\section*{שאלה 4}
	מצאו את עצמת קבוצת התת קבוצות הבנות מניה של מספרים ממשיים.
	\begin{proof}
		נסמן ב $A$ את הקבוצה הנ"ל. ונוכיח כל כיוון בנפרד:
		\begin{enumerate}
			\item $\boldsymbol{:\pip{A} \geq \aleph}$
			 נציג פונקציה חד חד ערכית $f: \RR \rightarrow A$ בצורה הבאה: $f(x) = \set{x}$, ז"א לכל $x \in \RR$ נחזיר את הקבוצה $\set{x}$ שהיא מוכלת ב $A$ מכיוון שהיא קבוצה בת מניה(איבר יחיד) של מספרים ממשיים.
			 \item $\boldsymbol{:\pip{A} \leq \aleph}$ נראה פונקציית על $f: \RR^{\NN} \rightarrow A$ בצורה הבאה: לכל $a \in A$, $a$ היא בעצם קבוצה בת מניה של ממשיים, ולכן קיימת $g \in \RR^{\NN}$ כך ש $g[\NN] = a$ ונגדיר $f(a) = g$ (היא קיימת מכיוון שהטבעיים הם בני מניה, וגם הקבוצה הזו בת מניה, ולכן קיימת פונקציה חח"ע  מהטבעיים לכל קבוצה בת מניה).
			 והוכנו כי $\aleph = \aleph^{\aleph_0} = \pip{\RR^{\NN}} \leq \pip{A}$ לפי שאלה 1 בפרק 5.
		\end{enumerate}
		וביחד קיבלנו כי: $\aleph \leq \pip{A} \leq \aleph$ ז"א $\pip{A} = \aleph$.
	\end{proof}


	\pagebreak
	\section*{שאלה 5}
	תהי $A$ תת קבוצה של $\RR$, מספר ממשי $x$ נקרא \dashuline{נקודת עיבוי} של $A$ אם"ם לכל $a, b \in \RR$ כך ש $a < x < b$ מתקיים $\pip{A \cap \brac{a,b}} > \aleph_0$.
	תהי $C$ קבוצת \dashuline{נקודות העיבוי} של $A$.
	\subsection*{סעיף א}
	הוכיחו או הפריכו: $C \subseteq A$.
	\begin{proof}
		נפריך את הטענה באצמעות דוגמא נגדית: $A = (0, 1) \sub \frac{1}{2}$.
		וניתן לראות כי $\frac{1}{2}$ היא נקודת עיבוי מכיוון שלכל $a, b \in \RR$ שעבורם מתקיים $a < \frac{1}{2} < b$
		מתקיים: $A \cap (a,b) = (max\set{0,a}, min\set{1, b}) \sub \frac{1}{2}$ ולכן: $\pip{A \cap (a,b)} = \aleph > \aleph_0$ (כי זה מקטע של מספרים ממשיים).
		ולכן $\frac{1}{2} \in C$ ומאידך $\frac{1}{2} \not\in A$, ז"א $C \not\subseteq A$.
	\end{proof}

	\subsection*{סעיף ב}
	הוכיחו או הפריכו: אם $A \subseteq B \subseteq \RR$ אז כל נקודת עיבוי של $A$ היא גם נקודת עיבוי של $B$
	\begin{proof}
		נוכיח את הטענה: יהיה $x$ נקודת עיבוי של $A$, אזי לכל $a,b \in \RR$ כך ש: $a < x < b$ מתקיים:
		$|A \cap (a,b)| > \aleph_0$, ולכל $z \in A \cap (a,b) \Rightarrow z \in A \land z \in (a,b)$  ומכיוון ש $A \subseteq B$ אזי $z \in B$ ז"א $z \in B \cap (a,b)$
		וקיבלנו ש: $\pip{B \cap (a,b)} \geq \pip{A \cap (a,b)} > \aleph_0$ ז"א $x$ היא גם נקודת עיבוי של $B$.
	\end{proof}

	\subsection*{סעיף ג}
	הוכיחו כי אם $A$ אינה בת מניה אז $A \cap C \neq \emptyset$.
	\begin{proof}
		נוכיח זו באמצעות סעיף ד,
		נניח בשלילה כי $A \cap C = \emptyset$ ומכאן ש: $A \sub C = A$ ולכן $A \sub C$ אינה בת מניה, בסתירה לסעיף ד.
		ומכאן ש $A \cap C \neq \emptyset$ כנדרש.
	\end{proof}

	\subsection*{סעיף ד}
	הוכיחו כי $A \sub C$ בת מניה.

	\begin{proof}
		יהי $x \in A \sub C$ ז"א $x$ אינה נקודת עיבוי, לפי שלילת ההגדרה נקבל שקיימים $a,b \in \RR$ כך ש $a < x < b$ ומתקיים $\pip{A \cap (a,b)} \leq \aleph_0$,
		מצפיפות הרציונליים בממשיים אנו יודעים כי קיימים $q,r \in \QQ$ כך ש $a < q < x < r < b$ ז"א $(q, r) \subseteq (a,b) \land x \in (q,r)$.
		ולכן מתקיים: $\pip{A \cap (q,r)} \leq \pip{A \cap (a,b)} \leq \aleph_0$.
		וקיבלנו שלכל $x$ שאינה נקודת עיבוי, קיים מקטע שנוכל לייצג באמצעות $\ang{q, r} \in \QQ \times \QQ$ שהחיתוך של $A$ איתו היא בת מניה.
		ומכיוון ש $\pip{\QQ \times \QQ} = \aleph_0 \cdot \aleph_0 = \aleph_0$ אזי כמות המקטעים הללו היא בת מניה. \\
		נסמן ב $V \subseteq \set{\ang{q,r} | \ang{q,r} \in \QQ \times \QQ}$ את קבוצת כל המקטעים הללו, ז"א שלכל $x$ שאינה נקודת עיבוי, קיימים $\ang{q,r} \in V$ כך ש $x \in \ang{q,r}$,
		ולכן: $A \sub C \subseteq \bigcup\limits_{v\in V}v$. וכעת נשים לב לעוצמות:
		\begin{align*}
			&A \sub C \subseteq \bigcup\limits_{v\in V}v \subseteq \QQ \times \QQ \\
			&\pip{A \sub C} \leq \pip{\bigcup\limits_{v\in V}v} \leq \pip{\QQ \times \QQ} = \aleph_0 \times \aleph_0 = \aleph_0 \\
		\end{align*}
		וקיבלנו ש: $\pip{A \sub C} \leq \aleph_0$ ז"א $A \sub C$ בת מניה כנדרש.
	\end{proof}


	\pagebreak
	\section*{שאלה 6}
	תהי $\kappa$ עוצמה שמקיימת: $2 \leq \kappa \leq \aleph'$ הוכיחו כי: $\kappa^{\aleph'} = 2^{2^{2^{\aleph_0}}}$
	\begin{proof}
		נפתח את העוצמות:
		\begin{align*}
			&2^{2^{2^{\aleph_0}}}
			= 2^{2^{\aleph}}
			= 2^{\aleph'} \\
			&\aleph'^{\aleph'}
			= {2^{\aleph}}^{\aleph'}
			= 2^{\aleph \cdot \aleph'}
			= 2^{\aleph'}
		\end{align*}
		מתוך הנתון ש: $2 \leq \kappa \leq \aleph'$ וממשפט 5.12 נקבל כי:
		\begin{align*}
			2^{\aleph'} \leq \kappa^{\aleph'} \leq \aleph'^{\aleph'} = 2^{\aleph'}
			\Rightarrow \kappa^{\aleph'} = 2^{\aleph'} = 2^{2^{2^{\aleph_0}}}
		\end{align*}
	\end{proof}

	\section*{שאלה 7}
	לכל $x \in \RR$ תהי $\kappa_x$ עוצמה. הוכיחו או הפריכו:

	\subsection*{סעיף א}
	אם לכל $x \in \RR$ מתקיים: $\kappa_x \leq \aleph$ אז $\prod\limits_{x \in \RR} \kappa_x \leq \aleph$
	\begin{proof}
		נפריח את המשפט, נסמן: $\kappa_x = \aleph$ לכל $x \in \RR$
		\begin{align*}
			\prod\limits_{x \in \RR} \kappa_x
			= \prod\limits_{x \in \RR} \aleph
			\overset{(5.20)} = \aleph^{\aleph}
			= {2^{\aleph_0}}^{\aleph}
			= 2^{\aleph_0 \cdot {\aleph}}
			= 2^{\aleph} = \aleph' > \aleph
		\end{align*}
	\end{proof}

	\subsection*{סעיף ב}
	אם לכל $x \in \RR$ מתקיים: $\kappa_x \leq \aleph'$ אז $\prod\limits_{x \in \RR} \kappa_x \leq \aleph'$

	\begin{proof}
		נוכיח את המשפט:
		\begin{align*}
			\prod\limits_{x \in \RR} \kappa_x
			\overset{5.18 \land \kappa_x \leq \aleph'}\leq \prod\limits_{x \in \RR} \aleph'
			\overset{(5.20)} = \aleph'^{\aleph}
			= {2^{\aleph}}^{\aleph}
			= 2^{\aleph\cdot {\aleph}}
			= 2^{\aleph}
			= \aleph'
		\end{align*}
	\end{proof}
\end{document}
